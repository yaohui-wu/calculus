\subsection{Other Convergence Tests}

\subsubsection*{Alternating Series}

An \textbf{alternating series} is a series whose terms are alternately
positive and negative.

If the terms of an alternating series decrease to 0 in absolute value, then
the series converges.
\begin{theorem}[Alternating Series Test]
    If the alternating series
    \[\sum_{n=1}^\infty (-1)^{n-1}b_n=b_1-b_2+b_3-b_4+b_5-b_6+\cdots,
    \quad b_n>0\]
    satisfies
    \[b_{n+1}\leq b_n\]
    for all \(n\) and
    \[\lim_{n\to\infty}b_n=0\]
    then the series is convergent.
\end{theorem}
\begin{theorem}[Alternating Series Estimation Theorem]
    If \(\displaystyle{s=\sum_{n=1}^\infty}(-1)^{n-1}b_n\) is the sum of an
    alternating series that satisfies
    \[0\leq b_{n+1}\leq b_n\]
    and
    \[\lim_{n\to\infty}b_n=0\]
    then
    \[|R_n|=|s-s_n|\leq b_{n+1}\]
\end{theorem}

\subsubsection*{Absolute Convergence}

\begin{definition}
    A series \(\displaystyle{\sum_{n=1}^\infty}a_n\) is called
    \textbf{absolutely convergent} if the series of absolute values
    \(\displaystyle{\sum_{n=1}^\infty}|a_n|\) is convergent.
\end{definition}
\begin{definition}
    A series \(\displaystyle{\sum_{n=1}^\infty}a_n\) is called
    \textbf{conditionally convergent} if it is convergent but not absolutely
    convergent.
\end{definition}
\begin{theorem}
    If a series \(\displaystyle{\sum_{n=1}^\infty}a_n\) is absolutely
    convergent, then it is convergent.
\end{theorem}

\subsubsection*{The Ratio Test}

\begin{theorem}[Ratio Test]
    \begin{enumerate}
        \item If
        \(\displaystyle{\lim_{n\to\infty}\left|\frac{a_{n+1}}{a_n}\right|}
        =L<1\),
        then the series \(\displaystyle{\sum_{n=1}^\infty}a_n\) is absolutely
        convergent (and therefore convergent).
        \item If
        \(\displaystyle{\lim_{n\to\infty}\left|\frac{a_{n+1}}{a_n}\right|}
        =L>1\)
        or
        \(\displaystyle{\lim_{n\to\infty}\left|\frac{a_{n+1}}{a_n}\right|}
        =\infty\),
        then the series \(\displaystyle{\sum_{n=1}^\infty}a_n\) is divergent.
        \item If
        \(\displaystyle{\lim_{n\to\infty}\left|\frac{a_{n+1}}{a_n}\right|}
        =1\),
        the Ratio Test is inconclusive; that is, no conclusion can be drawn
        about the convergence or divergence of
        \(\displaystyle{\sum_{n=1}^\infty}a_n\).
    \end{enumerate}
\end{theorem}

\begin{theorem}[Root Test]
    \begin{enumerate}
        \item If
        \(\displaystyle{\lim_{n\to\infty}\sqrt[n]{|a_n|}}=L<1\),
        then the series \(\displaystyle{\sum_{n=1}^\infty}a_n\) is absolutely
        convergent (and therefore convergent).
        \item If
        \(\displaystyle{\lim_{n\to\infty}\sqrt[n]{|a_n|}}=L>1\)
        or
        \(\displaystyle{\lim_{n\to\infty}\sqrt[n]{|a_n|}}
        =\infty\),
        then the series \(\displaystyle{\sum_{n=1}^\infty}a_n\) is divergent.
        \item If
        \(\displaystyle{\lim_{n\to\infty}\sqrt[n]{|a_n|}}=1\),
        the Root Test is inconclusive.
    \end{enumerate}
\end{theorem}