\subsection{Representing Functions as Power Series}

We start with the equation
\[\frac{1}{1-x}=1+x+x^2+x^3+\cdots=\sum_{n=0}^\infty x^n,\quad|x|<1\]
and find power series representations of similar functions.

\subsubsection*{Differentiation and Integration of Power Series}

We can differentiate and integrate a function represented by the sum of a
power series term by term.
This is called \textbf{term-by-term differentiation and integration}.
\begin{theorem}
    If the power series \(\displaystyle{\sum_{n=0}^\infty c_n(x-a)^n}\) has radius of convergence \(R>0\),
    then the function \(f\) defined by
    \[f(x)=c_0+c_1(x-a)+c_2(x-a)^2+\cdots=\sum_{n=0}^\infty c_n(x-a)^n\]
    is differentiable (and therefore continuous) on the interval
    \((a-R,a+R)\) and
    \[f'(x)=c_1+2c_2(x-a)+3c_3(x-a)^2+\cdots
    =\sum_{n=1}^\infty nc_n(x-a)^{n-1}\]
    and
    \[\int f(x)\,dx
    =C+c_0(x-a)+c_1\frac{(x-a)^2}{2}+c_2\frac{(x-a)^3}{3}+\cdots
    =C+\sum_{n=0}^\infty c_n\frac{(x-a)^{n+1}}{n+1}\]
    The radii of convergence of these power series are both \(R\).
\end{theorem}
In Leibniz's notation:
\[\frac{d}{dx}\left[\sum_{n=0}^\infty c_n(x-a)^n\right]
=\sum_{n=0}^\infty\frac{d}{dx}\big[c_n(x-a)^n\big]\]
\[\int\left[\sum_{n=0}^\infty c_n(x-a)^n\right]\,dx
=\sum_{n=0}^\infty\int c_n(x-a)^n\,dx\]
\[\ln(1+x)=x-\frac{x^2}{2}+\frac{x^3}{3}-\frac{x^4}{4}+\cdots
=\sum_{n=0}^\infty(-1)^{n-1}\frac{x^n}{n},\quad|x|<1,\quad R=1\]
\[\arctan x=x-\frac{x^3}{3}+\frac{x^5}{5}-\frac{x^7}{7}+\cdots
=\sum_{n=0}^\infty(-1)^n\frac{x^{2n+1}}{2n+1},\quad R=1\]