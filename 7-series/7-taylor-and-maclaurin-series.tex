\subsection{Taylor and Maclaurin Series}

\begin{theorem}
    If \(f\) has a power series expansion at \(a\), that is, if
    \[f(x)=\sum_{n=0}^\infty c_n(x-a)^n,\quad|x-a|<R\]
    then its coefficients are given by the formula
    \[c_n=\frac{f^{(n)}(a)}{n!}\]
\end{theorem}
The \textbf{Taylor series} of a function \(f\) centered at \(a\) is
\[f(x)=\sum_{n=0}^{\infty}\frac{f^{(n)}(a)}{n!}(x-a)^n
=f(a)+\frac{f'(a)}{1!}(x-a)+\frac{f''(a)}{2!}(x-a)^2+\frac{f'''(a)}{3!}(x-a)^2
+\cdots\]
For the special case \(a=0\), the Taylor series becomes the
\textbf{Maclaurin series}
\[f(x)=\sum_{n=0}^{\infty}\frac{f^{(n)}(0)}{n!}x^n
=f(0)+\frac{f'(0)}{1!}x+\frac{f''(0)}{2!}x^2+\cdots\]
\begin{problem}
    Find the Maclaurin series of the function \(f(x)=e^x\) and its radius of
    convergence.
\end{problem}
\begin{solution}
    Since \(f^{(n)}(x)=e^x\) and \(f^{(n)}(0)=e^0=1\) for all \(n\)
    therefore
    \[e^x=\sum_{n=0}^\infty\frac{x^n}{n!}
    =1+x+\frac{x^2}{2!}+\frac{x^3}{3!}+\cdots\]
    Let \(a_n=\dfrac{x^n}{n!}\), then
    \[\left|\frac{a_{n+1}}{a_n}\right|
    =\left|\frac{x^{n+1}}{(n+1)!}\frac{n!}{x^n}\right|=\frac{|x|}{n+1}\]
    and
    \[\lim_{n\to\infty}\frac{|x|}{n+1}=0<1\]
    so by the Ratio Test the series converges for all \(x\in\R\) and the
    radius of convergence is \(R=\infty\).
\end{solution}
The \(n\)th degree \textbf{Taylor polynomial} \(T_n\) of \(f\) at \(a\) is
\begin{align*}
    T_n(x) &= \sum_{k=0}^{n}\frac{f^{(k)}(a)}{k!}(x-a)^k \\
    &= f(a)+\frac{f'(a)}{1!}(x-a)+\frac{f''(a)}{2!}(x-a)^2+\cdots
    +\frac{f^{(n)}(a)}{n!}(x-a)^n
\end{align*}
In general, \(f(x)\) is the sum of its Taylor series if
\[f(x)=\lim_{n\to\infty}T_n(x)\]
Let
\[R_n(x)=f(x)-T_n(x)\]
so that
\[f(x)=T_n(x)+R_n(x)\]
then \(R_n(x)\) is the \textbf{remainder} of the Taylor series.
If \(\displaystyle{\lim_{n\to\infty}R_n(x)=0}\), then
\[\lim_{n\to\infty}T_n(x)=\lim_{n\to\infty}\Big[f(x)-R_n(x)\Big]=f(x)\]
\begin{theorem}
    If \(f(x)=T_n(x)+R_n(x)\) where \(T_n\) is the \(n\)th degree Taylor
    polynomial of \(f\) at \(a\) and
    \[\lim_{n\to\infty}R_n(x)=0\]
    for \(|x-a|<R\), then \(f\) is equal to the sum of its Taylor series on
    the interval \(|x-a|<R\).
\end{theorem}
\begin{theorem}[Taylor's Formula]
    If \(f\) has \(n+1\) derivatives in an interval \(I\) that contains the
    number \(a\), then for \(x\) in \(I\) there is a number \(z\) strictly
    between \(x\) and \(a\) such that the remainder term in the Taylor series
    is
    \[R_n(x)=\frac{f^{(n+1)}(z)}{(n+1)!}(x-a)^{n+1}\]
\end{theorem}
This expression for \(R_n(x)\) is
\textbf{Lagrange's form of the remainder term}.
\[\lim_{n\to\infty}\frac{x^n}{n!}=0\] for every real number \(x\in\R\).
\begin{problem}
    Prove that \(e^x\) is equal to the sum of its Taylor series.
\end{problem}
\begin{solution}
    If \(f(x)=e^x\), then \(f^{n+1}(x)=e^x\), so the remainder term in
    Taylor's Formula is
    \[R_n(x)=\frac{e^z}{(n+1)!}x^{n+1}\]
    where \(z\) is a number strictly between \(0\) and \(x\).
    Note that \(z\) depends on \(n\).
    If \(x>0\), then \(0<z<x\), so \(e^z<e^x\).
    Therefore
    \[0<R_n(x)=\frac{e^z}{(n+1)!}x^{n+1}<e^x\frac{x^{n+1}}{(n+1)!}\to 0\]
    so \(R_n(x)\to 0\) as \(n\to\infty\) by the Squeeze Theorem.
    If \(x<0\), then \(x<z<0\), so \(e^z<e^0=1\) and
    \[|R_n(x)|<\frac{|x|^{n+1}}{(n+1)!}\to 0\]
    Again \(R_n(x)\to 0\).
    Thus \(e^x\) is equal to the sum of its Taylor series.
\end{solution}
\[e=\sum_{n=0}^\infty\frac{1}{n!}=1+\frac{1}{1!}+\frac{1}{2!}+\frac{1}{3!}+\cdots\]

\[\sin x=x-\frac{x^3}{3!}+\frac{x^5}{5!}-\frac{x^7}{7!}+\cdots
=\sum_{n=0}^\infty(-1)^n\frac{x^{2n+1}}{(2n+1)!},\quad R=\infty\]

\[\cos x=1-\frac{x^2}{2!}+\frac{x^4}{4!}-\frac{x^6}{6!}+\cdots
=\sum_{n=0}^\infty(-1)^n\frac{x^{2n}}{(2n)!},\quad R=\infty\]

If \(k\) is any real number and \(|x|<1\), then the \textbf{binomial series}
is
\[(1+x)^k=\sum_{n=0}^\infty\binom{k}{n}x^n=1+kx+\frac{k(k-1)}{2!}x^2
+\frac{k(k-1)(k-2)}{3!}x^3+\cdots\]
where the \textbf{binomial coefficient} is
\[\binom{k}{n}=\frac{k(k-1)(k-2)\cdots(k-n+1)}{n!}\]
\begin{problem}
    Evaluate \(\displaystyle{\int e^{-x^2}}\,dx\) as an infinite series.
\end{problem}
\begin{solution}
    For all \(x\in\R\) we have
    \[e^{-x^2}=\sum_{n=0}^\infty\frac{(-x^2)^n}{n!}
    =\sum_{n=0}^\infty(-1)^n\frac{x^{2n}}{n!}
    =1-\frac{x^2}{1!}+\frac{x^4}{2!}-\frac{x^6}{3!}+\cdots\]
    Then we integrate term by term:
    \begin{align*}
        \int e^{-x^2}\,dx &= \int \left(1-\frac{x^2}{1!}+\frac{x^4}{2!}
        -\frac{x^6}{3!}+\cdots+(-1)^n\frac{x^{2n}}{n!}+\cdots\right)\,dx \\
        &= C+x-\frac{x^3}{3\cdot 1!}+\frac{x^5}{5\cdot 2!}
        +\frac{x^7}{7\cdot 3!}+\cdots+(-1)^n\frac{x^{2n+1}}{(2n+1)n!}+\cdots
    \end{align*}
    This series converges for all \(x\in\R\).
\end{solution}
Power series can be added, subtracted, multiplied, and divided like
polynomials.

\subsubsection*{Euler's Formula}

The set of all complex numbers \(\C\) is defined by
\[\C=\{z=a+bi\mid a,b\in\R,i^2=-1\}\]
where \(a\) and \(b\) are real numbers, and
\[i^2=-1\]
\begin{theorem}[Euler's Formula]
    \[e^{ix}=\cos x+i\sin x\]
\end{theorem}
\begin{proof}
    \begin{align*}
        e^x &= \sum_{n=0}^\infty\frac{x^n}{n!} \\
        e^{ix} &= \sum_{n=0}^\infty\frac{(ix)^n}{n!}
        =\sum_{n=0}^\infty\frac{(ix)^{2n}}{(2n)!}
        +\sum_{n=0}^\infty\frac{(ix)^{2n+1}}{(2n+1)!} \\
        &= \sum_{n=0}^\infty(i^2)^n\frac{x^{2n}}{(2n)!}
        +\sum_{n=0}^\infty i(i^2)^n\frac{x^{2n+1}}{(2n+1)!} \\
        &= \sum_{n=0}^\infty(-1)^n\frac{x^{2n}}{(2n)!}
        +i\sum_{n=0}^\infty(-1)^n\frac{x^{2n+1}}{(2n+1)!} \\
        &= \cos x+i\sin x
    \end{align*}
\end{proof}
\begin{theorem}[Euler's Identity]
    \[e^{i\pi}+1=0\]
\end{theorem}
\begin{proof}
    \begin{align*}
        e^{ix} &=\cos x+i\sin x \\
        e^{i\pi} &= \cos(\pi)+i\sin(\pi)=-1+i\cdot0=-1
        e^{i\pi}+1 &= 0
    \end{align*}
\end{proof}
This is the most beautiful equation in mathematics.