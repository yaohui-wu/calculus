\subsection{Applications of Taylor Polynomial}

\subsubsection*{Approximation Functions by Polynomials}

The Taylor polynomial of \(f\) at \(a\) can be used as an approximation to
\(f\):
\[f(x)\approx T_n(x)\]
The size of the error of the approximation is:
\[|R_n(x)|=|f(x)-T_n(x)|\]

\subsubsection*{Applications to Physics}

Physicist uses a Taylor polynomial as an approximation to the function.
\begin{problem}
    In Einstein's theory of special relativity the mass of an object moving
    with velocity \(v\) is
    \[m=\frac{m_0}{\sqrt{1-\dfrac{v^2}{c^2}}}\]
    where \(m_0\) is the mass of the object when at rest and \(c\) is the
    speed of light.
    The kinetic energy of the object is the difference between its total
    energy and its energy at rest:
    \[K=mc^2-m_0c^2\]
    Show that when \(v\) is very small compared with \(c\), this
    expression for \(K\) agrees with classical Newtonian physics:
    \[K=\frac{1}{2}m_0v^2\]
\end{problem}
\begin{solution}
    Using the expressions given for \(K\) and \(m\), we get
    \begin{align*}
        K &= mc^2-m_0c^2=\frac{m_0c^2}{\sqrt{1-\dfrac{v^2}{c^2}}}-m_0c^2 \\
        &= m_0c^2\left[\left(1-\frac{v^2}{c^2}\right)^{-1/2}-1\right]
    \end{align*}
    With \(x=-\dfrac{v^2}{c^2}\), then we have the binomial series with
    \(k=-\dfrac{1}{2}\).
    Notice that \(|x|<1\) because \(v<c\).
    Therefore we have
    \[(1+x)^{-1/2}=1-\frac{1}{2}x+\frac{3}{8}x^2-\frac{5}{16}x^3+\cdots\]
    and
    \[K=m_0c^2\left(\frac{1}{2}\cdot\frac{v^2}{c^2}
    +\frac{3}{8}\cdot\frac{v^2}{c^2}+\frac{5}{16}\cdot\frac{v^2}{c^2}\right)\]
    If \(v\) is much smaller than \(c\), then all terms after the first are
    very small when compared with the first term.
    If we omit them, we get
    \[K\approx m_0c^2\left(\frac{1}{2}\cdot\frac{v^2}{c^2}\right)
    =\frac{1}{2}m_0v^2\]
\end{solution}