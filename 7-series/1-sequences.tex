\subsection{Sequences}

An infinite \textbf{sequence} \(\{a_n\}\) is a list of numbers written in a
definite order:
\[a_1,a_2,a_3\dots,a_n,a_{n+1},\dots\]
where \(a_n\) is the \(n\)th term of the sequence.
Consider the sequence
\[a_n=\left\{\frac{1}{2},\frac{1}{4},\frac{1}{8},\frac{1}{16},\frac{1}{32},
\cdots\right\}\]
We can rewrite it as
\[a_n=\left\{\frac{1}{2},\frac{1}{2^2},\frac{1}{2^3},\frac{1}{2^4},
\frac{1}{2^5},\cdots\right\}\]
so the formula for the \(n\)th term is
\[a_n=\frac{1}{2^n}\]
for \(n=1,2,3,\cdots\) and
\[\{a_n\}=\left\{\frac{1}{2^n}\right\}_{n=1}^\infty\]
\begin{definition}
    A sequence \(\{a_n\}\) has the \textbf{limit} \(L\) and we write
    \[\lim_{n\to\infty}a_n=L\]
    or \(a_n\to L\) as \(n\to\infty\) if we can make the terms \(a_n\) as
    close to \(L\) as we like by taking \(n\) sufficiently large.
    If \(\displaystyle{\lim_{n\to\infty}a_n}\) exists, then the sequence is
    \textbf{convergent}, otherwise it is \textbf{divergent}.
\end{definition}
The precise definition of a limit of a sequence is:
\begin{definition}
    A sequence \(\{a_n\}\) has the \textbf{limit} \(L\) and we write
    \[\lim_{n\to\infty}a_n=L\]
    if for every \(\epsilon>0\) there is an corresponding integer \(N\) such
    that
    \[n>N\implies |a_n-L|<\epsilon\]
\end{definition}
\begin{theorem}
    If \(\displaystyle{\lim_{x\to\infty}f(x)=L}\) and \(f(n)=a_n\) when \(n\) is an integer,
    then \(\displaystyle{\lim_{n\to\infty}a_n=L}\). 
\end{theorem}
In particular, we have
\[\lim_{n\to\infty}\frac{1}{n^r}=0\]
when \(r>0\).
If \(a_n\) becomes large as \(n\) becomes large, we use the notation
\[\lim_{n\to\infty}a_n=\infty\]
\begin{definition}
    \[\lim_{n\to\infty}a_n=\infty\]
    means that for every positive number \(M\) there is a positive integer
    \(N\) such that
    \[n>N\implies a_n>M\]
\end{definition}
If \(\displaystyle{\lim_{n\to\infty}a_n=\infty}\) then the sequence
\(\{a_n\}\) is divergent and we say that \(\{a_n\}\) diverges to \(\infty\).
If \(a_n\) and \(b_n\) are convergent sequences and \(c\) is a constant, then
\begin{enumerate}
    \item
    \(\displaystyle{\lim_{n\to\infty}(a_n+b_n)
    =\lim_{n\to\infty}a_n+\lim_{n\to\infty}b_n}\)
    \item
    \(\displaystyle{\lim_{n\to\infty}(a_n-b_n)
    =\lim_{n\to\infty}a_n-\lim_{n\to\infty}b_n}\)
    \item \(\displaystyle{\lim_{n\to\infty}c\cdot a_n=c\lim_{n\to\infty}a_n}\)
    \item \(\displaystyle{\lim_{n\to\infty}c=c}\)
    \item
    \(\displaystyle{\lim_{n\to\infty}(a_n\cdot b_n)
    =\lim_{n\to\infty}a_n\cdot\lim_{n\to\infty}b_n}\)
    \item
    \(\displaystyle{\lim_{n\to\infty}\frac{a_n}{b_n}
    =\frac{\displaystyle{\lim_{n\to\infty}a_n}}
    {\displaystyle{\lim_{n\to\infty}b_n}}}\)
    if \(\displaystyle{\lim_{n\to\infty}b_n\neq 0}\).
    \item
    \(\displaystyle{\lim_{n\to\infty}a_n^p=\big[\lim_{n\to\infty}a_n\big]^p}\)
    if \(p>0\) and \(a_n>0\).
\end{enumerate}
\begin{theorem}
    If \(a_n\leq b_n\leq c_n\) for \(n\geq n_0\) and
    \(\displaystyle{\lim_{n\to\infty}a_n=\lim_{n\to\infty}c_n}=L\), then
    \(\displaystyle{\lim_{n\to\infty}b_n=L}\).
\end{theorem}
\begin{theorem}
    If \(\displaystyle{\lim_{n\to\infty}|a_n|=0}\), then
    \(\displaystyle{\lim_{n\to\infty}a_n=0}\).
\end{theorem}
\begin{theorem}[Continuity and Convergence Theorem]
    If \(\displaystyle{\lim_{n\to\infty}a_n=L}\) and the function \(f\) is
    continuous at \(L\), then
    \[\lim_{n\to\infty}f(a_n)=f(L)\]
\end{theorem}
The sequence \(\{r^n\}\) is convergent if \(-1<r\leq 1\) and divergent for all
other values of \(r\).
\[\lim_{n\to\infty}r^n=0\]
if \(-1<r<1\) and
\[\lim_{n\to\infty}r^n=1\]
if \(r=1\).
\begin{definition}
    A sequence \(\{a_n\}\) is \textbf{increasing} if \(a_n<a_{n+1}\) for all
    \(n\geq 1\), that is, \(a_1<a_2<a_3<\cdots\).
    it is \textbf{decreasing} if \(a_n>a_{n+1}\) for all \(n\geq 1\).
    A sequence is \textbf{monotonic} if it is either increasing or decreasing.
\end{definition}
\begin{definition}
    A sequence \(\{a_n\}\) is \textbf{bounded above} if there is a number
    \(M\) such that
    \[a_n\leq M\]
    for all \(n\geq 1\).
    It is \textbf{bounded below} if there is a number
    \(m\) such that
    \[m\leq a_n\]
    for all \(n\geq 1\).
    If it is bounded above and below, then it is a \textbf{bounded sequence}.
\end{definition}
\begin{theorem}[Monotonic Sequence Theorem]
    Every bounded, monotonic sequence is convergent.
\end{theorem}
The proof of the Monotonic Sequence Theorem is based on the
\textbf{Completeness Axiom} of the set of real numbers \(\R\).
The Completeness Axiom states if \(S\) is a nonempty set of real numbers that
has an upper bound \(M\) (\(x\leq M\) for all \(x\in S\)), then \(S\) has a
\textbf{least upper bound} \(b\).
(This means that \(b\) is an upper bound of \(S\),
but if \(M\) is any other upper bound, then \(b\leq M\).)
The Completeness Axiom is an expression of the fact that there is no gap or
hole in the real number line.
\begin{proof}
    Suppose \(\{a_n\}\) is an incerasing sequence.
    Since \(\{a_n\}\) is bounded, the set \\ \(S=\{a_n\mid n\geq 1\}\) has an
    upper bound.
    By the Completeness Axiom it has a least upper bound \(L\).
    Given \(\epsilon>0\), \(L-\epsilon\) is not an upper bound for \(S\)
    (since \(L\) is the least upper bound).
    Therefore
    \[a_N>L-\epsilon\]
    for some integer \(N\).
    But the sequence is incerasing so \(a_n\geq a_N\) for every \(n>N\).
    Thus if \(n>N\) we have
    \begin{align*}
        a_n &> L-\epsilon \\
        a_n+\epsilon &> L \\
        L &< a_n +\epsilon
    \end{align*}
    so
    \[0\leq L-a_n<\epsilon\]
    since \(a_n\leq L\).
    Thus
    \[|L-a_n|<\epsilon\]
    which implies that
    \[|a_n-L|<\epsilon\]
    whenever \(n>N\) so \(\displaystyle{\lim_{n\to\infty}a_n=L}\).
    A similar proof (using the greatest lower bound) works if \(\{a_n\}\) is
    decreasing.
\end{proof}
The proof of the Monotonic Sequence Theorem shows that an increasing sequence
that is bounded above is convergent and a decreasing sequence that is bounded
below is convergent.