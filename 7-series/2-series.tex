\subsection{Series}

In general, if we try to add the terms of an infinite sequence \(\{a_n\}\) we
get an expression of the form
\[a_1+a_2+a_3+\cdots+a_n+\cdots\]
which is called an \textbf{infinite series} (or just a \textbf{series}) and is
denoted by
\[\sum_{n=1}^\infty a_n\]
\begin{definition}
    Given a series
    \(\displaystyle{\sum_{n=1}^{\infty}a_n=a_1+a_2+a_3\cdots}\), let \(s_n\)
    be its \(n\)th partial sum:
    \[s_n=\sum_{k=1}^n a_i=a_1+a_2+\cdots+a_n\]
    If the sequence \(\{s_n\}\) is \textbf{convergent} and we write
    \(\displaystyle{\lim_{n\to\infty}s_n=s}\) exists as a real number, then the
    series \(\displaystyle{\sum_{n=1}^\infty a_n}\) is convergent and
    \[\sum_{n=1}^{\infty}a_n=a_1+a_2+\dots+a_n+\cdots=s\]
    The number \(s\) is the \textbf{sum} of the series.
    If the sequence \(\{s_n\}\) is divergent, then the series is
    \textbf{divergent}.
\end{definition}
Thus the sum of a series is the limit of the sequence of partial sums.
Notice that
\[\sum_{n=1}^\infty a_n=\lim_{n\to\infty}\sum_{k=1}^n a_i\]
Consider the \textbf{geometric series}
\[\sum_{n=1}^\infty ar^{n-1}=a+ar+ar^2+ar^3+\cdots+ar^{n-1}+\cdots,
\quad a\neq 0\]
with common ratio \(r\).

If \(r=1\), then \(s_n=a+a+\cdots+a=na\).
Thus \(\displaystyle{\lim_{n\to\infty}s_n=\pm\infty}\) so the limit does not
exist and the geometric series diverges when \(r=1\).

If \(r\neq 1\), then we have
\begin{align*}
    s_n &= a+ar+ar^2+\cdots+ar^{n-1} \\
    rs_n &= ar+ar^2+ar^3\cdots+ar^{n-1}+ar^n \\
    s_n-rs_n &= a-ar^n \\
    (1-r)s_n &= a(1-r^n) \\
    s_n &= \frac{a(1-r^n)}{1-r}
\end{align*}

If \(-1<r<1\), since \(\displaystyle{\lim_{n\to\infty}r^n}=0\), so
\[\lim_{n\to\infty}s_n=\lim_{n\to\infty}\frac{a(1-r^n)}{1-r}
=\frac{a}{1-r}-\frac{a}{1-r}\cdot\lim_{n\to\infty}r^n=\frac{a}{1-r}\]
Thus when \(|r|<1\) the geometric series is convergent and its sum is
\(\dfrac{a}{1-r}\).

If \(r\leq -1\) or \(r>1\), then the sequence \(\{r^n\}\) is divergent so
\(\displaystyle{\lim_{n\to\infty}s_n}\) does not exist thus the geometric
series diverges.

The geometric series
\[\sum_{n=1}^\infty ar^{n-1}=a+ar+ar^2+\cdots\]
is convergent if \(|r|<1\) and
its sum is
\[\sum_{n=1}^\infty ar^{n-1}=\frac{a}{1-r},\quad|r|<1\]
If \(|r|\geq 1\), then the geometric series is divergent.
\begin{problem}
    Find the sum of the series \(\displaystyle{\sum_{n=0}^\infty x^n}\), where
    \(|x|<1\).
\end{problem}
\begin{solution}
    Notice that
    \[\sum_{n=0}^\infty x^n=\sum_{n=1}^{\infty}x^{n-1}=1+x+x^2+x^3+\cdots\]
    This is the geometric series with \(a=1\) and \(r=x\).
    Since \(|r|=|x|<1\), it converges and gives
    \[\sum_{n=0}^\infty x^n=\frac{1}{1-x}\]
\end{solution}
\begin{problem}
    Show that the series \(\displaystyle{\sum_{n=1}^\infty\frac{1}{n(n+1)}}\)
    is convergent, and find its sum.
\end{problem}
\begin{solution}
    From the definition of convergent series we compute the partial sums.
    \[s_n=\sum_{k=1}^n\frac{1}{k(k+1)}
    =\frac{1}{1\cdot 2}+\frac{1}{2\cdot 3}+\frac{1}{3\cdot 4}+\cdots
    +\frac{1}{n(n+1)}\]
    We can simplify this expression if we use the partial fraction
    decomposition
    \[\frac{1}{k(k+1)}=\frac{1}{k}-\frac{1}{k+1}\]
    Thus we have
    \begin{align*}
        s_n &= \sum_{k=1}^n\frac{1}{k(k+1)}
        =\sum_{k=1}^n\left(\frac{1}{k}-\frac{1}{k+1}\right) \\
        &= \left(1-\frac{1}{2}\right)+\left(\frac{1}{2}-\frac{1}{3}\right)
        +\left(\frac{1}{3}-\frac{1}{4}\right)+\cdots
        +\left(\frac{1}{n}-\frac{1}{n+1}\right) \\
        &= 1-\frac{1}{n+1}
    \end{align*}
    Notice that the terms cancel in pairs.
    This series is an example of a \textbf{telescoping series}.
    Because of all the cancellations, the sum collapses into just two terms.
    Then
    \[\lim_{n\to\infty}s_n=\lim_{n\to\infty}\left(1-\frac{1}{n+1}\right)=1-0
    =1\]
    Therefore the given series is convergent and
    \[\sum_{n=1}^\infty\frac{1}{n(n+1)}=1\]
\end{solution}
\begin{problem}
    Show that the \textbf{harmonic series}
    \[\sum_{n=1}^\infty \frac{1}{n}=1+\frac{1}{2}+\frac{1}{3}+\frac{1}{4}
    +\cdots\]
    is divergent.
\end{problem}
\begin{solution}
    For this particular series it is convenient to consider the partial sums
    \[s_2,s_4,s_8,s_{16},s_{32},\dots\]
    and show that they become large.
    \begin{align*}
        s_2 &= 1+\frac{1}{2} \\
        s_4 &= 1+\frac{1}{2}+\left(\frac{1}{3}+\frac{1}{4}\right)
        >1+\frac{1}{2}+\left(\frac{1}{4}+\frac{1}{4}\right)=1+\frac{2}{2} \\
        s_8 &= 1+\frac{1}{2}+\left(\frac{1}{3}+\frac{1}{4}\right)
        +\left(\frac{1}{5}+\frac{1}{6}+\frac{1}{7}+\frac{1}{8}\right) \\
        &> 1+\frac{1}{2}+\left(\frac{1}{4}+\frac{1}{4}\right)
        +\left(\frac{1}{8}+\frac{1}{8}+\frac{1}{8}+\frac{1}{8}\right)
        =1+\frac{1}{2}+\frac{1}{2}+\frac{1}{2}=1+\frac{3}{2} \\
        s_{16} &= 1+\frac{1}{2}+\left(\frac{1}{3}+\frac{1}{4}\right)
        +\left(\frac{1}{5}+\cdots+\frac{1}{8}\right)
        +\left(\frac{1}{9}+\cdots+\frac{1}{16}\right) \\
        &> 1+\frac{1}{2}+\left(\frac{1}{4}+\frac{1}{4}\right)
        +\left(\frac{1}{8}+\cdots+\frac{1}{8}\right)
        +\left(\frac{1}{16}+\cdots+\frac{1}{16}\right) \\
        &= 1+\frac{1}{2}+\frac{1}{2}+\frac{1}{2}+\frac{1}{2}=1+\frac{4}{2}
    \end{align*}
    Similarly, \(s_{32}>1+\frac{5}{2},s_{64}>1+\frac{6}{2}\), and in general
    \[s_{2^n}>1+\frac{n}{2}\]
    This shows that \(s_{2^n}\to\infty\) as \(n\to\infty\) and so \(\{s_n\}\)
    is divergent.
    Therefore the harmonic series diverges.
\end{solution}

\begin{theorem}
    If the series \(\displaystyle{\sum_{n=1}^\infty a_n}\) is convergent, then
    \(\displaystyle{\lim_{n\to\infty}a_n=0}\).
\end{theorem}
\begin{proof}
    Let \(s_n=a_1+a_2+\cdots+a_n\).
    Then \(a_n=s_n-s_{n-1}\).
    Since \(\displaystyle{\sum_{n=1}^\infty a_n}\) is convergent, the sequence
    \(\{s_n\}\) is convergent.
    Let \(\displaystyle{\lim_{n\to\infty}s_n=s}\).
    Since \(n-1\to\infty\) as \(n\to\infty\), we also have
    \(\displaystyle{\lim_{n\to\infty}s_{n-1}=s}\).
    Therefore
    \[\lim_{n\to\infty}a_n=\lim_{n\to\infty}(s_n-s_{n-1})
    =\lim_{n\to\infty}s_n-\lim_{n\to\infty}s_{n-1}=s-s=0\]
\end{proof}
With any series \(\displaystyle{\sum_{n=1}^\infty a_n}\) we associate two
sequences: the sequence \(\{s_n\}\) of its partial sums and the sequence
\({a_n}\) of its terms.
If \(\displaystyle{\sum_{n=1}^\infty a_n}\) is convergent, then the limit of
the sequence \(\{s_n\}\) is \(s\) (the sum of the series) and the limit of the
sequence \(\{a_n\}\) is 0.
\begin{theorem}[Test for Divergence]
    If \(\displaystyle{\lim_{n\to\infty}a_n}\) does not exist or if
    \(\displaystyle{\lim_{n\to\infty}a_n\neq 0}\), then the series
    \(\displaystyle{\sum_{n=1}^\infty a_n}\) is divergent.
\end{theorem}
The Test of Divergence works because if the series is not divergent, then it
is convergent, and so \(\displaystyle{\lim_{n\to\infty}a_n=0}\).
\begin{theorem}
    If \(\displaystyle{\sum_{n=1}^\infty a_n}\) and
    \(\displaystyle{\sum_{n=1}^\infty b_n}\) are convergent series, then so
    are the following series:
    \begin{enumerate}
        \item
        \(\displaystyle{\sum_{n=1}^\infty c\cdot a_n
        =c\sum_{n=1}^\infty a_n}\) where \(c\) is a constant.
        \item
        \(\displaystyle{\sum_{n=1}^\infty(a_n+b_n)
        =\sum_{n=1}^\infty a_n+\sum_{n=1}^\infty b_n}\)
        \item
        \(\displaystyle{\sum_{n=1}^\infty(a_n-b_n)
        =\sum_{n=1}^\infty a_n-\sum_{n=1}^\infty b_n}\)
    \end{enumerate}
\end{theorem}
A finite number of terms does not affect the convergence or divergence of a
series.
If it is known that the series \(\displaystyle{\sum_{n=N+1}^\infty}a_n\)
converges, then the full series
\[\sum_{n=1}^\infty a_n=\sum_{n=1}^N a_n+\sum_{n=N+1}^\infty a_n\]
is also convergent.