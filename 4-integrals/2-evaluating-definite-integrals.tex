\subsection{Evaluating Definite Integrals}

\begin{theorem}[Evaluation Theorem]
    If \(f\) is continuous on the interval \([a,b]\), then
    \[\int_a^b f(x)\,dx=F(b)-F(a)\]
    where \(F\) is any antiderivative of \(f\), that is, \(F'=f\).
\end{theorem}
\begin{proof}
    We divide the interval \([a,b]\) into \(n\) subintervals with endpoints \\
    \(x_0=a,x_1,x_2,\cdots,x_n=b\) and with length
    \(\Delta x=\dfrac{b-a}{n}\).
    Let \(F\) be any antiderivative of \(f\).
    Then
    \begin{align*}
        & F(b)-F(a) \\
        &= F(x_n)-F(x_0) \\
        &= F(x_n)-F(x_{n-1})+F(x_{n-1})-F(x_{n-2})+\cdots+F(x_2)-F(x_1)+F(x_1)
        -F(x_0) \\
        &= \sum_{i=1}^n\big[F(x_i)-F(x_{i-1})\big]
    \end{align*}
    Now \(F\) is continuous because it's differentiable and so we can apply
    Lagrange's Mean Value Theorem to \(F\) on each subinterval
    \([x_{i-1},x_i]\).
    Thus there exists a number \(x_i^*\) between \(x_{i-1}\) and \(x_i\) such
    that
    \[F(x_i)-F(x_{i-1})=F'(x_i^*)(x_i-x_{i-1})=f(x_i^*)\Delta x\]
    Therefore
    \[F(b)-F(a)=\sum_{i=1}^n f(x_i^*)\Delta x\]
    Now we take the limit of each side of this equation as \(n\to\infty\).
    Then
    \[F(b)-F(a)=\lim_{n\to\infty}\sum_{i=1}^n f(x_i^*)\Delta x
    =\int_a^b f(x)\,dx\]
\end{proof}
When applying the Evaluation Theorem we use the notation
\[\int_a^b f(x)\,dx=F(b)-F(a)=\big[F(x)\big]_a^b=F(x)\Big|_a^b\]

\subsubsection*{Indefinite Integrals}
The notation
\[\int f(x)\,dx\]
is used for an antiderivative of \(f\) and is called an
\textbf{indefinite intergral}.
Thus
\[\int f(x)\,dx=F(x)\]
means that
\[F'(x)=f(x)\]

\subsubsection*{Applications}
\begin{theorem}[Net Change Theorem]
    The integral of a rate of change is the net change:
    \[\int_a^b F'(x)\,dx=F(b)-F(a)\]
\end{theorem}