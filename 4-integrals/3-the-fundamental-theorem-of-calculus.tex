\subsection{The Fundamental Theorem of Calculus}
\begin{theorem}[The Fundamental Theorem of Calculus (Newton-Leibniz Theorem)]
    Suppose \(f\) is continuous on \([a,b]\).
    If the function \(F\) is defined by
    \[F(x)=\int_a^xf(t)\,dt,\quad a\leq x\leq b\]
    then \(F\) is an antiderivative of \(f\) and \[F'(x)=f(x),\quad a<x<b\]
    and therefore\[\frac{d}{dx}\int_a^x f(t)\,dt=f(x)\]
    If \(F\) is an antiderivative of \(f\) such that \(F'=f\), then
    \[\int_a^b f(x)\,dx=F(b)-F(a)\qedhere\]
\end{theorem}
The Fresnel integrals named after \textbf{Augustin-Jean Fresnel} (1788--1827)
are
\begin{align*}
    S(x) &= \int_0^x\sin(t^2)\,dt & C(x) &= \int_0^x\cos(t^2)\,dt
\end{align*}
and by the fundamental theorem of calculus the derivatives are
\begin{align*}
    S'(x) &= \sin(x^2) & C'(x) &= \cos(x^2)
\end{align*}