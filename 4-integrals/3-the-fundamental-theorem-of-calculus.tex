\subsection{The Fundamental Theorem of Calculus}

The \textbf{Fundamental Theorem of Calculus} establishes a connection between the two \\
branches of calculus: differential calculus and integral calculus.
It says that differentiation and integration are inverse processes and gives
the precise inverse relationship between the derivative and the integral.
The first part of the Fundamental Theorem of Calculus deals with functions
defined by an equation of the form
\[g(x)=\int_a^x f(t)\,dt\]
where \(f\) is a continuous function on \([a,b]\) and \(x\) varies between
\(a\) and \(b\).
\begin{theorem}[Fundamental Theorem of Calculus Part 1]
    If \(f\) is continuous on \([a,b]\), then the function \(g\) defined by
    \[g(x)=\int_a^x f(t)\,dt,\quad a\leq x\leq b\]
    is an antiderivative of \(f\), that is, \(g'(x)=f(x)\) for \(a<x<b\).
\end{theorem}
\begin{proof}
    If \(x\) and \(x+h\) are in the open interval \((a,b)\), then
    \begin{align*}
        g(x+h)-g(x) &= \int_a^{x+h}f(t)\,dt-\int_a^x f(t)\,dt \\
        &= \left(\int_a^x f(t)\,dt+\int_x^{x+h}f(t)\,dt\right)
        -\int_a^x f(t)\,dt \\
        &= \int_x^{x+h}f(t)\,dt
    \end{align*}
    and so, for \(h\neq 0\),
    \[\frac{g(x+h)-g(x)}{h}=\frac{1}{h}\int_x^{x+h}f(t)\,dt\]
    Assume that \(h>0\).
    Since \(f\) is continuous on \([x,x+h]\), the Extreme Value Theorem says
    that there are numbers \(u\) and \(v\) in \([x,x+h]\) such that \(f(u)=m\)
    and \(f(v)=M\), where \(m\) and \(M\) are the absolute minimum and the
    absolute maximum values of \(f\) in \([x,x+h]\).
    Then
    \[m\cdot h\leq\int_x^{x+h}f(t)\,dt\leq M\cdot h\]
    that is,
    \[f(u)\cdot h\leq\int_x^{x+h}f(t)\,dt\leq f(v)\cdot h\]
    Since \(h>0\), we have
    \[f(u)\leq\frac{1}{h}\int_x^{x+h}f(t)\,dt\leq f(v)\]
    Thus
    \[f(u)\leq\frac{g(x+h)-g(x)}{h}\leq f(v)\]
    This inequality can be proved in a similar manner for the case \(h<0\).
    Let \(h\to 0\).
    Then \(u\to x\) and \(v\to x\), since \(u\) and \(v\) lie between \(x\)
    and \(x+h\).
    Thus
    \[\lim_{h\to 0}f(u)=\lim_{u\to x}f(u)=f(x)\]
    and
    \[\lim_{h\to 0}f(v)=\lim_{v\to x}f(v)=f(x)\]
    because \(f\) is continuous at \(x\).
    By the Squeeze Theorem,
    \[g'(x)=\lim_{h\to 0}\frac{g(x+h)-g(x)}{h}=f(x)\]
    If \(x=a\) or \(b\), then we can interpret this equation as a one-sided
    limit.
    Since \(g\) is differentiable, this shows that  \(g\) is continuous on
    \([a,b]\).
\end{proof}
Using Leibniz notation for derivatives, we can write the Fundamental Theorem
of Calculus Part 1 as
\[\frac{d}{dx}\int_a^x f(t)\,dt=f(x)\]
By the Fundamental Theorem of Calculus Part 1, the derivative of the Fresnel
function
\[S(x)=\int_0^x \sin\left(\frac{\pi t^2}{2}\right)\,dt\]
is
\[S'(x)=\sin\left(\frac{\pi x^2}{2}\right)\]
\begin{problem}
    Find \(\displaystyle{\frac{d}{dx}\int_1^{x^4}\sec t\,dt}\).
\end{problem}
\begin{solution}
    We use the Chain Rule and the Fundamental Theorem of Calculus Part 1.
    Let \(u=x^4\).
    Then
    \[\frac{d}{dx}\int_1^{x^4}\sec t\,dt
    =\frac{d}{du}\int_1^u\sec t\,dt\cdot\frac{du}{dx}
    =\sec u\cdot\frac{du}{dx}=\sec (x^4)\cdot 4x^3\]
\end{solution}

\subsubsection*{Differentiation and Integration as Inverse Processes}
\begin{theorem}[Fundamental Theorem of Calculus]
    Suppose \(f\) is continuous on \([a,b]\).
    \begin{enumerate}
        \item If
        \[g(x)=\int_a^x f(t)\,dt\]
        then
        \[g'(x)=f(x)\]
        \item
        \[\int_a^b f(x)\,dx=F(b)-F(a)\]
        where \(F\) is any antiderivative of \(f\), that is, \(F'=f\).
    \end{enumerate}
\end{theorem}
The Fundamental Theorem of Calculus is the most important theorem in calculus.
It is one of the greatest accomplishments of the human mind.

\subsubsection*{Average Value of a Function}
We define the \textbf{average value of} \(f\) on the interval \([a,b]\) as
\[\overline{f}=\frac{1}{b-a}\int_a^b f(x)\,dx\]\
\begin{theorem}[Mean Value Theorem for Integrals]
    If \(f\) is continuous on \([a,b]\), then there exists a number \(c\) in
    \([a,b]\) such that
    \[f(c)=\overline{f}=\frac{1}{b-a}\int_a^b f(x)\,dx\]
    that is,
    \[\int_a^b f(x)\,dx=f(c)(b-a)\]
\end{theorem}
\begin{proof}
    Let \(\displaystyle{F(x)=\int_a^x f(t)\,dt}\) for \\
    \(a\leq x\leq b\).
    By Lagrange's Mean Value Theorem for derivatives, there is a number \(c\)
    between \(a\) and \(b\) such that
    \[F(b)-F(a)=F'(c)(b-a)\]
    But \(F'(x)=f(x)\) by the Fundamental Theorem of Calculus Part 1.
    Therefore
    \[\int_a^b f(t)\,dt-0=f(c)(b-a)\]
\end{proof}