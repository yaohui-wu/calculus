\subsection{The Definite Integral}

\subsubsection*{The Area Problem}
\begin{definition}
    The area \(A\) of the region \(S\) that lies under the graph of the
    continuous function \(f\) is the limit of the sum of the areas of the
    approximating rectangles:
    \[A=\lim_{n\to\infty}R_n
    =\lim_{n\to\infty}\big[f(x_1)\Delta x+f(x_2)\Delta x+\cdots
    +f(x_n)\Delta x\big]\]
\end{definition}

\subsubsection*{The Distance Problem}
In general, suppose an object moves with velocity \(v=f(t)\), where
\(a\leq t\leq b\), and \(f(t)>0\).
The exact total distance \(d\) traveled is
\[d=\lim_{n\to\infty}\sum_{i=1}^n f(t_i)\Delta t\]

\subsubsection*{The Definite Integral}
In general, we start with any function \(f\) defined on \([a,b]\) and we
divide \([a,b]\) into \(n\) smaller subintervals by choosing partition points
\(x_0,x_1,x_2,\dots,x_n\) so that
\[a=x_0<x_1<x_2<\dots<x_{n-1}<x_n=b\]
The resulting collection of subintervals
\[[x_0,x_1],[x_1,x_2],[x_2,x_3],\dots,[x_{n-1},x_n]\]
is called a \textbf{partition} \(P\) of the \([a,b]\).
We use the notation \(\Delta x_i\) for the length of the \(i\)th subinterval
\([x_{i-1},x_i]\).
Thus
\[\Delta x_i=x_i-x_{i-1}\]
Then we choose \textbf{sample points} \(x_1^*,x_2^*,\dots,x_n^*\) in the
subintervals with \(x_i^*\) in the \(i\)th subinterval \([x_{i-1},x_i]\).
A \textbf{Riemann sum} associated with a partition \(P\) and a function \(f\)
is
\[\sum_{i=1}^n f(x_i^*)\Delta x_i=f(x_1^*)\Delta x_1+f(x_2^*)\Delta x_2
+\dots+f(x_n^*)\Delta x_n\]
\begin{definition}
    If \(f\) is a function defined on \([a,b]\),
    the \textbf{definite integral of} \(f\) \textbf{from} \(a\) \textbf{to}
    \(b\) is the number
    \[\int_a^b f(x)\,dx=\lim_{\max\Delta x_i\to 0}
    \sum_{i=1}^n f(x_i^*)\Delta x_i\]
    provided that this limit exists.
    If it does exist, we say that \(f\) is \textbf{integrable} on \([a,b]\).
\end{definition}
\begin{definition}
    \[\int_a^b f(x)\,dx=I\]
    means that for every \(\epsilon>0\) there is a corresponding number
    \(\delta>0\) such that
    \[\left\lvert I-\sum_{i=1}^n f(x_i^*)\Delta x_i\right\rvert<\epsilon\]
    for all partitions \(P\) of \([a,b]\) with \(\max\Delta x_i<\delta\) and
    for all possible choices of \(x_i^*\) in \([x_{i-1},x_i]\).
\end{definition}
The symbol \(\displaystyle{\int}\) was introduced by Leibniz and is called an
\textbf{integral sign}.
In the notation \(\displaystyle{\int_a^b f(x)\,dx}\), \(f(x)\) is called the
\textbf{integrand} and \(a\) and \(b\) are called the
\textbf{limits of integration}; \(a\) is the \textbf{lower limit} and \(b\) is the
\textbf{upper limit}.
The symbol \(dx\) indicates that the independent variable is \(x\).
The procedure of calculating an integral is called \textbf{integration}.
The definite integral is a number so it does not depend on \(x\).
In fact, we can use any letter in place without changing the value of the
integral:
\[\int_a^b f(x)\,dx=\int_a^b f(y)\,dy=\int_a^b f(t)\,dt\]
\begin{theorem}
    If \(f\) is continuous on \([a,b]\), or if \(f\) has only a finite number
    of jump discontinuities, then \(f\) is integrable on \([a,b]\); that is,
    the definite integral \(\displaystyle{\int_a^b f(x)\,dx}\) eixsts.
\end{theorem}
\begin{theorem}
    If \(f\) is integrable on \([a,b]\), then
    \[\int_a^b f(x)\,dx=\lim_{n\to\infty}\sum_{i=1}^nf(x_i)\Delta x\]
    where \(\Delta x=\dfrac{b-a}{n}\) and \(x_i=a+i\Delta x\).
\end{theorem}
A definite integral can be interpreted as a \textbf{net area}, that is, a
difference of areas:
\[\int_a^b f(x)\,dx=A_1-A_2\]

\subsubsection*{The Midpoint Rule}
The Midpoint Rule:
\[\int_a^b f(x)\,dx\approx \sum_{i=1}^n f(\overline{x}_i)\Delta x
=\Delta x\big[f(\overline{x}_1)+\dots+f(\overline{x}_n)\big]\]
where \(\Delta x=\dfrac{b-a}{n}\) and
\(\overline{x}_i=\dfrac{1}{2}(x_{i-1}+x_i)\).

\subsubsection*{Properties of The Definite Integrals}
If \(a>b\), then
\[\int_b^a f(x)\,dx=-\int_a^b f(x)\,dx\]
If \(a=b\), then
\[\int_a^a f(x)\,dx=0\]
Properties of the Integral: Suppose all of the following integrals exist.
\begin{enumerate}
    \item \(\displaystyle{\int_a^b c\,dx=c(b-a)}\), where \(c\) is any
    constant.
    \item \(\displaystyle{\int_a^b\big[f(x)+g(x)\big]\,dx
    =\int_a^b f(x)\,dx+\int_a^b g(x)\,dx}\)
    \item \(\displaystyle{\int_a^b cf(x)\,dx=c\int_a^b f(x)\,dx}\), where
    \(c\) is any constant.
    \item \(\displaystyle{\int_a^b\big[f(x)-g(x)\big]\,dx
    =\int_a^b f(x)\,dx-\int_a^b g(x)\,dx}\)
\end{enumerate}
\[\int_a^c f(x)\,dx+\int_c^b f(x)\,dx=\int_a^b f(x)\,dx\]
Comparison Properties of the Integral:
\begin{enumerate}
    \item If \(f(x)\geq 0\) for \(a\leq x\leq b\),
    then \(\displaystyle{\int_a^b f(x)\,dx\geq 0}\).
    \item If \(f(x)\geq g(x)\) for \(a\leq x\leq b\),
    then \(\displaystyle{\int_a^b f(x)\,dx\geq \int_a^b g(x)\,dx}\).
    \item If \(m\leq f(x)\leq M\) for \(a\leq x\leq b\),
    then
    \[m(b-a)\leq\int_a^b f(x)\,dx\leq M(b-a)\]
\end{enumerate}