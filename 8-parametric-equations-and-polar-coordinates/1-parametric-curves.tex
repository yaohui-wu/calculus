\subsection{Parametric Curves}

Suppose that \(x\) and \(y\) are both given as functions of a third variable
\(t\) (called a \textbf{parameter}) by the equations
\begin{align*}
    x &= f(t) & y &= g(t)
\end{align*}
(called \textbf{parametric equations}).
As \(t\) varies, the point \((x,y)=(f(t),g(t))\) varies and traces out a curve
\(C\) which we call a \textbf{parametric curve}.

In general, the curve with parametric equations
\begin{align*}
    x &= f(t) & y &= g(t) & a\leq t\leq b
\end{align*}
has \textbf{initial point} \((f(a),g(a))\) and \textbf{terminal point}
\((f(b),g(b))\).

The parametric equations tell us the position of a particle at time \(t\).
They also indicate the direction of the motion.

The parametric equation for the circle with center \((h,k)\) and radius \(r\)
is
\begin{align*}
    x &= h+r\cos t & y &= k+r\sin t & 0\leq t\leq 2\pi
\end{align*}

\subsubsection*{The Cycloid}

The curve traced out by a point \(P\) on the circumference of a circle as the
circle rolls along a straight line is called a \textbf{cycloid}.
THe parametric equations of the cycloid are
\begin{align*}
    x &= r(\theta-\sin\theta) & y &= r(1-\cos\theta) & \theta\in\R
\end{align*}