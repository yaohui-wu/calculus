\subsection{Conic Sections in Polar Coordinates}

\subsubsection*{Conics in Cartesian Coordinates}

A \textbf{parabola} is the set of points in a plane that are equidistant from
a fixed point \(F\) (called the \textbf{focus}) and a fixed line (called the
\textbf{directrix}).
Notice that the point halfway between the focus and the
directrix lies on the parabola; it is called the \textbf{vertex}.
The line through the focus perpendicular to the directrix is called the
\textbf{axis} of the parabola.

The equation of the parabola opening upward with vertex \((h,k)\), focus
\((h,k+p)\), and directrix \(y=k-p\) is
\[(x-h)^2=4p(y-k)\]

An \textbf{ellipse} is the set of points in a plane the sum of whose distances from two fixed
points \(F_1\) and \(F_2\) is a constant.
These two fixed points are called the \textbf{foci} (plural of focus).

The equation of the ellipse with center \((h,k)\), horizontal major axis, foci
\((h\pm c,k)\) where \(c^2=a^2-b^2\) and vertices \((\pm a,k)\) is
\[\frac{(x-h)^2}{a^2}+\frac{(y-k)^2}{b^2}=1\]

A \textbf{hyperbola} is the set of all points in a plane the difference of whose distances
from two fixed points \(F_1\) and \(F_2\) (the foci) is a constant.

The equation of the hyperbola with center \((h,k)\), horizontal major axis, foci
\((h\pm c,k)\) where \(c^2=a^2+b^2\) is
\[\frac{(x-h)^2}{a^2}-\frac{(y-k)^2}{b^2}=1\]
The equations of the asymptotes are given by
\[y=k\pm \frac{b}{a}(x-h)\]

\subsubsection*{Conics in Polar Coordinates}

\begin{theorem}
    Let \(F\) be a fixed point (called the \textbf{focus}) and \(l\) be a
    fixed line (called the \textbf{directrix}) in a plane.
    Let \(e\) be a fixed a fixed positive number (called the
    \textbf{eccentricity}).
    The set of all points \(P\) in the plane such that
    \[\frac{|PF|}{|Pl|}=e\]
    (that is, the ratio of the distance from \(F\) to the distance from \(l\)
    is the constant \(e\)) is a conic section.
    The conic is
    \begin{enumerate}
        \item An ellipse if \(e<1\).
        \item A parabola if \(e=1\).
        \item A hyperbola if \(e>1\).
    \end{enumerate}
\end{theorem}
\begin{theorem}
    A polar equation of the form
    \[r=\frac{ed}{1\pm e\cos\theta}\]
    or
    \[r=\frac{ed}{1\pm e\sin\theta}\]
    represents a conic section with eccentricity \(e\).
    The conic is an ellipse if \(e<1\), a parabola if \(e=1\), or a hyperbola
    if \(e>1\).
\end{theorem}