\subsection{Polar Coordinates}

A coordinate system represents a point in the plane by an ordered pair of
numbers called coordinates.
Usually we use Cartesian coordinates, which are directed distances from two
perpendicular axes.
Here we describe a coordinate system introduced by Newton, called the
\textbf{polar coordinate system}.

We choose a point \(O\) in the plane called the \textbf{pole} (or origin).
Then we draw a ray (half-line) starting at \(O\) called the
\textbf{polar axis}.
This axis is usually drawn horizontally to the right and
corresponds to the positive \(x\)-axis in Cartesian coordinates.

If \(P\) is any other point in the plane, let \(r\) be the distance from \(O\)
to \(P\) and let \(\theta\) be the angle  (usually measured in radians)
between the polar axis and line \(OP\).
Then the point \(P\) is represented by the ordered pair \((r,\theta)\) called
the \textbf{polar coordinates} of \(P\).
We use the convention that an angle is positive if measured in the
counterclockwise direction from the polar axis and negative in the clockwise
direction.
If \(P=O\), then \(r=0\) and we agree that \((0,\theta)\) represents the pole
for any value of \(\theta\).

We extend the meaning of polar coordinates \((r,\theta)\) to the case \(r\) is
negative by agreeing that, the points \((-r,\theta)\) and \((r,\theta)\) lie
on the same line through \(O\) and at the same distance \(|r|\) from \(O\),
but on opposite sides of \(O\).
If \(r>0\), the point \((r,\theta)\) lies in the same quadrant as \(\theta\);
if \(r<0\), it lies in the quadrant on the opposite side of the pole.
Note that \((-r,\theta)\) represents the same point as \((r,\theta+\pi)\).

In fact, since a complete counterclockwise rotation is given by an angle
\(2\pi\), the point represented by polar coordinates \((r,\theta)\) is also
represented by
\[(r,\theta+2n\pi)\]
and
\[(-r,\theta+(2n+1)\pi)\]
where \(n\) is any integer.

If the point \(P\) has Cartesian coordinates \((x,y)\) and polar
coordinates \((r,\theta)\), then we have
\begin{align*}
    \cos\theta &= \frac{x}{r} & \sin\theta &= \frac{y}{r}
\end{align*}
and so
\begin{align*}
    x &= r\cos\theta & y &= r\sin\theta
\end{align*}
To find \(r\) and \(\theta\) when \(x\) and \(y\) are known, we use the
equations
\begin{align*}
    r^2 &= x^2+y^2 & \tan\theta &= \frac{y}{x}
\end{align*}

\subsubsection*{Polar Curves}

The \textbf{graph of a polar equation} \(r=f(\theta)\), or more generally
\(F(r,\theta)=0\), consists of all points \(P\) that have at least one polar
representation \((r,\theta)\) whose coordinates satisfy the equation.