\subsection{Calculus with Parametric Curves}

\subsubsection*{Tangents}
Suppose \(f\) and \(g\) are differentiable functions and we want to find the
tangent line at a point on the parametric curve \(x=f(t),y=g(t)\), where \(y\)
is a differentiable function of \(x\).
Then the chain rule gives
\[\frac{dy}{dt}=\frac{dy}{dx}\cdot\frac{dx}{dt}\]
If \(dx/dt\neq0\), then
\[\frac{dy}{dx}=\frac{dy/dt}{dx/dt}\]
It follows that
\[\frac{d^2y}{dx^2}=\frac{d}{dx}\left(\frac{dy}{dx}\right)
=\frac{\dfrac{d}{dt}\left(\dfrac{dy}{dx}\right)}{\dfrac{dx}{dt}}\]

\subsubsection*{Areas}
If the curve \(y=F(x)\) from \(a\) to \(b\) is traced out once by the
parametric equations \(x=f(t),y=g(t),\alpha\leq t\leq\beta\), then the area
under the curve is
\[A=\int_{a}^{b}y\,dx=\int_{\alpha}^{\beta}g(t)f'(t)\,dt\]

\subsubsection*{Arc Length}
\begin{theorem}
    If a curve \(C\) is described by the parametric equations
    \(x=f(t),y=g(t),\alpha\leq t\leq\beta\), where \(f'\) and \(g'\) are
    continuous on \([\alpha,\beta]\) and \(C\) is traversed exactly once at
    \(t\) increases from \(\alpha\) to \(\beta\), then the length of \(C\) is
    \[L=\int_{\alpha}^{\beta}
    \sqrt{\left(\frac{dx}{dt}\right)^2+\left(\frac{dy}{dt}\right)^2}\,dt\]
\end{theorem}

\subsubsection*{Surface Area}
Suppose that a curve \(C\) is given by the parametric equations
\(x=f(t),y=g(t),\alpha\leq t\leq\beta\) where \(f',g'\) are continuous,
\(g(t)\geq0\), and \(C\) is traversed exactly once as \(t\) increases from
\(\alpha\) to \(\beta\).
If \(C\) is rotated about the \(x\)-axis, then the area of the resulting
surface is given by
\[S=\int_{\alpha}^{\beta}2\pi y
\sqrt{\left(\frac{dx}{dt}\right)^2+\left(\frac{dy}{dt}\right)^2}\,dt\]