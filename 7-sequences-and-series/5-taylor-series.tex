\subsection{Taylor Series}

\subsubsection*{Taylor Series and Maclaurin Series}
The \textbf{Taylor series} of a function \(f\) centered at \(a\) is
\[f(x)=\sum_{n=0}^{\infty}\frac{f^{(n)}(a)}{n!}(x-a)^n
=f(a)+f'(a)(x-a)+\frac{f''(a)}{2!}(x-a)^2+\cdots\]
For the special case of \(a=0\), the Taylor series becomes the
\textbf{Maclaurin series}
\[f(x)=\sum_{n=0}^{\infty}\frac{f^{(n)}(0)}{n!}x^n
=f(0)+f'(0)x+\frac{f''(0)}{2!}x^2+\cdots\]
\begin{problem}
    Find the Maclaurin series of the function \(f(x)=e^x\) and its radius of
    convergence.
\end{problem}
\begin{solution}
    We have
    \[f(x)=e^x=\sum_{n=0}^{\infty}\frac{f^{(n)}(0)}{n!}x^n
    =f(0)+f'(0)x+\frac{f''(0)}{2!}x^2+\cdots\]
    and since \(f^{(n)}(x)=e^x\) so \(f^{(n)}(0)=e^0=1\) for all \(n\)
    therefore
    \[e^x=1+x+\frac{x^2}{2!}+\frac{x^3}{3!}\cdots
    =\sum_{n=0}^\infty\frac{x^n}{n!}\]
    Let \(a_n=x^n/n!\), then
    \[\left|\frac{a_{n+1}}{a_n}\right|
    =\left|\frac{x^{n+1}}{(n+1)!}\frac{n!}{x^n}\right|=\frac{|x|}{n+1}\]
    and
    \[\lim_{n\to\infty}\frac{|x|}{n+1}=0<1\]
    so by the ratio test the series converges for all \(x\in\R\) and the
    radius of convergence is \(R=\infty\).
\end{solution}

\subsubsection*{When Is a Function Represented by Its Taylor Series?}
The \(n\)th degree Taylor polynomial \(T_n\) of \(f\) at \(a\) is the partial
sum
\begin{align*}
    T_n(x) &= \sum_{i=0}^{n}\frac{f^{(i)}(a)}{i!}(x-a)^i \\
    &= f(a)+f'(a)(x-a)+\frac{f''(a)}{2!}(x-a)^2+\cdots
    +\frac{f^{(n)}(a)}{n!}(x-a)^n
\end{align*}
Let \(R_n(x)=f(x)-T_n(x)\) so that \(f(x)=T_n(x)+R_n(x)\), then \(R_n(x)\) is
the remainder of the Taylor series.
If \(\lim_{n\to\infty}R_n(x)=0\), then
\[\lim_{n\to\infty}T_n(x)=\lim_{n\to\infty}[f(x)-R_n(x)]=f(x)\]
\begin{theorem}
    If \(f(x)=T_n(x)+R_n(x)\) where \(T_n\) is the \(n\)th degree Taylor
    polynomial of \(f\) at \(a\) and \(\lim_{n\to\infty}R_n(x)=0\) for
    \(|x-a|<R\), then \(f\) is equal to the sum of its Taylor series on the
    interval \(|x-a|<R\).
\end{theorem}
\begin{theorem}[Taylor's Formula]
    If \(f\) has \(n+1\) derivatives in an interval \(I\) that contains the
    number \(a\), then for \(x\) in \(I\) there is a number \(z\) strictly
    between \(x\) and \(a\) such that the remainder term in the Taylor series
    is the expression
    \[R_n(x)=\frac{f^{(n+1)}(z)}{(n+1)!}(x-a)^{n+1}\]
    which is the Lagrange's form of the remainder term.
\end{theorem}
\[\lim_{n\to\infty}\frac{x^n}{n!}=0\] for all \(x\in\R\).

\subsubsection*{Taylor Series of Important Functions}
\[e^x=\sum_{n=0}^\infty=1+x+\frac{x^2}{2!}+\frac{x^3}{3!}+\cdots\]
\[e=\sum_{n=0}^\infty\frac{1}{n!}=1+1+\frac{1}{2!}+\frac{1}{3!}+\cdots\]

\[\sin x=x-\frac{x^3}{3!}+\frac{x^5}{5!}-\frac{x^7}{7!}+\cdots
=\sum_{n=0}^\infty(-1)^n\frac{x^{2n+1}}{(2n+1)!}\]

\[\cos x=1-\frac{x^2}{2!}+\frac{x^4}{4!}-\frac{x^6}{6!}+\cdots
=\sum_{n=0}^\infty(-1)^n\frac{x^{2n}}{(2n)!}\]

If \(k\) is any real number and \(|x|<1\), then the binomial series is
\[(1+x)^k=\sum_{n=0}^\infty\binom{k}{n}x^n=1+kx+\frac{k(k-1)}{2}x^2
+\frac{k(k-1)(k-2)}{3!}x^3+\cdots\]
\begin{problem}
    Evaluate \(\displaystyle{\int e^{-x^2}}\,dx\) as an infinite series.
\end{problem}

\subsubsection*{New Taylor Series from Old}

\subsubsection*{Multiplication and Division of Power Series}