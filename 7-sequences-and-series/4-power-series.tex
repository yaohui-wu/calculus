\subsection{Power Series}
A \textbf{power series} centered at \(a\) is a series of the form
\[\sum_{n=0}^{\infty}c_n(x-a)^n=a_0+c_1(x-a)+c_2(x-a)^2+\cdots\]
where \(x\) is a variable and the coefficient \(c_n\) is a constant.
\begin{theorem}
    For a given power series \(\sum_{n=0}^\infty c_n(x-a)^n\) there are only
    three possibilities:
    \begin{enumerate}
        \item The series converges only when \(x=a\).
        \item The series converges for all \(x\in\R\).
        \item There is a positive number \(R\) such that the series converges
        if \(|x-a|<R\) and diverges if \(|x-a|>R\).
    \end{enumerate}
\end{theorem}
The number \(R\) in cases 3 is the radius of convergence of the power series.
The radius of convergence is \(R=0\) in case 1 and \(R=\infty\) in case 2.
The interval of convergence of a power series is the interval that consists of
all values of \(x\) for which the series converges.

\subsubsection*{Function Representation of Power Series}

\subsubsection*{Differentiation and Integration of Power Series}