\subsection{Sequences}
An infinite \textbf{sequence} \(\{a_n\}\) is a list of numbers defined by
\[a_1,a_2,a_3\dots,a_n,\dots\]
where \(a_n\) is the \(n\)th term of the sequence.
\begin{definition}
    A sequence \(\{a_n\}\) has the limit \(L\)
    \[\lim_{n\to\infty}a_n=L\]
    if we can make the terms \(a_n\) as close to \(L\) as we like by taking
    \(n\) sufficiently large.
    If \(\lim_{n\to\infty}a_n\) exists, then the sequence is convergent,
    otherwise it is divergent.
\end{definition}
\begin{definition}
    A sequence \(\{a_n\}\) has the limit \(L\)
    \[\lim_{n\to\infty}a_n=L\]
    if for every \(\epsilon>0\) there eixsts an integer \(N\) such that
    \[n>N\implies |a_n-L|<\epsilon\qedhere\]
\end{definition}
\begin{theorem}
    If \(\lim_{x\to\infty}f(x)=L\) and \(f(n)=a_n\) when \(n\) is an integer,
    then \(\lim_{n\to\infty}a_n=L\).
\end{theorem}
\begin{definition}
    \(\lim_{n\to\infty}a_n=\infty\) if for every positive number \(M\) there
    is a positive integer \(N\) such that
    \[n>N\implies a_n>M\qedhere\]
\end{definition}
\begin{theorem}[Continuity and Convergence Theorem]
    If \(\lim_{n\to\infty}a_n=L\) and the function \(f\) is continuous at
    \(L\), then
    \[\lim_{n\to\infty}f(a_n)=f(L)\qedhere\]
\end{theorem}