\subsection{Sequences}
An infinite \textbf{sequence} \(\{a_n\}\) is a list of numbers is a definite
number
\[a_1,a_2,a_3\dots,a_n,a_{n+1},\dots\]
where \(a_n\) is the \(n\)th term of the sequence.
Consider the sequence
\[a_n=\left\{\frac{1}{2},\frac{1}{4},\frac{1}{8},\frac{1}{16},\frac{1}{32},
\cdots\right\}\]
we can rewrite it as
\[a_n=\left\{\frac{1}{2},\frac{1}{2^2},\frac{1}{2^3},\frac{1}{2^4},
\frac{1}{2^5},\cdots\right\}\]
so the formula for the \(n\)th term is
\[a_n=\frac{1}{2^n}\]
for \(n=1,2,3,\cdots\) and
\[\{a_n\}=\left\{\frac{1}{2^n}\right\}\]

\subsubsection*{The Limit of a Sequence}
The intuitive definition of a limit of a sequence is:
\begin{definition}
    A sequence \(\{a_n\}\) has the limit \(L\)
    \[\lim_{n\to\infty}a_n=L\]
    if we can make the terms \(a_n\) as close to \(L\) as we like by taking
    \(n\) sufficiently large.
    If \(\lim_{n\to\infty}a_n\) exists, then the sequence is convergent,
    otherwise it is divergent.
\end{definition}
The precise definition of a limit of a sequence is:
\begin{definition}
    A sequence \(\{a_n\}\) has the limit \(L\)
    \[\lim_{n\to\infty}a_n=L\]
    if for every \(\epsilon>0\) there eixsts an integer \(N\) such that
    \[n>N\implies |a_n-L|<\epsilon\]
\end{definition}
\begin{theorem}
    If \(\lim_{x\to\infty}f(x)=L\) and \(f(n)=a_n\) when \(n\) is an integer,
    then \(\lim_{n\to\infty}a_n=L\).
\end{theorem}
The precise definition of an infinite limit of a sequence is:
\begin{definition}
    \(\lim_{n\to\infty}a_n=\infty\) if for every positive number \(M\) there
    is a positive integer \(N\) such that
    \[n>N\implies a_n>M\]
\end{definition}

\subsubsection*{Properties of Convergent Sequences}
\begin{theorem}[Continuity and Convergence Theorem]
    If \(\lim_{n\to\infty}a_n=L\) and the function \(f\) is continuous at
    \(L\), then
    \[\lim_{n\to\infty}f(a_n)=f(L)\]
\end{theorem}
The sequence \(\{r^n\}\) is convergent if \(-1<r\leq 1\) and divergent for all
other values of \(r\).

\subsubsection*{Monotonic and Bounded Sequences}
\begin{definition}
    A sequence \(\{a_n\}\) is \textbf{increasing} if \(a_n<a_{n+1}\) for all
    \(n\geq 1\), that is, \(a_1<a_2<a_3<\cdots\).
    A sequence is \textbf{decreasing} if \(a_n>a_{n+1}\) for all \(n\geq 1\).
    A sequence is \textbf{monotonic} if it is either increasing or decreasing.
\end{definition}
\begin{definition}
    A sequence \(\{a_n\}\) is \textbf{bounded above} if there is a number
    \(M\) such that \(a_n\leq M\) for all \(n\geq 1\).
    A sequence is \textbf{bounded below} if there is a number
    \(m\) such that \(m\leq a_n\) for all \(n\geq 1\).
    If a sequence is bounded above and below, then it is a
    \textbf{bounded sequence}.
\end{definition}
\begin{theorem}[Monotonic Sequence Theorem]
    Every bounded, monotonic sequence is convergent.
    In particular, a sequence that is increasing and bounded above converges,
    and a sequence that is decreasing and bounded below converges.
\end{theorem}
The proof of the monotonic sequence theorem is based on the
\textbf{completeness axiom} of the set of real numbers \(\R\).
The completeness axiom states if \(S\) is a nonempty set of real numbers that
has an upper bound \(M\) so \(x\leq M\) for all \(x\in S\), then \(S\) has a
\textbf{least upper bound} \(b\).
(This means that \(b\) is an upper bound of \(S\),
but if \(M\) is any other upper bound, then \(b\leq M\).)
The completeness axiom is an expression of the intuitive fact that there is no
gap or hole in the real number line.
\begin{proof}
    Suppose \(\{a_n\}\) is an incerasing sequence.
    Since \(\{a_n\}\) is an bounded, the set \(S=\{a_n\mid n\geq 1\}\) has an
    upper bound.
    By the completeness axiom, it has a least upper bound \(L\).
    Given \(\epsilon>0\), \(L-\epsilon\) is not an upper bound for \(S\) since
    \(L\) is the least upper bound.
    Therefore we have \(a_N>L-\epsilon\) for some integer \(N\) but the
    sequence is incerasing so \(a_n\geq a_N\) for every \(n>N\).
    Thus if \(n>N\), we have
    \begin{align*}
        a_n &> L-\epsilon \\
        a_n+\epsilon &> L \\
        L &< a_n +\epsilon \\
        0 \leq L-a_n &< \epsilon
    \end{align*}
    since \(a_n\leq L\).
\end{proof}