\subsection{Limits at Infinity and Horizontal Asymptotes}

\subsubsection*{Limits at Infinity and Horizontal Asymptotes}
\begin{definition}
    Let \(f\) be a function defined on some interval \((a,\infty)\).
    Then
    \[\lim_{x\to\infty}f(x)=L\]
    or \(f(x)\to L\) as \(x\to\infty\) means that the values of \(f(x)\) can
    be made arbitrarily close to \(L\) by requiring \(x\) to be sufficiently
    large.
\end{definition}
We say that the limit of \(f(x)\),
as \(x\) approaches infinity,
is \(L\).
\begin{definition}
    Let \(f\) be a function defined on some interval \((-\infty,a)\).
    Then
    \[\lim_{x\to-\infty}f(x)=L\]
    means that the values of \(f(x)\) can be made arbitrarily close to \(L\) by requiring
    \(x\) to be sufficiently large negative.
\end{definition}
We say that the limit of \(f(x)\),
as \(x\) approaches negative infinity,
is \(L\).
\begin{definition}
    The line \(y=L\) is called a \textbf{horizontal asymptote} of the curve
    \(y=f(x)\) if either
    \[\lim_{x\to\infty}f(x)=L\]
    or
    \[\lim_{x\to-\infty}f(x)=L\]
\end{definition}

\subsubsection*{Evaluating Limits at Infinity}
\begin{theorem}
    If \(r>0\) is a rational number,
    then
    \[\lim_{x\to\infty}\frac{1}{x^r}=0\]
    If \(r>0\) is a rational number such that \(x^r\) is defined for all \(x\),
    then
    \[\lim_{x\to-\infty}\frac{1}{x^r}=0\]
\end{theorem}

\subsubsection*{Infinite Limits at Infinity}
The notation
\[\lim_{x\to\infty}f(x)=\infty\]
is used to indicate that the values of \(f(x)\) become large as \(x\) becomes
large.
Similar meanings are attached to the following symbols:
\begin{align*}
    \lim_{x\to\infty}f(x) &= \infty & \lim_{x\to\infty}f(x) &= -\infty
    & \lim_{x\to-\infty}f(x) &= -\infty
\end{align*}

\subsubsection*{Precise Definitions}
\begin{definition}
    Let \(f\) be a function defined on some interval \((a,\infty)\).
    Then
    \[\lim_{x\to\infty}f(x)=L\]
    means that for every \(\epsilon>0\) there is a corresponding number \(N\)
    such that
    \[x>N\implies|f(x)-L|<\epsilon\]
\end{definition}
\begin{definition}
    Let \(f\) be a function defined on some interval \((-\infty,a)\).
    Then
    \[\lim_{x\to-\infty}f(x)=L\]
    means that for every \(\epsilon>0\) there is a corresponding number \(N\)
    such that
    \[x<N\implies|f(x)-L|<\epsilon\]
\end{definition}
\begin{problem}
    Prove that \(\displaystyle{\lim_{x\to \infty}\frac{1}{x}=0}\).
\end{problem}
\begin{solution}
    Given \(\epsilon>0\), we want to find an \(N\) such that
    \[x>N\implies\left|\frac{1}{x}-0\right|<\epsilon\]
    Since \(x\to \infty\),
    we can that \(x>0\) in computing the limit.
    Then \(1/x<\epsilon\iff x>1/\epsilon\).
    Let \(N=1/\epsilon\),
    so
    \[x>N=\frac{1}{\epsilon}\implies\left|\frac{1}{x}-0\right|=\frac{1}{x}
    <\epsilon\]
    Therefore, by definition,
    \[\lim_{x\to \infty}\frac{1}{x}=0\]
\end{solution}
\begin{definition}
    Let \(f\) be a function defined on some interval \((a,\infty)\).
    Then
    \[\lim_{x\to\infty}f(x)=\infty\]
    means that for every positive number \(M\) there is a corresponding
    positive number \(N\) such that
    \[x>N\implies f(x)>M\]
\end{definition}
Similar definitions apply when the symbol \(\infty\) is replaced by \(-\infty\).