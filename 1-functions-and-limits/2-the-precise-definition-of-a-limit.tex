\subsection{The Precise Definition of a Limit}

\subsubsection*{The Precise Definition of a Limit}
Let \(f\) be a function defined on some open interval that contains the number
\(a\), except possibly at \(a\) itself.
\begin{definition}
    The limit of \(f(x)\) as \(x\) approaches \(a\) is \(L\), and we write
    \[\lim_{x\to a}f(x)=L\]
    if for every number \(\epsilon>0\) there is a number \(\delta>0\) such that
    \[0<|x-a|<\delta\implies|f(x)-L|<\epsilon\]
\end{definition}
\begin{problem}
    Prove that \(\displaystyle{\lim_{x\to 3}(4x-5)=7}\).
\end{problem}
\begin{solution}
    Let \(\epsilon>0\) be given, we want to find a number \(\delta>0\) such
    that
    \[0<|x-3|<\delta \implies|(4x-5)-7|<\epsilon\]
    We simplify to get \(|(4x-5)-7|=|4x-12|=4|x-3|\) so we have
    \[4|x-3|<\epsilon\iff|x-3|<\frac{\epsilon}{4}\]
    Let \(\delta=\epsilon/4\), we have
    \[0<|x-3|<\frac{\epsilon}{4}\implies4|x-3|<\epsilon\implies
    |(4x-5)-7|<\epsilon\]
    Therefore, by the definition of a limit, it is proved that
    \(\displaystyle{\lim_{x\to 3}(4x-5)=7}\).
\end{solution}
\begin{problem}
    Prove that \(\displaystyle{\lim_{x\to 3}x^2=9}\).
\end{problem}
\begin{solution}
    Let \(\epsilon>0\) be given, we want to find a number \(\delta>0\) such
    that
    \[0<|x-3|<\delta\implies|x^2-9|<\epsilon\]
    We simplify to get \[|x^2-9|=|x+3|\,|x-3|<\epsilon\]
    Let \(C\) be a positive constant such that
    \[|x+3|\,|x-3|<C\,|x-3|<\epsilon \iff |x-3|<\frac{\epsilon}{C}\]
    Since we are interested only in values of \(x\) that are close to 3, it is
    reasonable to assume that \(|x-3|<1\) then \(|x+3|<7\) so \(C=7\).
    Let \(\delta=\min\{1,\epsilon/7\}\).
    If \(0<|x-3|<\delta\), then
    \[|x^2-9|=|x+3|\,|x-3|<7\cdot\frac{\epsilon}{7}=\epsilon\]
    It is proved that \(\displaystyle{\lim_{x\to 3}x^2=9}\).
\end{solution}

\subsubsection*{One-Sided Limits}
\begin{definition}
    \[\lim_{x\to a^-}f(x)=L\]
    if for every number \(\epsilon>0\) there is a number \(\delta>0\) such
    that
    \[a-\delta<x<a\implies|f(x)-L|<\epsilon\]
\end{definition}
\begin{definition}
    \[\lim_{x\to a^+}f(x)=L\]
    if for every number \(\epsilon>0\) there is a number \(\delta>0\) such
    that
    \[a<x<a+\delta\implies|f(x)-L|<\epsilon\]
\end{definition}
\begin{problem}
    Prove that \(\displaystyle{\lim_{x\to 0^+}\sqrt{x}=0}\).
\end{problem}
\begin{solution}
    Let \(\epsilon>0\) be given, we want to find a number \(\delta>0\) such
    that
    \[0<x<\delta\implies|\sqrt{x}-0|<\epsilon\]
    We simplify to get \(\sqrt{x}<\epsilon\iff x<\epsilon^2\).
    Let \(\delta=\epsilon^2\).
    If \(0<x<\delta\), then
    \(\sqrt{x}<\sqrt{\delta}=\sqrt{\epsilon^2}=\epsilon\)
    so \(|\sqrt{x}-0|<\epsilon\).
    It is proved that \(\displaystyle{\lim_{x\to 0^+}\sqrt{x}=0}\).
\end{solution}

\subsubsection*{Infinite Limits}
\begin{definition}
    \[\lim_{x\to a}f(x)=\infty\]
    means that for every positive number \(M\) there is a positive number
    \(\delta\) such that
    \[0<|x-a|<\delta\implies f(x)>M\]
\end{definition}
\begin{problem}
    Prove that \(\displaystyle{\lim_{x\to 0}\frac{1}{x^2}=\infty}\).
\end{problem}
\begin{solution}
    Let \(M\) be a given positive number
    We want to find a number \(\delta\) such that
    \[0<|x|<\delta\implies\frac{1}{x^2}>M\]
    We have
    \[\frac{1}{x^2}>M\iff x^2<\frac{1}{M}\iff\sqrt{x^2}<\sqrt{\frac{1}{M}}
    \iff|x|<\frac{1}{\sqrt{M}}\]
    Let \(\delta=1/\sqrt{M}\), then
    \[0<|x|<\delta=\frac{1}{\sqrt{M}}\implies\frac{1}{x^2}>M\]
    This shows that
    \(\displaystyle{\lim_{x\to 0}\frac{1}{x^2}=\infty}\).
\end{solution}
\begin{definition}
    \[\lim_{x\to a}f(x)=-\infty\]
    if for every negative number \(N\) there is a positive number \(\delta\)
    such that
    \[0<|x-a|<\delta\implies f(x)<N\]
\end{definition}