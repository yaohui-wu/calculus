\subsection{The Precise Definition of a Limit}

\subsubsection*{Precise Definition of a Limit}
Let \(f\) be a function defined on some open interval that contains the number
\(a\), except possibly at \(a\) itself.
\begin{definition}
    We say that limit of \(f(x)\) as \(x\) approaches \(a\) is \(L\),
    and we write
    \[\lim_{x\to a}f(x)=L\]
    if for every number \(\epsilon>0\) there is a corresponding number
    \(\delta>0\) such that
    \[0<|x-a|<\delta\implies|f(x)-L|<\epsilon\]
\end{definition}
\begin{problem}
    Prove that \(\displaystyle{\lim_{x\to 3}(4x-5)=7}\).
\end{problem}
\begin{solution}
    Let \(\epsilon>0\) be a given positive number.
    We want to find a number \(\delta\) such that
    \[0<|x-3|<\delta \implies|(4x-5)-7|<\epsilon\]
    But \(|(4x-5)-7|=|4x-12|=4|x-3|\).
    Note that \(4|x-3|<\epsilon\iff|x-3|<\epsilon/4\).
    Let \(\delta=\epsilon/4\), we have
    \[0<|x-3|<\frac{\epsilon}{4}\implies4|x-3|<\epsilon\implies
    |(4x-5)-7|<\epsilon\]
    Therefore, by the definition of a limit,
    \[\lim_{x\to 3}(4x-5)=7\]
\end{solution}
\begin{problem}
    Prove that \(\displaystyle{\lim_{x\to 3}x^2=9}\).
\end{problem}
\begin{solution}
    Let \(\epsilon\) be a given positive number.
    We want to find a number \(\delta\) such that
    \[0<|x-3|<\delta\implies|x^2-9|<\epsilon\]
    We simplify to get \[|x^2-9|=|x+3|\,|x-3|<\epsilon\]
    Let \(C\) be a positive constant such that
    \[|x+3|\,|x-3|<C\,|x-3|<\epsilon \iff |x-3|<\frac{\epsilon}{C}\]
    Since we are interested only in values of \(x\) that are close to 3,
    it is reasonable to assume that \(|x-3|<1\).
    Then we have \(|x+3|<7\), and so \(C=7\).
    Let \(\delta=\min\{1,\epsilon/7\}\).
    If \(0<|x-3|<\delta\),
    then
    \[|x^2-9|=|x+3|\,|x-3|<7\cdot\frac{\epsilon}{7}=\epsilon\]
    This shows that \(\displaystyle{\lim_{x\to 3}x^2=9}\).
\end{solution}
\begin{definition}
    \[\lim_{x\to a^-}f(x)=L\]
    if for every number \(\epsilon>0\) there is a number \(\delta>0\) such
    that
    \[a-\delta<x<a\implies|f(x)-L|<\epsilon\]
\end{definition}
\begin{definition}
    \[\lim_{x\to a^+}f(x)=L\]
    if for every number \(\epsilon>0\) there is a number \(\delta>0\) such
    that
    \[a<x<a+\delta\implies|f(x)-L|<\epsilon\]
\end{definition}
\begin{problem}
    Prove that \(\displaystyle{\lim_{x\to 0^+}\sqrt{x}=0}\).
\end{problem}
\begin{solution}
    Let \(\epsilon\) be a given positive number
    We want to find a number \(\delta\) such that
    \[0<x<\delta\implies|\sqrt{x}-0|<\epsilon\]
    But \(\sqrt{x}<\epsilon\iff x<\epsilon^2\).
    Let \(\delta=\epsilon^2\).
    If \(0<x<\delta\), then
    \(\sqrt{x}<\sqrt{\delta}=\sqrt{\epsilon^2}=\epsilon\)
    so \(|\sqrt{x}-0|<\epsilon\).
    This shows that \(\displaystyle{\lim_{x\to 0^+}\sqrt{x}=0}\).
\end{solution}