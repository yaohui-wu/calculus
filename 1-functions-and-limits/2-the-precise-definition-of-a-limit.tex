\subsection{The Precise Definition of a Limit}

\subsubsection{Epsilon-Delta Definition of a Limit}
\textbf{Augustin-Louis Cauchy} (1789--1857) and \textbf{Karl Weierstrass}
(1815--1897) developed a rigorous definition of a limit.
\begin{definition}
    \[\lim_{x\to a}f(x)=L\] if for every number \(\varepsilon>0\), there is a
    number \(\delta>0\) such that \[0<|x-a|<\delta\implies|f(x)-L|<\varepsilon\]
\end{definition}
\begin{definition}
    \[\lim_{x\to a^-}f(x)=L\] if for every number \(\varepsilon>0\), there is a
    number \(\delta>0\) such that \[a-\delta<x<a\implies|f(x)-L|<\varepsilon\]
\end{definition}
\begin{definition}
    \[\lim_{x\to a^+}f(x)=L\] if for every number \(\varepsilon>0\), there is a
    number \(\delta>0\) such that \[a<x<a+\delta\implies|f(x)-L|<\varepsilon\]
\end{definition}
\begin{problem}
    Prove that \[\lim_{x\to 3}(4x-5)=7\]
\end{problem}
\begin{solution}
    Let \(\varepsilon>0\) be given, we want to find a number \(\delta>0\) such
    that \[0<|x-3|<\delta \implies |(4x-5)-7|<\varepsilon\]
    We simplify to get \(|(4x-5)-7|=|4x-12|=4|x-3|\) so we have
    \[4|x-3|<\varepsilon \iff |x-3|<\frac{\varepsilon}{4}\]
    Let \(\delta=\varepsilon/4\), we have
    \[0<|x-3|<\delta=\frac{\varepsilon}{4} \iff 0<4|x-3|<\varepsilon \iff
    0<|(4x-5)-7|<\varepsilon\]
    Therefore, by the definition of a limit, \[\lim_{x\to 3}(4x-5)=7\]
\end{solution}