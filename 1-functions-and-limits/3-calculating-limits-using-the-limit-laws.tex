\subsection{Calculating Limits Using the Limit Laws}

We have the following properties of limits called the \textbf{limit laws} to
calculate limits.
Suppose that \(c\) is a constant and the limits
\begin{align*}
    \lim_{x\to a}f(x) &= L & \lim_{x\to a}g(x) &= M
\end{align*}
exist.
Then
\begin{enumerate}
    \item Sum Law: \(\displaystyle{\lim_{x\to a}\bigl[f(x)+g(x)\bigr]=L+M}\)
    \item Difference Law: \(\displaystyle{\lim_{x\to a}\bigl[f(x)-g(x)\bigr]=L-M}\)
    \item Constant Multiple Law: \(\displaystyle{\lim_{x\to a}\bigl[cf(x)\bigr]=cL}\)
    \item Product Law: \(\displaystyle{\lim_{x\to a}\bigl[f(x)g(x)\bigr]=LM}\)
    \item Quotient Law:
    \(\displaystyle{\lim_{x\to a}\frac{f(x)}{g(x)}=\frac{L}{M}}\) if
    \(M\neq 0\).
    \item Power Law: \(\displaystyle{\lim_{x\to a}\bigl[f(x)\bigr]^n
    =\Bigl[\lim_{x\to a}f(x)\Bigr]^n}\) where \(n\) is a positive integer.
    \item Root Law: \(\displaystyle{\lim_{x\to a}\sqrt[n]{f(x)}
    =\sqrt[n]{\lim_{x\to a}f(x)}}\) where \(n\) is a positive integer.
    If \(n\) is even,
    we assume that \(\displaystyle{\lim_{x\to a}f(x)>0}\).
    \item \(\displaystyle{\lim_{x\to a}c=c}\)
    \item \(\displaystyle{\lim_{x\to a}x=a}\)
    \item \(\displaystyle{\lim_{x\to a}x^n=a^n}\) where \(n\) is a positive
    integer.
    \item \(\displaystyle{\lim_{x\to a}\sqrt[n]{x}=\sqrt[n]{a}}\) where \(n\)
    is a positive integer.
    If \(n\) is even,
    we assume that \(a>0\).
\end{enumerate}
\begin{proof}
    Proof of limit law 8:
    Let \(\epsilon>0\) be given,
    we want to find a number \(\delta>0\) such that
    \[0<|x-a|<\delta\implies|c-c|<\epsilon\]
    We have \(|c-c|=0<\epsilon\) so the trivial inequality is always true for
    any number \(\delta>0\).
    It is proved that \(\displaystyle{\lim_{x\to a}c=c}\).
\end{proof}
\begin{proof}
    Proof of limit law 9:
    Let \(\epsilon>0\) be given,
    we want to find a number \(\delta>0\) such
    that \[0<|x-a|<\delta\implies|x-a|<\epsilon\]
    Let \(\delta=\epsilon\),
    we have
    \[0<|x-a|<\delta=\epsilon\implies|x-a|<\epsilon\]
    It is proved that \(\displaystyle{\lim_{x\to a}x=a}\).
\end{proof}
\begin{proof}
    Proof of the sum law:
    Let \(\epsilon>0\) be given,
    we want to find a number \(\delta>0\) such that
    \[0<|x-a|<\delta\implies|f(x)+g(x)-(L+M)|<\epsilon\]
    By the triangle inequality,
    we have
    \[|f(x)+g(x)-(L+M)|=|f(x)-L+g(x)-M|\leq|f(x)-L|+|g(x)-M|\]
    We make \(|f(x)-L|+|g(x)-M|\) less than \(\epsilon\) by making each of the
    terms \(|f(x)-L|\) and \(|g(x)-M|\) less than \(\epsilon/2\).
    Since \(\displaystyle{\lim_{x\to a}f(x)=L}\),
    there is a number \(\delta_1>0\) such that
    \[0<|x-a|<\delta_1\implies|f(x)-L|<\frac{\epsilon}{2}\]
    Similarly,
    there is a number \(\delta_2>0\) such that
    \[0<|x-a|<\delta_2\implies|g(x)-M|<\frac{\epsilon}{2}\]
    Let \(\delta=\min\{\delta_1,\delta_2\}\).
    If \(0<|x-a|<\delta\),
    then \(0<|x-a|<\delta_1\) and \(0<|x-a|<\delta_2\)
    and so
    \[|f(x)-L|+|g(x)-M|<\frac{\epsilon}{2}+\frac{\epsilon}{2}=\epsilon\]
    Then
    \[0<|x-a|<\delta\implies|f(x)+g(x)-(L+M)|<\epsilon\]
    Thus,
    by the definition of a limit,
    \[\lim_{x\to a}\bigl[f(x)+g(x)\bigr]=L+M\]
\end{proof}
\begin{proof}
    Proof of the product law:
    Let \(\epsilon>0\) be given,
    we want to find a number \(\delta>0\) such
    that \[0<|x-a|<\delta\implies|f(x)g(x)-LM|<\epsilon\]
    In order to get terms that contain \(|f(x)-L|\) and \(|g(x)-M|\),
    we add
    and subtract \(Lg(x)\) as follows and use the triangle inequality:
    \begin{align*}
        |f(x)g(x)-LM| &= |f(x)g(x)-Lg(x)+Lg(x)-LM| \\
        &= \left|\bigl[f(x)-L\bigr]g(x)+L[g(x)-M]\right| \\
        &\leq \left|\bigl[f(x)-L\bigr]g(x)\right|
        +\left|L\bigl[g(x)-M\bigr]\right| \\
        &= |f(x)-L|\,|g(x)|+|L|\,|g(x)-M|
    \end{align*}
    We want to make both of the terms less than \(\epsilon/2\).
    Since \(\displaystyle{\lim_{x\to a}g(x)=M}\),
    there is a number \(\delta_1>0\) such that
    \[0<|x-a|<\delta_1\implies|g(x)-M|<\frac{\epsilon}{2(1+|L|)}\]
    Also,
    there is a number \(\delta_2>0\) such that
    \[0<|x-a|<\delta_2\implies|g(x)-M|<1\] and therefore
    \[|g(x)|=|g(x)-M+M|\leq|g(x)-M|+|M|<1+|M|\]
    Since \(\displaystyle{\lim_{x\to a}f(x)=L}\),
    there is a number \(\delta_3>0\) such that
    \[0<|x-a|<\delta_3\implies|f(x)-L|<\frac{\epsilon}{2(1+|M|)}\]
    Let \(\delta=\min\{\delta_1,\delta_2,\delta_3\}\).
    If \(0<|x-a|<\delta\),
    then \(0<|x-a|<\delta_1\),
    \(0<|x-a|<\delta_2\),
    and \(0<|x-a|<\delta_3\).
    Then we can combine the inequalities to get
    \begin{align*}
        |f(x)g(x)-LM|
        &\leq |f(x)-L|\,|g(x)|+|L|\,|g(x)-M| \\
        &< \frac{\epsilon}{2(1+|M|)}(1+|M|)+|L|\frac{\epsilon}{2(1+|L|)} \\
        &< \frac{\epsilon}{2(1+|M|)}(1+|M|)
        +(1+|L|)\frac{\epsilon}{2(1+|L|)} \\
        &= \frac{\epsilon}{2}+\frac{\epsilon}{2}=\epsilon
    \end{align*}
    This shows that
    \[\lim_{x\to a}\bigl[f(x)g(x)\bigr]=LM\]
\end{proof}
\begin{proof}
    Proof of the constant multiple law:
    If we take \(g(x)=c\) then by the product law and limit law 8,
    we get
    \[\lim_{x\to a}\bigl[cf(x)\bigr]=\lim_{x\to a}c\cdot\lim_{x\to a}f(x)
    =c\lim_{x\to a}f(x)=cL\]
    We can prove the constant multiple law using the precise the definition.
    Note that if \(c=0\),
    then \(cf(x)=0\) and we have
    \[\lim_{x\to a}\bigl[0\cdot f(x)\bigr]=\lim_{x\to a}0=0
    =0\cdot\lim_{x\to a}f(x)=0\cdot L\]
    Let \(\epsilon>0\) and \(c\neq0\) be given,
    we want to find a number \(\delta>0\) such that
    \[0<|x-a|<\delta\implies|cf(x)-cL|<\epsilon\]
    We simplify to get
    \[|f(x)-L|<\frac{\epsilon}{|c|}\]
    By the definition of a limit,
    there is a number \(\delta_1>0\) such that
    \[0<|x-a|<\delta_1\implies|f(x)-L|<\frac{\epsilon}{|c|}\]
    Let \(\delta=\delta_1\),
    we have
    \[0<|x-a|<\delta\implies|cf(x)-cL|<\epsilon\]
    This shows that \(\displaystyle{\lim_{x\to a}\bigl[cf(x)\bigr]=cL}\).
\end{proof}
\begin{proof}
    Proof of the difference law:
    Using the sum law and the constant multiple law with \(c=-1\),
    we have
    \begin{align*}
        \lim_{x\to a}\bigl[f(x)-g(x)\bigr]
        &= \lim_{x\to a}\bigl[f(x)+(-1)g(x)\bigr]
        =\lim_{x\to a}f(x)+\lim_{x\to a}(-1)g(x) \\
        &=\lim_{x\to a}f(x)+(-1)\lim_{x\to a}g(x)
        =\lim_{x\to a}f(x)-\lim_{x\to a}g(x)=L-M
    \end{align*}
\end{proof}
\begin{proof}
    Proof of the quotient law:
    First we prove that \[\lim_{x\to a}\frac{1}{g(x)}=\frac{1}{M}\]
    Let \(\epsilon>0\) be given,
    we want to find a number \(\delta>0\) such that
    \[0<|x-a|<\delta\implies\left|\frac{1}{g(x)}-\frac{1}{M}\right|<\epsilon\]
    Observe that \[\left|\frac{1}{g(x)}-\frac{1}{M}\right|
    =\frac{|M-g(x)|}{|Mg(x)|}=\frac{|g(x)-M|}{|Mg(x)|}\]
    Since \(\displaystyle{\lim_{x\to a}g(x)=M}\),
    there is a number \(\delta_1\) such that
    \[0<|x-a|<\delta_1\implies|g(x)-M|<\frac{|M|}{2}\] and therefore
    \[|M|=|M-g(x)+g(x)|\leq|M-g(x)|+|g(x)|=|g(x)-M|+|g(x)|
    <\frac{|M|}{2}+|g(x)|\]
    This shows that
    \[0<|x-a|<\delta_1\implies\frac{|M|}{2}<|g(x)|\iff\frac{1}{|g(x)|}
    <\frac{2}{|M|}\]
    and so,
    for these values of \(x\),
    \[\frac{1}{|Mg(x)|}=\frac{1}{|M|\,|g(x)|}<\frac{1}{|M|}\cdot\frac{2}{|M|}
    =\frac{2}{M^2}\]
    Also,
    there is a number \(\delta_2>0\) such that
    \[0<|x-a|<\delta_2\implies|g(x)-M|<\frac{M^2}{2}\epsilon\]
    Let \(\delta=\min\{\delta_1,\delta_2\}\).
    If \(0<|x-a|<\delta\),
    then
    \[\left|\frac{1}{g(x)}-\frac{1}{M}\right|=\frac{|M-g(x)|}{|Mg(x)|}
    =\frac{1}{|Mg(x)|}|g(x)-M|<\frac{2}{M^2}\frac{M^2}{2}\epsilon=\epsilon\]
    which is what we want to show.
    We apply the product law to prove the quotient law
    \[\lim_{x\to a}\frac{f(x)}{g(x)}
    =\lim_{x\to a}\left(f(x)\cdot\frac{1}{g(x)}\right)
    =\lim_{x\to a}f(x)\lim_{x\to a}\frac{1}{g(x)}=L\cdot\frac{1}{M}
    =\frac{L}{M}\]
\end{proof}
We have the following \textbf{direct substitution property} to calculate
limits.
If \(f\) is a polynomial or a rational function and \(a\) is in the domain of
\(f\),
then
\[\lim_{x\to a}f(x)=f(a)\]
\begin{problem}
    Find \(\displaystyle{\lim_{x\to 1}\frac{x^2-1}{x-1}}\).
\end{problem}
If \(f(x)=g(x)\) when \(x\neq a\),
then \(\displaystyle{\lim_{x\to a}f(x)=\lim_{x\to a}g(x)}\),
provided that this limit exists.
When computing one-sided limits,
we use the fact that the limit laws also hold for one-sided limits.
\begin{problem}
    Show that \(\displaystyle{\lim_{x\to 0}|x|=0}\).
\end{problem}
\begin{solution}
    Since \(|x|=x\) for \(x>0\),
    we have
    \[\lim_{x\to 0^+}|x|=\lim_{x\to 0^+}x=0\]
    For \(x<0\) we have \(|x|=-x\) so
    \[\lim_{x\to 0^-}|x|=\lim_{x\to 0^-}(-x)=0\]
    Therefore,
    it is shown that \(\displaystyle{\lim_{x\to 0}|x|=0}\).
\end{solution}

\subsubsection*{The Squeeze Theorem}
\begin{theorem}
    If \(f(x)\leq g(x)\) for all \(x\) in an open interval that contains
    \(a\) (except possibly at \(a\)) and
    \begin{align*}
        \lim_{x\to a}f(x) &= L & \lim_{x\to a}g(x) &= M
    \end{align*}
    then \(L\leq M\).
\end{theorem}
\begin{proof}
    We use the method of proof by contradiction.
    Suppose  that \(L>M\),
    then we have
    \[\lim_{x\to a}\bigl[g(x)-f(x)\bigr]=M-L\]
    Therefore,
    for any number \(\epsilon>0\),
    there is a number \(\delta>0\) such that
    \[0<|x-a|<\delta\implies
    \left|\bigl[g(x)-f(x)\bigr]-(M-L)\right|<\epsilon\]
    Note that \(L-M>0\) by the hypothesis.
    Let \(\epsilon=L-M\),
    there is a number \(\delta>0\) such that
    \[0<|x-a|<\delta\implies\left|\bigl[g(x)-f(x)\bigr]-(M-L)\right|<L-M\]
    Since \(a\leq|a|\) for any number \(a\),
    we have
    \[0<|x-a|<\delta\implies \bigl[g(x)-f(x)\bigr]-(M-L)<L-M\]
    which simplifies to
    \[0<|x-a|<\delta\implies g(x)<f(x)\]
    But this contradicts \(f(x)\leq g(x)\).
    Thus the inequality \(L>M\) must be false.
    Therefore \(L\leq M\).
\end{proof}
\begin{theorem}[Squeeze Theorem]
    If \(f(x)\leq g(x)\leq h(x)\) for all \(x\) in an open interval that
    contains \(a\) (except possibly at \(a\)) and
    \[\lim_{x\to a}f(x)=\lim_{x\to a}h(x)=L\]
    then
    \[\lim_{x\to a}g(x)=L\]
\end{theorem}
\begin{proof}
    Let \(\epsilon>0\) be given.
    Since \(\displaystyle{\lim_{x\to a}f(x)=L}\),
    there is a number \(\delta_1>0\) such that
    \[0<|x-a|<\delta_1\implies|f(x)-L|<\epsilon\implies L-\epsilon<f(x)
    <L+\epsilon\]
    Since \(\displaystyle{\lim_{x\to a}h(x)=L}\),
    there is a number \(\delta_2>0\) such that
    \[0<|x-a|<\delta_2\implies|h(x)-L|<\epsilon\implies L-\epsilon<h(x)
    <L+\epsilon\]
    Let \(\delta=\min\{\delta_1,\delta_2\}\).
    If \(0<|x-a|<\delta\),
    then \(0<|x-a|<\delta_1\) and \(0<|x-a|<\delta_2\),
    so
    \[L-\epsilon<f(x)\leq g(x)\leq h(x)<L+\epsilon\]
    In particular,
    \[L-\epsilon<g(x)<L+\epsilon\]
    and so \(|g(x)-L|<\epsilon\).
    Therefore \(\displaystyle{\lim_{x\to a}g(x)=L}\).
\end{proof}
\begin{problem}
    Show that
    \(\displaystyle{\lim_{x\to 0}x^2\sin\frac{1}{x}=0}\).
\end{problem}

If \(0<\theta<\pi/2\),
then \[\sin\theta<\theta\implies\frac{\sin\theta}{\theta}<1\]
and \(\theta\leq\tan\theta\).
Therefore we have
\[\theta<\tan\theta=\frac{\sin\theta}{\cos\theta}
\implies\cos\theta<\frac{\sin\theta}{\theta}<1\]
We know that \(\displaystyle{\lim_{\theta\to 0}1=1}\) and
\(\displaystyle{\lim_{\theta\to 0}\cos\theta=1}\),
so by the squeeze theorem,
we have
\[\lim_{\theta\to 0^+}\frac{\sin\theta}{\theta}=1\]
But the function \((\sin\theta)/\theta\) is an even function,
so its left and right limits must be equal.
Hence we have
\[\lim_{\theta\to 0}\frac{\sin\theta}{\theta}=1\]
\begin{problem}
    Find \(\displaystyle{\lim_{x\to 0}\frac{\sin 7x}{4x}}\)
\end{problem}
\begin{problem}
    Evaluate \(\displaystyle{\lim_{\theta\to 0}\frac{\cos\theta-1}{\theta}}\).
\end{problem}
\begin{solution}
    We have
    \[\frac{\cos\theta-1}{\theta}
    =\frac{\cos\theta-1}{\theta}\left(\frac{\cos\theta+1}{\cos\theta+1}\right)
    =\frac{\cos^2\theta-1}{\theta(\cos\theta+1)}
    =\frac{-\sin^2\theta}{\theta(\cos\theta+1)}
    =\frac{\sin\theta}{\theta}\left(\frac{-\sin\theta}{\cos\theta+1}\right)\]
    We take the limit then
    \begin{align*}
        \lim_{\theta\to 0}\frac{\cos\theta-1}{\theta}
        &= \lim_{\theta\to 0}\left[\frac{\sin\theta}{\theta}
        \left(\frac{-\sin\theta}{\cos\theta+1}\right)\right]
        =\left(\lim_{\theta\to 0}\frac{\sin\theta}{\theta}\right)
        \left(\lim_{\theta\to 0}\frac{-\sin\theta}{\cos\theta+1}\right) \\
        &= 1\left(\frac{(-1)(0)}{1+1}\right)=0 
    \end{align*}
\end{solution}