\subsection{Computing Limits}

\subsubsection{Limit Laws}
Suppose that \(c\) is a constant and the limits
\begin{align*}
    \lim_{x\to a}f(x)&=L&\lim_{x\to a}g(x)&=M
\end{align*}
exist.
We have the following properties of limits called the \textbf{limit laws} to
compute limits.
\begin{theorem}
    \[\lim_{x\to a}c=c\]
\end{theorem}
\begin{proof}
    Let \(\epsilon>0\) be given, we want to find a number \(\delta>0\) such
    that \[0<|x-a|<\delta\implies|c-c|<\epsilon\]
    We have \(|c-c|=0<\epsilon\) so the trivial inequality is always true for
    any number \(\delta>0\).
\end{proof}
\begin{theorem}
    \[\lim_{x\to a}x=a\]
\end{theorem}
\begin{proof}
    Let \(\epsilon>0\) be given, we want to find a number \(\delta>0\) such
    that \[0<|x-a|<\delta\implies|x-a|<\epsilon\]
    Let \(\delta=\epsilon\) so we have
    \[0<|x-a|<\delta=\epsilon\implies|x-a|<\epsilon\]
\end{proof}
\begin{theorem}[Constant Multiple Law]
    The limit of a constant times a function is the constant times the limit
    of the function.
    \[\lim_{x\to a}[c\,f(x)]=c\lim_{x\to a}f(x)=cL\]
\end{theorem}
\begin{proof}
    Note that if \(c=0\), then \(cf(x)=0\) and we have
    \[\lim_{x\to a}[0\cdot f(x)]=\lim_{x\to a}0=0=0\cdot\lim_{x\to a}f(x)\]
    Let \(\epsilon>0\) and \(c\neq0\) be given, we want to find a number
    \(\delta>0\) such that
    \[0<|x-a|<\delta\implies|cf(x)-c\lim_{x\to a}f(x)|<\epsilon\]
    We simplify to get \[|f(x)-L|<\frac{\epsilon}{|c|}\]
    By the definition of the limit, there is a number \(\delta_1>0\) such that
    \[0<|x-a|<\delta_1\implies|f(x)-L|<\frac{\epsilon}{|c|}\]
    Let \(\delta=\delta_1\), we have \[0<|x-a|<\delta\implies|cf(x)-c\lim_{x\to a}f(x)|<\epsilon\]
\end{proof}
\begin{theorem}[\textbf{Sum and Difference Law}]
    The limit of a sum or difference is the sum or difference of the limits.
    \[\lim_{x\to a}[f(x)\pm g(x)]=\lim_{x\to a}f(x)\pm\lim_{x\to a}g(x)=L\pm M\]
\end{theorem}
\begin{proof}
    We prove the sum law first.
    Let \(\epsilon>0\) be given, we want to find a number \(\delta>0\) such
    that \[0<|x-a|<\delta\implies|f(x)+g(x)-(L+M)|<\epsilon\]
    By the \textbf{triangle inequality} \(|a+b|\leq|a|+|b|\), we have
    \[|f(x)+g(x)-(L+M)|=|f(x)-L+g(x)-M|\leq|f(x)-L|+|g(x)-M|\]
    Since \(\lim_{x\to a}f(x)=L\), there is a number \(\delta_1\) such that
    \[0<|x-a|<\delta_1\implies|f(x)-L|<\frac{\epsilon}{2}\]
    Similary, there is a number \(\delta_2\) such that
    \[0<|x-a|<\delta_2\implies|g(x)-M|<\frac{\epsilon}{2}\]
    Let \(\delta=\min\{\delta_1,\delta_2\}\) such that
    \[0<|x-a|<\delta\implies0<|x-a|<\delta_1,\delta_2\]
    \[|f(x)-L|+|g(x)-M|<\frac{\epsilon}{2}+\frac{\epsilon}{2}=\epsilon\]
    Then we have \[0<|x-a|<\delta\implies|f(x)+g(x)-(L+M)|<\epsilon\]
    Therefore, by the definition of the limit, it is proved that
    \[\lim_{x\to a}[f(x)+g(x)]=\lim_{x\to a}f(x)+\lim_{x\to a}g(x)=L+M\]
    We prove the difference law using the sum law and the constant multiple
    law with \(c=-1\).
    \begin{align*}
        \lim_{x\to a}[f(x)-g(x)]&=\lim_{x\to a}[f(x)+(-1)g(x)]
        =\lim_{x\to a}f(x)+\lim_{x\to a}(-1)g(x) \\
        &=\lim_{x\to a}f(x)+(-1)\lim_{x\to a}g(x) \\
        &=\lim_{x\to a}f(x)-\lim_{x\to a}g(x) = L-M
    \end{align*}
\end{proof}
\begin{theorem}[Product Law]
    The limit of a product is the product of the limits.
    \[\lim_{x\to a}[f(x)g(x)]=\lim_{x\to a}f(x)\cdot\lim_{x\to a}g(x)=L\cdot M\]
\end{theorem}
\begin{proof}
    Let \(\epsilon>0\) be given, we want to find a number \(\delta>0\) such
    that \[0<|x-a|<\delta\implies|f(x)g(x)-LM|<\epsilon\]
    By the triangle inequality, we have
    \begin{align*}
        |f(x)g(x)-LM|&=|f(x)g(x)-Lg(x)+Lg(x)-LM| \\ &=|[f(x)-L]g(x)+L[g(x)-M]|
        \\ &\leq |[f(x)-L]g(x)|+|L[g(x)-M]| \\ &=|[f(x)-L]|\,|g(x)|+|L|\,|[g(x)-M]|
    \end{align*}
    Since \(\lim_{x\to a}g(x)=M\), there is a number \(\delta_1>0\) such that
    \[0<|x-a|<\delta_1\implies|g(x)-M|<\frac{\epsilon}{2(1+|L|)}\]
    Also, there is a number \(\delta_2>0\) such that
    \[0<|x-a|<\delta_2\implies|g(x)-M|<1\] and therefore
    \[|g(x)|=|g(x)-M+M|\leq|g(x)-M|+|M|<1+|M|\]
    Since \(\lim_{x\to a}f(x)=L\), there is a number \(\delta_3>0\) such that
    \[0<|x-a|<\delta_3\implies|f(x)-L|<\frac{\epsilon}{2(1+|M|)}\]
    Let \(\delta=\min\{\delta_1,\delta_2,\delta_3\}\) such that
    \[0<|x-a|<\delta\implies0<|x-a|<\delta_1,\delta_2,\delta_3\]
    so we can combine the inequalities to get
    \begin{align*}
        |f(x)g(x)-LM|&\leq|[f(x)-L]|\,|g(x)|+|L|\,|[g(x)-M]| \\
        &<\frac{\epsilon}{2(1+|M|)}(1+|M|)+|L|\frac{\epsilon}{2(1+|L|)} \\
        &<\frac{\epsilon}{2}+\frac{\epsilon}{2}=\epsilon
    \end{align*}
    It is proved that
    \[\lim_{x\to a}[f(x)g(x)]=\lim_{x\to a}f(x)\cdot\lim_{x\to a}g(x)=L\cdot M\]
\end{proof}
\begin{theorem}[Quotient Law]
    The limit of a quotient is the quotient of the limits (if that the
    limit of the denominator is not 0).
    \[\lim_{x\to a}\frac{f(x)}{g(x)}=\frac{\lim_{x\to a}f(x)}{\lim_{x\to a}g(x)}
    =\frac{L}{M}\iff\lim_{x\to a}g(x)=M\neq0\]
\end{theorem}
\begin{theorem}[Power Law]
    \[\lim_{x\to a}[f(x)]^n=[\lim_{x\to a}f(x)]^n,n\in\R\]
\end{theorem}
\begin{theorem}[Root Law]
    \[\lim_{x\to a}\sqrt[n]{f(x)}=\sqrt[n]{\lim_{x\to a}f(x)},n\in\R\]
\end{theorem}
If \(f(x)=g(x)\) when \(x\neq a\), then \(\lim_{x\to a}f(x)=\lim_{x\to a}g(x)\)
if the limits exist.
\begin{theorem}
    If \(f(x)\leq g(x)\) when \(x\) is near \(a\) (except possibly at \(a\))
    and the limits of \(f\) and \(g\) exist, then
    \[\lim_{x\to a}f(x)\leq\lim_{x\to a}g(x)\]
\end{theorem}
\begin{theorem}[Squeeze Theorem]
    If \(f(x)\leq g(x)\leq h(x)\) when \(x\) is near \(a\) (except possibly at
    \(a\)) and \[\lim_{x\to a}f(x)=\lim_{x\to a}h(x)=L\] then \[\lim_{x\to a}g(x)=L\]
\end{theorem}