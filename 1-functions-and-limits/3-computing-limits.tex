\subsection{Computing Limits}

\subsubsection{Limit Laws}
Suppose that \(c\) is a constant and the limits
\begin{align*}
    \lim_{x\to a}f(x)&=L&\lim_{x\to a}g(x)&=M
\end{align*}
exist.
We have the following properties of limits called the \textbf{limit laws} to
compute limits.
\begin{theorem}
    \[\lim_{x\to a}c=c\]
\end{theorem}
\begin{proof}
    Let \(\epsilon>0\) be given, we want to find a number \(\delta>0\) such
    that \[0<|x-a|<\delta\implies|c-c|<\epsilon\]
    We have \(|c-c|=0<\epsilon\) so the trivial inequality is always true for
    any number \(\delta>0\).
    Therefore, by the definition of the limit, it is proved that
    \[\lim_{x\to a}c=c\qedhere\]
\end{proof}
\begin{theorem}
    \[\lim_{x\to a}x=a\]
\end{theorem}
\begin{proof}
    Let \(\epsilon>0\) be given, we want to find a number \(\delta>0\) such
    that \[0<|x-a|<\delta\implies|x-a|<\epsilon\]
    Let \(\delta=\epsilon\), we have
    \[0<|x-a|<\delta=\epsilon\implies|x-a|<\epsilon\]
    Therefore, by the definition of the limit, it is proved that
    \[\lim_{x\to a}x=a\qedhere\]
\end{proof}
\begin{theorem}[Constant Multiple Law]
    The limit of a constant times a function is the constant times the limit
    of the function.
    \[\lim_{x\to a}[c\,f(x)]=c\lim_{x\to a}f(x)=cL\]
\end{theorem}
\begin{proof}
    Note that if \(c=0\), then \(cf(x)=0\) and we have
    \[\lim_{x\to a}[0\cdot f(x)]=\lim_{x\to a}0=0=0\cdot\lim_{x\to a}f(x)\]
    Let \(\epsilon>0\) and \(c\neq0\) be given, we want to find a number
    \(\delta>0\) such that
    \[0<|x-a|<\delta\implies|cf(x)-c\lim_{x\to a}f(x)|<\epsilon\]
    We simplify to get \[|f(x)-L|<\frac{\epsilon}{|c|}\]
    By the definition of the limit, there is a number \(\delta_1>0\) such that
    \[0<|x-a|<\delta_1\implies|f(x)-L|<\frac{\epsilon}{|c|}\]
    Let \(\delta=\delta_1\), we have
    \[0<|x-a|<\delta\implies|cf(x)-c\lim_{x\to a}f(x)|<\epsilon\]
    Therefore, by the definition of the limit, it is proved that
    \[\lim_{x\to a}[c\,f(x)]=c\lim_{x\to a}f(x)=cL\qedhere\]
\end{proof}
\begin{theorem}[\textbf{Sum and Difference Law}]
    The limit of a sum or difference is the sum or difference of the limits.
    \[\lim_{x\to a}[f(x)\pm g(x)]=\lim_{x\to a}f(x)\pm\lim_{x\to a}g(x)=L\pm M\]
\end{theorem}
\begin{proof}
    First we prove the sum law.
    Let \(\epsilon>0\) be given, we want to find a number \(\delta>0\) such
    that \[0<|x-a|<\delta\implies|f(x)+g(x)-(L+M)|<\epsilon\]
    By the \textbf{triangle inequality} \(|a+b|\leq|a|+|b|\), we have
    \[|f(x)+g(x)-(L+M)|=|f(x)-L+g(x)-M|\leq|f(x)-L|+|g(x)-M|\]
    Since \(\lim_{x\to a}f(x)=L\), there is a number \(\delta_1\) such that
    \[0<|x-a|<\delta_1\implies|f(x)-L|<\frac{\epsilon}{2}\]
    Similarly, there exists \(delta_2>0\), there is a number \(\delta_2\) such that such that
    \[0<|x-a|<\delta_2\implies|h(x)-L|<\epsilon\implies L\]
    \[0<|x-a|<\delta_2\implies|g(x)-M|<\frac{\epsilon}{2}\]
    Let \(\delta=\min\{\delta_1,\delta_2\}\) such that
    \[0<|x-a|<\delta\implies0<|x-a|<\delta_1,\delta_2\]
    \[|f(x)-L|+|g(x)-M|<\frac{\epsilon}{2}+\frac{\epsilon}{2}=\epsilon\]
    Then we have \[0<|x-a|<\delta\implies|f(x)+g(x)-(L+M)|<\epsilon\]
    Therefore, by the definition of the limit, it is proved that
    \[\lim_{x\to a}[f(x)+g(x)]=\lim_{x\to a}f(x)+\lim_{x\to a}g(x)=L+M\]
    We prove the difference law using the sum law and the constant multiple
    law with \(c=-1\).
    \begin{align*}
        \lim_{x\to a}[f(x)-g(x)]&=\lim_{x\to a}[f(x)+(-1)g(x)]
        =\lim_{x\to a}f(x)+\lim_{x\to a}(-1)g(x) \\
        &=\lim_{x\to a}f(x)+(-1)\lim_{x\to a}g(x) \\
        &=\lim_{x\to a}f(x)-\lim_{x\to a}g(x) = L-M
    \end{align*}
    Therefore, it is proved that
    \[\lim_{x\to a}[f(x)\pm g(x)]=\lim_{x\to a}f(x)\pm\lim_{x\to a}g(x)=L
    \pm M\qedhere\]
\end{proof}
\begin{theorem}[Product Law]
    The limit of a product is the product of the limits.
    \[\lim_{x\to a}[f(x)g(x)]=\lim_{x\to a}f(x)\cdot\lim_{x\to a}g(x)=L\cdot M\]
\end{theorem}
\begin{proof}
    Let \(\epsilon>0\) be given, we want to find a number \(\delta>0\) such
    that \[0<|x-a|<\delta\implies|f(x)g(x)-LM|<\epsilon\]
    By the triangle inequality, we have
    \begin{align*}
        |f(x)g(x)-LM|&=|f(x)g(x)-Lg(x)+Lg(x)-LM|=|[f(x)-L]g(x)+L[g(x)-M]| \\
        &\leq |[f(x)-L]g(x)|+|L[g(x)-M]|=|f(x)-L|\,|g(x)|+|L|\,|g(x)-M|
    \end{align*}
    We want to make both of the terms less than \(\epsilon/2\).
    Since \(\lim_{x\to a}f(x)=L\), there is a number \(\delta_1>0\) such that
    \[0<|x-a|<\delta_1\implies|f(x)-L|<\frac{\epsilon}{2(1+|M|)}\]
    Since \(\lim_{x\to a}g(x)=M\), there is a number \(\delta_2>0\) such that
    \[0<|x-a|<\delta_2\implies|g(x)-M|<\frac{\epsilon}{2(1+|L|)}\]
    Also, there is a number \(\delta_3>0\) such that
    \[0<|x-a|<\delta_3\implies|g(x)-M|<1\] and therefore
    \[|g(x)|=|g(x)-M+M|\leq|g(x)-M|+|M|<1+|M|\]
    Let \(\delta=\min\{\delta_1,\delta_2,\delta_3\}\) such that
    \[0<|x-a|<\delta\implies0<|x-a|<\delta_1,\delta_2,\delta_3\]
    so we can combine the inequalities to get
    \begin{align*}
        |f(x)g(x)-LM|&\leq|f(x)-L|\,|g(x)|+|L|\,|g(x)-M| \\
        &<\frac{\epsilon}{2(1+|M|)}(1+|M|)+(1+|L|)\frac{\epsilon}{2(1+|L|)} \\
        &<\frac{\epsilon}{2}+\frac{\epsilon}{2}=\epsilon
    \end{align*}
    Therefore, it is proved that
    \[\lim_{x\to a}[f(x)g(x)]=\lim_{x\to a}f(x)\cdot\lim_{x\to a}g(x)
    =L\cdot M\qedhere\]
\end{proof}
\begin{theorem}[Quotient Law]
    The limit of a quotient is the quotient of the limits (if that the
    limit of the denominator is not 0).
    \[\lim_{x\to a}\frac{f(x)}{g(x)}=\frac{\lim_{x\to a}f(x)}{\lim_{x\to a}g(x)}
    =\frac{L}{M}\iff\lim_{x\to a}g(x)=M\neq0\]
\end{theorem}
\begin{proof}
    First we prove that \[\lim_{x\to a}\frac{1}{g(x)}=\frac{1}{M}\]
    Let \(\epsilon>0\) be given, we want to find a number \(\delta>0\) such that
    \[0<|x-a|<\delta\implies\left|\frac{1}{g(x)}-\frac{1}{M}\right|<\epsilon\]
    Notice that \[\left|\frac{1}{g(x)}-\frac{1}{M}\right|=\frac{|M-g(x)|}{|Mg(x)|}
    =\frac{|g(x)-M|}{|Mg(x)|}\]
    Since \(\lim_{x\to a}g(x)=M\), there is a number \(\delta_1\) such that
    \[0<|x-a|<\delta_1\implies|g(x)-M|<\frac{|M|}{2}\] and therefore
    \[|M|=|M-g(x)+g(x)|\leq|M-g(x)|+|g(x)|=|g(x)-M|+|g(x)|<\frac{|M|}{2}+|g(x)|\]
    It is shown that
    \[0<|x-a|<\delta_1\implies\frac{|M|}{2}<|g(x)|\implies\frac{1}{|g(x)|}
    <\frac{2}{|M|}\]
    It follows that for these values of \(x\),
    \[\frac{1}{|Mg(x)|}=\frac{1}{|M|\,|g(x)|}<\frac{1}{|M|}\frac{2}{|M|}
    =\frac{2}{M^2}\]
    Also, there is a number \(\delta_2>0\) such that
    \[0<|x-a|<\delta_2\implies|g(x)-M|<\frac{M^2}{2}\epsilon\]
    Let \(\delta=\min\{\delta_1,\delta_2\}\), if \(0<|x-a|<\delta\), then
    \[\left|\frac{1}{g(x)}-\frac{1}{M}\right|=\frac{|M-g(x)|}{|Mg(x)|}
    =\frac{1}{|Mg(x)|}|g(x)-M|<\frac{2}{M^2}\frac{M^2}{2}\epsilon=\epsilon\]
    which is what we want to show.
    We apply the product law to prove the quotient law
    \[\lim_{x\to a}\frac{f(x)}{g(x)}=\lim_{x\to a}f(x)\left(\frac{1}{g(x)}\right)
    =\lim_{x\to a}f(x)\lim_{x\to a}\frac{1}{g(x)}=L\cdot\frac{1}{M}
    =\frac{L}{M}\qedhere\]
\end{proof}
\begin{theorem}[Power Law]
    \[\lim_{x\to a}[f(x)]^n=[\lim_{x\to a}f(x)]^n,n\in\R\]
\end{theorem}
\begin{theorem}[Root Law]
    \[\lim_{x\to a}\sqrt[n]{f(x)}=\sqrt[n]{\lim_{x\to a}f(x)},n\in\R\]
\end{theorem}
\begin{theorem}[Direct Substitution Property]
    If \(f\) is a polynomial function, rational function, or trigonometric
    function and \(a\) is in the domain of \(f\), then
    \[\lim_{x\to a}f(x)=f(a)\]
\end{theorem}
Thus, we have the following limits
\begin{align*}
    \lim_{\theta\to 0}\sin\theta&=0 & \lim_{\theta\to 0}\cos\theta&=1
\end{align*}
If \(f(x)=g(x)\) when \(x\neq a\), then \(\lim_{x\to a}f(x)=\lim_{x\to a}g(x)\)
if the limits exist.
\begin{problem}
    Show that \[\lim_{x\to 0}|x|=0\]
\end{problem}
\begin{solution}
    Since \(|x|=x\) for \(x>0\), we have
    \[\lim_{x\to 0^+}|x|=\lim_{x\to 0^+}x=0\]
    For \(x<0\) we have \(|x|=-x\) so
    \[\lim_{x\to 0^-}|x|=\lim_{x\to 0^-}(-x)=0\]
    Therefore, it is shown that \[\lim_{x\to 0}|x|=0\qedhere\]
\end{solution}
\begin{theorem}
    If \(f(x)\leq g(x)\) for all \(x\) in an open interval that contains \(a\),
    except possibly at \(a\), and
    \begin{align*}
        \lim_{x\to a}f(x)&=L&\lim_{x\to a}g(x)&=M
    \end{align*}
    then \[\lim_{x\to a}f(x)\leq\lim_{x\to a}g(x)\iff L\leq M\]
\end{theorem}
\begin{proof}
    We use the method of proof by contradiction.
    Suppose  that \(L>M\), then we have
    \[\lim_{x\to a}[g(x)-f(x)]=M-L\]
    Therefore, for any number \(\epsilon>0\), there exists \(\delta>0\) such that
    \[0<|x-a|<\delta\implies|g(x)-f(x)-(M-L)|<\epsilon\]
    Note that \(L-M>0\) by the hypothesis.
    Let \(\epsilon=L-M\), there exists \(\delta>0\) such that
    \[0<|x-a|<\delta\implies|g(x)-f(x)-(M-L)|<L-M\]
    Since \(a\leq|a|\) for any number \(a\), we have
    \[0<|x-a|<\delta\implies g(x)-f(x)-(M-L)<L-M\]
    which simplifies to
    \[0<|x-a|<\delta\implies g(x)<f(x)\]
    but this is a contradiction since given \(f(x)\leq g(x)\).
    Then the inequality \(L>M\) must be false so \(L\leq M\) must be true.
    Therefore, it is proved that
    \[\lim_{x\to a}f(x)\leq\lim_{x\to a}g(x)\qedhere\]
\end{proof}
\begin{theorem}[Squeeze Theorem]
    If \(f(x)\leq g(x)\leq h(x)\) for all \(x\) in an open interval that
    contains \(a\), except possibly at \(a\), and
    \[\lim_{x\to a}f(x)=\lim_{x\to a}h(x)=L\] then \[\lim_{x\to a}g(x)=L\]
\end{theorem}
\begin{proof}
    Let \(\epsilon>0\) be given. Since \(\lim_{x\to a}f(x)=L\), there exists
    \(\delta_1>0\) such that
    \[0<|x-a|<\delta_1\implies|f(x)-L|<\epsilon\implies L-\epsilon<f(x)<L+\epsilon\]
    Since \(\lim_{x\to a}h(x)=L\), there exists \(\delta_2>0\) such that
    \[0<|x-a|<\delta_2\implies|h(x)-L|<\epsilon\implies L-\epsilon<h(x)<L+\epsilon\]
    Let \(\delta=\min\{\delta_1,\delta_2\}\). If \(0<|x-a|<\delta\), then we have
    \[L-\epsilon<f(x)\leq g(x)\leq h(x)<L+\epsilon\implies L-\epsilon<g(x)<L+\epsilon
    \implies|g(x)-L|<\epsilon\] which is what we want to prove.
    Therefore, it is proved that \[\lim_{x\to a}g(x)=L\qedhere\]
\end{proof}
By \textbf{algebra}, \textbf{geometry}, and \textbf{trigonometry}, we can get
the following result by the \textbf{Pythagorean theorem} \(a^2+b^2=c^2\).
If \(0<\theta<\pi/2\), then
\begin{align*}
    \sin\theta<\theta&\implies\frac{\sin\theta}{\theta}<1
    &\theta<\tan\theta=\frac{\sin\theta}{\cos\theta}\implies\cos\theta
    <\frac{\sin\theta}{\theta}
\end{align*}
so we have the following inequality
\[\cos\theta<\frac{\sin\theta}{\theta}<1\]
Since \((\sin\theta)/\theta\) is an even function, its left and right limits
must be equal.
Therefore, we have the following limit by the squeeze theorem.
\[\lim_{\theta\to 0}\frac{\sin\theta}{\theta}=1\]
\begin{problem}
    Evaluate \[\lim_{\theta\to 0}\frac{\cos\theta-1}{\theta}\]
\end{problem}
\begin{solution}
    By the \textbf{Pythagorean identity} \(\sin^2\theta+\cos^2\theta=1\), we have
    \[\frac{\cos\theta-1}{\theta}
    =\frac{\cos\theta-1}{\theta}\left(\frac{\cos\theta+1}{\cos\theta+1}\right)
    =\frac{\cos^2\theta-1}{\theta(\cos\theta+1)}
    =\frac{-\sin^2\theta}{\theta(\cos\theta+1)}
    =\frac{\sin\theta}{\theta}\left(\frac{-\sin\theta}{\cos\theta+1}\right)\]
    We take the limit and we have
    \[\lim_{\theta\to 0}\frac{\cos\theta-1}{\theta}
    =\lim_{\theta\to 0}\left[\frac{\sin\theta}{\theta}
    \left(\frac{-\sin\theta}{\cos\theta+1}\right)\right]
    =\left(\lim_{\theta\to 0}\frac{\sin\theta}{\theta}\right)
    \left(\lim_{\theta\to 0}\frac{-\sin\theta}{\cos\theta+1}\right)
    =0\left(\frac{-1\cdot0}{1+1}\right)=0\]
    Therefore, it is shown that
    \[\lim_{\theta\to 0}\frac{\cos\theta-1}{\theta}=0\qedhere\]
\end{solution}