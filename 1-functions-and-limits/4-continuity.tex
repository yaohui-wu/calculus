\subsection{Continuity}

\begin{definition}
    A function \(f\) is \textbf{continuous at a number} \(a\) if
    \[\lim_{x\to a}f(x)=f(a)\]
\end{definition}
Note that \(f\)  is continuous at \(a\) requires that \(f(a)\) is defined and
the limit exists.
We say that \(f\) is \textbf{discontinuous} at \(a\) if \(f\) is not
continuous at \(a\).
\begin{definition}
    A function \(f\) is \textbf{continuous from the right at a number} \(a\)
    if
    \[\lim_{x\to a^+}f(x)=f(a)\]
    and \(f\) is \textbf{continuous from the left at} \(a\) if
    \[\lim_{x\to a^-}f(x)=f(a)\]
\end{definition}
\begin{definition}
    A function \(f\) is \textbf{continuous on an interval} if it is continuous
    at every number in the interval.
\end{definition}
\begin{theorem}
    If \(f\) and \(g\) are continuous at \(a\) and \(c\) is a constant,
    then
    the following functions are also continuous at \(a\):
    \begin{enumerate}
        \item \(f+g\)
        \item \(f-g\)
        \item \(cf\)
        \item \(fg\)
        \item \(\dfrac{f}{g}\) if \(g(a)\neq 0\).
    \end{enumerate}
\end{theorem}
\begin{proof}
    Each of the five parts of this theorem follows from the corresponding
    limit law.
    We give the proof of part 1.
    Since \(f\) and \(g\) are continuous at \(a\),
    we have
    \begin{align*}
        \lim_{x\to a}f(x) &= f(a) & \lim_{x\to a}g(x) &= g(a)
    \end{align*}
    Then
    \[\lim_{x\to a}(f+g)(x)=\lim_{x\to a}\bigl[f(x)+g(x)\bigr]
    =\lim_{x\to a}f(x)+\lim_{x\to a}g(x)=f(a)+g(a)=(f+g)(a)\]
    This shows that \(f+g\) is continuous at \(a\).
\end{proof}
\begin{theorem}
    Any polynomial is continuous on \(\R=(-\infty,\infty)\).
    Any rational function is continuous on its domain.
\end{theorem}
\begin{proof}
    A polynomial is a function of the form
    \[P(x)=a_nx^n+a_{n-1}x^{n-1}+\dotsb +a_{2}x^{2}+a_1x+a_0\]
    where the coefficients \(a_0,a_1,\dots,a_n\) are constants.
    \(P(x)\) is the sum of power functions with a constant multiple and
    therefore it is continuous.
    A rational function is a function of the form
    \[f(x)=\frac{P(x)}{Q(x)}\]
    where \(P\) and \(Q\) are polynomials.
    The domain of \(f\) is \(D=\{x\in\R\mid Q(x)\neq 0\}\).
    We know that polynomials are continuous on \(\R\) so the rational function
    \(f\) is continuous at every number in \(D\).
\end{proof}
\begin{theorem}
    The following types of functions are continuous at every number in their
    domains:
    \begin{itemize}
        \item Polynomials
        \item Rational functions
        \item Root functions
        \item Trigonometric functions
        \item Inverse trigonometric functions
        \item Exponential functions
        \item Logarithmic functions
    \end{itemize}
\end{theorem}
\begin{theorem}
    If \(f\) is continuous at \(b\) and
    \(\displaystyle{\lim_{x\to a}g(x)=b}\),
    then
    \[\lim_{x\to a}f(g(x))=f(b)\]
\end{theorem}
\begin{proof}
    Let \(\epsilon>0\) be given.
    We want to find a number \(\delta>0\) such that
    \[0<|x-a|<\delta\implies|f(g(x))-f(b)|<\epsilon\]
    Since \(f\) is continuous at \(b\),
    we have
    \[\lim_{y\to b}f(y)=f(b)\]
    and so there exists \(\delta_1>0\) such that
    \[0<|y-b|<\delta_1\implies|f(y)-f(b)|<\epsilon\]
    Since \(\displaystyle{\lim_{x\to a}g(x)=b}\),
    there exists \(\delta>0\) such that
    \[0<|x-a|<\delta\implies|g(x)-b|<\delta_1\]
    Combining these two statements,
    we see that when \(0<|x-a|<\delta\) we have \(|g(x)-b|<\delta_1\),
    which implies that \(|f(g(x))-f(b)|<\epsilon\).
    Therefore we have proved that \(\displaystyle{\lim_{x\to a}f(g(x))=f(b)}\).
\end{proof}
\begin{theorem}
    If \(g\) is continuous at \(a\) and \(f\) is continuous at \(g(a)\),
    then the composite function \(f\circ g\) given by \(f\circ g=f(g(x))\) is
    continuous at \(a\).
\end{theorem}
\begin{proof}
    Since \(g\) is continuous at \(a\),
    we have
    \[\lim_{x\to a}g(x)=g(a)\]
    Since \(f\) is continuous at \(b=g(a)\),
    we have
    \[\lim_{x\to a}f(g(x))=f(g(a))\]
    which is precisely the statement that the function \(f(g(x))\) is
    continuous at \(a\).
\end{proof}
\begin{theorem}[Intermediate Value Theorem]
    Suppose that \(f\) is continuous on the closed interval \([a,b]\) and let
    \(N\) be any number between \(f(a)\) and \(f(b)\),
    where \(f(a)\neq f(b)\).
    Then there exists a number \(c\) in the open interval \((a,b)\) such that
    \(f(c)=N\).
\end{theorem}
The intermediate value theorem states that a continuous function takes on
every intermediate value between the function values \(f(a)\) and \(f(b)\).
If a continuous function \(f(x)\) has values of opposite sign in an interval
\((a,b)\),
then by the intermediate value theorem there exists a root of \(f(x)\) in \((a,b)\).