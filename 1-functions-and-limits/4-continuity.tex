\subsection{Continuity}

\subsubsection*{Continuity of a Function}
\begin{definition}
    A function \(f\) is \textbf{continuous at a number} \(a\) if
    \[\lim_{x\to a}f(x)=f(a)\]
\end{definition}
Note that \(f\)  is continuous at \(a\) requires that \(f(a)\) is defined and
the limit exists.
We say that \(f\) is \textbf{discontinuous} at \(a\) if \(f\) is not
continuous at \(a\).
If we can remove the discontinuity by redefining \(f\) at \(a\),
then it is a \textbf{removable discontinuity}.
If \(f\) has a vertical asymptote,
then it has an \textbf{infinite discontinuity}.
If the left-hand limit is not equal to the right-hand limit,
then \(f\) has a \textbf{jump discontinuity}.
\begin{definition}
    A function \(f\) is \textbf{continuous from the right at a number} \(a\)
    if
    \[\lim_{x\to a^+}f(x)=f(a)\]
    and \(f\) is \textbf{continuous from the left at} \(a\) if
    \[\lim_{x\to a^-}f(x)=f(a)\]
\end{definition}
\begin{definition}
    A function \(f\) is \textbf{continuous on an interval} if it is continuous
    at every number in the interval.
\end{definition}

\subsubsection*{Properties of Continuous Functions}
\begin{theorem}
    If \(f\) and \(g\) are continuous at \(a\) and \(c\) is a constant, then
    the following functions are also continuous at \(a\):
    \begin{enumerate}
        \item \(f+g\)
        \item \(f-g\)
        \item \(cf\)
        \item \(fg\)
        \item \(\dfrac{f}{g}\) if \(g(a)\neq 0\).
    \end{enumerate}
\end{theorem}
\begin{theorem}
    Let \(P(x)\) be any polynomial, then \(P(x)\) is continuous on
    \(\R=(-\infty,\infty)\).
\end{theorem}
\begin{proof}
    A polynomial \(P(x)\) is a function of the form
    \[a_nx^n+a_{n-1}x^{n-1}+\dotsb +a_{2}x^{2}+a_1x+a_0\]
    where the coefficients \(a_0,a_1,\dots,a_n\) are constants.
    \(P(x)\) is the sum of power functions with a constant multiple and
    therefore it is continuous.
\end{proof}
\begin{theorem}
    Let \(f\) be any rational function,
    then \(f\) is continuous on its domain.
\end{theorem}
\begin{proof}
    A rational function \(f\) is a function of the form
    \[f(x)=\frac{P(x)}{Q(x)}\] where \(P\) and \(Q\) are polynomials.
    We know that polynomials are continuous so a rational function is
    continuous on its domain.
\end{proof}
\begin{theorem}
    The following types of functions are continuous at every number in their
    domains:
    \begin{itemize}
        \item Polynomials
        \item Rational functions
        \item Root functions
        \item Trigonometric functions
        \item Inverse trigonometric functions
        \item Exponential functions
        \item Logarithmic functions
    \end{itemize}
\end{theorem}
\begin{theorem}
    If \(f\) is continuous at \(b\) and
    \(\displaystyle{\lim_{x\to a}g(x)=b}\), then
    \[\lim_{x\to a}f(g(x))=f\Bigl(\lim_{x\to a}g(x)\Bigr)=f(b)\]
\end{theorem}
\begin{proof}
    Let \(\epsilon>0\) be given, we want to find \(\delta>0\) such that
    \[0<|x-a|<\delta\implies|f(g(x))-f(b)|<\epsilon\]
    Since \(f\) is continuous at \(b\),
    then we have \(\lim_{y\to b}f(y)=f(b)\).
    There exists \(\delta_1>0\) such that
    \[0<|y-b|<\delta_1\implies|f(y)-f(b)|<\epsilon\]
    Since \(\lim_{x\to a}g(x)=b\), there exists \(\delta>0\) such that
    \[0<|x-a|<\delta\implies|g(x)-b|<\delta_1\implies|f(g(x))-f(b)|<\epsilon\]
    It is proved that
    \[\lim_{x\to a}f(g(x))=f\Bigl(\lim_{x\to a}g(x)\Bigr)=f(b)\]
\end{proof}
\begin{theorem}
    If \(g\) is continuous at \(a\) and \(f\) is continuous at \(g(a)\), then
    the composite function \(f(g(x))\) is continuous at \(a\).
\end{theorem}
\begin{proof}
    Since \(g\) is continuous at \(a\), we have \(\lim_{x\to a}g(x)=g(a)\).
    Since \(f\) is continuous at \(g(a)\), we have
    \[\lim_{x\to a}f(g(x))=f(g(a))\]
    Therefore, \(f(g(x))\) is continuous at \(a\).
\end{proof}

\subsubsection*{The Intermediate Value Theorem}
\begin{theorem}[Intermediate Value Theorem]
    Suppose that \(f\) is continuous on the closed interval \([a,b]\) and let
    \(N\) be any number between \(f(a)\) and \(f(b)\),
    where \(f(a)\neq f(b)\).
    Then there exists a number \(c\) in the open interval \((a,b)\) such that
    \(f(c)=N\).
\end{theorem}
If a continuous function \(f(x)\) has values of opposite sign in an interval
\((a,b)\), then there exists a root of \(f(x)\) in \((a,b)\) by the
intermediate value theorem.