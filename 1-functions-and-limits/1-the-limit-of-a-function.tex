\subsection{The Limit of a Function}

\subsubsection*{Functions}
\begin{definition}
    A \textbf{function} \(f\) is a rule that assigns to each element \(x\) in
    a set \(D\) exactly one element \(f(x)\) in a set \(E\).
\end{definition}
\begin{definition}
    A function \(f\) is \textbf{injective} (or one-to-one) if
    \(f(x_1)\neq f(x_2)\) when \(x_1\neq x_2\).
\end{definition}
\begin{definition}
    A function \(f\) is \textbf{surjective} (or onto) if for all \(y\) in
    range \(Y\),
    there exists an \(x\) in domain \(X\) such that \(f(x)=y\).
\end{definition}
\begin{definition}
    A function \(f\) is \textbf{bijective} if \(f\) is injective and surjective.
\end{definition}
\begin{definition}
    Let \(f\) be an injective function with domain \(A\) and range \(B\).
    Then its \textbf{inverse function} \(f^{-1}\) has domain \(B\) and range
    \(A\) and is defined by
    \[f^{-1}(y)=x\iff f(x)=y\]
    for all \(y\) in \(B\).
\end{definition}

\subsubsection*{Intuitive Definition of a Limit}
Let \(f(x)\) be a function defined near the number \(a\), that is, \(f(x)\) is
defined on some open interval that contains the number \(a\),
except possibly at \(a\) itself.
\begin{definition}
    The \textbf{limit} of \(f(x)\) as \(x\) approaches \(a\) equals \(L\), and
    we write
    \[\lim_{x\to a}f(x)=L\]
    or \(f(x)\to L\) as \(x\to a\)
    if we can make the values of \(f(x)\) arbitrarily close to \(L\) by
    restricting \(x\) to be sufficiently close to \(a\) but \(x\neq a\).
\end{definition}
\subsubsection*{One-Sided Limits}
\begin{definition}
    We write
    \[\lim_{x\to a^-}f(x)=L\]
    and say that the \textbf{left-hand limit} of \(f(x)\) as \(x\) approaches
    \(a\) equals \(L\) if we can make \(f(x)\) arbitrarily close to \(L\) by
    taking \(x\) sufficiently close to \(a\) with \(x<a\).
\end{definition}
\begin{definition}
    We write
    \[\lim_{x\to a^+}f(x)=L\]
    and say that the \textbf{right-hand limit} of \(f(x)\) as \(x\) approaches
    \(a\) equals \(L\) if we can make \(f(x)\) arbitrarily close to \(L\) by
    taking \(x\) sufficiently close to \(a\) with \(x>a\).
\end{definition}
\begin{theorem}
    \(\displaystyle{\lim_{x\to a}f(x)=L}\) if and only if
    \(\displaystyle{\lim_{x\to a^-}f(x)}=L\) and
    \(\displaystyle{\lim_{x\to a^+}f(x)}=L\).
\end{theorem}
The limit exists if the left-hand limit and the right-hand limit of \(f(x)\)
as \(x\) approaches \(a\) equal \(L\),
otherwise the limit does not exist.

\subsubsection*{Infinite Limits and Vertical Asymptotes}
Let \(f\) be a function defined on both sides of \(a\),
except possibly at \(a\) itself.
\begin{definition}
    \[\lim_{x\to a}f(x)=\infty\]
    or \(f(x)\to\infty\) as \(x\to a\) means that the values of \(f(x)\) can
    be made arbitrarily large by taking \(x\) sufficiently close to \(a\) but
    \(x\neq a\).
\end{definition}
\begin{definition}
    \[\lim_{x\to a}f(x)=-\infty\]
    means that the values of \(f(x)\) can be made arbitrarily large negative
    by taking \(x\) sufficiently close to \(a\) but \(x\neq a\).
\end{definition}
Similar definitions can be given for the one-sided infinite limits
\begin{align*}
    \lim_{x\to a^-}f(x) &= \infty & \lim_{x\to a^+}f(x) &= \infty \\
    \lim_{x\to a^-}f(x) &= -\infty & \lim_{x\to a^+}f(x) &= -\infty
\end{align*}
\begin{definition}
    The vertical line \(x=a\) is the \textbf{vertical asymptote} of the curve
    \(y=f(x)\) if at least one of the following statements is true:
    \begin{align*}
        \lim_{x\to a}f(x) &= \infty & \lim_{x\to a^-}f(x) &= \infty
        & \lim_{x\to a^+}f(x) &= \infty \\
        \lim_{x\to a}f(x) &= -\infty & \lim_{x\to a^-}f(x) &= -\infty
        & \lim_{x\to a^+}f(x) &= -\infty
    \end{align*}
\end{definition}