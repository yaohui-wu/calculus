\subsection{The Limit of a Function}

\subsubsection*{Functions}  
A function \(f:X\mapsto Y\) is a rule that assigns each element \(x\) in set
\(X\) to exactly one element \(y\) in set \(Y\).
\begin{definition}
    A \textbf{function} \(f\) is a binary relation \(R\) between domain \(X\)
    and codomain \(Y\) that satisfies:
    \begin{itemize}
        \item \(R\) is a subset of the Cartesian product of \(X\) and \(Y\).
        \[R\subset\{(x,y)\mid x\in X,y\in Y\}\]
        \item For every \(x\) in \(X\), there exists a \(y\) in \(Y\) such
        that \((x,y)\) is in \(R\).
        \[\forall x\in X,\exists y\in Y,(x,y)\in R\]
        \item If \((x,y)\) and \((x,z)\) are in \(R\), then \(y=z\).
        \[(x,y),(x,z)\in R\implies y=z\qedhere\]
    \end{itemize}
\end{definition}
The real line is the one-dimensional (1D) Eculidean space defined as the set
of all real numbers \(\R\).
The \(xy\)-plane in the Cartesian coordinate system is the two-dimensional
(2D) Eculidean space defined as the set of all ordered pairs of real numbers
\((x,y)\in\R^2\).
A function of a real variable is a function whose domain is the set of real
numbers \(\R\).
A real function is a real-valued function of a real variable whose domain and
codomain is \(\R\).
\begin{definition}
    A function \(f\) is injective, or one-to-one if \(f(x_1)\neq f(x_2)\) when
    \(x_1\neq x_2\).
\end{definition}
\begin{definition}
    A function \(f\) is surjective, or onto if for all \(y\) in range \(Y\),
    there exists an \(x\) in domain \(X\) such that \(f(x)=y\).
\end{definition}
\begin{definition}
    A function \(f\) is bijective if \(f\) is injective and surjective.
\end{definition}
\begin{definition}
    Let \(f\) be an injective function with domain \(X\) and range \(Y\).
    Then its inverse function \(f^{-1}\) has domain \(Y\) and range \(X\) and
    is defined by
    \[f^{-1}(y)=x\iff f(x)=y\]
    for all \(y\) in \(Y\).
\end{definition}

\subsubsection*{Intuitive Definition of a Limit}
Let \(f(x)\) be a function defined on some open interval that contains the
number \(a\), except possibly at \(a\) itself.
\begin{definition}
    The \textbf{limit} of \(f(x)\) as \(x\) approaches \(a\) equals \(L\) if
    we can make \(f(x)\) arbitrarily close to \(L\) by taking \(x\)
    sufficiently close to \(a\) from the left and the right but \(x\neq a\).
    \[\lim_{x\to a}f(x)=L\qedhere\]
\end{definition}
\begin{definition}
    The left-hand limit of \(f(x)\) as \(x\) approaches \(a\) from the left
    equals \(L\) if we can make \(f(x)\) arbitrarily close to \(L\) by taking
    \(x\) sufficiently close to \(a\) where \(x<a\).
    \[\lim_{x\to a^-}f(x)=L\qedhere\]
\end{definition}
\begin{definition}
    The right-hand limit of \(f(x)\) as \(x\) approaches \(a\) from the right
    equals \(L\) if we can make \(f(x)\) arbitrarily close to \(L\) by taking
    \(x\) sufficiently close to \(a\) where \(x>a\).
    \[\lim_{x\to a^+}f(x)=L\qedhere\]
\end{definition}
The limit exists if the left-hand limit and the right-hand limit of \(f(x)\)
as \(x\) approaches \(a\) equal \(L\), otherwise the limit does not exist.
\[\lim_{x\to a}f(x)=L\iff\lim_{x\to a^-}f(x)=\lim_{x\to a^+}f(x)=L\]