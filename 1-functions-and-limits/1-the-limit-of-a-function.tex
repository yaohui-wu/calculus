\subsection{The Limit of a Function}

\subsubsection{Functions}
A function \(f:X\mapsto Y\) is a rule that assigns each element \(x\) in set
\(X\) to exactly one element \(y\) in set \(Y\).
We have a formal definition of a function.
\begin{definition}
    A \textbf{function} \(f\) is a binary relation \(R\) between domain \(X\)
    and codomain \(Y\) that satisfies:
    \begin{itemize}
        \item \(R\) is a subset of the \textbf{Cartesian product} of \(X\) and \(Y\).
        \[R\subset\{(x,y)\mid x\in X,y\in Y\}\]
        \item For every \(x\) in \(X\), there exists a \(y\) in \(Y\) such
        that \((x,y)\) is in \(R\).
        \[\forall x\in X,\exists y\in Y,(x,y)\in R\]
        \item If \((x,y)\) and \((x,z)\) are in \(R\), then \(y=z\).
        \[(x,y)\in R \wedge (x,z)\in R \implies y=z\]
    \end{itemize}
\end{definition}
The real line is the 1-dimensional \textbf{Eculidean space} defined as the set
of real numbers \(\R\).
The \(xy\)-plane in the \textbf{Cartesian coordinate system} by
\textbf{René Descartes} (1596--1650) is the 2-dimensional Eculidean space defined
as the set of all ordered pairs of real numbers \((x,y)\in\R^2\).
A \textbf{function of a real variable} is a function whose domain is the set
of real numbers \(\R\). A real function is a real-valued function of a real
variable whose domain and codomain is \(\R\).

\subsubsection{Intuitive Definition of a Limit}
Newton and Leibniz introduced a working definition of a limit.
Let \(f(x)\) be a function defined on some open interval that contains the
number \(a\), except possibly at \(a\) itself.
\begin{definition}
    The \textbf{limit} of \(f(x)\) as \(x\) approaches \(a\) equals \(L\) if
    we can make \(f(x)\) arbitrarily close to \(L\) by taking \(x\)
    sufficiently close to \(a\) from the left and the right but \(x\neq a\).
    \[\lim_{x\to a}f(x)=L\]
\end{definition}
\begin{definition}
    The \textbf{left-hand limit} of \(f(x)\) as \(x\) approaches \(a\) from
    the left equals \(L\) if we can make \(f(x)\) arbitrarily close to \(L\)
    by taking \(x\) sufficiently close to \(a\) where \(x<a\).
    \[\lim_{x\to a^-}f(x)=L\]
\end{definition}
\begin{definition}
    The \textbf{right-hand limit} of \(f(x)\) as \(x\) approaches \(a\) from
    the right equals \(L\) if we can make \(f(x)\) arbitrarily close to \(L\)
    by taking \(x\) sufficiently close to \(a\) where \(x>a\).
    \[\lim_{x\to a^+}f(x)=L\]
\end{definition}
The limit \textbf{exists} if the left-hand limit and the right-hand limit of
\(f(x)\) as \(x\) approaches \(a\) equal \(L\), otherwise the limit
\textbf{does not exist}.
\[\lim_{x\to a}f(x)=L \iff \lim_{x\to a^-}f(x) = \lim_{x\to a^+}f(x)=L\]