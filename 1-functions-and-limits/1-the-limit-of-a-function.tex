\subsection{The Limit of a Function}

\subsubsection*{Functions}
\begin{definition}
    A \textbf{function} \(f\) is a rule that assigns to each element \(x\) in
    a set \(D\) exactly one element \(f(x)\) in a set \(E\).
\end{definition}
We usually consider functions for which the sets \(D\) and \(E\) are sets of
real numbers.
The set \(D\) is called the \textbf{domain} of the function.
The \textbf{range} of \(f\) is the set of all possible values of \(f(x)\) as
\(x\) varies throughout the domain.
A symbol that represents an arbitrary number in the domain of a function \(f\)
is called an \textbf{independent variable}.
A symbol that represents a number in the range of \(f\) is called a
\textbf{dependent variable}.
If \(f\) is a function with domain \(D\), then its \textbf{graph} is the set
of ordered pairs
\[\{(x,f(x))\mid x\in D\}\]

\subsubsection*{Symmetry}
If a function \(f\) satisfies \(f(-x)=f(x)\) for every number \(x\) in its
domain, then \(f\) is called an \textbf{even function}.
If \(f\) satisfies \(f(-x)=-f(x)\) for every number \(x\) in its
domain, then \(f\) is called an \textbf{odd function}.

\subsubsection*{Increasing and Decreasing Functions}
A function \(f\) is called \textbf{increasing} on an interval \(I\) if
\[f(x_1)<f(x_2)\]
whenever \(x_1<x_2\) in \(I\).
It is called \textbf{decreasing} on \(I\) if
\[f(x_1)>f(x_2)\]
whenever \(x_1<x_2\) in \(I\).

\subsubsection*{Inverse Functions}
\begin{definition}
    A function \(f\) is \textbf{injective} (or \textbf{one-to-one}) if
    \(f(x_1)\neq f(x_2)\) when \(x_1\neq x_2\).
\end{definition}
\begin{definition}
    A function \(f\) is \textbf{surjective} (or \textbf{onto}) if for all
    \(y\) in range \(Y\),
    there exists an \(x\) in domain \(X\) such that \(f(x)=y\).
\end{definition}
\begin{definition}
    A function \(f\) is \textbf{bijective} if \(f\) is injective and
    surjective (or one-to-one and onto).
\end{definition}
\begin{definition}
    Let \(f\) be a one-to-one function with domain \(A\) and range \(B\).
    Then its \textbf{inverse function} \(f^{-1}\) has domain \(B\) and range
    \(A\) and is defined by
    \[f^{-1}(y)=x\iff f(x)=y\]
    for all \(y\) in \(B\).
\end{definition}
We usually reverse the roles of \(x\) and \(y\) and write
\[f^{-1}(x)=y\iff f(y)=x\]

\subsubsection*{Intuitive Definition of a Limit}
Suppose \(f(x)\) is defined near the number \(a\).
(This means that \(f(x)\) is defined on some open interval that contains the
number \(a\),
except possibly at \(a\) itself.)
\begin{definition}
    We write
    \[\lim_{x\to a}f(x)=L\]
    and say that the \textbf{limit} of \(f(x)\),
    as \(x\) approaches \(a\),
    equals \(L\),
    if we can make the values of \(f(x)\) arbitrarily close to \(L\) by taking
    \(x\) to be sufficiently close to \(a\) but \(x\neq a\).
\end{definition}
An alternative notation for the limit is \(f(x)\to L\) as \(x\to a\).

\subsubsection*{One-Sided Limits}
\begin{definition}
    We write
    \[\lim_{x\to a^-}f(x)=L\]
    and say that the \textbf{left-hand limit} of \(f(x)\) as \(x\) approaches
    \(a\) is equal to \(L\) if we can make the values of \(f(x)\) arbitrarily
    close to \(L\) by taking \(x\) sufficiently close to \(a\) and \(x<a\).
\end{definition}
\begin{definition}
    We write
    \[\lim_{x\to a^+}f(x)=L\]
    and say that the \textbf{right-hand limit} of \(f(x)\) as \(x\) approaches
    \(a\) is equal to \(L\) if we can make the values of \(f(x)\) arbitrarily
    close to \(L\) by taking \(x\) sufficiently close to \(a\) and \(x>a\).
\end{definition}
\begin{theorem}
    \(\displaystyle{\lim_{x\to a}f(x)=L}\) if and only if
    \(\displaystyle{\lim_{x\to a^-}f(x)}=L\) and
    \(\displaystyle{\lim_{x\to a^+}f(x)}=L\).
\end{theorem}
The limit exists if and only if the left-hand limit and the right-hand limit of \(f(x)\)
as \(x\) approaches \(a\) are equal to \(L\),
otherwise the limit does not exist.