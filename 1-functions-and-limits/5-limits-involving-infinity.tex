\subsection{Limits Involving Infinity}

\subsubsection*{Infinite Limits}
\begin{definition}
    The notation
    \[\lim_{x\to a}f(x)=\infty\]
    means that the values of \(f(x)\) can be made arbitrarily large by taking
    \(x\) sufficiently close to \(a\) but \(x\neq a\).
\end{definition}
Another notation for the limit is \(f(x)\to\infty\) as \(x\to a\).
We say that the limit of \(f(x)\),
as \(x\) approaches \(a\),
is infinity.
\begin{definition}
    \[\lim_{x\to a}f(x)=-\infty\]
    means that the values of \(f(x)\) can be made arbitrarily large negative
    by taking \(x\) sufficiently close to \(a\) but \(x\neq a\).
\end{definition}
We say that the limit of \(f(x)\),
as \(x\) approaches \(a\),
is negative infinity.
Similar definitions can be given for the one-sided infinite limits
\begin{align*}
    \lim_{x\to a^-}f(x) &= \infty & \lim_{x\to a^+}f(x) &= \infty \\
    \lim_{x\to a^-}f(x) &= -\infty & \lim_{x\to a^+}f(x) &= -\infty
\end{align*}
\begin{definition}
    The vertical line \(x=a\) is called a \textbf{vertical asymptote} of the
    curve \(y=f(x)\) if at least one of the following statements is true:
    \begin{align*}
        \lim_{x\to a}f(x) &= \infty & \lim_{x\to a^-}f(x) &= \infty
        & \lim_{x\to a^+}f(x) &= \infty \\
        \lim_{x\to a}f(x) &= -\infty & \lim_{x\to a^-}f(x) &= -\infty
        & \lim_{x\to a^+}f(x) &= -\infty
    \end{align*}
\end{definition}

\subsubsection*{Limits at Infinity}
\begin{definition}
    Let \(f\) be a function defined on some interval \((a,\infty)\).
    Then
    \[\lim_{x\to\infty}f(x)=L\]
    means that the values of \(f(x)\) can be made arbitrarily close to \(L\)
    by requiring \(x\) to be sufficiently large.
\end{definition}
Another notation is \(f(x)\to L\) as \(x\to\infty\).
We say that the limit of \(f(x)\),
as \(x\) approaches infinity,
is \(L\).
\begin{definition}
    Let \(f\) be a function defined on some interval \((-\infty,a)\).
    Then
    \[\lim_{x\to-\infty}f(x)=L\]
    means that the values of \(f(x)\) can be made arbitrarily close to \(L\) by requiring
    \(x\) to be sufficiently large negative.
\end{definition}
We say that the limit of \(f(x)\),
as \(x\) approaches negative infinity,
is \(L\).
\begin{definition}
    The line \(y=L\) is called a \textbf{horizontal asymptote} of the curve
    \(y=f(x)\) if either
    \[\lim_{x\to\infty}f(x)=L\]
    or
    \[\lim_{x\to-\infty}f(x)=L\]
\end{definition}
If \(n\) is a positive integer,
then
\begin{align*}
    \lim_{x\to\infty}\frac{1}{x^n} &= 0 & \lim_{x\to-\infty}\frac{1}{x^n} &= 0
\end{align*}
\begin{problem}
    Evaluate \(\displaystyle{\lim_{x\to\infty}\sin x}\).
\end{problem}
\begin{solution}
    As \(x\) increases,
    the values of \(sin x\) oscillate between 1 and \(-1\) infinitely often.
    Thus \(\displaystyle{\lim_{x\to\infty}\sin x}\) does not exist.
\end{solution}

\subsubsection*{Infinite Limits at Infinity}
The notation
\[\lim_{x\to\infty}f(x)=\infty\]
is used to indicate that the values of \(f(x)\) become large as \(x\) becomes
large.
Similar meanings are attached to the following symbols:
\begin{align*}
    \lim_{x\to\infty}f(x) &= \infty & \lim_{x\to\infty}f(x) &= -\infty
    & \lim_{x\to-\infty}f(x) &= -\infty
\end{align*}

\subsubsection*{Precise Definitions}
Let \(f\) be a function defined on some open interval that contains the number
\(a\),
except possibly at \(a\) itself.
\begin{definition}
    \[\lim_{x\to a}f(x)=\infty\]
    means that for every positive number \(M\) there is a positive number
    \(\delta\) such that
    \[0<|x-a|<\delta\implies f(x)>M\]
\end{definition}
\begin{problem}
    Prove that \(\displaystyle{\lim_{x\to 0}\frac{1}{x^2}=\infty}\).
\end{problem}
\begin{solution}
    Let \(M\) be a given positive number
    We want to find a number \(\delta\) such that
    \[0<|x|<\delta\implies\frac{1}{x^2}>M\]
    But
    \[\frac{1}{x^2}>M\iff x^2<\frac{1}{M}\iff\sqrt{x^2}<\sqrt{\frac{1}{M}}
    \iff|x|<\frac{1}{\sqrt{M}}\]
    Let \(\delta=1/\sqrt{M}\),
    then
    \[0<|x|<\delta=\frac{1}{\sqrt{M}}\implies\frac{1}{x^2}>M\]
    This shows that
    \(\displaystyle{\lim_{x\to 0}\frac{1}{x^2}=\infty}\).
\end{solution}
\begin{definition}
    \[\lim_{x\to a}f(x)=-\infty\]
    if for every negative number \(N\) there is a positive number \(\delta\)
    such that
    \[0<|x-a|<\delta\implies f(x)<N\]
\end{definition}

\begin{definition}
    Let \(f\) be a function defined on some interval \((a,\infty)\).
    Then
    \[\lim_{x\to\infty}f(x)=L\]
    means that for every \(\epsilon>0\) there is a corresponding number \(N\)
    such that
    \[x>N\implies|f(x)-L|<\epsilon\]
\end{definition}
\begin{definition}
    Let \(f\) be a function defined on some interval \((-\infty,a)\).
    Then
    \[\lim_{x\to-\infty}f(x)=L\]
    means that for every \(\epsilon>0\) there is a corresponding number \(N\)
    such that
    \[x<N\implies|f(x)-L|<\epsilon\]
\end{definition}
\begin{problem}
    Prove that \(\displaystyle{\lim_{x\to \infty}\frac{1}{x}=0}\).
\end{problem}
\begin{solution}
    Given \(\epsilon>0\),
    we want to find an \(N\) such that
    \[x>N\implies\left|\frac{1}{x}-0\right|<\epsilon\]
    Since \(x\to \infty\),
    we can that \(x>0\) in computing the limit.
    Then \(1/x<\epsilon\iff x>1/\epsilon\).
    Let \(N=1/\epsilon\),
    so
    \[x>N=\frac{1}{\epsilon}\implies\left|\frac{1}{x}-0\right|=\frac{1}{x}
    <\epsilon\]
    Therefore,
    by definition,
    \[\lim_{x\to \infty}\frac{1}{x}=0\]
\end{solution}
\begin{definition}
    Let \(f\) be a function defined on some interval \((a,\infty)\).
    Then
    \[\lim_{x\to\infty}f(x)=\infty\]
    means that for every positive number \(M\) there is a corresponding
    positive number \(N\) such that
    \[x>N\implies f(x)>M\]
\end{definition}
Similar definitions apply when the symbol \(\infty\) is replaced by \(-\infty\).