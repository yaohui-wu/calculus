\subsection{Indeterminate Forms and L'Hôpital's Rule}
\subsubsection{Indeterminate Forms}
Consider the limit of the form \[\lim_{x\to a}\frac{f(x)}{g(x)}\]
If \(f(x)\to 0\) and \(g(x)\to 0\), then the limit is an
\textbf{indeterminate form} of type \(0/0\).
If \(f(x)\to\infty\) or \(-\infty\) and \(g(x)\to\infty\) or \(-\infty\),
then the limit is an indeterminate form of type \(\infty/\infty\).

\subsubsection{L'Hôpital's Rule}
We have the \textbf{L'Hôpital's Rule} discoverd by
\textbf{Johann Bernoulli} (1667--1748) and named after
\textbf{Guillaume de l'Hôpital} (1661--1704) to evaluate limits of indeterminate
forms of type \(0/0\) and \(\infty/\infty\).
\begin{theorem}[L'Hôpital's Rule]
    Suppose that \(f\) and \(g\) are differentiable and \(g'(x)\neq 0\) near \(a\),
    except possibly at \(a\).
    If
    \begin{align*}
        &\lim_{x\to a}f(x)=0&\lim_{x\to a}g(x)=0
    \end{align*}
    or
    \begin{align*}
        &\lim_{x\to a}f(x)=\pm\infty&\lim_{x\to a}g(x)=\pm\infty
    \end{align*}
    Then
    \[\lim_{x\to a}\frac{f(x)}{g(x)}=\lim_{x\to a}\frac{f'(x)}{g'(x)}\]
    if the limit on the right side exists, is \(\infty\), or is \(-\infty\).
\end{theorem}
L'Hôpital's rule is also valid for one-sided limits and for limits at infinity
or negative infinity.