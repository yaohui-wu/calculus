\subsection{Indeterminate Forms and l'Hospital Rule}

\subsubsection*{Indeterminate Forms and l'Hospital Rule}
Consider the limit of the form \[\lim_{x\to a}\frac{f(x)}{g(x)}\]
If \(f(x)\to 0\) and \(g(x)\to 0\) as \(x\to a\), then the limit is an
\textbf{indeterminate form} of type \(0/0\).
If \(f(x)\to\infty\) or \(-\infty\) and \(g(x)\to\infty\) or \(-\infty\),
then the limit is an indeterminate form of type \(\infty/\infty\).
We have the \textbf{l'Hospital Rule} discoverd by
\textbf{Johann Bernoulli} (1667--1748) and named after
\textbf{Guillaume de l'Hôpital} (1661--1704) to evaluate limits of indeterminate
forms of type \(0/0\) and \(\infty/\infty\).
\begin{theorem}[l'Hospital Rule]
    Suppose that \(f\) and \(g\) are differentiable and \(g'(x)\neq 0\) near \(a\),
    except possibly at \(a\).
    If
    \begin{align*}
        &\lim_{x\to a}f(x)=0&\lim_{x\to a}g(x)=0
    \end{align*}
    or
    \begin{align*}
        &\lim_{x\to a}f(x)=\pm\infty&\lim_{x\to a}g(x)=\pm\infty
    \end{align*}
    Then
    \[\lim_{x\to a}\frac{f(x)}{g(x)}=\lim_{x\to a}\frac{f'(x)}{g'(x)}\]
    if the limit on the right side exists, is \(\infty\), or is \(-\infty\).
\end{theorem}
l'Hospital rule is also valid for one-sided limits and for limits at infinity
or negative infinity.
\begin{problem}
    Find \(\displaystyle{\lim_{x\to 1}\frac{\ln x}{x-1}}\).
\end{problem}
\begin{solution}
    Notice that the limit is an indeterminate form of \(0/0\).
    We apply l'Hospital rule then we have
    \[\lim_{x\to 1}\dfrac{\ln x}{x-1}=\lim_{x\to 1}\frac{1/x}{1}
    =\lim_{x\to 1}\frac{1}{x}=1\]
\end{solution}
\begin{problem}
    Find \(\displaystyle{\lim_{x\to 0}\frac{\tan x-x}{x^3}}\).
\end{problem}
\begin{proof}
    Notice that the limit is an indeterminate form of \(0/0\).
    We apply l'Hospital rule then we have
    \[\lim_{x\to 0}\frac{\tan x-x}{x^3}=\lim_{x\to 0}\frac{\sec^2 x-1}{3x^2}\]
    The limit still is an indeterminate form of \(0/0\) so we apply
    l'Hospital rule again then we have
    \[\lim_{x\to 0}\frac{\tan x-x}{x^3}
    =\lim_{x\to 0}\frac{2\sec^2 x\tan x}{6x}\]
    Similarly, we apply l'Hospital rule again then we have
    \[\lim_{x\to 0}\frac{\tan x-x}{x^3}
    =\lim_{x\to 0}\frac{2\sec^4 x+4\sec^2 x\tan^2 x}{6}
    =\frac{2(1)+4(1)(0)}{6}=\frac{1}{3}\]
\end{proof}
Note that l'Hospital rule only applies if the limit is one of the
indeterminate forms of type \(0/0\) or \(\infty/\infty\).
If \(\lim_{x\to a}f(x)=0\) and \(\lim_{x\to a}g(x)=\infty\) (or \(-\infty\)),
then the limit \(\lim_{x\to a}f(x)g(x)\) is an indeterminate form of type
\(0\cdot\infty\).
We can express the product as a quotient then use l'Hospital rule to
calculate the limit.
\begin{problem}
    Evaluate \(\lim_{x\to 0^+}x\ln x\).
\end{problem}
\begin{solution}
    \[\lim_{x\to 0^+}x\ln x=\lim_{x\to 0^+}\frac{\ln x}{1/x}
    =\lim_{x\to 0^+}\frac{1/x}{-1/x^2}=\lim_{x\to 0^+}-x=0\]
\end{solution}
If \(\lim_{x\to a}f(x)=\infty\) and \(\lim_{x\to a}g(x)=\infty\), then the
limit \(\lim_{x\to a}[f(x)-g(x)]\) is an indeterminate form of type
\(\infty-\infty\).
We can convert the difference into a quotient to apply l'Hospital rule to
calculate the limit.
\begin{problem}
    Compute \(\lim_{x\to (\pi/2)^-}(\sec x-\tan x)\).
\end{problem}
\begin{solution}
    \[\lim_{x\to (\pi/2)^-}(\sec x-\tan x)
    =\lim_{x\to (\pi/2)^-}\frac{1-\sin x}{\cos x}
    =\lim_{x\to (\pi/2)^-}\frac{\cos x}{\sin x}=0\]
\end{solution}
Consider the limit of the form \[\lim_{x\to a}[f(x)^{g(x)}]\]
We have the following indeterminate forms arise from the limit:
\begin{itemize}
    \item Type \(0^0\) if \(\lim_{x\to a}f(x)=0\) and \(\lim_{x\to a}g(x)=0\).
    \item Type \(\infty^0\) if \(\lim_{x\to a}f(x)=\infty\) and
    \(\lim_{x\to a}g(x)=0\).
    \item Type \(1^\infty\) if \(\lim_{x\to a}f(x)=1\) and
    \(\lim_{x\to a}g(x)=\pm\infty\).
\end{itemize}
Let \(y=f(x)^{g(x)}\), then we can write \(\ln y=\ln f(x)^{g(x)}\) and
\(y=e^{g(x)\ln f(x)}\) to find the limit.
\begin{problem}
    Calculate \(\lim_{x\to 0^+}(1+\sin 4x)^{\cot x}\).
\end{problem}
\begin{solution}
    Let \(y=(1+\sin 4x)^{\cot x}\), then \(\ln y=\ln[(1+\sin 4x)^{\cot x}]=\cot x\ln(1+\sin 4x)\).
    We apply l'Hospital rule then we have
    \begin{align*}
    \lim_{x\to 0^+}\ln y &= \lim_{x\to 0^+}\frac{\ln(1+\sin 4x)}{\tan x}
    =\lim_{x\to 0^+}\frac{[1/(1+\sin 4x)](\cos 4x)(4)}{\sec^2 x} \\
    &= \lim_{x\to 0^+}\frac{4\cos 4x\cos^2 x}{1+\sin 4x}=\frac{4(1)(1)}{1+0}=4
    \end{align*}
    Therefore, we have
    \[\lim_{x\to 0^+}(1+\sin 4x)^{\cot x}=\lim_{x\to 0^+}y
    =\lim_{x\to 0^+}e^{\ln y}=e^4\]
\end{solution}
\begin{problem}
    Calculate \(\lim_{x\to 0^+}x^x\).
\end{problem}
\begin{solution}
    We have shown that \(\lim_{x\to 0^+}x\ln x=0\) and therefore we have
    \[\lim_{x\to 0^+}x^x=\lim_{x\to 0^+}e^{x\ln x}=e^0=1\]
\end{solution}