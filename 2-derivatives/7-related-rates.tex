\subsection{Related Rates}

In a related rates problem the idea is to compute the rate of change of one
quantity in terms of the rate of change of another quantity.
The procedure is to find an equation that relates the two quantities and then
use the Chain Rule to differentiate both sides with respect to time.
\begin{problem}
    Air is being pumped into a spherical balloon so that its volume increases
    at a rate of 100 \(\text{cm}^3/\text{s}\).
    How fast is the radius of the balloon increasing when the diameter is 50
    cm?
\end{problem}
\begin{solution}
    Let \(V\) be the volume of the balloon and let \(r\) be its radius.
    The rate of increase of the volume with respect to time is
    \(\dfrac{dV}{dt}\), and the rate of increase of the radius is
    \(\dfrac{dr}{dt}\).
    Then, we are given that
    \[\frac{dV}{dt}=100\ \text{cm}^3/\text{s}\]
    and we want to find \(\dfrac{dr}{dt}\) when \(r=25\) cm.
    We relate \(V\) and \(r\) by the formula of the volume of a sphere:
    \[V=\frac{4}{3}\pi r^3\]
    Then by the Chain Rule,
    \[\frac{dV}{dt}=\frac{dV}{dr}\cdot\frac{dr}{dt}=4\pi r^2\frac{dr}{dt}\]
    If \(r=25\) and \(\dfrac{dV}{dt}=100\), then
    \[\frac{dr}{dt}=\frac{1}{4\pi r^2}\frac{dV}{dt}
    =\frac{1}{4\pi(25)^2}\cdot100=\frac{1}{25\pi}\]
    The radius of the balloon is increasing at the rate of
    \(\dfrac{1}{25\pi}\) cm/s.
\end{solution}