\subsection{Linear Approximations and Differentials}

We use the tangent line at \((a,f(a))\) as an approximation to the curve
\(y=f(x)\) when \(x\) is near \(a\).
An equation of this tangent line is
\[y=f(a)+f'(a)(x-a)\]
and the approximation
\[f(x)\approx f(a)+f'(a)(x-a)\]
is called the \textbf{linear approximation} of \(f\) at \(a\).
The linear function whose graph is this tangent line, that is,
\[L(x)=f(a)+f'(a)(x-a)\]
is called the \textbf{linearization} of \(f\) at \(a\).

\subsubsection*{Applications to Physics}
The linear approximations
\begin{align*}
    \sin\theta &\approx \theta & \cos\theta &\approx 1
\end{align*}
are used in physics when \(\theta\) is close to 0.

\subsubsection*{Differentials}
If \(y=f(x)\), where \(f\) is a differentiable function, then the
\textbf{differential} \(dx\) is an independent variable.
The differential \(dy\) is then defined by
\[dy=f'(x)\,dx\]
so \(dy\) is an independent variable.