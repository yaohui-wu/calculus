\subsection{Derivatives of Inverse Functions}

\subsubsection*{The Calculus of Inverse Functions}
\begin{theorem}
    If \(f\) is a one-to-one continuous function defined on an interval,
    then its inverse function \(f^{-1}\) is also continuous.
\end{theorem}
\begin{theorem}
    If \(f\) is a one-to-one differentiable function with inverse function
    \(f^{-1}\) and \(f'(f^{-1}(a))\neq 0\), then the inverse function is
    differentiable at \(a\) and
    \[(f^{-1})'(a)=\frac{1}{f'(f^{-1}(a))}\]
\end{theorem}
In general, for any number \(x\) we have
\[(f^{-1})'(x)=\frac{1}{f'(f^{-1}(x))}\]
If we write \(y=f^{-1}(x)\), then \(f(y)=x\), in Leibniz notation we have
\[\frac{dy}{dx}=\frac{1}{\dfrac{dx}{dy}}\]

\subsubsection*{Derivatives of Logarithmic Functions}
The \textbf{Euler's number} \(e\) is the base of the natural exponential
function \(y=e^x\).
It is also the base of the the natural logarithmic function \(y=\ln x\).
\begin{definition}[Euler's Number]
    The Euler's number \(e\) is defined as
    \[e=\lim_{n\to\infty}\left(1+\frac{1}{n}\right)^n
    =\lim_{x\to 0}(1+x)^{1/x}\]
\end{definition}
Note that the approximate value of \(e\) is \(e\approx2.71828\).
\begin{theorem}
    The exponential function \(f(x)=\log_a x\) is differentiable and
    \[f'(x)=\frac{1}{x}\log_a e\]
\end{theorem}
\begin{proof}
    \begin{align*}
        f'(x) &= \lim_{h\to 0}\frac{f(x+h)-f(x)}{h}
        = \lim_{h\to 0}\frac{\log_a(x+h)-\log_a x}{h}
        =\lim_{h\to 0}\frac{\log_a\left(\dfrac{x+h}{x}\right)}{h} \\
        &= \lim_{h\to 0}\frac{1}{x}\cdot\frac{x}{h}
        \log_a\left(1+\frac{h}{x}\right)
        = \frac{1}{x}
            \lim_{h\to 0}\log_a\left(1+\frac{h}{x}\right)^{x/h} \\
        &=\frac{1}{x}
            \lim_{h\to 0}\log_a\left(1+\frac{h}{x}\right)^{1/(h/x)}
        =\frac{1}{x}\log_a e
    \end{align*}
\end{proof}
We know from the change of base formula that
\[\log_a e=\frac{\ln e}{\ln a}=\frac{1}{\ln a}\]
and therefore
\[\frac{d}{dx}(\log_a x)=\frac{1}{x}\log_a e=\frac{1}{x\ln a}\]
The derivative of the natural logarithmic function \(f(x)=\ln x\) is:
\[\frac{d}{dx}(\ln x)=\frac{1}{x}\]
\begin{proof}
    \[\frac{d}{dx}(\ln x)=\frac{1}{x\ln e}=\frac{1}{x}\]
\end{proof}
In general, if \(u=f(x)\), then
\[\frac{d}{dx}(\ln u)=\frac{1}{u}\cdot\frac{du}{dx}\]
\begin{problem}
    Find \(f'(x)\) if \(f(x)=\ln|x|\).
\end{problem}
\begin{solution}
    Since \(f(x)=\ln x\) for \(x>0\) and \(f(x)=\ln(-x)\) for \(x<0\),
    it follows that
    \begin{align*}
        f'(x) &= \frac{d}{dx}(\ln x)=\frac{1}{x},\quad x>0 \\
        f'(x) &= \frac{d}{dx}\big[\ln(-x)\big]=\frac{1}{-x}(-1)
        =\frac{1}{x},\quad x<0
    \end{align*}
    Therefore \(\dfrac{d}{dx}(\ln|x|)=\dfrac{1}{x}\) for all \(x\neq 0\).
\end{solution}

\subsubsection*{Logarithmic Differentiation}
The steps in \textbf{logarithmic differentiation} are
\begin{enumerate}
    \item Take natural logarithms of both sides of an equation \(y=f(x)\) and
    use the Laws of Logarithms to simplify.
    \item Differentiate implicitly with respect to \(x\).
    \item Solve the resulting equation for \(y'\).
\end{enumerate}
The proof of the Power Rule (general version):
\begin{proof}
    Let \(y=x^n\) and we use logarithmic differentiation:
    \[\ln|y|=\ln|x|^n=n\ln|x|,\quad x\neq 0\]
    Therefore
    \[\frac{1}{y}\cdot\frac{dy}{dx}=\frac{n}{x}\]
    Hence
    \[\frac{dy}{dx}=n\frac{y}{x}=n\frac{x^n}{x}=nx^{n-1}\]
\end{proof}
If \(x=0\), we can show that \(f'(0)=0\) for \(n>1\) directly from the
definition of the derivative.

\subsubsection*{Derivatives of Exponential Functions}
\begin{theorem}
    The exponential function \(f(x)=a^x,a>0,\) is differentiable and
    \[\frac{d}{dx}(a^x)=a^x\ln a\]
\end{theorem}
\begin{proof}
    We know that the logarithmic function \(y=\log_a x\) is differentiable
    (and its derivative is nonzero) so its inverse function \(y=a^x\) is
    differentiable.
    If \(y=a^x\), then \(\log_a y=x\).
    By implicit differentiation we have
    \begin{align*}
        \log_a y &= x \\
        \frac{1}{y\ln a}\cdot\frac{dy}{dx} &= 1 \\
        \frac{dy}{dx} &= y\ln a=a^x\ln a 
    \end{align*}
\end{proof}
The derivative of the natural exponential function \(f(x)=e^x\) is:
\[\frac{d}{dx}(e^x)=e^x\]
\begin{proof}
    \[\frac{d}{dx}(e^x)=e^x\ln e=e^x\]
\end{proof}
In general, if \(u=f(x)\), then
\[\frac{d}{dx}(e^u)=e^u\cdot\frac{du}{dx}\]

\subsubsection*{Inverse Trigonometric Functions}
\[\frac{d}{dx}(\arcsin x)=\frac{1}{\sqrt{1-x^2}},\quad -1<x<1\]
\begin{proof}
    Let \(y=\arcsin x \).
    Then \(\sin y=x\) and \(-\pi/2\leq y\leq\pi/2\).
    By implicit differentiation we have
    \begin{align*}
        \frac{d}{dx}(\sin y) &= \frac{d}{dx}(x) \\
        \cos y\cdot\frac{dy}{dx} &= 1 \\
        \frac{dy}{dx} &= \frac{1}{\cos y}
    \end{align*}
    Now \(\cos y\geq0\) since \(-\pi/2\leq y\leq\pi/2\), so
    \[\cos y=\sqrt{1-\sin^2 y}\]
    Therefore
\[\frac{d}{dx}(\arcsin x)=\frac{1}{\sqrt{1-x^2}},\quad -1<x<1\]
\end{proof}
\[\frac{d}{dx}(\arccos x)=-\frac{1}{\sqrt{1-x^2}},\quad -1<x<1\]
\begin{proof}
    Let \(y=\arccos x\).
    Then \(\cos y=x\) and \(0\leq y\leq\pi\).
    By implicit differentiation we have
    \begin{align*}
        \frac{d}{dx}(\cos y) &= \frac{d}{dx}(x) \\
        -\sin y\cdot\frac{dy}{dx} &= 1 \\
        \frac{dy}{dx} &= -\frac{1}{\sin y}
    \end{align*}
    Now \(\sin y\geq0\) since \(0\leq y\leq\pi\), so
    \[\sin y=\sqrt{1-\cos^2 y}\]
    Therefore
    \[\frac{d}{dx}(\arccos x)=-\frac{1}{\sqrt{1-x^2}},\quad -1<x<1\]
\end{proof}
\[\frac{d}{dx}(\arctan x)=\frac{1}{1+x^2}\]
\begin{proof}
    Let \(y=\arctan x\).
    Then \(\tan y=x\) and \(-\pi/2<y<\pi/2\).
    We have
    \begin{align*}
        \frac{d}{dx}(\tan y) &= \frac{d}{dx}(x) \\
        \sec^2 y\cdot\frac{dy}{dx} &= 1 \\
        \frac{dy}{dx} &= \frac{1}{\sec^2 y}=\frac{1}{\tan^2 y+1}
        =\frac{1}{1+x^2}
    \end{align*}
    Therefore
    \[\frac{d}{dx}(\arctan x)=\frac{1}{1+x^2} \]
\end{proof}
\[\frac{d}{dx}(\arccsc x)=-\frac{1}{x\sqrt{x^2-1}}\]
\[\frac{d}{dx}(\arcsec x)=\frac{1}{x\sqrt{x^2-1}}\]
\[\frac{d}{dx}(\arccot x)=\frac{1}{x\sqrt{1+x^2}}\]