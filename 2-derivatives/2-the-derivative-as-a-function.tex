\subsection{The Derivative as a Function}

\begin{definition}
    The \textbf{derivative of a function} \(f\) is the function
    \[f'(x)=\lim_{h\to 0}\frac{f(x+h)-f(x)}{h}\]
\end{definition}
\begin{problem}
    Find the derivative of \(f(x)=\sqrt{x}\).
\end{problem}
\begin{solution}
    \begin{align*}
        f'(x) &= \lim_{h\to 0}\frac{\sqrt{x+h}-\sqrt{x}}{h}=\lim_{h\to 0}
        \left(\frac{\sqrt{x+h}-\sqrt{x}}{h}\cdot
        \frac{\sqrt{x+h}+\sqrt{x}}{\sqrt{x+h}+\sqrt{x}}\right) \\
        &= \lim_{h\to 0}\frac{x+h-x}{h(\sqrt{x+h}+\sqrt{x})}
        =\lim_{h\to 0}\frac{1}{\sqrt{x+h}+\sqrt{x}}
        =\frac{1}{\sqrt{x}+\sqrt{x}}=\frac{1}{2\sqrt{x}} 
    \end{align*}
\end{solution}

\subsubsection*{Notations}
The following notations for the derivative of the function \(y=f(x)\) with
respect to \(x\) are equivalent:
\[y'=f'(x)=\frac{dy}{dx}=\frac{d}{dx}f(x)=D_x f(x)\]
The symbols \(\dfrac{d}{dx}\) and \(D_x\) are called \textbf{differential operators}
because they indicate the operation of \textbf{differentiation},
which is the process of calculating a derivative.
The symbol \(\dfrac{dy}{dx}\) is callled the Leibniz notation.
We can rewrite the definition of the derivative in Leibniz notation in the
form
\[\frac{dy}{dx}=\lim_{\Delta x\to0}\frac{\Delta y}{\Delta x}\]
The following notations for the value of the derivative of \(y=f(x)\)
evaluated at the number \(a\) are equivalent:
\[y'(a)=f'(a)=\left.\frac{dy}{dx}\right|_{x=a}
=\left[\frac{dy}{dx}\right]_{x=a}\]


\subsubsection*{Differentiable Functions}
\begin{definition}
    A function \(f\) is \textbf{differentiable at} \(a\) if \(f'(a)\) exists.
    It is \textbf{differentiable on an open interval} \((a,b)\)
    (or \((a,\infty)\) or \((-\infty,a)\) or \((-\infty,\infty)\))
    if it is differentiable at every number in the interval.
\end{definition}
\begin{theorem}
    If \(f\) is differentiable at \(a\), then \(f\) is continuous at \(a\).
\end{theorem}
\begin{proof}
    To prove that \(f\) is continuous at \(a\),
    we want to show that
    \[\lim_{x\to a}f(x)=f(a)\]
    Given that \(f\) is differentiable at \(a\) so
    \[f'(a)=\lim_{x\to a}\frac{f(x)-f(a)}{x-a}\]
    exists.
    Then
    \[\lim_{x\to a}\big[f(x)-f(a)\big]
    =\lim_{x\to a}\frac{f(x)-f(a)}{x-a}(x-a)
    =\lim_{x\to a}\frac{f(x)-f(a)}{x-a}\cdot\lim_{x\to a}(x-a)
    =f'(a)\cdot 0=0\]
    and
    \[\lim_{x\to a}f(x)=\lim_{x\to a}\big[f(a)+(f(x)-f(a))\big]
    =\lim_{x\to a}f(a)+\lim_{x\to a}(f(x)-f(a))=f(a)+0=f(a)\]
    Therefore \(f\) is continuous at \(a\).
\end{proof}
Note that there are functions that are continuous but not differentiable.
The function \(y=|x|\) is continuous at 0 but it is not differentiable at 0.
Since
\[f'(0)=\lim_{h\to 0}\frac{|0+h|-|0|}{h}\] if the limit exists.
But
\begin{align*}
    \lim_{h\to 0^-}\frac{|0+h|-|0|}{h} &= \lim_{h\to 0^-}\frac{|h|}{h}
    =\lim_{h\to 0^-}\frac{-h}{h}=-1 \\
    \lim_{h\to 0^+}\frac{|0+h|-|0|}{h} &= \lim_{h\to 0^+}\frac{|h|}{h}
    =\lim_{h\to 0^+}\frac{h}{h}=1
\end{align*}
then the limit does not exist so \(f'(0)\) does not exist.
Thus \(y=|x|\) is differentiable at all \(x\) except 0.

\subsubsection*{Higher Order Derivatives}
 If \(y=f(x)\) is a differentiable function and its derivative \(y'=f'(x)\) is
 also a differentiable function,
 then the \textbf{second derivative} of \(y=f(x)\) is
\[y''=f''(x)=\frac{d}{dx}\left(\frac{dy}{dx}\right)=\frac{d^2y}{dx^2}\]
\(f''(x)\) is the slope of the curve \(y=f'(x)\) at the point \((x,f'(x))\),
which is the rate of change of the slope of the original curve \(y=f(x)\).
In general, the second derivative is the rate of change of the rate of change.
If \(s=s(t)\) is the position function of an object,
then its first derivative is the velocity \(v(t)\) of the object as a function
of time \(t\):
\[v(t)=s'(t)=\frac{ds}{dt}\]
The instantaneous rate of change of velocity with respect to time is called 
the \textbf{acceleration} \(a(t)\) of the object.
Thus the acceleration function is the derivative of the velocity
function and is therefore the second derivative of the position function:
\[a(t)=v'(t)=\frac{dv}{dt}=s''(t)=\frac{d^2s}{dt^2}\]
In general, the \(n\)th derivative of \(f\) is obtained from \(f\) by
differentiating \(n\) times.
If \(y=f(x)\), we write
\[y^{(n)}=f^{(n)}(x)=\frac{d^ny}{dx^n}\]
\begin{problem}
    Find the first and the second derivatives of \(f(x)=x^3\).
\end{problem}
\begin{solution}
    The first derivative is
    \begin{align*}
        f'(x) &= \lim_{h\to 0}\frac{(x+h)^3-x^3}{h}=\lim_{h\to 0}
        \frac{x^3+3x^2h+3xh^2+h^3-x^3}{h}
        =\lim_{h\to 0}\frac{3x^2h+3xh^2+h^3}{h} \\
        &= \lim_{h\to 0}(3x^2+3xh+h^2)=3x^2
    \end{align*}
    The second derivative is
    \begin{align*}
        f''(x) &= \lim_{h\to 0}\frac{3(x+h)^2-3x^2}{h}
        =\lim_{h\to 0}\frac{3(x^2+2hx+h^2)-3x^2}{h}
        =\lim_{h\to 0}\frac{3x^2+6hx+3h^2-3x^2}{h} \\
        &= \lim_{h\to 0}\frac{6hx+3h^2}{h}=\lim_{h\to 0}(6x+3h)=6x
    \end{align*}
\end{solution}