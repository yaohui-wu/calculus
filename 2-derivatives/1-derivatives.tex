\subsection{Derivatives and Rates of Change}

\subsubsection*{Tangents}
\begin{definition}
    The \textbf{tangent line} of the curve \(y=f(x)\) at the point
    \(P(a, f(a))\) is the line through \(P\) with slope
    \[m=\lim_{x\to a}\frac{f(x)-f(a)}{x-a}\]
    provided that this limit exists.
\end{definition}
If \(h=x-a\),
then
\[m=\lim_{h\to 0}\frac{f(a+h)-f(a)}{h}\]
Let \(s=f(t)\) be a \textbf{position function} that describes the motion of an
object where \(s\) is the displacement of the object from the origin at time
\(t\).
In the time interval from \(t=a\) to \(t=a+h\) the change in position is
\(f(a+h)-f(a)\).
The average velocity over this time interval is
\[\text{average velocity}
=\frac{\text{displacement}}{\text{time}}
=\frac{f(a+h)-f(a)}{h}\]
The \textbf{velocity} (or \textbf{instantaneous velocity}) of the object at
time \(t=a\) is the limit of the average velocities:
\[v(a)=\lim_{h\to 0}\frac{f(a+h)-f(a)}{h}\]

\subsubsection*{Derivatives}
\begin{definition}
    The \textbf{derivative of a function} \(f\) \textbf{at a number} \(a\) is
    \[f'(a)=\lim_{h\to 0}\frac{f(a+h)-f(a)}{h}\]
    if this limit exists.
\end{definition}
The tangent line to \(y=f(x)\) at the point \((a,f(a))\) is the line through
\((a,f(a))\) whose slope is equal to \(f'(a)\),
the derivative of \(f\) at \(a\).
The equation of the tangent line in point-slope form is
\[y-f(a)=f'(a)(x-a)\]

\subsubsection*{Rates of Change}
Suppose \(y\) is a quantity that depends on another quantity \(x\).
Thus \(y\) is a function of \(x\) and we write \(y=f(x)\).
If \(x\) changes from \(x_1\) to \(x_2\),
then the change in \(x\) (also called the \textbf{increment} of \(x\)) is
\[\Delta x=x_2-x_1\]
and the corresponding change in \(y\) is
\[\Delta y=f(x_2)-f(x_1)\]
The difference quotient
\[\frac{\Delta y}{\Delta x}=\frac{f(x_2)-f(x_1)}{x_2-x_1}\]
is called the
\textbf{average rate of change of} \(y\) \textbf{with respect to} \(x\) over
the interval \([x_1,x_2]\).
The limit of these average rates of change is called the
(\textbf{instantaneous})
\textbf{rate of change of} \(y\) \textbf{with respect to} \(x\)
at \(x=x_1\):
\[\text{instantaneous rate of change}
=\lim_{\Delta x\to 0}\frac{\Delta y}{\Delta x}
=\lim_{x_1\to x_2}\frac{f(x_2)-f(x_1)}{x_2-x_1}\]
We recognize this limit as being the derivative \(f'(x_1)\).
The derivative \(f'(a)\) is the instantaneous rate of change of \(y=f(x)\)
with respect to \(x\) when \(x=a\).
If \(s=f(t)\) is a position function of a particle,
then \(f'(a)\) is the rate of change of the displacement \(s\) with respect to
time \(t\).
\(f'(a)\) is the velocity of the particle at time \(t=a\).
The \textbf{speed} of the particle is \(|f'(a)|\),
the absolute value of the velocity.