\subsection{Derivatives}

\subsubsection{Derivatives and Rates of Change}
\begin{definition}
    The tangent line of the curve \(y=f(x)\) at the point \(P(a, f(a))\) is
    the line through \(P\) with slope
    \[m=\lim_{x\to a}\frac{f(x)-f(a)}{x-a}
    =\lim_{h\to 0}\frac{f(a+h)-f(a)}{h}\]
    where \(h=x-a\), if this limit exists.
\end{definition}
\begin{definition}
    The velocity at time \(t=a\) of a position function \(s=f(t)\) is
    \[v(a)=\lim_{h\to 0}\frac{f(a+h)-f(a)}{h}\qedhere\]
\end{definition}
\begin{definition}
    The \textbf{derivative} of a function \(f\) at a number \(a\) is
    \[f'(a)=\lim_{h\to 0}\frac{f(a+h)-f(a)}{h}\]
    if this limit exists.
\end{definition}
The slope of the tangent line to \(y=f(x)\) at the point \((a,f(a))\) is
\(f'(a)\), the derivative of \(f\) at \(a\).
The equation of the tangent line is
\[y-f(a)=f'(a)(x-a)\]
The instantaneous rate of change of \(y=f(x)\) with respect to \(x\) at
\(x=x_0\) in the interval \([x_0,x_1]\) is
\[\lim_{\Delta x\to 0}\frac{\Delta y}{\Delta x}
=\lim_{x_1 \to x_0}\frac{f(x_1)-f(x_0)}{x_1-x_0}\]
The derivative \(f'(a)\) is the instantaneous rate of change of \(y=f(x)\)
with respect to \(x\) when \(x=a\).
If \(s=f(t)\) is a position function of an object, then the velocity of the
object at time \(t=a\) is \(v(a)=f'(a)\) and the speed of the object is
\(|f'(a)|\), the magnitude of the velocity.

\subsubsection{The Derivative as a Function}
\begin{definition}
    The derivative of a function \(f(x)\) is the function
    \[f'(x)=\lim_{h\to 0}\frac{f(x+h)-f(x)}{h}\qedhere\]
\end{definition}
The following notations of the derivative of \(y=f(x)\) with respect to \(x\)
are equivalent.
\[y'=f'(x)=\frac{dy}{dx}=\frac{d}{dx}f(x)=D_x f(x)\]
where \(f'(x)\) is Newton's notation and \(dy/dx\) is Leibniz's notation.
The notation \(d/dx\) is the differential operator that indicates the
operation of differentiation.
The notations of the derivative of \(f(x)\) at \(a\) are
\[f'(a)=\left.\frac{dy}{dx}\right|_{x=a}=\left[\frac{dy}{dx}\right]_{x=a}\]
\begin{problem}
    Find the derivative of \(f(x)=\sqrt{x}\).
\end{problem}
\begin{solution}
    \begin{align*}
        f'(x) &= \lim_{h\to 0}\frac{\sqrt{x+h}-\sqrt{x}}{h}=\lim_{h\to 0}
        \left(\frac{\sqrt{x+h}-\sqrt{x}}{h}\cdot
        \frac{\sqrt{x+h}+\sqrt{x}}{\sqrt{x+h}+\sqrt{x}}\right) \\
        &= \lim_{h\to 0}\frac{x+h-x}{h(\sqrt{x+h}+\sqrt{x})}
        =\lim_{h\to 0}\frac{1}{\sqrt{x+h}+\sqrt{x}}
        =\frac{1}{\sqrt{x}+\sqrt{x}}=\frac{1}{2\sqrt{x}} \qedhere
    \end{align*}
\end{solution}

\subsubsection{Differentiable Functions}
\begin{definition}
    A function \(f\) is \textbf{differentiable} at \(a\) if \(f'(a)\) exists.
    \(f\) is differentiable on an open interval if it is differentiable at
    every number in the interval.
\end{definition}
\begin{theorem}
    If \(f\) is differentiable at \(a\), then \(f\) is continuous at \(a\).
\end{theorem}
\begin{proof}
    Given that \(f\) is differentiable at \(a\), we want to show that
    \[\lim_{x\to a}f(x)=f(a)\]
    Since \(f'(a)\) exists, we have
    \[f'(a)=\lim_{x\to a}\frac{f(x)-f(a)}{x-a}\]
    Then we have
    \[\lim_{x\to a}[f(x)-f(a)]
    =\lim_{x\to a}\left(\frac{f(x)-f(a)}{x-a}(x-a)\right)
    =\lim_{x\to a}\left(\frac{f(x)-f(a)}{x-a}\right)\lim_{x\to a}(x-a)
    =f'(a)\cdot 0=0\]
    Then we have
    \[\lim_{x\to a}f(x)=\lim_{x\to a}(f(a)+f(x)-f(a))
    =\lim_{x\to a}f(a)+\lim_{x\to a}(f(x)-f(a))=f(a)+0=f(a)\]
    Therefore, it is proved that \(f\) is continuous at \(a\).
\end{proof}
Note that there are functions that are continuous but not differentiable.
The function \(y=|x|\) is continuous at 0 but not differentiable at 0 since
\[f'(0)=\lim_{h\to 0}\frac{|0+h|-|0|}{h}\] if the limit exists but
\begin{align*}
    \lim_{h\to 0^-}\frac{|0+h|-|0|}{h} &= \lim_{h\to 0^-}\frac{|h|}{h}
    =\lim_{h\to 0^-}\frac{-h}{h}=-1 \\
    \lim_{h\to 0^+}\frac{|0+h|-|0|}{h} &= \lim_{h\to 0^+}\frac{|h|}{h}
    =\lim_{h\to 0^+}\frac{h}{h}=1
\end{align*}
thus the limit does not exist so \(f'(0)\) does not exist.
If a function is differentiable, then it has no sharp points, it is
continuous, and it has no vertical tangent lines.

\subsubsection{Higher Order Derivatives}
 If \(y=f(x)\) is a differentiable function and its derivative \(f'(x)\) is
 differentiable, then the second derivative of \(f\) is
\[y''=f''(x)=\frac{d^2y}{dx^2}\]
We can interpret \(f''(x)\) as the slope of the curve \(y=f'(x)\) at the point
\((x,f'(x))\), which is the rate of change of the slope of the original curve
\(y=f(x)\).
Let \(s=s(t)\) be a position function of an object with respect to time \(t\).
The velocity function \(v(t)\) of the object is \[v(t)=s'(t)=\frac{ds}{dt}\]
The instantaneous rate of change of velocity is the acceleration.
Thus the acceleration function \(a(t)\) is the derivative of the velocity
function and is therefore the second derivative of the position function.
\[a(t)=v'(t)=\frac{dv}{dt}=s''(t)=\frac{d^2s}{dt^2}\]
In general, the \(n\)th derivative of \(y=f(x)\) is
\[f^{(n)}(x)=\frac{d^ny}{dx^n}\]
\begin{problem}
    Find the first and the second derivatives of \(f(x)=x^3\).
\end{problem}
\begin{solution}
    We apply the binomial theorem by Newton
    \[(x+y)^n=\sum_{k=0}^n \binom{n}{k}x^{n-k}y^k\]
    For the first derivative we have
    \begin{align*}
        f'(x) &= \lim_{h\to 0}\frac{(x+h)^3-x^3}{h}=\lim_{h\to 0}
        \frac{x^3+3x^2h+3xh^2+h^3-x^3}{h}
        =\lim_{h\to 0}\frac{3x^2h+3xh^2+h^3}{h} \\
        &= \lim_{h\to 0}(3x^2+3xh+h^2)=3x^2
    \end{align*}
    For the second derivative we have
    \begin{align*}
        f''(x) &= \lim_{h\to 0}\frac{3(x+h)^2-3x^2}{h}
        =\lim_{h\to 0}\frac{3(x^2+2hx+h^2-x^2)}{h}
        =\lim_{h\to 0}\frac{6hx+3h^2}{h} \\
        &= \lim_{h\to 0}(6x+3h)=6x \qedhere
    \end{align*}
\end{solution}