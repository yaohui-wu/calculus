\subsection{Indeterminate Forms and l'Hospital's Rule}

In general, if we have a limit of the form
\[\lim_{x\to a}\frac{f(x)}{g(x)}\]
where both \(f(x)\to 0\) and \(g(x)\to 0\) as \(x\to a\), then this limit may
or may not exist and is called an \textbf{indeterminate form of type}
\(\dfrac{0}{0}\).
In general, if we have a limit of the form
\[\lim_{x\to a}\frac{f(x)}{g(x)}\]
where both \(f(x)\to\infty\) (or \(-\infty\)) and \(g(x)\to\infty\)
(or \(-\infty\)), then this limit may or may not exist and is called an
\textbf{indeterminate form of type} \(\dfrac{\infty}{\infty}\).
\begin{theorem}[l'Hospital's Rule]
    Suppose \(f\) and \(g\) are differentiable and \(g'(x)\neq 0\) near \(a\)
    (except possibly at \(a\)).
    Suppose that
    \begin{align*}
        \lim_{x\to a}f(x) &= 0 & \lim_{x\to a}g(x) &= 0
    \end{align*}
    or that
    \begin{align*}
        \lim_{x\to a}f(x) &= \pm\infty & \lim_{x\to a}g(x) &= \pm\infty
    \end{align*}
    In other words, we have an indeterminate form of type \(\dfrac{0}{0}\) or
    \(\dfrac{\infty}{\infty}\).
    Then
    \[\lim_{x\to a}\frac{f(x)}{g(x)}=\lim_{x\to a}\frac{f'(x)}{g'(x)}\]
    if the limit on the right side exists (or is \(\infty\) or \(-\infty\)).
\end{theorem}
It is important to verify the conditions regarding the limits of \(f\) and \(g\) before using
l'Hospital's Rule.
L'Hospital's Rule is also valid for one-sided limits and for limits at
infinity or negative infinity.
For the special case in which \(f(a)=g(a)=0\), \(f'\) and \(g'\) is
continuous, and \(g'(a)\neq 0\), we have
\[\lim_{x\to a}\frac{f'(x)}{g'(x)}=\frac{f'(a)}{g'(a)}
= \frac{\displaystyle{\lim_{x\to a}\frac{f(x)-f(a)}{x-a}}}
{\displaystyle{\lim_{x\to a}\frac{g(x)-g(a)}{x-a}}}
=\lim_{x\to a}\frac{f(x)-f(a)}{g(x)-g(a)}=\lim_{x\to a}\frac{f(x)}{g(x)}\]

\begin{problem}
    Find \(\displaystyle{\lim_{x\to 1}\frac{\ln x}{x-1}}\).
\end{problem}
\begin{solution}
    Notice that we have an indeterminate form of type \(\dfrac{0}{0}\).
    We apply l'Hospital's rule then
    \[\lim_{x\to 1}\dfrac{\ln x}{x-1}=\lim_{x\to 1}\frac{1/x}{1}
    =\lim_{x\to 1}\frac{1}{x}=1\]
\end{solution}
\begin{problem}
    Calculate \(\displaystyle{\lim_{x\to\infty}\frac{e^x}{x^2}}\).
\end{problem}
\begin{solution}
    Notice that we have an indeterminate form of type
    \(\dfrac{\infty}{\infty}\).
    Then l'Hospital's Rule gives
    \[\lim_{x\to\infty}\frac{e^x}{x^2}=\lim_{x\to\infty}\frac{e^x}{2x}\]
    Since we still have an indeterminate form of \(\dfrac{\infty}{\infty}\), a
    second application of l'Hospital's Rule gives
    \[\lim_{x\to\infty}\frac{e^x}{x^2}=\lim_{x\to\infty}\frac{e^x}{2x}
    =\lim_{x\to\infty}\frac{e^x}{2}=\infty\]
\end{solution}
\begin{problem}
    Find \(\displaystyle{\lim_{x\to 0}\frac{\tan x-x}{x^3}}\).
\end{problem}
\begin{solution}
    Notice that we have an indeterminate form of type \(\dfrac{0}{0}\).
    We apply l'Hospital's rule then
    \begin{align*}
        \lim_{x\to 0}\frac{\tan x-x}{x^3}
        &= \lim_{x\to 0}\frac{\sec^2 x-1}{3x^2}
        =\lim_{x\to 0}\frac{2\sec^2 x\tan x}{6x}
        =\lim_{x\to 0}\frac{2\sec^4 x+4\sec^2 x\tan^2 x}{6} \\
        &= \frac{2(1)+4(1)(0)}{6}=\frac{1}{3}
    \end{align*}
\end{solution}

\subsubsection*{Indeterminate Products}
If \(\displaystyle{\lim_{x\to a}f(x)=0}\) and
\(\displaystyle{\lim_{x\to a}g(x)=\infty}\) (or \(-\infty\)),
then the limit
\[\lim_{x\to a}f(x)g(x)\]
is called an \textbf{indeterminate form of type} \(0\cdot\infty\).
We can deal with it by writing the product \(fg\) as a quotient:
\[fg=\frac{f}{1/g}\]
or
\[fg=\frac{g}{1/f}\]
This converts the given limit into an indeterminate form of type
\(\dfrac{0}{0}\) or \(\dfrac{\infty}{\infty}\) so that we can use l'Hospital's
Rule.
\begin{problem}
    Evaluate \(\displaystyle{\lim_{x\to 0^+}x\ln x}\).
\end{problem}
\begin{solution}

    \[\lim_{x\to 0^+}x\ln x=\lim_{x\to 0^+}\frac{\ln x}{1/x}
    =\lim_{x\to 0^+}\frac{1/x}{-1/x^2}=\lim_{x\to 0^+}(-x)=0\]
\end{solution}

\subsubsection*{Indeterminate Differences}
If \(\displaystyle{\lim_{x\to a}f(x)=\infty}\) and
\(\displaystyle{\lim_{x\to a}g(x)=\infty}\), then the limit
\[\lim_{x\to a}\big[f(x)-g(x)\big]\]
is called an \textbf{indeterminate form of type} \(\infty-\infty\).
We can convert the difference into a quotient so that we have an indeterminate
form of type \(\dfrac{0}{0}\) or \(\dfrac{\infty}{\infty}\).
\begin{problem}
    Compute \(\displaystyle{\lim_{x\to (\pi/2)^-}(\sec x-\tan x)}\).
\end{problem}
\begin{solution}
    \[\lim_{x\to (\pi/2)^-}(\sec x-\tan x)
    =\lim_{x\to (\pi/2)^-}\frac{1-\sin x}{\cos x}
    =\lim_{x\to (\pi/2)^-}\frac{-\cos x}{-\sin x}=0\]
\end{solution}

\subsubsection*{Indeterminate Powers}
Several indeterminate forms arise from the limit
\[\lim_{x\to a}\big[f(x)\big]^{g(x)}\]
\begin{itemize}
    \item Type \(0^0\): \(\displaystyle{\lim_{x\to a}f(x)=0}\) and
    \(\displaystyle{\lim_{x\to a}g(x)=0}\).
    \item Type \(\infty^0\): \(\displaystyle{\lim_{x\to a}f(x)=\infty}\) and
    \(\displaystyle{\lim_{x\to a}g(x)=0}\).
    \item Type \(1^\infty\): \(\displaystyle{\lim_{x\to a}f(x)=1}\) and
    \(\displaystyle{\lim_{x\to a}g(x)=\pm\infty}\).
\end{itemize}
Note that the form \(0^\infty\) is not indeterminate.
Each of these three cases can be treated either by taking the natural
logarithm: let \(y=\big[f(x)\big]^{g(x)}\), then \(\ln y=g(x)\ln f(x)\) or
by writing the function as an exponential:
\[\big[f(x)\big]^{g(x)}=e^{g(x)\ln f(x)}\]
In either method we are led to the indeterminate product \(g(x)\ln f(x)\),
which is of type \(0\cdot\infty\).
\begin{problem}
    Calculate \(\displaystyle{\lim_{x\to 0^+}(1+\sin 4x)^{\cot x}}\).
\end{problem}
\begin{solution}
    Let
    \[y=(1+\sin 4x)^{\cot x}\]
    Then
    \[\ln y=\ln\big[(1+\sin 4x)^{\cot x}\big]=\cot x\ln(1+\sin 4x)\]
    so l'Hospital's Rule gives
    \[\lim_{x\to 0^+}\ln y=\lim_{x\to 0^+}\frac{\ln(1+\sin 4x)}{\tan x}
    =\lim_{x\to 0^+}\frac{\dfrac{4\cos 4x}{1+\sin 4x}}{\sec^2 x}=4\]
    Then
    \[\lim_{x\to 0^+}(1+\sin 4x)^{\cot x}=\lim_{x\to 0^+}y
    =\lim_{x\to 0^+}e^{\ln y}=e^4\]
\end{solution}
\begin{problem}
    Find \(\displaystyle{\lim_{x\to 0^+}x^x}\).
\end{problem}
\begin{solution}
    We used l'Hospital's Rule to show that
   \[\lim_{x\to 0^+}x\ln x=0\]
   Therefore
   \[\lim_{x\to 0^+}x^x=\lim_{x\to 0^+}e^{x\ln x}=e^0=1\]
\end{solution}