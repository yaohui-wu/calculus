\subsection{Basic Differentiation Formulas}

The derivative of the constant function \(f(x)=c\) is
\[\frac{d}{dx}(c)=0\]
\begin{proof}
    If \(f(x)=c\) where \(c\) is a constant, then
    \[f'(x)=\lim_{h\to 0}\frac{f(x+h)-f(x)}{h}
    =\lim_{h\to 0}\frac{c-c}{h}=\lim_{h\to 0}0=0\]
\end{proof}

\subsubsection*{Power Functions}
\[\frac{d}{dx}(x)=1\]
\begin{proof}
    If \(f(x)=x\), then
    \[f'(x)=\lim_{h\to 0}\frac{f(x+h)-f(x)}{h}=\lim_{h\to 0}\frac{x+h-x}{h}
    =\lim_{h\to 0}\frac{h}{h}=\lim_{h\to 0}1=1\]
\end{proof}
The power rule:
If \(n\) is a positive integer, then
\[\frac{d}{dx}(x^n)=nx^{n-1}\]
\begin{proof}
    Let \(f(x)=x^n\), then
    \[f'(x)=\lim_{h\to 0}\frac{f(x+h)-f(x)}{h}
    =\lim_{h\to 0}\frac{(x+h)^n-x^n}{h}\]
    We use the binomial theorem to expand \((x+h)^n\) then
    \begin{align*}
        f'(x)
        &= \lim_{h\to 0}
        \frac{
            \left(x^n+nx^{n-1}h+\dfrac{n(n-1)}{2}x^{n-2}h^2+\cdots+nxh^{n-1}
            +h^n\right)-x^n}{h} \\
        &= \lim_{h\to 0}
        \frac{
            nx^{n-1}h+\dfrac{n(n-1)}{2}x^{n-2}h^2+\cdots+nxh^{n-1}+h^n}{h} \\
        &= \lim_{h\to 0}
        (nx^{n-1}+\frac{n(n-1)}{2}x^{n-2}h+\dots+nxh^{n-2}+h^{n-1}) \\
        &= nx^{n-1}
    \end{align*}
    because every term except the first has \(h\) as a factor and therefore
    approaches 0.
\end{proof}
The power rule (general version):
If \(n\) is any real number, then
\[\frac{d}{dx}(x^n)=nx^{n-1}\]
\begin{problem}
    Differentiate \(f(x)=1/x\).
\end{problem}
\begin{solution}
    \[\frac{d}{dx}\left(\frac{1}{x}\right)=\frac{d}{dx}x^{-1}=(-1)x^{-1-1}
    =-x^{-2}=-\frac{1}{x^2}\]
\end{solution}
The \textbf{normal line} to a curve \(C\) at a point \(P\) is the line through
\(P\) that is perpendicular to the tangent line at \(P\).
The constant multiple rule:
If \(c\) is a constant and \(f\) is a differentiable function, then
\[\frac{d}{dx}\bigl[cf(x)\bigr]=c\,\frac{d}{dx}f(x)\]
\begin{proof}
    Let \(g(x)=cf(x)\).
    Then
    \begin{align*}
        g'(x) &= \lim_{h\to 0}\frac{cf(x+h)-cf(x)}{h}
        =\lim_{h\to 0}c\left(\frac{f(x+h)-f(x)}{h}\right) \\
        &= c\lim_{h\to 0}\frac{f(x+h)-f(x)}{h}
        =c\,\frac{d}{dx}f(x)
    \end{align*}
\end{proof}
The sum rule: If \(f\) and \(g\) are both differentiable, then
\[\frac{d}{dx}\bigl[f(x)+g(x)\bigr]=\frac{d}{dx}f(x)+\frac{d}{dx}g(x)\]
\begin{proof}
    Let \(F(x)=f(x)+g(x)\).
    Then
    \begin{align*}
        F'(x) &= \lim_{h\to 0}
        \frac{\bigl[f(x+h)+g(x+h)\bigr]-\bigl[f(x)+g(x)\bigr]}{h} \\
        &= \lim_{h\to 0}
        \left[\frac{f(x+h)-f(x)}{h}+\frac{g(x+h)-g(x)}{h}\right] \\
        &= \lim_{h\to 0}\frac{f(x+h)-f(x)}{h}
        +\lim_{h\to 0}\frac{g(x+h)-g(x)}{h} \\
        &= \frac{d}{dx}f(x)+\frac{d}{dx}g(x)
    \end{align*}
\end{proof}
The sum rule can be extended to the sum of any number of functions.
By writing \(f-g\) as \(f+(-1)g\) and applying the sum rule and the constant
multiple rule, we get the following formula.
The difference rule: If \(f\) and \(g\) are both differentiable, then
\[\frac{d}{dx}\bigl[f(x)-g(x)\bigr]=\frac{d}{dx}f(x)-\frac{d}{dx}g(x)\]
\begin{proof}
    Let \(F(x)=f(x)-g(x)\).
    Then
    \begin{align*}
        F'(x) &= \lim_{h\to 0}
        \frac{\bigl[f(x+h)-g(x+h)\bigr]-\bigl[f(x)-g(x)\bigr]}{h} \\
        &= \lim_{h\to 0}\frac{f(x+h)-f(x)-g(x+h)+g(x)}{h} \\
        &= \lim_{h\to 0}\frac{f(x+h)-f(x)-\bigl[g(x+h)-g(x)\bigr]}{h} \\
        & =\lim_{h\to 0}
        \left[\frac{f(x+h)-f(x)}{h}-\frac{g(x+h)-g(x)}{h}\right] \\
        &= \lim_{h\to 0}
        \frac{f(x+h)-f(x)}{h}-\lim_{h\to 0}\frac{g(x+h)-g(x)}{h} \\
        &= \frac{d}{dx}f(x)-\frac{d}{dx}g(x)
    \end{align*}
\end{proof}

The Constant Multiple Rule, the Sum Rule, and the Difference Rule can be
combined with the Power Rule to differentiate any polynomial.

\subsubsection*{The Sine and Cosine Functions}
\[\frac{d}{dx}(\sin x)=\cos x\]
\begin{proof}
    If \(f(x)=\sin x\), then
    \begin{align*}
        f'(x) &= \lim_{h\to 0}\frac{f(x+h)-f(x)}{h}
        =\lim_{h\to 0}\frac{\sin(x+h)-\sin x}{h}
        =\lim_{h\to 0}\frac{\sin x\cos h+\cos x\sin h-\sin x}{h} \\
        &= \lim_{h\to 0}
        \left[\frac{\sin x\cos h-\sin x}{h}+\frac{\cos x\sin h}{h}\right]
        =\lim_{h\to 0}
        \left[\sin x\left(\frac{\cos h-1}{h}\right)
        +\cos x\left(\frac{\sin h}{h}\right)\right] \\
        &= \sin x\cdot\lim_{h\to 0}\left(\frac{\cos h-1}{h}\right)
        +\cos x\cdot\lim_{h\to 0}\left(\frac{\sin h}{h}\right) \\
        &=(\sin x)\cdot 0+(\cos x)\cdot 1=\cos x
    \end{align*}
\end{proof}
\[\frac{d}{dx}(\cos x)=-\sin x\]
\begin{proof}
    If \(f(x)=\cos x\), then
    \begin{align*}
        f'(x) &= \lim_{h\to 0}\frac{f(x+h)-f(x)}{h}
        =\lim_{h\to 0}\frac{\cos(x+h)-\cos x}{h}
        =\lim_{h\to 0}\frac{\cos x\cos h-\sin x\sin h-\cos x}{h} \\
        &= \lim_{h\to 0}
        \left[\frac{\cos x\cos h-\cos x}{h}-\frac{\sin x\sin h}{h}\right]
        =\lim_{h\to 0}
        \left[\cos x\left(\frac{\cos h-1}{h}\right)
        -\sin x\left(\frac{\sin h}{h}\right)\right] \\
        &= \cos x\cdot\lim_{h\to 0}\left(\frac{\cos h-1}{h}\right)
        -\sin x\cdot\lim_{h\to 0}\left(\frac{\sin h}{h}\right) \\
        &=(\cos x)\cdot 0-(\sin x)\cdot 1=-\sin x
    \end{align*}
\end{proof}
If \(f(x)=\sin x\), then
\begin{align*}
    f'(x) &= \cos x & f''(x) &= -\sin x & f'''(x) &= -\cos x
    & f^{(4)}(x) &= \sin x
\end{align*}
In general, if \(f(x)=\sin x\), then for \(n=0,1,2,\cdots\) we have
\begin{align*}
    f^{(4n)}(x) &= \sin x & f^{(4n+1)}(x) &= \cos x & f^{(4n+2)}(x) &= -\sin x
    & f^{(4n+3)}(x) &= -\cos x
\end{align*}
If \(f(x)=\cos x\), then
\begin{align*}
    f'(x) &= -\sin x & f''(x) &= -\cos x & f'''(x) &= \sin x
    & f^{(4)}(x) &= \cos x
\end{align*}
In general, if \(f(x)=\cos x\), then for \(n=0,1,2,\cdots\) we have
\begin{align*}
    f^{(4n)}(x) &= \cos x & f^{(4n+1)}(x) &= -\sin x
    & f^{(4n+2)}(x) &= -\cos x
    & f^{(4n+3)}(x) &= \sin x
\end{align*}