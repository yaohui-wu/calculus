\subsection{Differentiation}
\subsubsection{Differentiation Formulas}
Let \(f(x)\) and \(g(x)\) be differentiable functions, then we have the
following differentiation formulas.
\begin{theorem}
    Let \(f(x)=c\) where \(c\) is a constant, then \[\frac{d}{dx}(c)=0\qedhere\]
\end{theorem}
\begin{proof}
    \[f'(x)=\lim_{h\to 0}\frac{c-c}{h}=\lim_{h\to 0}0=0\qedhere\]
\end{proof}
\begin{theorem}[Power Rule]
    \[\frac{d}{dx}(x^n)=nx^{n-1},n\in\R\qedhere\]
\end{theorem}
Note the special case when \(n=1\), then we have \[\frac{d}{dx}(x)=1\]
\begin{theorem}[Constant Multiple Rule]
    If \(c\) is a constant, then \[\frac{d}{dx}[cf(x)]=c\,\frac{d}{dx}f(x)\qedhere\]
\end{theorem}
\begin{theorem}[Sum and Difference Rule]
    \[\frac{d}{dx}[f(x)\pm g(x)]=\frac{d}{dx}f(x)\pm \frac{d}{dx}g(x)\qedhere\]
\end{theorem}

\subsubsection{Product and Quotient Rules}
Let \(f(x)\) and \(g(x)\) be differentiable functions, then we have the
product rule by Leibniz and the quotient rule.
\begin{theorem}[Product Rule]
    \[\frac{d}{dx}[f(x)g(x)]=f(x)\left[\frac{d}{dx}g(x)\right]
    +\left[\frac{d}{dx}f(x)\right]g(x)\qedhere\]
\end{theorem}
\begin{theorem}[Quotient Rule]
    \[\frac{d}{dx}\left[\frac{f(x)}{g(x)}\right]=\frac{\left[\cfrac{d}{dx}f(x)\right]g(x)
    -f(x)\left[\cfrac{d}{dx}g(x)\right]}{[g(x)]^2}\qedhere\]
\end{theorem}

\subsubsection{Trigonometric Functions}
\begin{theorem}
    \[\frac{d}{dx}\sin x=\cos x\qedhere\]
\end{theorem}
\begin{theorem}
    \[\frac{d}{dx}\cos x=-\sin x\qedhere\]
\end{theorem}
\begin{theorem}
    \[\frac{d}{dx}\tan x=\sec^2 x\qedhere\]
\end{theorem}

\subsubsection{Chain Rule}
We have the chain rule by \textbf{James Gregory} (1638--1675) to find the
derivative of a composite function.
\begin{theorem}[Chain Rule]
    If \(f\) and \(g\) are differentiable functions and \(F=f(g(x))\),
    then \(F\) is differentiable and \(F'\) is
    \[F'(x)=f'(g(x))\cdot g'(x)\]
    If \(y=f(u)\) and \(u=g(x)\), then \[\frac{dy}{dx}=\frac{dy}{du}\frac{du}{dx}\qedhere\]
\end{theorem}