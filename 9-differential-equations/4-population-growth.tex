\subsection{Population Growth}
In general, if \(y(t)\) is the value of a quantity \(y\) at time \(t\) and if
the rate of change of \(y\) with respect to \(t\) is proportional to its size
\(y(t)\) at any time, then
\[\frac{dy}{dt}=ky\]
where \(k\) is a constant.
This differential equation is called the \textbf{law of natural growth}
(if \(k>0\)) or the \textbf{law of natural decay} (if \(k<0\)).
Since the differential equation is separable, then
\begin{align*}
    \frac{dy}{dt} &= ky \\
    \int\frac{dy}{y} &= \int k\,dt \\
    \ln|y| &= kt+C \\
    |y| &= e^{kt+C}=e^Ce^{kt} \\
    y &= \pm e^Ce^{kt}=Ae^{kt}
\end{align*}
so \(y=Ae^{kt}\) is the general solution of the differential equation.
It follows that
\[\frac{dy}{dt}=kAe^{kt}=ky\]
and \(y(0)=Ae^0=A\) is the initial value of the function \(y(t)\).

\subsubsection*{Logistic Growth}
If \(M\) is the carrying capacity and \(0<y<M\), then the logistic
differential equation is
\[\frac{dy}{dt}=ky(M-y)\]
We use separation of variables then
\begin{align*}
    \frac{dy}{dt} &= ky(M-y) \\
    \int\frac{dy}{y(M-y)} &= \int k\,dt
\end{align*}
and using partial fractions
\begin{align*}
    \frac{1}{y(M-y)} &= \frac{A}{y}+\frac{B}{M-y} \\
    1 &= A(M-y)+By \\
    y=0 &\iff A=\frac{1}{M} \\
    y=M &\iff B=\frac{1}{M}
\end{align*}
therefore
\begin{align*}
    \frac{1}{M}\int\left(\frac{1}{y}+\frac{1}{M-y}\right)\,dy &= \int k\,dt \\
    \frac{1}{M}(\ln|y|-\ln|M-y|) &= kt+C_1 \\
    \ln\frac{y}{M-y} &= kMt+C_2 \\
    \frac{y}{M-y} &= e^{kMt+C_2}=e^{C_2}e^{kMt}=Ae^{kMt}
\end{align*}
If the population at time \(t=0\) is \(y(0)=y_0\),
then \(A=y_0/(M-y_0)\) and so
\begin{align*}
    \frac{y}{M-y} &= \frac{y_0}{M-y_0}e^{kMt} \\
    (M-y_0)y &= y_0e^{kMt}(M-y)=y_0Me^{kMt}-y_0e^{kMt}y \\
    (M-y_0)y+y_0e^{kMt}y=(M-y_0+y_0e^{kMt})y &= y_0Me^{kMt} \\
    y &= \frac{y_0Me^{kMt}}{M-y_0+y_0e^{kMt}}
\end{align*}
then
\[y=\frac{y_0M}{(M-y_0+y_0e^{kMt})e^{-kMt}}=\frac{y_0M}{y_0+(M-y_0)e^{-kMt}}\]
is the solution of the differential equation and
\[\lim_{t\to\infty}y(t)=\frac{y_0M}{y_0+0}=M\]
We can show that
\begin{align*}
    \frac{d^2y}{dt^2} &= \frac{d}{dt}(kMy-ky^2)=kM\frac{dy}{dt}-2ky\frac{dy}{dt}
    =k(M-2y)\frac{dy}{dt}=k(M-2y)ky(M-y) \\
    &= k^2y(M-y)(M-2y)
\end{align*}
then
\[k^2y(M-y)(M-2y)=0\]
and
\begin{align*}
    k^2y &= 0 & M-y &= 0 & M-2y &= 0 \\
    y &= 0 & y &= M & y &= \frac{M}{2}
\end{align*}
so
\[\left.\frac{dy}{dt}\right|_{y=0}=k(0)(M-0)=0\]
\[\left.\frac{dy}{dt}\right|_{y=M}=kM(M-M)=0\]
\[\left.\frac{dy}{dt}\right|_{y=M/2}=k\frac{M}{2}\left(M-\frac{M}{2}\right)
=\frac{kM^2}{4}\]
We deduce that a population grows fastest when it reaches half its carrying
capacity.

\begin{problem}
    An object of mass \(m\) is moving horizontally through a medium which resists the
    motion with a force that is a function of the velocity
    \[m\frac{d^2s}{dt^2}=m\frac{dv}{dt}=f(v)\]
    where \(v=v(t)\) and \(s=s(t)\) represent the velocity and position of the
    object at time \(t\), respectively.
    Let \(v(0)=v_0\) and \(s(0)=s_0\) be the initial values of \(v\) and \(s\).
    Suppose that the resisting force is proportional to the velocity for small
    values of \(v\) so
    \[f(v)=-kv\]
    where \(k\) is a positive constant.
    For large values of \(v\) a better model is
    \[f(v)=-kv^2\]
    where \(k\) is a positive constant.
    Determine \(v\) and \(s\) at any time \(t\) for each of the two models.
    What is the total distance that the object travels in each case?
\end{problem}
\begin{solution}
    For small values of \(v\) we have
    \begin{align*}
        m\frac{dv}{dt} &= -kv \\
        \int\frac{dv}{v} &= -\int\frac{k}{m}\,dt \\
        \ln(|v|) &= -\frac{k}{m}t+C \\
        |v| &= e^{-(k/m)t+C}=e^Ce^{-(k/m)t} \\
        v &= \pm e^Ce^{-(k/m)t}=Ae^{-(k/m)t}
    \end{align*}
    and since \(v(0)=A=v_0\) hence \(v=v_0e^{-(k/m)t}\).
    Then
    \[s=\int v_0e^{-(k/m)t}\,dt=-\frac{m}{k}v_0e^{-(k/m)t}+C\]
    and since \(s(0)=-(m/k)v_0+C=s_0\iff C=s_0+(m/k)v_0\) hence
    \[s=-\frac{m}{k}v_0e^{-(k/m)t}+s_0+\frac{m}{k}v_0
    =s_0+\frac{m}{k}v_0(1-e^{-(k/m)t})\]
    The total distance traveled by the object is
    \begin{align*}
        \lim_{t\to\infty}(s(t)-s(0))
        &= \lim_{t\to\infty}\left(s_0+\frac{m}{k}v_0(1-e^{-(k/m)t})-s_0\right)
        =\lim_{t\to\infty}\frac{m}{k}v_0(1-e^{-(k/m)t}) \\
        &= \frac{m}{k}v_0(1-0)=\frac{m}{k}v_0
    \end{align*}
    For large values of \(v\) we have
    \begin{align*}
        m\frac{dv}{dt} &= -kv^2 \\
        \int\frac{dv}{v^2} &= -\int\frac{k}{m}\,dt \\
        -\frac{1}{v} &= -\frac{k}{m}t+C_1 \\
        \frac{1}{v} &= \frac{k}{m}t+C=\frac{kt+Cm}{m} \\
        v &= \frac{m}{kt+Cm}
    \end{align*}
    and since \(v(0)=m/(Cm)=v_0\iff C=1/v_0\) hence
    \[v=\frac{m}{kt+(m/v_0)}\]
    Then
    \begin{align*}
        s &= \int\frac{m}{kt+(m/v_0)}\,dt=\frac{m}{k}\int\frac{du}{u}
        =\frac{m}{k}\ln(|u|)+C
        =\frac{m}{k}\ln\left(\left|kt+\frac{m}{v_0}\right|\right)+C
    \end{align*}
    and since \(s(0)=(m/k)\ln(|m/v_0|)+C=s_0\iff C=s_0-(m/k)\ln(|m/v_0|)\)
    hence
    \begin{align*}
        s &= \frac{m}{k}\ln\left(\left|kt+\frac{m}{v_0}\right|\right)+s_0
        -\frac{m}{k}\ln\left(\left|\frac{m}{v_0}\right|\right)
        =s_0+\frac{m}{k}\left(\ln\left(\left|kt+\frac{m}{v_0}\right|\right)
        -\ln\left(\left|\frac{m}{v_0}\right|\right)\right) \\
        &= s_0+\frac{m}{k}
        \ln\left(\left|\frac{kt+(m/v_0)}{m/v_0}\right|\right)
        =s_0+\frac{m}{k}\ln\left(\left|\frac{k}{m}v_0t+1\right|\right)
    \end{align*}
    Since \(m\) is the mass so \(m\) is positive and given \(k\) is positive
    therefore
    \[s=s_0+\frac{m}{k}\ln\left(\frac{k}{m}v_0t+1\right)\]
    and the distance traveled by the object is
    \[\lim_{t\to\infty}(s(t)-s(0))
    =\lim_{t\to\infty}\left(s_0
    +\frac{m}{k}\ln\left(\frac{k}{m}v_0t+1\right)-s_0\right)=
    \lim_{t\to\infty}\frac{m}{k}\ln\left(\frac{k}{m}v_0t+1\right)
    =\infty\qedhere\]
\end{solution}
\begin{problem}
    According to Newton's law of universal gravitation,
    the gravitational force on an object of mass \(m\) that has been
    projected vertically upward from the earth's surface is
    \[F=\frac{mgR^2}{(x+R)^2}\]
    where \(x=x(t)\) is the object's distance
    above the surface at time \(t\), \(R\) is the
    earth's radius, and \(g\) is the acceleration due to gravity.
    Also, by Newton's second law,
    \[F=ma=m\frac{dv}{dt}=-\frac{mgR^2}{(x+R)^2}\]
    Suppose a rocket is fired vertically upward with an initial velocity \(v_0\).
    Let \(h\) be the maximum height above the surface reached by the object.
    Show that \(v_0=\sqrt{\dfrac{2gRh}{R+h}}\) and compute
    \(v_e=\lim_{h\to\infty}v_0\), the escape velocity of the earth, using
    \(R=6,378\) km and \(g=9.8\ \text{m/s}^2\).
\end{problem}
\begin{solution}
    By the chain rule
    \[\frac{dv}{dt}=\frac{dx}{dt}\frac{dv}{dx}=v\frac{dv}{dx}\]
    then
    \begin{align*}
        m\frac{dv}{dt}=mv\frac{dv}{dx} &= -\frac{mgR^2}{(x+R)^2} \\
        v\,dv &= -\frac{gR^2}{(x+R)^2}\,dx
    \end{align*}
    Since the height is zero when \(v=v_0\) and the height is maximum when
    \(v=0\) then
    \begin{align*}
        \int_{v_0}^0 v\,dv &= -\int_0^h \frac{gR^2}{(x+R)^2}\,dx \\
        \left[\frac{v^2}{2}\right]_{v_0}^0
        &= \left[\frac{gR^2}{x+R}\right]_0^h \\
        -\frac{(v_0)^2}{2} &= gR^2\left(\frac{1}{R+h}-\frac{1}{R}\right)
        =gR^2\left(\frac{R-(R+h)}{R(R+h)}\right)=-\frac{gRh}{R+h} \\
        v_0= &= \sqrt{\dfrac{2gRh}{R+h}}
    \end{align*}
    The escape velocity of the earth is
    \begin{align*}
        v_e &= \lim_{h\to\infty}\sqrt{\dfrac{2gRh}{R+h}}
        =\sqrt{\lim_{h\to\infty}\frac{2gR}{(R/h)+1}}=\sqrt{\frac{2gR}{0+1}}
        =\sqrt{2gR} \\
        &=\sqrt{2(9.8)(6.378\times10^6)}\ \text{m/s}
        \approx 11.1807\ \text{km/s}
    \end{align*}
\end{solution}