\subsection{Ordinary Differential Equations}

A \textbf{differential equation }is an equation that contains an unknown
function and one or more of its derivatives.
An \textbf{ordinary differential equation} (ODE) is a differential equation
that relates one or more functions of a single variable and their ordinary
derivatives.The importance of differential equations lies in the fact that
when a scientist or engineer formulates a physical law in mathematical terms,
it frequently turns out to be a differential equation.

The \textbf{order} of a differential equation is the order of the highest
derivative that occurs in the equation.

A function \(f\) is called a \textbf{solution} of a differential equation if
the equation is satisfied when \(y=f(x)\) and its derivatives are substituted
into the equation.

When we are asked to solve a differential equation we are expected to find all
possible solutions of the equation.

In many physical problems we need to find the particular solution that
satisfies a condition of the form \(y(x_0)=y_0\).
This is called an \textbf{initial condition}, and the problem of finding a
solution of the differential equation that satisfies the initial condition is
called an \textbf{initial value problem} (IVP).

% Newton's second law of motion, the force \(F\) acting on an object with mass
% \(m\) and acceleration \(a\) is \(F=ma\) can be written as an ordinary
% differential equation
% \[F=m\frac{dv}{dt}\]
% which is a first order differential equation, or
% \[F=m\frac{d^2s}{dt^2}\]
%  which is a second order differential equation.
% By Newton's second law of motion, the net force \(F\) acting on a falling
% object with mass \(m\), velocity \(v\), and air resistance force \(\gamma v\)
% can be modeled by the differential equation
% \[F=m\frac{dv}{dt}=mg-\gamma v\]
% where \(g\) is the acceleration due to gravity and \(\gamma\) is the drag
% coefficient.
% A function \(f\) is a \textbf{solution} of a differential equation if the
% function and its derivatives satisfy the equation for all values of \(x\) in
% some open interval \(a<x<b\).
% It is possible that there are many solutions of a differential equation.
% An initial condition is a condition \(y(x_0)=y_0\) or
% \(y^{(n)}(x_0)=y_n\) on the solution.
% An \textbf{initial value problem} is solving a differential equation with initial conditions.
% The interval of validity is the largest possible interval on which
% the solution is valid and contains \(x_0\) in the initial conditions.
% The general solution of a differential equation is the set of all solutions
% and the particular solution is the solution that satisfies the initial
% conditions.
% An explicit solution is any solution in the form \(y=y(x)\), otherwise it is
% an implicit solution.
% The particular solution of the differential equation
% \[\frac{dy}{dx}=y\]
% with initial
% condition \(y(0)=1\) is \(y=e^x\) since
% \[\frac{dy}{dx}=\frac{d}{dx}e^x=e^x=y\]
% and \(y(0)=e^0=1\).
% The \textbf{existence} and \textbf{uniqueness} problem asks that given a
% differential equation, does there exist a solution and if any is there only
% one solution.