\documentclass[12pt]{article}
\usepackage{mystyle}

\title{Calculus}
\author{Yaohui Wu}
\date{August 7, 2024}

\begin{document}
\maketitle

\section*{Introduction}

\textbf{Calculus} is the mathematical study of continuous change established
by Issac Newton and Gottfried Wilhelm Leibniz.
\textbf{Single variable calculus} studies \textbf{derivatives} and
\textbf{integrals} of functions of one variable and their relationship stated
by the \textbf{Fundamental Theorem of Calculus}.
\[\int_a^b f(x)\,dx=F(b)-F(a)\]

\tableofcontents
\newpage

\section{Functions and Limits}

\subsection{The Limit of a Function}

\subsubsection*{Functions}  
A function \(f:X\mapsto Y\) is a rule that assigns each element \(x\) in set
\(X\) to exactly one element \(y\) in set \(Y\).
\begin{definition}
    A \textbf{function} \(f\) is a binary relation \(R\) between domain \(X\)
    and codomain \(Y\) that satisfies:
    \begin{itemize}
        \item \(R\) is a subset of the Cartesian product of \(X\) and \(Y\).
        \[R\subset\{(x,y)\mid x\in X,y\in Y\}\]
        \item For every \(x\) in \(X\), there exists a \(y\) in \(Y\) such
        that \((x,y)\) is in \(R\).
        \[\forall x\in X,\exists y\in Y,(x,y)\in R\]
        \item If \((x,y)\) and \((x,z)\) are in \(R\), then \(y=z\).
        \[(x,y),(x,z)\in R\implies y=z\qedhere\]
    \end{itemize}
\end{definition}
The real line is the one-dimensional (1D) Eculidean space defined as the set
of all real numbers \(\R\).
The \(xy\)-plane in the Cartesian coordinate system is the two-dimensional
(2D) Eculidean space defined as the set of all ordered pairs of real numbers
\((x,y)\in\R^2\).
A function of a real variable is a function whose domain is the set of real
numbers \(\R\).
A real function is a real-valued function of a real variable whose domain and
codomain is \(\R\).
\begin{definition}
    A function \(f\) is injective, or one-to-one if \(f(x_1)\neq f(x_2)\) when
    \(x_1\neq x_2\).
\end{definition}
\begin{definition}
    A function \(f\) is surjective, or onto if for all \(y\) in range \(Y\),
    there exists an \(x\) in domain \(X\) such that \(f(x)=y\).
\end{definition}
\begin{definition}
    A function \(f\) is bijective if \(f\) is injective and surjective.
\end{definition}
\begin{definition}
    Let \(f\) be an injective function with domain \(X\) and range \(Y\).
    Then its inverse function \(f^{-1}\) has domain \(Y\) and range \(X\) and
    is defined by
    \[f^{-1}(y)=x\iff f(x)=y\]
    for all \(y\) in \(Y\).
\end{definition}

\subsubsection*{Intuitive Definition of a Limit}
Let \(f(x)\) be a function defined on some open interval that contains the
number \(a\), except possibly at \(a\) itself.
\begin{definition}
    The \textbf{limit} of \(f(x)\) as \(x\) approaches \(a\) equals \(L\) if
    we can make \(f(x)\) arbitrarily close to \(L\) by taking \(x\)
    sufficiently close to \(a\) from the left and the right but \(x\neq a\).
    \[\lim_{x\to a}f(x)=L\qedhere\]
\end{definition}
\begin{definition}
    The left-hand limit of \(f(x)\) as \(x\) approaches \(a\) from the left
    equals \(L\) if we can make \(f(x)\) arbitrarily close to \(L\) by taking
    \(x\) sufficiently close to \(a\) where \(x<a\).
    \[\lim_{x\to a^-}f(x)=L\qedhere\]
\end{definition}
\begin{definition}
    The right-hand limit of \(f(x)\) as \(x\) approaches \(a\) from the right
    equals \(L\) if we can make \(f(x)\) arbitrarily close to \(L\) by taking
    \(x\) sufficiently close to \(a\) where \(x>a\).
    \[\lim_{x\to a^+}f(x)=L\qedhere\]
\end{definition}
The limit exists if the left-hand limit and the right-hand limit of \(f(x)\)
as \(x\) approaches \(a\) equal \(L\), otherwise the limit does not exist.
\[\lim_{x\to a}f(x)=L\iff\lim_{x\to a^-}f(x)=\lim_{x\to a^+}f(x)=L\]
\subsection{The Precise Definition of a Limit}

\subsubsection*{The Precise Definition of a Limit}
Let \(f\) be a function defined on some open interval that contains the number
\(a\), except possibly at a itself.
\begin{definition}
    The limit of \(f(x)\) as \(x\) approaches \(a\) is \(L\) and we write
    \[\lim_{x\to a}f(x)=L\]
    if for every number \(\epsilon>0\) there is a number \(\delta>0\) such that
    \[0<|x-a|<\delta\implies|f(x)-L|<\epsilon\]
\end{definition}
\begin{definition}
    \[\lim_{x\to a^-}f(x)=L\] if for every number \(\epsilon>0\), there exists
    a number \(\delta>0\) such that
    \[a-\delta<x<a\implies|f(x)-L|<\epsilon\]
\end{definition}
\begin{definition}
    \[\lim_{x\to a^+}f(x)=L\] if for every number \(\epsilon>0\), there exists
    a number \(\delta>0\) such that
    \[a<x<a+\delta\implies|f(x)-L|<\epsilon\]
\end{definition}
\begin{problem}
    Prove that \(\displaystyle{\lim_{x\to 3}(4x-5)=7}\).
\end{problem}
\begin{solution}
    Let \(\epsilon>0\) be given, we want to find a number \(\delta>0\) such
    that
    \[0<|x-3|<\delta \implies|(4x-5)-7|<\epsilon\]
    We simplify to get \(|(4x-5)-7|=|4x-12|=4|x-3|\) so we have
    \[4|x-3|<\epsilon\iff|x-3|<\frac{\epsilon}{4}\]
    Let \(\delta=\epsilon/4\), we have
    \[0<|x-3|<\frac{\epsilon}{4}\implies4|x-3|<\epsilon\implies
    |(4x-5)-7|<\epsilon\]
    Therefore, by the definition of a limit, it is proved that
    \(\lim_{x\to 3}(4x-5)=7\).
\end{solution}
\begin{problem}
    Prove that \(\displaystyle{\lim_{x\to 3}x^2=9}\).
\end{problem}
\begin{solution}
    Let \(\epsilon>0\) be given, we want to find a number \(\delta>0\) such
    that
    \[0<|x-3|<\delta\implies|x^2-9|<\epsilon\]
    We simplify to get \[|x^2-9|=|x+3|\,|x-3|<\epsilon\]
    Let \(C\) be a positive constant such that
    \[|x+3|\,|x-3|<C\,|x-3|<\epsilon \iff |x-3|<\frac{\epsilon}{C}\]
    Since we are interested only in values of \(x\) that are close to 3, it is
    reasonable to assume that \(|x-3|<1\) such that \(|x+3|<7\) so \(C=7\).
    Let \(\delta=\min\{1,\epsilon/7\}\), we have
    \begin{align*}
        0<|x-3|<1 \iff |x+3| &< 7 \\
        0<|x-3|<\frac{\epsilon}{7} \iff 7\,|x-3| &< \epsilon \\
        |x+3|\,|x-3|<7\,|x-3|<\epsilon\implies|x^2-9| &< \epsilon
    \end{align*}
    Therefore, it is proved that \(\lim_{x\to 3}x^2=9\).
\end{solution}
\begin{problem}
    Prove that \(\displaystyle{\lim_{x\to 0^+}\sqrt{x}=0}\).
\end{problem}
\begin{solution}
    Let \(\epsilon>0\) be given, we want to find a number \(\delta>0\) such
    that
    \[0<x<\delta\implies|\sqrt{x}-0|<\epsilon\]
    We simplify to get \(\sqrt{x}<\epsilon\iff x<\epsilon^2\).
    Let \(\delta=\epsilon^2\), we have
    \[0<x<\epsilon^2\implies|\sqrt{x}-0|<\epsilon\]
    It is proved that \(\lim_{x\to 0^+}\sqrt{x}=0\).
\end{solution}
\subsection{Calculating Limits Using the Limit Laws}

We have the following properties of limits called the \textbf{limit laws} to
calculate limits.
Suppose that \(c\) is a constant and the limits
\begin{align*}
    \lim_{x\to a}f(x) &= L & \lim_{x\to a}g(x) &= M
\end{align*}
exist.
Then
\begin{enumerate}
    \item Sum Law: \(\displaystyle{\lim_{x\to a}\bigl[f(x)+g(x)\bigr]=L+M}\)
    \item Difference Law: \(\displaystyle{\lim_{x\to a}\bigl[f(x)-g(x)\bigr]=L-M}\)
    \item Constant Multiple Law: \(\displaystyle{\lim_{x\to a}\bigl[cf(x)\bigr]=cL}\)
    \item Product Law: \(\displaystyle{\lim_{x\to a}\bigl[f(x)g(x)\bigr]=LM}\)
    \item Quotient Law:
    \(\displaystyle{\lim_{x\to a}\frac{f(x)}{g(x)}=\frac{L}{M}}\) if
    \(M\neq 0\).
    \item Power Law: \(\displaystyle{\lim_{x\to a}\bigl[f(x)\bigr]^n
    =\Bigl[\lim_{x\to a}f(x)\Bigr]^n}\) where \(n\) is a positive integer.
    \item Root Law: \(\displaystyle{\lim_{x\to a}\sqrt[n]{f(x)}
    =\sqrt[n]{\lim_{x\to a}f(x)}}\) where \(n\) is a positive integer.
    If \(n\) is even,
    we assume that \(\displaystyle{\lim_{x\to a}f(x)>0}\).
    \item \(\displaystyle{\lim_{x\to a}c=c}\)
    \item \(\displaystyle{\lim_{x\to a}x=a}\)
    \item \(\displaystyle{\lim_{x\to a}x^n=a^n}\) where \(n\) is a positive
    integer.
    \item \(\displaystyle{\lim_{x\to a}\sqrt[n]{x}=\sqrt[n]{a}}\) where \(n\)
    is a positive integer.
    If \(n\) is even,
    we assume that \(a>0\).
\end{enumerate}
\begin{proof}
    Proof of limit law 8:
    Let \(\epsilon>0\) be given,
    we want to find a number \(\delta>0\) such that
    \[0<|x-a|<\delta\implies|c-c|<\epsilon\]
    We have \(|c-c|=0<\epsilon\) so the trivial inequality is always true for
    any number \(\delta>0\).
    It is proved that \(\displaystyle{\lim_{x\to a}c=c}\).
\end{proof}
\begin{proof}
    Proof of limit law 9:
    Let \(\epsilon>0\) be given,
    we want to find a number \(\delta>0\) such
    that \[0<|x-a|<\delta\implies|x-a|<\epsilon\]
    Let \(\delta=\epsilon\),
    we have
    \[0<|x-a|<\delta=\epsilon\implies|x-a|<\epsilon\]
    It is proved that \(\displaystyle{\lim_{x\to a}x=a}\).
\end{proof}
\begin{proof}
    Proof of the sum law:
    Let \(\epsilon>0\) be given,
    we want to find a number \(\delta>0\) such that
    \[0<|x-a|<\delta\implies|f(x)+g(x)-(L+M)|<\epsilon\]
    By the triangle inequality,
    \[|f(x)+g(x)-(L+M)|=|f(x)-L+g(x)-M|\leq|f(x)-L|+|g(x)-M|\]
    We make \(|f(x)-L|+|g(x)-M|\) less than \(\epsilon\) by making each of the
    terms \(|f(x)-L|\) and \(|g(x)-M|\) less than \(\epsilon/2\).
    Since \(\displaystyle{\lim_{x\to a}f(x)=L}\),
    there is a number \(\delta_1>0\) such that
    \[0<|x-a|<\delta_1\implies|f(x)-L|<\frac{\epsilon}{2}\]
    Similarly,
    there is a number \(\delta_2>0\) such that
    \[0<|x-a|<\delta_2\implies|g(x)-M|<\frac{\epsilon}{2}\]
    Let \(\delta=\min\{\delta_1,\delta_2\}\).
    If \(0<|x-a|<\delta\),
    then \(0<|x-a|<\delta_1\) and \(0<|x-a|<\delta_2\)
    and so
    \[|f(x)-L|+|g(x)-M|<\frac{\epsilon}{2}+\frac{\epsilon}{2}=\epsilon\]
    Then
    \[0<|x-a|<\delta\implies|f(x)+g(x)-(L+M)|<\epsilon\]
    Thus,
    by the definition of a limit,
    \[\lim_{x\to a}\bigl[f(x)+g(x)\bigr]=L+M\]
\end{proof}
\begin{proof}
    Proof of the product law:
    Let \(\epsilon>0\) be given,
    we want to find a number \(\delta>0\) such
    that \[0<|x-a|<\delta\implies|f(x)g(x)-LM|<\epsilon\]
    In order to get terms that contain \(|f(x)-L|\) and \(|g(x)-M|\),
    we add
    and subtract \(Lg(x)\) as follows and use the triangle inequality:
    \begin{align*}
        |f(x)g(x)-LM| &= |f(x)g(x)-Lg(x)+Lg(x)-LM| \\
        &= \left|\bigl[f(x)-L\bigr]g(x)+L[g(x)-M]\right| \\
        &\leq \left|\bigl[f(x)-L\bigr]g(x)\right|
        +\left|L\bigl[g(x)-M\bigr]\right| \\
        &= |f(x)-L|\,|g(x)|+|L|\,|g(x)-M|
    \end{align*}
    We want to make both of the terms less than \(\epsilon/2\).
    Since \(\displaystyle{\lim_{x\to a}g(x)=M}\),
    there is a number \(\delta_1>0\) such that
    \[0<|x-a|<\delta_1\implies|g(x)-M|<\frac{\epsilon}{2(1+|L|)}\]
    Also,
    there is a number \(\delta_2>0\) such that
    \[0<|x-a|<\delta_2\implies|g(x)-M|<1\] and therefore
    \[|g(x)|=|g(x)-M+M|\leq|g(x)-M|+|M|<1+|M|\]
    Since \(\displaystyle{\lim_{x\to a}f(x)=L}\),
    there is a number \(\delta_3>0\) such that
    \[0<|x-a|<\delta_3\implies|f(x)-L|<\frac{\epsilon}{2(1+|M|)}\]
    Let \(\delta=\min\{\delta_1,\delta_2,\delta_3\}\).
    If \(0<|x-a|<\delta\),
    then \(0<|x-a|<\delta_1\),
    \(0<|x-a|<\delta_2\),
    and \(0<|x-a|<\delta_3\).
    Then we can combine the inequalities to get
    \begin{align*}
        |f(x)g(x)-LM|
        &\leq |f(x)-L|\,|g(x)|+|L|\,|g(x)-M| \\
        &< \frac{\epsilon}{2(1+|M|)}(1+|M|)+|L|\frac{\epsilon}{2(1+|L|)} \\
        &< \frac{\epsilon}{2(1+|M|)}(1+|M|)
        +(1+|L|)\frac{\epsilon}{2(1+|L|)} \\
        &= \frac{\epsilon}{2}+\frac{\epsilon}{2}=\epsilon
    \end{align*}
    This shows that
    \[\lim_{x\to a}\bigl[f(x)g(x)\bigr]=LM\]
\end{proof}
\begin{proof}
    Proof of the constant multiple law:
    If we take \(g(x)=c\) then by the product law and limit law 8,
    we get
    \[\lim_{x\to a}\bigl[cf(x)\bigr]=\lim_{x\to a}c\cdot\lim_{x\to a}f(x)
    =c\lim_{x\to a}f(x)=cL\]
    We can prove the constant multiple law using the precise the definition.
    Note that if \(c=0\),
    then \(cf(x)=0\) and we have
    \[\lim_{x\to a}\bigl[0\cdot f(x)\bigr]=\lim_{x\to a}0=0
    =0\cdot\lim_{x\to a}f(x)=0\cdot L\]
    Let \(\epsilon>0\) and \(c\neq0\) be given,
    we want to find a number \(\delta>0\) such that
    \[0<|x-a|<\delta\implies|cf(x)-cL|<\epsilon\]
    We simplify to get
    \[|f(x)-L|<\frac{\epsilon}{|c|}\]
    By the definition of a limit,
    there is a number \(\delta_1>0\) such that
    \[0<|x-a|<\delta_1\implies|f(x)-L|<\frac{\epsilon}{|c|}\]
    Let \(\delta=\delta_1\),
    we have
    \[0<|x-a|<\delta\implies|cf(x)-cL|<\epsilon\]
    This shows that \(\displaystyle{\lim_{x\to a}\bigl[cf(x)\bigr]=cL}\).
\end{proof}
\begin{proof}
    Proof of the difference law:
    Using the sum law and the constant multiple law with \(c=-1\),
    we have
    \begin{align*}
        \lim_{x\to a}\bigl[f(x)-g(x)\bigr]
        &= \lim_{x\to a}\bigl[f(x)+(-1)g(x)\bigr]
        =\lim_{x\to a}f(x)+\lim_{x\to a}(-1)g(x) \\
        &=\lim_{x\to a}f(x)+(-1)\lim_{x\to a}g(x)
        =\lim_{x\to a}f(x)-\lim_{x\to a}g(x)=L-M
    \end{align*}
\end{proof}
\begin{proof}
    Proof of the quotient law:
    First we prove that \[\lim_{x\to a}\frac{1}{g(x)}=\frac{1}{M}\]
    Let \(\epsilon>0\) be given,
    we want to find a number \(\delta>0\) such that
    \[0<|x-a|<\delta\implies\left|\frac{1}{g(x)}-\frac{1}{M}\right|<\epsilon\]
    Observe that \[\left|\frac{1}{g(x)}-\frac{1}{M}\right|
    =\frac{|M-g(x)|}{|Mg(x)|}=\frac{|g(x)-M|}{|Mg(x)|}\]
    Since \(\displaystyle{\lim_{x\to a}g(x)=M}\),
    there is a number \(\delta_1\) such that
    \[0<|x-a|<\delta_1\implies|g(x)-M|<\frac{|M|}{2}\] and therefore
    \[|M|=|M-g(x)+g(x)|\leq|M-g(x)|+|g(x)|=|g(x)-M|+|g(x)|
    <\frac{|M|}{2}+|g(x)|\]
    This shows that
    \[0<|x-a|<\delta_1\implies\frac{|M|}{2}<|g(x)|\iff\frac{1}{|g(x)|}
    <\frac{2}{|M|}\]
    and so,
    for these values of \(x\),
    \[\frac{1}{|Mg(x)|}=\frac{1}{|M|\,|g(x)|}<\frac{1}{|M|}\cdot\frac{2}{|M|}
    =\frac{2}{M^2}\]
    Also,
    there is a number \(\delta_2>0\) such that
    \[0<|x-a|<\delta_2\implies|g(x)-M|<\frac{M^2}{2}\epsilon\]
    Let \(\delta=\min\{\delta_1,\delta_2\}\).
    If \(0<|x-a|<\delta\),
    then
    \[\left|\frac{1}{g(x)}-\frac{1}{M}\right|=\frac{|M-g(x)|}{|Mg(x)|}
    =\frac{1}{|Mg(x)|}|g(x)-M|<\frac{2}{M^2}\frac{M^2}{2}\epsilon=\epsilon\]
    which is what we want to show.
    We apply the product law to prove the quotient law
    \[\lim_{x\to a}\frac{f(x)}{g(x)}
    =\lim_{x\to a}\left(f(x)\cdot\frac{1}{g(x)}\right)
    =\lim_{x\to a}f(x)\lim_{x\to a}\frac{1}{g(x)}=L\cdot\frac{1}{M}
    =\frac{L}{M}\]
\end{proof}
We have the following \textbf{direct substitution property} to calculate
limits.
If \(f\) is a polynomial or a rational function and \(a\) is in the domain of
\(f\),
then
\[\lim_{x\to a}f(x)=f(a)\]
\begin{problem}
    Find \(\displaystyle{\lim_{x\to 1}\frac{x^2-1}{x-1}}\).
\end{problem}
\begin{solution}
    \[\lim_{x\to 1}\frac{x^2-1}{x-1}
    =\lim_{x\to 1}\frac{(x+1)(x-1)}{x-1}
    =\lim_{x\to 1}(x+1)=1+1=2\]
\end{solution}
If \(f(x)=g(x)\) when \(x\neq a\),
then \(\displaystyle{\lim_{x\to a}f(x)=\lim_{x\to a}g(x)}\),
provided that this limit exists.
When computing one-sided limits,
we use the fact that the limit laws also hold for one-sided limits.
\begin{problem}
    Show that \(\displaystyle{\lim_{x\to 0}|x|=0}\).
\end{problem}
\begin{solution}
    Since \(|x|=x\) for \(x>0\),
    we have
    \[\lim_{x\to 0^+}|x|=\lim_{x\to 0^+}x=0\]
    For \(x<0\) we have \(|x|=-x\) so
    \[\lim_{x\to 0^-}|x|=\lim_{x\to 0^-}(-x)=0\]
    Therefore,
    it is shown that \(\displaystyle{\lim_{x\to 0}|x|=0}\).
\end{solution}

\subsubsection*{The Squeeze Theorem}
\begin{theorem}
    If \(f(x)\leq g(x)\) for all \(x\) in an open interval that contains
    \(a\) (except possibly at \(a\)) and
    \begin{align*}
        \lim_{x\to a}f(x) &= L & \lim_{x\to a}g(x) &= M
    \end{align*}
    then \(L\leq M\).
\end{theorem}
\begin{proof}
    We use the method of proof by contradiction.
    Suppose  that \(L>M\),
    then we have
    \[\lim_{x\to a}\bigl[g(x)-f(x)\bigr]=M-L\]
    Therefore,
    for any number \(\epsilon>0\),
    there is a number \(\delta>0\) such that
    \[0<|x-a|<\delta\implies
    \left|\bigl[g(x)-f(x)\bigr]-(M-L)\right|<\epsilon\]
    Note that \(L-M>0\) by the hypothesis.
    Let \(\epsilon=L-M\),
    there is a number \(\delta>0\) such that
    \[0<|x-a|<\delta\implies\left|\bigl[g(x)-f(x)\bigr]-(M-L)\right|<L-M\]
    Since \(a\leq|a|\) for any number \(a\),
    we have
    \[0<|x-a|<\delta\implies \bigl[g(x)-f(x)\bigr]-(M-L)<L-M\]
    which simplifies to
    \[0<|x-a|<\delta\implies g(x)<f(x)\]
    But this contradicts \(f(x)\leq g(x)\).
    Thus the inequality \(L>M\) must be false.
    Therefore \(L\leq M\).
\end{proof}
\begin{theorem}[Squeeze Theorem]
    If \(f(x)\leq g(x)\leq h(x)\) for all \(x\) in an open interval that
    contains \(a\) (except possibly at \(a\)) and
    \[\lim_{x\to a}f(x)=\lim_{x\to a}h(x)=L\]
    then
    \[\lim_{x\to a}g(x)=L\]
\end{theorem}
\begin{proof}
    Let \(\epsilon>0\) be given.
    Since \(\displaystyle{\lim_{x\to a}f(x)=L}\),
    there is a number \(\delta_1>0\) such that
    \[0<|x-a|<\delta_1\implies|f(x)-L|<\epsilon\implies L-\epsilon<f(x)
    <L+\epsilon\]
    Since \(\displaystyle{\lim_{x\to a}h(x)=L}\),
    there is a number \(\delta_2>0\) such that
    \[0<|x-a|<\delta_2\implies|h(x)-L|<\epsilon\implies L-\epsilon<h(x)
    <L+\epsilon\]
    Let \(\delta=\min\{\delta_1,\delta_2\}\).
    If \(0<|x-a|<\delta\),
    then \(0<|x-a|<\delta_1\) and \(0<|x-a|<\delta_2\),
    so
    \[L-\epsilon<f(x)\leq g(x)\leq h(x)<L+\epsilon\]
    In particular,
    \[L-\epsilon<g(x)<L+\epsilon\]
    and so \(|g(x)-L|<\epsilon\).
    Therefore \(\displaystyle{\lim_{x\to a}g(x)=L}\).
\end{proof}
\begin{problem}
    Show that
    \(\displaystyle{\lim_{x\to 0}x^2\sin\frac{1}{x}=0}\).
\end{problem}
\begin{solution}
    Since
    \[-1\leq\sin\frac{1}{x}\leq 1\]
    then
    \[-x^2\leq x^2\sin\frac{1}{x}\leq x^2\]
    We know that
    \begin{align*}
        \lim_{x\to 0}(-x^2) &= 0 & \lim_{x\to 0}x^2 &= 0
    \end{align*}
    By the squeeze theorem,
    \[\lim_{x\to 0}x^2\sin\frac{1}{x}=0\]
\end{solution}

Note the approximate value of \(\pi\) is \(\pi\approx 3.14159\).
If \(0<\theta<\pi/2\),
then \[\sin\theta<\theta\implies\frac{\sin\theta}{\theta}<1\]
and \(\theta\leq\tan\theta\).
Therefore we have
\[\theta<\tan\theta=\frac{\sin\theta}{\cos\theta}
\implies\cos\theta<\frac{\sin\theta}{\theta}<1\]
We know that \(\displaystyle{\lim_{\theta\to 0}1=1}\) and
\(\displaystyle{\lim_{\theta\to 0}\cos\theta=1}\),
so by the squeeze theorem,
we have
\[\lim_{\theta\to 0^+}\frac{\sin\theta}{\theta}=1\]
But the function \((\sin\theta)/\theta\) is an even function,
so its left and right limits must be equal.
Hence we have
\[\lim_{\theta\to 0}\frac{\sin\theta}{\theta}=1\]
\begin{problem}
    Find \(\displaystyle{\lim_{x\to 0}\frac{\sin 7x}{4x}}\).
\end{problem}
\begin{solution}
    \[\lim_{x\to 0}\frac{\sin 7x}{4x}
    =\lim_{x\to 0}\frac{7x\cdot\sin 7x}{4x\cdot7x}
    =\frac{7}{4}\lim_{x\to 0}\frac{\sin 7x}{7x}=\frac{7}{4}\]
\end{solution}
\begin{problem}
    Evaluate \(\displaystyle{\lim_{\theta\to 0}\frac{\cos\theta-1}{\theta}}\).
\end{problem}
\begin{solution}
    We have
    \[\frac{\cos\theta-1}{\theta}
    =\frac{\cos\theta-1}{\theta}\left(\frac{\cos\theta+1}{\cos\theta+1}\right)
    =\frac{\cos^2\theta-1}{\theta(\cos\theta+1)}
    =\frac{-\sin^2\theta}{\theta(\cos\theta+1)}
    =\frac{\sin\theta}{\theta}\left(\frac{-\sin\theta}{\cos\theta+1}\right)\]
    We take the limit then
    \begin{align*}
        \lim_{\theta\to 0}\frac{\cos\theta-1}{\theta}
        &= \lim_{\theta\to 0}\left(\frac{\cos\theta-1}{\theta}
        \cdot\frac{\cos\theta+1}{\cos\theta+1}\right)
        = \lim_{\theta\to 0}\frac{\cos^2\theta-1}{\theta(\cos\theta+1)}
        = \lim_{\theta\to 0}\frac{-\sin^2\theta}{\theta(\cos\theta+1)} \\
        &= -\lim_{\theta\to 0}\left(\frac{\sin\theta}{\theta}
        \cdot\frac{\sin\theta}{\cos\theta+1}\right)
        =-\lim_{\theta\to 0}\frac{\sin\theta}{\theta}
        \cdot\lim_{\theta\to 0}\frac{\sin\theta}{\cos\theta+1}
        = -1\cdot\frac{0}{1+1}=0
    \end{align*}
\end{solution}
\subsection{Continuity}

\subsubsection*{Continuous Functions}
Let \(f(x)\) be a function and the number \(a\) is in the domain of \(f\) so
\(f(a)\) is defined.
If the limit exists, then we have the following definition.
\begin{definition}
    A function \(f\) is \textbf{continuous} at the number \(a\) if
    \[\lim_{x\to a}f(x)=f(a)\]
\end{definition}
A function \(f\) is continuous from the left at \(a\) if the left-hand limit
equals \(f(a)\) and it is continuous from the right at \(a\) if the right-hand
limit equals \(f(a)\).
A function \(f\) is continuous on an interval if it is continuous
at every number in the interval.
If \(f\) is not continuous at \(a\),
then it is a discontinuous function at \(a\).
If \(f\) and \(g\) are continuous functions at \(a\) and \(c\) is a constant,
then the following functions are also continuous at \(a\).
    \begin{align*}
        &f+g & f-g && cf && f\cdot g && \dfrac{f}{g}\iff g(x)\neq 0 &
    \end{align*}
\begin{theorem}
    Let \(P(x)\) be any polynomial, then \(P(x)\) is continuous on
    \(\R=(-\infty,\infty)\).
\end{theorem}
\begin{proof}
    A polynomial \(P(x)\) is a function of the form
    \[a_nx^n+a_{n-1}x^{n-1}+\dotsb +a_{2}x^{2}+a_1x+a_0\]
    where the coefficients \(a_i\) are constants.
    \(P(x)\) is the sum of power functions with a constant multiple and
    therefore it is continuous.
\end{proof}
\begin{theorem}
    Let \(f\) be any rational function,
    then \(f\) is continuous on its domain.
\end{theorem}
\begin{proof}
    A rational function \(f\) is a function of the form
    \[f(x)=\frac{P(x)}{Q(x)}\] where \(P\) and \(Q\) are polynomials.
    We know that polynomials are continuous so a rational function is
    continuous on its domain.
\end{proof}
Polynomials, rational functions, root functions, trigonometric
functions, inverse trigonometric functions, logarithmic functions, and
exponential functions are continuous on their domain.
\begin{theorem}
    If \(f\) is continuous at \(b\) and \(\lim_{x\to a}g(x)=b\), then
    the limit of the composite function \(f\circ g\) is
    \[\lim_{x\to a}f(g(x))=f\left(\lim_{x\to a}g(x)\right)=f(b)\]
\end{theorem}
\begin{proof}
    Let \(\epsilon>0\) be given, we want to find \(\delta>0\) such that
    \[0<|x-a|<\delta\implies|f(g(x))-f(b)|<\epsilon\]
    Since \(f\) is continuous at \(b\),
    then we have \(\lim_{y\to b}f(y)=f(b)\).
    There exists \(\delta_1>0\) such that
    \[0<|y-b|<\delta_1\implies|f(y)-f(b)|<\epsilon\]
    Since \(\lim_{x\to a}g(x)=b\), there exists \(\delta>0\) such that
    \[0<|x-a|<\delta\implies|g(x)-b|<\delta_1\implies|f(g(x))-f(b)|<\epsilon\]
    Therefore, it is proved that
    \[\lim_{x\to a}f(g(x))=f\left(\lim_{x\to a}g(x)\right)=f(b)\]
\end{proof}
\begin{theorem}
    If \(g\) is continuous at \(a\) and \(f\) is continuous at \(g(a)\), then
    \(f\circ g\) is continuous at \(a\).
\end{theorem}
\begin{proof}
    Since \(g\) is continuous at \(a\), we have \(\lim_{x\to a}g(x)=g(a)\).
    Since \(f\) is continuous at \(g(a)\), we have
    \[\lim_{x\to a}f(g(x))=f(g(a))\]
    Therefore, \(f(g(x))\) is continuous at \(a\).
\end{proof}
An important property of continuous functions is formulated by the following
theorem proved by \textbf{Bernard Bolzano} (1781--1848).
\begin{theorem}[Intermediate Value Theorem]
    Suppose that \(f\) is continuous on the closed interval \([a,b]\) and let
    \(N\) be any number between \(f(a)\) and \(f(b)\) where \(f(a)\neq f(b)\)
    such that \[\min\{f(a),f(b)\}<N<\max\{f(a),f(b)\}\]
    Then there exists a number \(c\) in the open interval \((a,b)\) such that
    \(f(c)=N\).
\end{theorem}
If a continuous function \(f(x)\) has values of opposite sign in an interval
\((a,b)\), then there exists a root of \(f(x)\) in \((a,b)\) which follows
immediately from the intermediate value theorem.
\subsection{Limits Involving Infinity}

\subsubsection*{Infinite Limits}
\begin{definition}
    The notation
    \[\lim_{x\to a}f(x)=\infty\]
    means that the values of \(f(x)\) can be made arbitrarily large by taking
    \(x\) sufficiently close to \(a\) but \(x\neq a\).
\end{definition}
Another notation for the limit is \(f(x)\to\infty\) as \(x\to a\).
We say that the limit of \(f(x)\),
as \(x\) approaches \(a\),
is infinity.
\begin{definition}
    \[\lim_{x\to a}f(x)=-\infty\]
    means that the values of \(f(x)\) can be made arbitrarily large negative
    by taking \(x\) sufficiently close to \(a\) but \(x\neq a\).
\end{definition}
We say that the limit of \(f(x)\),
as \(x\) approaches \(a\),
is negative infinity.
Similar definitions can be given for the one-sided infinite limits
\begin{align*}
    \lim_{x\to a^-}f(x) &= \infty & \lim_{x\to a^+}f(x) &= \infty \\
    \lim_{x\to a^-}f(x) &= -\infty & \lim_{x\to a^+}f(x) &= -\infty
\end{align*}
\begin{definition}
    The vertical line \(x=a\) is called a \textbf{vertical asymptote} of the
    curve \(y=f(x)\) if at least one of the following statements is true:
    \begin{align*}
        \lim_{x\to a}f(x) &= \infty & \lim_{x\to a^-}f(x) &= \infty
        & \lim_{x\to a^+}f(x) &= \infty \\
        \lim_{x\to a}f(x) &= -\infty & \lim_{x\to a^-}f(x) &= -\infty
        & \lim_{x\to a^+}f(x) &= -\infty
    \end{align*}
\end{definition}

\subsubsection*{Limits at Infinity}
\begin{definition}
    Let \(f\) be a function defined on some interval \((a,\infty)\).
    Then
    \[\lim_{x\to\infty}f(x)=L\]
    means that the values of \(f(x)\) can be made arbitrarily close to \(L\)
    by requiring \(x\) to be sufficiently large.
\end{definition}
Another notation is \(f(x)\to L\) as \(x\to\infty\).
We say that the limit of \(f(x)\),
as \(x\) approaches infinity,
is \(L\).
\begin{definition}
    Let \(f\) be a function defined on some interval \((-\infty,a)\).
    Then
    \[\lim_{x\to-\infty}f(x)=L\]
    means that the values of \(f(x)\) can be made arbitrarily close to \(L\) by requiring
    \(x\) to be sufficiently large negative.
\end{definition}
We say that the limit of \(f(x)\),
as \(x\) approaches negative infinity,
is \(L\).
\begin{definition}
    The line \(y=L\) is called a \textbf{horizontal asymptote} of the curve
    \(y=f(x)\) if either
    \[\lim_{x\to\infty}f(x)=L\]
    or
    \[\lim_{x\to-\infty}f(x)=L\]
\end{definition}
If \(n\) is a positive integer,
then
\begin{align*}
    \lim_{x\to\infty}\frac{1}{x^n} &= 0 & \lim_{x\to-\infty}\frac{1}{x^n} &= 0
\end{align*}
\begin{problem}
    Evaluate \(\displaystyle{\lim_{x\to\infty}\sin x}\).
\end{problem}
\begin{solution}
    As \(x\) increases,
    the values of \(sin x\) oscillate between 1 and \(-1\) infinitely often.
    Thus \(\displaystyle{\lim_{x\to\infty}\sin x}\) does not exist.
\end{solution}

\subsubsection*{Infinite Limits at Infinity}
The notation
\[\lim_{x\to\infty}f(x)=\infty\]
is used to indicate that the values of \(f(x)\) become large as \(x\) becomes
large.
Similar meanings are attached to the following symbols:
\begin{align*}
    \lim_{x\to\infty}f(x) &= \infty & \lim_{x\to\infty}f(x) &= -\infty
    & \lim_{x\to-\infty}f(x) &= -\infty
\end{align*}

\subsubsection*{Precise Definitions}
Let \(f\) be a function defined on some open interval that contains the number
\(a\),
except possibly at \(a\) itself.
\begin{definition}
    \[\lim_{x\to a}f(x)=\infty\]
    means that for every positive number \(M\) there is a positive number
    \(\delta\) such that
    \[0<|x-a|<\delta\implies f(x)>M\]
\end{definition}
\begin{problem}
    Prove that \(\displaystyle{\lim_{x\to 0}\frac{1}{x^2}=\infty}\).
\end{problem}
\begin{solution}
    Let \(M\) be a given positive number
    We want to find a number \(\delta\) such that
    \[0<|x|<\delta\implies\frac{1}{x^2}>M\]
    But
    \[\frac{1}{x^2}>M\iff x^2<\frac{1}{M}\iff\sqrt{x^2}<\sqrt{\frac{1}{M}}
    \iff|x|<\frac{1}{\sqrt{M}}\]
    Let \(\delta=1/\sqrt{M}\),
    then
    \[0<|x|<\delta=\frac{1}{\sqrt{M}}\implies\frac{1}{x^2}>M\]
    This shows that
    \(\displaystyle{\lim_{x\to 0}\frac{1}{x^2}=\infty}\).
\end{solution}
\begin{definition}
    \[\lim_{x\to a}f(x)=-\infty\]
    if for every negative number \(N\) there is a positive number \(\delta\)
    such that
    \[0<|x-a|<\delta\implies f(x)<N\]
\end{definition}

\begin{definition}
    Let \(f\) be a function defined on some interval \((a,\infty)\).
    Then
    \[\lim_{x\to\infty}f(x)=L\]
    means that for every \(\epsilon>0\) there is a corresponding number \(N\)
    such that
    \[x>N\implies|f(x)-L|<\epsilon\]
\end{definition}
\begin{definition}
    Let \(f\) be a function defined on some interval \((-\infty,a)\).
    Then
    \[\lim_{x\to-\infty}f(x)=L\]
    means that for every \(\epsilon>0\) there is a corresponding number \(N\)
    such that
    \[x<N\implies|f(x)-L|<\epsilon\]
\end{definition}
\begin{problem}
    Prove that \(\displaystyle{\lim_{x\to \infty}\frac{1}{x}=0}\).
\end{problem}
\begin{solution}
    Given \(\epsilon>0\),
    we want to find an \(N\) such that
    \[x>N\implies\left|\frac{1}{x}-0\right|<\epsilon\]
    Since \(x\to \infty\),
    we can that \(x>0\) in computing the limit.
    Then \(1/x<\epsilon\iff x>1/\epsilon\).
    Let \(N=1/\epsilon\),
    so
    \[x>N=\frac{1}{\epsilon}\implies\left|\frac{1}{x}-0\right|=\frac{1}{x}
    <\epsilon\]
    Therefore,
    by definition,
    \[\lim_{x\to \infty}\frac{1}{x}=0\]
\end{solution}
\begin{definition}
    Let \(f\) be a function defined on some interval \((a,\infty)\).
    Then
    \[\lim_{x\to\infty}f(x)=\infty\]
    means that for every positive number \(M\) there is a corresponding
    positive number \(N\) such that
    \[x>N\implies f(x)>M\]
\end{definition}
Similar definitions apply when the symbol \(\infty\) is replaced by \(-\infty\).

\section{Derivatives}

\subsection{Derivatives and Rates of Change}

\subsubsection*{Tangents}
\begin{definition}
    The \textbf{tangent line} of the curve \(y=f(x)\) at the point
    \(P(a, f(a))\) is the line through \(P\) with slope
    \[m=\lim_{x\to a}\frac{f(x)-f(a)}{x-a}\]
    provided that this limit exists.
\end{definition}
If \(h=x-a\),
then
\[m=\lim_{h\to 0}\frac{f(a+h)-f(a)}{h}\]
Let \(s=f(t)\) be a \textbf{position function} that describes the motion of an
object where \(s\) is the displacement of the object from the origin at time
\(t\).
In the time interval from \(t=a\) to \(t=a+h\) the change in position is
\(f(a+h)-f(a)\).
The average velocity over this time interval is
\[\text{average velocity}
=\frac{\text{displacement}}{\text{time}}
=\frac{f(a+h)-f(a)}{h}\]
The \textbf{velocity} (or \textbf{instantaneous velocity}) of the object at
time \(t=a\) is the limit of the average velocities:
\[v(a)=\lim_{h\to 0}\frac{f(a+h)-f(a)}{h}\]

\subsubsection*{Derivatives}
\begin{definition}
    The \textbf{derivative of a function} \(f\) \textbf{at a number} \(a\) is
    \[f'(a)=\lim_{h\to 0}\frac{f(a+h)-f(a)}{h}\]
    if this limit exists.
\end{definition}
The tangent line to \(y=f(x)\) at the point \((a,f(a))\) is the line through
\((a,f(a))\) whose slope is equal to \(f'(a)\),
the derivative of \(f\) at \(a\).
The equation of the tangent line in point-slope form is
\[y-f(a)=f'(a)(x-a)\]

\subsubsection*{Rates of Change}
Suppose \(y\) is a quantity that depends on another quantity \(x\).
Thus \(y\) is a function of \(x\) and we write \(y=f(x)\).
If \(x\) changes from \(x_1\) to \(x_2\),
then the change in \(x\) (also called the \textbf{increment} of \(x\)) is
\[\Delta x=x_2-x_1\]
and the corresponding change in \(y\) is
\[\Delta y=f(x_2)-f(x_1)\]
The difference quotient
\[\frac{\Delta y}{\Delta x}=\frac{f(x_2)-f(x_1)}{x_2-x_1}\]
is called the
\textbf{average rate of change of \textit{y} with respect to \textit{x}} over
the interval \([x_1,x_2]\).
The limit of these average rates of change is called the
(\textbf{instantaneous})
\textbf{rate of change of \textit{y} with respect to \textit{x}}
at \(x=x_1\):
\[\text{instantaneous rate of change}
=\lim_{\Delta x\to 0}\frac{\Delta y}{\Delta x}
=\lim_{x_1\to x_2}\frac{f(x_2)-f(x_1)}{x_2-x_1}\]
We recognize this limit as being the derivative \(f'(x_1)\).
The derivative \(f'(a)\) is the instantaneous rate of change of \(y=f(x)\)
with respect to \(x\) when \(x=a\).
If \(s=f(t)\) is a position function of a particle,
then \(f'(a)\) is the rate of change of the displacement \(s\) with respect to
time \(t\).
\(f'(a)\) is the velocity of the particle at time \(t=a\).
The \textbf{speed} of the particle is \(|f'(a)|\),
the absolute value of the velocity.
\subsection{The Derivative as a Function}

\subsubsection*{The Derivative Function}
\begin{definition}
    The \textbf{derivative of a function \(f(x)\)} is the function
    \[f'(x)=\lim_{h\to 0}\frac{f(x+h)-f(x)}{h}\]
\end{definition}
\begin{problem}
    Find the derivative of \(f(x)=\sqrt{x}\).
\end{problem}
\begin{solution}
    \begin{align*}
        f'(x) &= \lim_{h\to 0}\frac{\sqrt{x+h}-\sqrt{x}}{h}=\lim_{h\to 0}
        \left(\frac{\sqrt{x+h}-\sqrt{x}}{h}\cdot
        \frac{\sqrt{x+h}+\sqrt{x}}{\sqrt{x+h}+\sqrt{x}}\right) \\
        &= \lim_{h\to 0}\frac{x+h-x}{h(\sqrt{x+h}+\sqrt{x})}
        =\lim_{h\to 0}\frac{1}{\sqrt{x+h}+\sqrt{x}}
        =\frac{1}{\sqrt{x}+\sqrt{x}}=\frac{1}{2\sqrt{x}} 
    \end{align*}
\end{solution}

\subsubsection*{Notations}
The following notations of the derivative of \(y=f(x)\) with respect to \(x\)
are equivalent:
\[y'=f'(x)=\frac{dy}{dx}=\frac{d}{dx}f(x)=D_x f(x)\]
The symbols \(d/dx\) and \(D_x\) are called \textbf{differential operators}
because they indicate the operation of \textbf{differentiation},
which is the process of calculating a derivative.
The symbol \(dy/dx\) is callled the Leibniz notation.
We can rewrite the definition of the derivative in Leibniz notation in the
form
\[\frac{dy}{dx}=\lim_{\Delta x\to0}\frac{\Delta y}{\Delta x}\]
The following notations for the value of the derivative of \(y=f(x)\)
evaluated at the number \(a\) are equivalent:
\[f'(a)=\left.\frac{dy}{dx}\right|_{x=a}=\left[\frac{dy}{dx}\right]_{x=a}\]


\subsubsection*{Differentiable Functions}
\begin{definition}
    A function \(f\) is \textbf{differentiable at \(a\)} if \(f'(a)\) exists.
    It is \textbf{differentiable on an open interval} if it is differentiable
    at every number in the interval.
\end{definition}
\begin{theorem}
    If \(f\) is differentiable at \(a\), then \(f\) is continuous at \(a\).
\end{theorem}
\begin{proof}
    Given that \(f\) is differentiable at \(a\), we want to show that
    \[\lim_{x\to a}f(x)=f(a)\]
    Since \(f'(a)\) exists, we have
    \[f'(a)=\lim_{x\to a}\frac{f(x)-f(a)}{x-a}\]
    Then we have
    \[\lim_{x\to a}[f(x)-f(a)]
    =\lim_{x\to a}\left(\frac{f(x)-f(a)}{x-a}(x-a)\right)
    =\lim_{x\to a}\left(\frac{f(x)-f(a)}{x-a}\right)\lim_{x\to a}(x-a)
    =f'(a)\cdot 0=0\]
    Then we have
    \[\lim_{x\to a}f(x)=\lim_{x\to a}(f(a)+f(x)-f(a))
    =\lim_{x\to a}f(a)+\lim_{x\to a}(f(x)-f(a))=f(a)+0=f(a)\]
    Therefore, it is proved that \(f\) is continuous at \(a\).
\end{proof}
Note that there are functions that are continuous but not differentiable.
The function \(y=|x|\) is continuous at 0 but not differentiable at 0 since
\[f'(0)=\lim_{h\to 0}\frac{|0+h|-|0|}{h}\] if the limit exists but
\begin{align*}
    \lim_{h\to 0^-}\frac{|0+h|-|0|}{h} &= \lim_{h\to 0^-}\frac{|h|}{h}
    =\lim_{h\to 0^-}\frac{-h}{h}=-1 \\
    \lim_{h\to 0^+}\frac{|0+h|-|0|}{h} &= \lim_{h\to 0^+}\frac{|h|}{h}
    =\lim_{h\to 0^+}\frac{h}{h}=1
\end{align*}
thus the limit does not exist so \(f'(0)\) does not exist.
If a function \(f\) has a sharp point at \(a\),
or \(f\) is not continuous at \(a\),
or the curve has a vertical tangent line when \(x=a\),
then \(f\) is not differentiable at \(a\).

\subsubsection*{Higher-Order Derivatives}
 If \(y=f(x)\) is a differentiable function and its derivative \(y'=f'(x)\) is
 also a differentiable function,
 then the \textbf{second derivative} of \(y=f(x)\) is
\[y''=f''(x)=\frac{d}{dx}\left(\frac{dy}{dx}\right)=\frac{d^2y}{dx^2}\]
We can interpret \(f''(x)\) as the slope of the curve \(y=f'(x)\) at the point
\((x,f'(x))\), which is the rate of change of the slope of the original curve
\(y=f(x)\).
Let \(s=s(t)\) be a position function of an object with respect to time \(t\).
The velocity function \(v(t)\) of the object is \[v(t)=s'(t)=\frac{ds}{dt}\]
The instantaneous rate of change of velocity is the acceleration.
Thus the acceleration function \(a(t)\) is the derivative of the velocity
function and is therefore the second derivative of the position function.
\[a(t)=v'(t)=\frac{dv}{dt}=s''(t)=\frac{d^2s}{dt^2}\]
In general, the \(n\)th derivative of \(y=f(x)\) is
\[f^{(n)}(x)=\frac{d^ny}{dx^n}\]
\begin{problem}
    Find the first and the second derivatives of \(f(x)=x^3\).
\end{problem}
\begin{solution}
    We apply the binomial theorem by Newton
    \[(x+y)^n=\sum_{k=0}^n \binom{n}{k}x^{n-k}y^k\]
    For the first derivative we have
    \begin{align*}
        f'(x) &= \lim_{h\to 0}\frac{(x+h)^3-x^3}{h}=\lim_{h\to 0}
        \frac{x^3+3x^2h+3xh^2+h^3-x^3}{h}
        =\lim_{h\to 0}\frac{3x^2h+3xh^2+h^3}{h} \\
        &= \lim_{h\to 0}(3x^2+3xh+h^2)=3x^2
    \end{align*}
    For the second derivative we have
    \begin{align*}
        f''(x) &= \lim_{h\to 0}\frac{3(x+h)^2-3x^2}{h}
        =\lim_{h\to 0}\frac{3(x^2+2hx+h^2-x^2)}{h}
        =\lim_{h\to 0}\frac{6hx+3h^2}{h} \\
        &= \lim_{h\to 0}(6x+3h)=6x 
    \end{align*}
\end{solution}
\subsection{Basic Differentiation Formulas}

The derivative of the constant function \(f(x)=c\) is
\[\frac{d}{dx}(c)=0\]
\begin{proof}
    If \(f(x)=c\) where \(c\) is a constant, then
    \[f'(x)=\lim_{h\to 0}\frac{f(x+h)-f(x)}{h}
    =\lim_{h\to 0}\frac{c-c}{h}=\lim_{h\to 0}0=0\]
\end{proof}

\subsubsection*{Power Functions}
\[\frac{d}{dx}(x)=1\]
\begin{proof}
    If \(f(x)=x\), then
    \[f'(x)=\lim_{h\to 0}\frac{f(x+h)-f(x)}{h}=\lim_{h\to 0}\frac{x+h-x}{h}
    =\lim_{h\to 0}\frac{h}{h}=\lim_{h\to 0}1=1\]
\end{proof}
The power rule:
If \(n\) is a positive integer, then
\[\frac{d}{dx}(x^n)=nx^{n-1}\]
\begin{proof}
    Let \(f(x)=x^n\), then
    \[f'(x)=\lim_{h\to 0}\frac{f(x+h)-f(x)}{h}
    =\lim_{h\to 0}\frac{(x+h)^n-x^n}{h}\]
    We use the binomial theorem to expand \((x+h)^n\) then
    \begin{align*}
        f'(x)
        &= \lim_{h\to 0}
        \frac{
            \left(x^n+nx^{n-1}h+\dfrac{n(n-1)}{2}x^{n-2}h^2+\cdots+nxh^{n-1}
            +h^n\right)-x^n}{h} \\
        &= \lim_{h\to 0}
        \frac{
            nx^{n-1}h+\dfrac{n(n-1)}{2}x^{n-2}h^2+\cdots+nxh^{n-1}+h^n}{h} \\
        &= \lim_{h\to 0}
        (nx^{n-1}+\frac{n(n-1)}{2}x^{n-2}h+\dots+nxh^{n-2}+h^{n-1}) \\
        &= nx^{n-1}
    \end{align*}
    because every term except the first has \(h\) as a factor and therefore
    approaches 0.
\end{proof}
The power rule (general version):
If \(n\) is any real number, then
\[\frac{d}{dx}(x^n)=nx^{n-1}\]
\begin{problem}
    Differentiate \(f(x)=1/x\).
\end{problem}
\begin{solution}
    \[\frac{d}{dx}\left(\frac{1}{x}\right)=\frac{d}{dx}x^{-1}=(-1)x^{-1-1}
    =-x^{-2}=-\frac{1}{x^2}\]
\end{solution}
The \textbf{normal line} to a curve \(C\) at a point \(P\) is the line through
\(P\) that is perpendicular to the tangent line at \(P\).
The constant multiple rule:
If \(c\) is a constant and \(f\) is a differentiable function, then
\[\frac{d}{dx}\bigl[cf(x)\bigr]=c\,\frac{d}{dx}f(x)\]
\begin{proof}
    Let \(g(x)=cf(x)\).
    Then
    \begin{align*}
        g'(x) &= \lim_{h\to 0}\frac{cf(x+h)-cf(x)}{h}
        =\lim_{h\to 0}c\left(\frac{f(x+h)-f(x)}{h}\right) \\
        &= c\lim_{h\to 0}\frac{f(x+h)-f(x)}{h}
        =c\,\frac{d}{dx}f(x)
    \end{align*}
\end{proof}
The sum rule: If \(f\) and \(g\) are both differentiable, then
\[\frac{d}{dx}\bigl[f(x)+g(x)\bigr]=\frac{d}{dx}f(x)+\frac{d}{dx}g(x)\]
\begin{proof}
    Let \(F(x)=f(x)+g(x)\).
    Then
    \begin{align*}
        F'(x) &= \lim_{h\to 0}
        \frac{\bigl[f(x+h)+g(x+h)\bigr]-\bigl[f(x)+g(x)\bigr]}{h} \\
        &= \lim_{h\to 0}
        \left[\frac{f(x+h)-f(x)}{h}+\frac{g(x+h)-g(x)}{h}\right] \\
        &= \lim_{h\to 0}\frac{f(x+h)-f(x)}{h}
        +\lim_{h\to 0}\frac{g(x+h)-g(x)}{h} \\
        &= \frac{d}{dx}f(x)+\frac{d}{dx}g(x)
    \end{align*}
\end{proof}
The sum rule can be extended to the sum of any number of functions.
By writing \(f-g\) as \(f+(-1)g\) and applying the sum rule and the constant
multiple rule, we get the following formula.
The difference rule: If \(f\) and \(g\) are both differentiable, then
\[\frac{d}{dx}\bigl[f(x)-g(x)\bigr]=\frac{d}{dx}f(x)-\frac{d}{dx}g(x)\]
\begin{proof}
    Let \(F(x)=f(x)-g(x)\).
    Then
    \begin{align*}
        F'(x) &= \lim_{h\to 0}
        \frac{\bigl[f(x+h)-g(x+h)\bigr]-\bigl[f(x)-g(x)\bigr]}{h} \\
        &= \lim_{h\to 0}\frac{f(x+h)-f(x)-g(x+h)+g(x)}{h} \\
        &= \lim_{h\to 0}\frac{f(x+h)-f(x)-\bigl[g(x+h)-g(x)\bigr]}{h} \\
        & =\lim_{h\to 0}
        \left[\frac{f(x+h)-f(x)}{h}-\frac{g(x+h)-g(x)}{h}\right] \\
        &= \lim_{h\to 0}
        \frac{f(x+h)-f(x)}{h}-\lim_{h\to 0}\frac{g(x+h)-g(x)}{h} \\
        &= \frac{d}{dx}f(x)-\frac{d}{dx}g(x)
    \end{align*}
\end{proof}

The Constant Multiple Rule, the Sum Rule, and the Difference Rule can be
combined with the Power Rule to differentiate any polynomial.

\subsubsection*{The Sine and Cosine Functions}
\[\frac{d}{dx}(\sin x)=\cos x\]
\begin{proof}
    If \(f(x)=\sin x\), then
    \begin{align*}
        f'(x) &= \lim_{h\to 0}\frac{f(x+h)-f(x)}{h}
        =\lim_{h\to 0}\frac{\sin(x+h)-\sin x}{h}
        =\lim_{h\to 0}\frac{\sin x\cos h+\cos x\sin h-\sin x}{h} \\
        &= \lim_{h\to 0}
        \left[\frac{\sin x\cos h-\sin x}{h}+\frac{\cos x\sin h}{h}\right]
        =\lim_{h\to 0}
        \left[\sin x\left(\frac{\cos h-1}{h}\right)
        +\cos x\left(\frac{\sin h}{h}\right)\right] \\
        &= \sin x\cdot\lim_{h\to 0}\left(\frac{\cos h-1}{h}\right)
        +\cos x\cdot\lim_{h\to 0}\left(\frac{\sin h}{h}\right) \\
        &=(\sin x)\cdot 0+(\cos x)\cdot 1=\cos x
    \end{align*}
\end{proof}
\[\frac{d}{dx}(\cos x)=-\sin x\]
\begin{proof}
    If \(f(x)=\cos x\), then
    \begin{align*}
        f'(x) &= \lim_{h\to 0}\frac{f(x+h)-f(x)}{h}
        =\lim_{h\to 0}\frac{\cos(x+h)-\cos x}{h}
        =\lim_{h\to 0}\frac{\cos x\cos h-\sin x\sin h-\cos x}{h} \\
        &= \lim_{h\to 0}
        \left[\frac{\cos x\cos h-\cos x}{h}-\frac{\sin x\sin h}{h}\right]
        =\lim_{h\to 0}
        \left[\cos x\left(\frac{\cos h-1}{h}\right)
        -\sin x\left(\frac{\sin h}{h}\right)\right] \\
        &= \cos x\cdot\lim_{h\to 0}\left(\frac{\cos h-1}{h}\right)
        -\sin x\cdot\lim_{h\to 0}\left(\frac{\sin h}{h}\right) \\
        &=(\cos x)\cdot 0-(\sin x)\cdot 1=-\sin x
    \end{align*}
\end{proof}
If \(f(x)=\sin x\), then
\begin{align*}
    f'(x) &= \cos x & f''(x) &= -\sin x & f'''(x) &= -\cos x
    & f^{(4)}(x) &= \sin x
\end{align*}
In general, if \(f(x)=\sin x\), then for \(n=0,1,2,\cdots\) we have
\begin{align*}
    f^{(4n)}(x) &= \sin x & f^{(4n+1)}(x) &= \cos x & f^{(4n+2)}(x) &= -\sin x
    & f^{(4n+3)}(x) &= -\cos x
\end{align*}
If \(f(x)=\cos x\), then
\begin{align*}
    f'(x) &= -\sin x & f''(x) &= -\cos x & f'''(x) &= \sin x
    & f^{(4)}(x) &= \cos x
\end{align*}
In general, if \(f(x)=\cos x\), then for \(n=0,1,2,\cdots\) we have
\begin{align*}
    f^{(4n)}(x) &= \cos x & f^{(4n+1)}(x) &= -\sin x
    & f^{(4n+2)}(x) &= -\cos x
    & f^{(4n+3)}(x) &= \sin x
\end{align*}
\subsection{The Product and Quotient Rules}

\subsubsection*{The Product Rule}
The \textbf{Product Rule}: If \(f\) and \(g\) are both differentiable, then
\[\frac{d}{dx}\big[f(x)g(x)\big]
=\left[\frac{d}{dx}\big[f(x)\big]\right]g(x)+f(x)\frac{d}{dx}\big[g(x)\big]\]
\begin{proof}
    Let \(F(x)=f(x)g(x)\).
    Then
    \begin{align*}
        F'(x) 
        &= \lim_{h\to 0}\frac{F(x+h)-F(x)}{h}
        =\lim_{h\to 0}\frac{f(x+h)g(x+h)-f(x)g(x)}{h} \\
        &= \lim_{h\to 0}\frac{f(x+h)g(x+h)-f(x)g(x)+f(x+h)g(x)-f(x+h)g(x)}{h} \\
        &= \lim_{h\to 0}\frac{f(x+h)g(x)-f(x)g(x)+f(x+h)g(x+h)-f(x+h)g(x)}{h} \\
        &= \lim_{h\to 0}
        \left[\frac{f(x+h)-f(x)}{h}g(x)+f(x+h)\frac{g(x+h)-g(x)}{h}\right] \\
        &= \lim_{h\to 0}\frac{f(x+h)-f(x)}{h}\cdot\lim_{h\to 0}g(x)
        +\lim_{h\to 0}f(x+h)\lim_{h\to 0}\frac{g(x+h)-g(x)}{h} \\
        &= f'(x)g(x)+f(x)g'(x)
    \end{align*}
\end{proof}

\subsubsection*{The Quotient Rule}
The \textbf{Quotient Rule}: If \(f\) and \(g\) are both differentiable, then
\[\frac{d}{dx}\left[\frac{f(x)}{g(x)}\right]
=\frac{\left[\dfrac{d}{dx}\big[f(x)\big]\right]g(x)
-f(x)\dfrac{d}{dx}\big[g(x)\big]}{\big[g(x)\big]^2}\]
\begin{proof}
    Let \(F(x)=\dfrac{f(x)}{g(x)}\).
    Then
    \allowdisplaybreaks
    \begin{align*}
        F'(x) 
        &= \lim_{h\to 0}\frac{F(x+h)-F(x)}{h}
        =\lim_{h\to 0}\frac{\dfrac{f(x+h)}{g(x+h)}-\dfrac{f(x)}{g(x)}}{h}
        =\lim_{h\to 0}\frac{f(x+h)g(x)-f(x)g(x+h)}{hg(x+h)g(x)} \\
        &= \lim_{h\to 0}
        \frac{f(x+h)g(x)-f(x)g(x+h)+f(x)g(x)-f(x)g(x)}{hg(x+h)g(x)} \\
        &= \lim_{h\to 0}
        \frac{f(x+h)g(x)-f(x)g(x)-f(x)g(x+h)+f(x)g(x)}{hg(x+h)g(x)} \\
        &= \lim_{h\to 0}
        \frac{f(x+h)g(x)-f(x)g(x)
        -\big[f(x)g(x+h)-f(x)g(x)\big]}{hg(x+h)g(x)} \\
        &= \lim_{h\to 0}
        \frac{\dfrac{f(x+h)-f(x)}{h}g(x)
        -f(x)\dfrac{g(x+h)-g(x)}{h}}{g(x+h)g(x)} \\
        &= {\frac{\displaystyle{\lim_{h\to 0}}
        \dfrac{f(x+h)-f(x)}{h}\cdot\displaystyle{\lim_{h\to 0}}g(x)
        -\displaystyle{\lim_{h\to 0}}f(x)\cdot\displaystyle{\lim_{h\to 0}}
        \dfrac{g(x+h)-g(x)}{h}}{\displaystyle{\lim_{h\to 0}}g(x+h)
        \cdot\displaystyle{\lim_{h\to 0}}g(x)}} \\
        &= \frac{f'(x)g(x)-f(x)g'(x)}{\big[g(x)^2\big]}
    \end{align*}
\end{proof}

\subsubsection*{Trigonometric Functions}
\[\frac{d}{dx}(\tan x)=\sec^2 x\]
\begin{proof}
    \begin{align*}
        \frac{d}{dx}(\tan x)
        &= \frac{d}{dx}\left(\frac{\sin x}{\cos x}\right)
        =\frac{\left[\dfrac{d}{dx}(\sin x)\right]\cos x
        -\sin x\dfrac{d}{dx}(\cos x)}{\cos^2 x} \\
        &= \frac{\cos x\cdot\cos x-\sin x(-\sin x)}{\cos^2 x}
        =\frac{\cos^2 x+\sin^2 x}{\cos^2 x}=\frac{1}{\cos^2 x}=\sec^2 x
    \end{align*}
\end{proof}
\[\frac{d}{dx}(\csc x)=-\csc x\cot x\]
\begin{proof}
    \begin{align*}
        \frac{d}{dx}(\csc x)
        &= \frac{d}{dx}\left(\frac{1}{\sin x}\right)
        =\frac{\left[\dfrac{d}{dx}(1)\right]\sin x
        -1\cdot\dfrac{d}{dx}(\sin x)}{\sin^2 x}
        =\frac{0\cdot\sin x-1\cdot\cos x}{\sin^2 x} \\
        &= \frac{-\cos x}{\sin^2 x}
        =-\frac{1}{\sin x}\cdot\frac{\cos x}{\sin x}=-\csc x\cot x
    \end{align*}
\end{proof}
\[\frac{d}{dx}(\sec x)=\sec x\tan x\]
\begin{proof}
    \begin{align*}
        \frac{d}{dx}(\sec x)
        &= \frac{d}{dx}\left(\frac{1}{\cos x}\right)
        =\frac{\left[\dfrac{d}{dx}(1)\right]\cos x
        -1\cdot\dfrac{d}{dx}(\cos x)}{\cos^2 x}
        =\frac{0\cdot\cos x-1\cdot(-\sin x)}{\cos^2 x} \\
        &= \frac{\sin x}{\cos^2 x}=\frac{1}{\cos x}\cdot\frac{\sin x}{\cos x}
        =\sec x\tan x
    \end{align*}
\end{proof}
\[\frac{d}{dx}(\cot x)=-\csc^2 x\]
\begin{proof}
    \begin{align*}
        \frac{d}{dx}(\cot x)
        &= \frac{d}{dx}\left(\frac{\cos x}{\sin x}\right)
        =\frac{\left[\dfrac{d}{dx}(\cos x)\right]\sin x
        -\cos x\dfrac{d}{dx}(\sin x)}{\sin^2 x} \\
        &= \frac{(-\sin x)\cdot\sin x-\cos x\cdot\cos x}{\sin^2 x}
        =\frac{-\sin^2 x-\cos^2 x}{\sin^2x} \\
        &= \frac{-(\sin^2 x+\cos^2 x)}{\sin^2 x}=-\frac{1}{\sin^2 x}=-\csc^2 x
    \end{align*}
\end{proof}
\subsection{The Chain Rule}

The \textbf{Chain Rule}: If \(f\) and \(g\) are both differentiable and
\(F=f\circ g\) is the composite function defined by \(F(x)=f(g(x))\), then
\(F\) is differentiable and \(F'\) is given by the product
\[F'(x)=f'(g(x))\cdot g'(x)\]
In Leibniz notation, if \(y=f(u)\) and \(u=g(x)\) are both differentiable
functions, then
\[\frac{dy}{dx}=\frac{dy}{du}\cdot\frac{du}{dx}\]
Comments on the proof of the Chain Rule: Let \(\Delta u\) be the change in
\(u\) corresponding to a change of \(\Delta x\) in \(x\), that is,
\[\Delta u=g(x+\Delta x)-g(x)\]
Then the corresponding change in \(y\) is
\[\Delta y=f(u+\Delta u)-f(u)\]
It is tempting to write
\begin{align*}
    \frac{dy}{dx} &= \lim_{\Delta x\to 0}\frac{\Delta y}{\Delta x}
    =\lim_{\Delta x\to 0}
    \frac{\Delta y}{\Delta u}\cdot\frac{\Delta u}{\Delta x}
    =\lim_{\Delta x\to 0}\frac{\Delta y}{\Delta u}
    \cdot\lim_{\Delta x\to 0}\frac{\Delta u}{\Delta x} \\
    &= \lim_{\Delta u\to 0}\frac{\Delta y}{\Delta u}
    \cdot\lim_{\Delta x\to 0}\frac{\Delta u}{\Delta x}
    =\frac{dy}{du}\cdot\frac{du}{dx}
\end{align*}
Note that \(\Delta u\to 0\) as \(\Delta x\to 0\) since \(g\) is continuous.
The only flaw in this reasoning is that it might happen that \(\Delta u=0\)
(even when \(\Delta x\neq 0\)) and, of course, we cannot divide by 0.
Nonetheless, this reasoning does at least suggest that the Chain Rule is true.
Note that \(\dfrac{dy}{dx}\) is the derivative of \(y\) with respect to \(x\),
whereas \(\dfrac{dy}{du}\) is the derivative of \(y\) with respect to \(u\).
The Power Rule combined with the Chain Rule: If \(n\) is any real number and
\(u=g(x)\) is differentiable, then
\[\frac{d}{dx}(u^n)=nu^{n-1}\frac{du}{dx}\]
Alternatively,
\[\frac{d}{dx}\big[g(x)\big]^n=n\big[g(x)\big]^{n-1}\cdot g'(x)\]
Suppose that \(y=f(u),u=g(x)\), and \(x=h(t)\) where \(f,g,h\) are
differentiable functions.
Then, to compute the derivative of \(y\) with respect to \(t\), we use the
Chain Rule twice:
\[\frac{dy}{dt}=\frac{dy}{dx}\cdot\frac{dx}{dt}
=\frac{dy}{du}\cdot\frac{du}{dx}\cdot\frac{dx}{dt}\]

\subsubsection*{Proof of the Chain Rule}
If we denote \(\epsilon\) the difference between the difference quotient and
the derivative, we obtain
\[\lim_{\Delta x\to 0}\epsilon
=\lim_{\Delta x\to 0}\left(\frac{\Delta y}{\Delta x}-f'(a)\right)
=f'(a)-f'(a)=0\]
But
\[\epsilon=\frac{\Delta y}{\Delta x}-f'(a)
\implies \Delta y=f'(a)\Delta x+\epsilon\Delta x\]
If we define \(\epsilon\) to be 0 when \(\Delta x=0\), then \(\epsilon\)
becomes a continuous function of \(\Delta x\).
Thus, for a differentiable function \(f\), we can write
\[\Delta y=f'(a)\Delta x+\epsilon\Delta x\]
where \(\epsilon\to 0\) as \(\Delta x\to 0\) and \(\epsilon\) is a continuous
function of \(\Delta x\). \\
The proof of the Chain Rule:
\begin{proof}
    Suppose \(u=g(x)\) is differentiable at \(a\) and \(y=f(u)\) is
    differentiable at \(b=g(a)\).
    If \(\Delta x\) is an increment in \(x\) and \(\Delta u\) and \(\Delta y\)
    are the corresponding increments in \(u\) and \(y\), then we can write
    \[\Delta u=g'(a)\Delta x+\epsilon_1\Delta x
    =\big[g'(a)+\epsilon_1\big]\Delta x\]
    where \(\epsilon_1\to 0\) as \(\Delta x\to 0\).
    Similarly,
    \[\Delta y=f'(b)+\epsilon_2\Delta u=\big[f'(b)+\epsilon_2\big]\Delta x\]
    where \(\epsilon_2\to 0\) as \(\Delta u\to 0\)
    If we now substitute the expression for \(\Delta u\), we get
    \[\Delta y=\big[f'(b)+\epsilon_2\big]\big[g'(a)+\epsilon_1\big]\Delta x\]
    so
    \[\frac{\Delta y}{\Delta x}
    =\big[f'(b)+\epsilon_2\big]\big[g'(a)+\epsilon_1\big]\]
    As \(\Delta x\to 0\), it shows that \(\Delta u\to 0\).
    So both \(\epsilon_1\to 0\) and \(\epsilon_2\to 0\) as \(\Delta x\to 0\).
    Therefore
    \[\frac{dy}{dx}=\lim_{\Delta x\to 0}\frac{\Delta y}{\Delta x}
    =\lim_{\Delta x\to 0}\big[f'(b)+\epsilon_2\big]\big[g'(a)+\epsilon_1\big]
    =f'(b)g'(a)=f'(g(a))g'(a)\]
    This proves the Chain Rule.
\end{proof}
\subsection{Implicit Differentiation}

An explicit function \(y=f(x)\) is defined by expressing one variable
explicitly in terms of another variable.
Some functions are defined implicitly by a relation between \(x\) and \(y\).
For example, the equation of the circle is defined by \(x^2+y^2=r^2\)
where the raidus \(r\) is a constant.
In some cases it is possible to solve an implicit function to get an explicit function.
We can use the method of \textbf{implicit differentiation} to find the derivative
of \(y\).
This consists of differentiating both sides of the equation with respect to
\(x\) and then solving the resulting equation for \(y'\).
\begin{problem}
    Find \(dy/dx\) of the unit circle \(x^2+y^2=1\).
\end{problem}
\begin{solution}
    We differentiate on both sides of the equation then we have
    \begin{align*}
        \frac{d}{dx}(x^2+y^2) &= \frac{d}{dx}(1) \\
        \frac{d}{dx}(x^2)+\frac{d}{dx}(y^2) &= 0
    \end{align*}
    Since \(y=f(x)\), we use the chain rule then we have
    \begin{align*}
        2x+\frac{d}{dy}(y^2)\frac{dy}{dx} &= 0 \\
        2x+2y\frac{dy}{dx} &= 0
    \end{align*}
    We solve this equation for \(dy/dx\):
    \[\frac{dy}{dx}=-\frac{x}{y}\]
\end{solution}
\begin{problem}
    Find \(dy/dx\) of the folium of Descartes \(x^3+y^3=6xy\).
\end{problem}
\begin{solution}
    \begin{align*}
        x^3+y^3 &= 6xy \\
        \frac{d}{dx}x^3+\frac{d}{dy}y^3\left(\frac{dy}{dx}\right)
        &= \left[\frac{d}{dx}(6x)\right]y+6x\frac{dy}{dx} \\
        3x^2+3y^2\frac{dy}{dx} &= 6y+6x\frac{dy}{dx} \\
        x^2+y^2\frac{dy}{dx} &= 2y+2x\frac{dy}{dx} \\
        (y^2-2x)\frac{dy}{dx} &= 2y-x^2 \\
        \frac{dy}{dx} &= \frac{2y-x^2}{y^2-2x} 
    \end{align*}
\end{solution}
\begin{problem}
    Find \(y'\) if \(\sin(x+y)=y^2\cos x\).
\end{problem}
\begin{solution}
    \begin{align*}
        \sin(x+y) &= y^2\cos x \\
        \cos(x+y)(1+\frac{dy}{dx}) &= 2y\frac{dy}{dx}\cos x-y^2\sin x \\
        [2y\cos x-\cos(x+y)]\frac{dy}{dx} &= \cos(x+y)+y^2\sin x \\
        \frac{dy}{dx}
        &= \frac{\cos(x+y)+y^2\sin x}{2y\cos x-\cos(x+y)} 
    \end{align*}
\end{solution}
\begin{problem}
    Find \(y''\) if \(x^4+y^4=16\).
\end{problem}
\begin{solution}
    First we find \(y'\) then we have
    \begin{align*}
        x^4+y^4 &= 16 \\ 4x^3+4y^3\frac{dy}{dx} &= 0 \\
        \frac{dy}{dx} &= -\frac{x^3}{y^3}
    \end{align*}
    Then we find \(y''\) and we have
    \begin{align*}
        \frac{d^2y}{dx^2} &= \frac{d}{dx}\left(-\frac{x^3}{y^3}\right)
        =-3x^2y^{-3}+(-x^3)(-3y^{-4})\frac{dy}{dx}
        =3x^3y^{-4}(-x^3y^{-3})-3x^2y^{-3} \\
        &= -3x^6y^{-7}-3x^2y^{-3}
        =-3x^2y^{-7}(x^4+y^4)=-48\frac{x^2}{y^7} 
    \end{align*}
\end{solution}
\subsection{Related Rates}

In a related rates problem the idea is to compute the rate of change of one
quantity in terms of the rate of change of another quantity.
The procedure is to find an equation that relates the two quantities and then
use the Chain Rule to differentiate both sides with respect to time.
\begin{problem}
    Air is being pumped into a spherical balloon so that its volume increases
    at a rate of 100 \(\text{cm}^3/\text{s}\).
    How fast is the radius of the balloon increasing when the diameter is 50
    cm?
\end{problem}
\begin{solution}
    Let \(V\) be the volume of the balloon and let \(r\) be its radius.
    The rate of increase of the volume with respect to time is
    \(\dfrac{dV}{dt}\), and the rate of increase of the radius is
    \(\dfrac{dr}{dt}\).
    Then, we are given that
    \[\frac{dV}{dt}=100\ \text{cm}^3/\text{s}\]
    and we want to find \(\dfrac{dr}{dt}\) when \(r=25\) cm.
    we relate \(V\) and \(r\) by the formula of the volume of a sphere:
    \[V=\frac{4}{3}\pi r^3\]
    Then by the Chain Rule,
    \[\frac{dV}{dt}=\frac{dV}{dr}\cdot\frac{dr}{dt}=4\pi r^2\frac{dr}{dt}\]
    If \(r=25\) and \(\dfrac{dV}{dt}=100\), then
    \[\frac{dr}{dt}=\frac{1}{4\pi r^2}\frac{dV}{dt}
    =\frac{1}{4\pi(25)^2}\cdot100=\frac{1}{25\pi}\]
    The radius of the balloon is increasing at the rate of
    \(\dfrac{1}{25\pi}\) cm/s.
\end{solution}
\subsection{Linear Approximations and Differentials}

We use the tangent line at \((a,f(a))\) as an approximation to the curve
\(y=f(x)\) when \(x\) is near \(a\).
An equation of this tangent line is
\[y=f(a)+f'(a)(x-a)\]
and the approximation
\[f(x)\approx f(a)+f'(a)(x-a)\]
is called the \textbf{linear approximation} of \(f\) at \(a\).
The linear function whose graph is this tangent line, that is,
\[L(x)=f(a)+f'(a)(x-a)\]
is called the \textbf{linearization} of \(f\) at \(a\).

\subsubsection*{Applications to Physics}
The linear approximations
\begin{align*}
    \sin\theta &\approx \theta & \cos\theta &\approx 1
\end{align*}
are used in physics when \(\theta\) is close to 0.

\subsubsection*{Differentials}
If \(y=f(x)\), where \(f\) is a differentiable function, then the
\textbf{differential} \(dx\) is an independent variable.
The differential \(dy\) is then defined by
\[dy=f'(x)\,dx\]
so \(dy\) is an independent variable.
\subsection{Derivatives of Inverse Functions}

\subsubsection*{The Calculus of Inverse Functions}
\begin{theorem}
    If \(f\) is a one-to-one continuous function defined on an interval,
    then its inverse function \(f^{-1}\) is also continuous.
\end{theorem}
\begin{theorem}
    If \(f\) is a one-to-one differentiable function with inverse function
    \(f^{-1}\) and \(f'(f^{-1}(a))\neq 0\), then the inverse function is
    differentiable at \(a\) and
    \[(f^{-1})'(a)=\frac{1}{f'(f^{-1}(a))}\]
\end{theorem}
In general, for any number \(x\) we have
\[(f^{-1})'(x)=\frac{1}{f'(f^{-1}(x))}\]
If we write \(y=f^{-1}(x)\), then \(f(y)=x\), in Leibniz notation we have
\[\frac{dy}{dx}=\frac{1}{\dfrac{dx}{dy}}\]

\subsubsection*{Derivatives of Logarithmic Functions}
The \textbf{Euler's number} \(e\) is the base of the natural exponential
function \(y=e^x\).
It is also the base of the the natural logarithmic function \(y=\ln x\).
\begin{definition}[Euler's Number]
    The Euler's number \(e\) is defined as
    \[e=\lim_{n\to\infty}\left(1+\frac{1}{n}\right)^n
    =\lim_{x\to 0}(1+x)^{1/x}\]
\end{definition}
Note that the approximate value of \(e\) is \(e\approx2.71828\).
\begin{theorem}
    The exponential function \(f(x)=\log_a x\) is differentiable and
    \[f'(x)=\frac{1}{x}\log_a e\]
\end{theorem}
\begin{proof}
    \begin{align*}
        f'(x) &= \lim_{h\to 0}\frac{f(x+h)-f(x)}{h}
        = \lim_{h\to 0}\frac{\log_a(x+h)-\log_a x}{h}
        =\lim_{h\to 0}\frac{\log_a\left(\dfrac{x+h}{x}\right)}{h} \\
        &= \lim_{h\to 0}\frac{1}{x}\cdot\frac{x}{h}
        \log_a\left(1+\frac{h}{x}\right)
        = \frac{1}{x}
            \lim_{h\to 0}\log_a\left(1+\frac{h}{x}\right)^{x/h} \\
        &=\frac{1}{x}
            \lim_{h\to 0}\log_a\left(1+\frac{h}{x}\right)^{1/(h/x)}
        =\frac{1}{x}\log_a e
    \end{align*}
\end{proof}
We know from the change of base formula that
\[\log_a e=\frac{\ln e}{\ln a}=\frac{1}{\ln a}\]
and therefore
\[\frac{d}{dx}(\log_a x)=\frac{1}{x}\log_a e=\frac{1}{x\ln a}\]
The derivative of the natural logarithmic function \(f(x)=\ln x\) is:
\[\frac{d}{dx}(\ln x)=\frac{1}{x}\]
\begin{proof}
    \[\frac{d}{dx}(\ln x)=\frac{1}{x\ln e}=\frac{1}{x}\]
\end{proof}
In general, if \(u=f(x)\), then
\[\frac{d}{dx}(\ln u)=\frac{1}{u}\cdot\frac{du}{dx}\]
\begin{problem}
    Find \(f'(x)\) if \(f(x)=\ln|x|\).
\end{problem}
\begin{solution}
    Since \(f(x)=\ln x\) for \(x>0\) and \(f(x)=\ln(-x)\) for \(x<0\),
    it follows that
    \begin{align*}
        f'(x) &= \frac{d}{dx}(\ln x)=\frac{1}{x},\quad x>0 \\
        f'(x) &= \frac{d}{dx}\big[\ln(-x)\big]=\frac{1}{-x}(-1)
        =\frac{1}{x},\quad x<0
    \end{align*}
    Therefore \(\dfrac{d}{dx}(\ln|x|)=\dfrac{1}{x}\) for all \(x\neq 0\).
\end{solution}

\subsubsection*{Logarithmic Differentiation}
The steps in \textbf{logarithmic differentiation} are
\begin{enumerate}
    \item Take natural logarithms of both sides of an equation \(y=f(x)\) and
    use the Laws of Logarithms to simplify.
    \item Differentiate implicitly with respect to \(x\).
    \item Solve the resulting equation for \(y'\).
\end{enumerate}
The proof of the Power Rule (general version):
\begin{proof}
    Let \(y=x^n\) and we use logarithmic differentiation:
    \[\ln|y|=\ln|x|^n=n\ln|x|,\quad x\neq 0\]
    Therefore
    \[\frac{1}{y}\cdot\frac{dy}{dx}=\frac{n}{x}\]
    Hence
    \[\frac{dy}{dx}=n\frac{y}{x}=n\frac{x^n}{x}=nx^{n-1}\]
\end{proof}
If \(x=0\), we can show that \(f'(0)=0\) for \(n>1\) directly from the
definition of the derivative.

\subsubsection*{Derivatives of Exponential Functions}
\begin{theorem}
    The exponential function \(f(x)=a^x,a>0,\) is differentiable and
    \[\frac{d}{dx}(a^x)=a^x\ln a\]
\end{theorem}
\begin{proof}
    We know that the logarithmic function \(y=\log_a x\) is differentiable
    (and its derivative is nonzero) so its inverse function \(y=a^x\) is
    differentiable.
    If \(y=a^x\), then \(\log_a y=x\).
    By implicit differentiation we have
    \begin{align*}
        \log_a y &= x \\
        \frac{1}{y\ln a}\cdot\frac{dy}{dx} &= 1 \\
        \frac{dy}{dx} &= y\ln a=a^x\ln a 
    \end{align*}
\end{proof}
The derivative of the natural exponential function \(f(x)=e^x\) is:
\[\frac{d}{dx}(e^x)=e^x\]
\begin{proof}
    \[\frac{d}{dx}(e^x)=e^x\ln e=e^x\]
\end{proof}
In general, if \(u=f(x)\), then
\[\frac{d}{dx}(e^u)=e^u\cdot\frac{du}{dx}\]

\subsubsection*{Inverse Trigonometric Functions}
\[\frac{d}{dx}(\arcsin x)=\frac{1}{\sqrt{1-x^2}},\quad -1<x<1\]
\begin{proof}
    Let \(y=\arcsin x \).
    Then \(\sin y=x\) and \(-\pi/2\leq y\leq\pi/2\).
    By implicit differentiation we have
    \begin{align*}
        \frac{d}{dx}(\sin y) &= \frac{d}{dx}(x) \\
        \cos y\cdot\frac{dy}{dx} &= 1 \\
        \frac{dy}{dx} &= \frac{1}{\cos y}
    \end{align*}
    Now \(\cos y\geq0\) since \(-\pi/2\leq y\leq\pi/2\), so
    \[\cos y=\sqrt{1-\sin^2 y}\]
    Therefore
\[\frac{d}{dx}(\arcsin x)=\frac{1}{\sqrt{1-x^2}},\quad -1<x<1\]
\end{proof}
\[\frac{d}{dx}(\arccos x)=-\frac{1}{\sqrt{1-x^2}},\quad -1<x<1\]
\begin{proof}
    Let \(y=\arccos x\).
    Then \(\cos y=x\) and \(0\leq y\leq\pi\).
    By implicit differentiation we have
    \begin{align*}
        \frac{d}{dx}(\cos y) &= \frac{d}{dx}(x) \\
        -\sin y\cdot\frac{dy}{dx} &= 1 \\
        \frac{dy}{dx} &= -\frac{1}{\sin y}
    \end{align*}
    Now \(\sin y\geq0\) since \(0\leq y\leq\pi\), so
    \[\sin y=\sqrt{1-\cos^2 y}\]
    Therefore
    \[\frac{d}{dx}(\arccos x)=-\frac{1}{\sqrt{1-x^2}},\quad -1<x<1\]
\end{proof}
\[\frac{d}{dx}(\arctan x)=\frac{1}{1+x^2}\]
\begin{proof}
    Let \(y=\arctan x\).
    Then \(\tan y=x\) and \(-\pi/2<y<\pi/2\).
    We have
    \begin{align*}
        \frac{d}{dx}(\tan y) &= \frac{d}{dx}(x) \\
        \sec^2 y\cdot\frac{dy}{dx} &= 1 \\
        \frac{dy}{dx} &= \frac{1}{\sec^2 y}=\frac{1}{\tan^2 y+1}
        =\frac{1}{1+x^2}
    \end{align*}
    Therefore
    \[\frac{d}{dx}(\arctan x)=\frac{1}{1+x^2} \]
\end{proof}
\[\frac{d}{dx}(\arccsc x)=-\frac{1}{x\sqrt{x^2-1}}\]
\[\frac{d}{dx}(\arcsec x)=\frac{1}{x\sqrt{x^2-1}}\]
\[\frac{d}{dx}(\arccot x)=\frac{1}{x\sqrt{1+x^2}}\]
\subsection{Hyperbolic Functions}

Definition of \textbf{hyperbolic functions}:
\begin{align*}
    \sinh x &= \frac{e^x-e^{-x}}{2} & \csch x &= \frac{1}{\sinh x} \\
    \cosh x &= \frac{e^x+e^{-x}}{2} & \sech x &= \frac{1}{\cosh x} \\
    \tanh x &= \frac{\sinh x}{\cosh x} & \coth x &= \frac{\cosh x}{\sinh x}
\end{align*}
Hyperbolic Identities:
\begin{itemize}
    \item \(\sinh(-x)=-\sinh x\)
    \item \(\cosh(-x)=\cosh x\)
    \item \(\cosh^2 x-\sinh^2 x=1\)
    \item \(1-\tanh^2 x=\sech^2 x\)
    \item \(\sinh(x+y)=\sinh x\cosh x+\cosh x\sinh y\)
    \item \(\cosh(x+y)=\cosh x\cosh x+\sinh x\sinh y\)
\end{itemize}
Derivatives of hyperbolic functions:
\begin{align*}
    \frac{d}{dx}(\sinh x) &= \cosh x
    & \frac{d}{dx}(\csch x) &= -\csch x\coth x \\
    \frac{d}{dx}(\cosh x) &= \sinh x
    & \frac{d}{dx}(\sech x) &= -\sech x\tanh x \\
    \frac{d}{dx}(\tanh x) &= \sech^2 x
    & \frac{d}{dx}(\coth x) &= -\csch^2 x
\end{align*}

\subsubsection*{Inverse Hyperbolic Functions}
\begin{align*}
    \arsinh x &= \ln(x+\sqrt{x^2+1}),\quad x\in\R \\
    \arcosh x &= \ln(x+\sqrt{x^2-1}),\quad x\geq 1 \\
    \artanh x &= \frac{1}{2}\ln\left(\frac{1+x}{1-x}\right),
    \quad -1\leq x\leq 1
\end{align*}
Derivatives of inverse hyperbolic functions:
\begin{align*}
    \frac{d}{dx}(\arsinh x) &= \frac{1}{\sqrt{1+x^2}}
    & \frac{d}{dx}(\arcsch x) &= -\frac{1}{|x|\sqrt{x^2+1}} \\
    \frac{d}{dx}(\arcosh x) &= \frac{1}{\sqrt{x^2-1}}
    & \frac{d}{dx}(\arsech x) &= -\frac{1}{x\sqrt{1-x^2}} \\
    \frac{d}{dx}(\artanh x) &= \frac{1}{1-x^2}
    & \frac{d}{dx}(\arcoth x) &= \frac{1}{1-x^2}
\end{align*}
\subsection{Indeterminate Forms and l'Hospital's Rule}

In general, if we have a limit of the form
\[\lim_{x\to a}\frac{f(x)}{g(x)}\]
where both \(f(x)\to 0\) and \(g(x)\to 0\) as \(x\to a\), then this limit may
or may not exist and is called an \textbf{indeterminate form of type}
\(\mathbf{\dfrac{0}{0}}\).
In general, if we have a limit of the form
\[\lim_{x\to a}\frac{f(x)}{g(x)}\]
where both \(f(x)\to\infty\) (or \(-\infty\)) and \(g(x)\to\infty\)
(or \(-\infty\)), then this limit may or may not exist and is called an
\textbf{indeterminate form of type} \(\mathbf{\dfrac{\infty}{\infty}}\).
\begin{theorem}[l'Hospital's Rule]
    Suppose \(f\) and \(g\) are differentiable and \(g'(x)\neq 0\) near \(a\)
    (except possibly at \(a\)).
    Suppose that
    \begin{align*}
        \lim_{x\to a}f(x) &= 0 & \lim_{x\to a}g(x) &= 0
    \end{align*}
    or that
    \begin{align*}
        \lim_{x\to a}f(x) &= \pm\infty & \lim_{x\to a}g(x) &= \pm\infty
    \end{align*}
    In other words, we have an indeterminate form of type \(\dfrac{0}{0}\) or
    \(\dfrac{\infty}{\infty}\).
    Then
    \[\lim_{x\to a}\frac{f(x)}{g(x)}=\lim_{x\to a}\frac{f'(x)}{g'(x)}\]
    if the limit on the right side exists (or is \(\infty\) or \(-\infty\)).
\end{theorem}
It is important to verify the conditions regarding the limits of \(f\) and \(g\) before using
l'Hospital's Rule.
L'Hospital's Rule is also valid for one-sided limits and for limits at
infinity or negative infinity.
For the special case in which \(f(a)=g(a)=0\), \(f'\) and \(g'\) is
continuous, and \(g'(a)\neq 0\), we have
\[\lim_{x\to a}\frac{f'(x)}{g'(x)}=\frac{f'(a)}{g'(a)}
= \frac{\displaystyle{\lim_{x\to a}\frac{f(x)-f(a)}{x-a}}}
{\displaystyle{\lim_{x\to a}\frac{g(x)-g(a)}{x-a}}}
=\lim_{x\to a}\frac{f(x)-f(a)}{g(x)-g(a)}=\lim_{x\to a}\frac{f(x)}{g(x)}\]

\begin{problem}
    Find \(\displaystyle{\lim_{x\to 1}\frac{\ln x}{x-1}}\).
\end{problem}
\begin{solution}
    Notice that we have an indeterminate form of type \(\dfrac{0}{0}\).
    We apply l'Hospital's rule then
    \[\lim_{x\to 1}\dfrac{\ln x}{x-1}=\lim_{x\to 1}\frac{1/x}{1}
    =\lim_{x\to 1}\frac{1}{x}=1\]
\end{solution}
\begin{problem}
    Calculate \(\displaystyle{\lim_{x\to\infty}\frac{e^x}{x^2}}\).
\end{problem}
\begin{solution}
    Notice that we have an indeterminate form of type
    \(\dfrac{\infty}{\infty}\).
    Then l'Hospital's Rule gives
    \[\lim_{x\to\infty}\frac{e^x}{x^2}=\lim_{x\to\infty}\frac{e^x}{2x}\]
    Since we stil have an indeterminate form of \(\dfrac{\infty}{\infty}\), a
    second application of l'Hospital's Rule gives
    \[\lim_{x\to\infty}\frac{e^x}{x^2}=\lim_{x\to\infty}\frac{e^x}{2x}
    =\lim_{x\to\infty}\frac{e^x}{2}=\infty\]
\end{solution}
\begin{problem}
    Find \(\displaystyle{\lim_{x\to 0}\frac{\tan x-x}{x^3}}\).
\end{problem}
\begin{solution}
    Notice that we have an indeterminate form of type \(\dfrac{0}{0}\).
    We apply l'Hospital's rule then
    \begin{align*}
        \lim_{x\to 0}\frac{\tan x-x}{x^3}
        &= \lim_{x\to 0}\frac{\sec^2 x-1}{3x^2}
        =\lim_{x\to 0}\frac{2\sec^2 x\tan x}{6x}
        =\lim_{x\to 0}\frac{2\sec^4 x+4\sec^2 x\tan^2 x}{6} \\
        &= \frac{2(1)+4(1)(0)}{6}=\frac{1}{3}
    \end{align*}
\end{solution}

\subsubsection*{Indeterminate Products}
If \(\displaystyle{\lim_{x\to a}f(x)=0}\) and
\(\displaystyle{\lim_{x\to a}g(x)=\infty}\) (or \(-\infty\)),
then the limit
\[\lim_{x\to a}f(x)g(x)\]
is called an \textbf{indeterminate form of type} \(\mathbf{0\cdot\infty}\).
We can deal with it by writing the product \(fg\) as a quotient:
\[fg=\frac{f}{1/g}\]
or
\[fg=\frac{g}{1/f}\]
This converts the given limit into an indeterminate form of type
\(\dfrac{0}{0}\) or \(\dfrac{\infty}{\infty}\) so that we can use l'Hospital's
Rule.
\begin{problem}
    Evaluate \(\displaystyle{\lim_{x\to 0^+}x\ln x}\).
\end{problem}
\begin{solution}

    \[\lim_{x\to 0^+}x\ln x=\lim_{x\to 0^+}\frac{\ln x}{1/x}
    =\lim_{x\to 0^+}\frac{1/x}{-1/x^2}=\lim_{x\to 0^+}(-x)=0\]
\end{solution}

\subsubsection*{Indeterminate Differences}
If \(\displaystyle{\lim_{x\to a}f(x)=\infty}\) and
\(\displaystyle{\lim_{x\to a}g(x)=\infty}\), then the limit
\[\lim_{x\to a}\bigl[f(x)-g(x)\bigr]\]
is called an \textbf{indeterminate form of type} \(\infty-\infty\).
We can convert the difference into a quotient so that we have an indeterminate
form of type \(\dfrac{0}{0}\) or \(\dfrac{\infty}{\infty}\).
\begin{problem}
    Compute \(\displaystyle{\lim_{x\to (\pi/2)^-}(\sec x-\tan x)}\).
\end{problem}
\begin{solution}
    \[\lim_{x\to (\pi/2)^-}(\sec x-\tan x)
    =\lim_{x\to (\pi/2)^-}\frac{1-\sin x}{\cos x}
    =\lim_{x\to (\pi/2)^-}\frac{-\cos x}{-\sin x}=0\]
\end{solution}

\subsubsection*{Indeterminate Powers}
Several indeterminate forms arise from the limit
\[\lim_{x\to a}\bigl[f(x)\bigr]^{g(x)}\]
\begin{itemize}
    \item Type \(0^0\): \(\displaystyle{\lim_{x\to a}f(x)=0}\) and
    \(\displaystyle{\lim_{x\to a}g(x)=0}\).
    \item Type \(\infty^0\): \(\displaystyle{\lim_{x\to a}f(x)=\infty}\) and
    \(\displaystyle{\lim_{x\to a}g(x)=0}\).
    \item Type \(1^\infty\): \(\displaystyle{\lim_{x\to a}f(x)=1}\) and
    \(\displaystyle{\lim_{x\to a}g(x)=\pm\infty}\).
\end{itemize}
Note that the form \(0^\infty\) is not indeterminate.
Each of these three cases can be treated either by taking the natural
logarithm: let \(y=\bigl[f(x)\bigr]^{g(x)}\), then \(\ln y=g(x)\ln f(x)\) or
by writing the function as an exponential:
\[\bigl[f(x)\bigr]^{g(x)}=e^{g(x)\ln f(x)}\]
In either method we are led to the indeterminate product \(g(x)\ln f(X)\),
which is of type \(0\cdot\infty\).
\begin{problem}
    Calculate \(\displaystyle{\lim_{x\to 0^+}(1+\sin 4x)^{\cot x}}\).
\end{problem}
\begin{solution}
    Let
    \[y=(1+\sin 4x)^{\cot x}\]
    Then
    \[\ln y=\ln\bigl[(1+\sin 4x)^{\cot x}\bigr]=\cot x\ln(1+\sin 4x)\]
    so l'Hospital's Rule gives
    \[\lim_{x\to 0^+}\ln y=\lim_{x\to 0^+}\frac{\ln(1+\sin 4x)}{\tan x}
    =\lim_{x\to 0^+}\frac{\dfrac{4\cos 4x}{1+\sin 4x}}{\sec^2 x}=4\]
    Then
    \[\lim_{x\to 0^+}(1+\sin 4x)^{\cot x}=\lim_{x\to 0^+}y
    =\lim_{x\to 0^+}e^{\ln y}=e^4\]
\end{solution}
\begin{problem}
    Find \(\displaystyle{\lim_{x\to 0^+}x^x}\).
\end{problem}
\begin{solution}
    We used l'Hospital's Rule to show that
   \[\lim_{x\to 0^+}x\ln x=0\]
   Therefore
   \[\lim_{x\to 0^+}x^x=\lim_{x\to 0^+}e^{x\ln x}=e^0=1\]
\end{solution}

\section{Applications of Differentiation}

\subsection{Maximum and Minimum Values}

\begin{definition}
    Let \(c\) be a number in the domain \(D\) of a function \(f\).
    Then \(f(c)\) is the
    \begin{itemize}
        \item \textbf{absolute maximum} value of \(f\) on \(D\) if
        \(f(c)\geq f(x)\) for all \(x\) in \(D\).
        \item \textbf{absolute minimum} value of \(f\) on \(D\) if
        \(f(c)\leq f(x)\) for all \(x\) in \(D\).
    \end{itemize}
\end{definition}
An absolute maximum or minimum is sometimes called a \textbf{global} maximum
or minimum.
The maximum and minimum values of \(f\) are called \textbf{extreme values} of
\(f\).
\begin{definition}
    The number \(f(c)\) is a
    \begin{itemize}
        \item \textbf{local maximum} value of \(f\) if \(f(c)\geq f(x)\) when
        \(x\) is near \(c\).
        \item \textbf{local minimum} value of \(f\) if \(f(c)\leq f(x)\) when
        \(x\) is near \(c\).
    \end{itemize}
\end{definition}
\begin{theorem}[Extreme Value Theorem]
    If \(f\) is continuous on a closed interval \([a,b]\), then \(f\) attains
    an absolute maximum value \(f(c)\) and absolute minimum value \(f(d)\) at
    some numbers \(c\) and \(d\) in \([a,b]\).
\end{theorem}
\begin{theorem}[Fermat's Theorem]
    If \(f\) has a local maximum or minimum at \(c\), and if \(f'(c)\) exists,
    then \(f'(c)=0\).
\end{theorem}
\begin{proof}
    Suppose that \(f\) has a local maximum at \(c\).
    Then \(f(c)\geq f(x)\) if \(x\) is sufficiently close to \(c\).
    This implies that if \(h\) is sufficiently close to 0, with \(h\) being
    positive or negative, then
    \[f(c)\geq f(c+h)\]
    and theorefore
    \[f(c+h)-f(c)\leq 0\]
    If \(h>0\) and \(h\) is sufficiently small, we have
    \[\frac{f(c+h)-f(c)}{h}\leq 0\]
    Taking the right-hand limit, we get
    \[\lim_{h\to 0^+}\frac{f(c+h)-f(c)}{h}\leq\lim_{h\to 0^+}0=0\]
    But since \(f'(c)\) exists, we have
    \[f'(c)=\lim_{h\to 0}\frac{f(c+h)-f(c)}{h}
    =\lim_{h\to 0^+}\frac{f(c+h)-f(c)}{h}\]
    and so we have shown that \(f'(c)\leq 0\).
    If \(h<0\), then
    \[\frac{f(c+h)-f(c)}{h}\geq 0\]
    Taking the left-hand limit, we have
    \[f'(c)=\lim_{h\to 0}\frac{f(c+h)-f(c)}{h}
    =\lim_{h\to 0^-}\frac{f(c+h)-f(c)}{h}\geq 0\]
    We have shown that \(f'(c)\leq 0\) and \(f'(c)\geq 0\).
    Therefore \(f'(c)=0\).
    The case of a local minimum can be proved in a similar manner.
\end{proof}
\begin{definition}
    A \textbf{critical number} of a function \(f\) is a number \(c\) in the
    domain of \(f\) such that either \(f'(c)=0\) or \(f'(c)\) does not exist.
\end{definition}
If \(f\) has a local maximum or minimum at \(c\), then \(c\) is a critical
number of \(f\).
The Closed Interval Method: To find the absolute maximum and minimum values of
a continuous function \(f\) on a closed interval \([a,b]\):
\begin{enumerate}
    \item Find the values of \(f\) at the critical numbers of \(f\) in
    \((a,b)\).
    \item Find the values of \(f\) at the endpoints of the interval.
    \item The largest of these values is the absolute maximum value; the
    smallest of these values is the absolute minimum value.
\end{enumerate}
\subsection{The Mean Value Theorem}

\begin{theorem}[Rolle's Theorem]
    Let \(f\) be a function that satisfies the following three hypotheses:
    \begin{enumerate}
        \item \(f\) is continuous on the closed interval \([a,b]\).
        \item \(f\) is differentiable on the open interval \((a,b)\).
        \item \(f(a)=f(b)\)
    \end{enumerate}
    Then there is a number \(c\) in \((a,b)\) such that \(f'(c)=0\).
\end{theorem}
\begin{proof}
    There are three cases: \\
    Case 1: \(f(x)=k\), a constant \\
    Then \(f'(x)=0\) so the number \(c\) can be any number in \((a,b)\). \\
    Case 2: \(f(x)>f(a)\) for some \(x\) in \((a,b)\) \\
    By the Extreme Value Theorem, \(f\) has a maximum value somewhere in
    \([a,b]\).
    Since \(f(a)=f(b)\), it must attain this maximum value at a number \(c\)
    in the open interval \((a,b)\).
    Then \(f\) has a local maximum at \(c\).
    Therefore \(f'(c)=0\) by Fermat's Theorem. \\
    Case 3: \(f(x)<f(a)\) for some \(x\) in \((a,b)\) \\
    By the Extreme Value Theorem, \(f\) has a minimum value somewhere in
    \([a,b]\), and since \(f(a)=f(b)\), it attains this minimum value at a
    number \(c\) in \((a,b)\).
    Then \(f'(c)=0\) by Fermat's Theorem.
\end{proof}
\begin{theorem}[Lagrange's Mean Value Theorem]
    Let \(f\) be a function that satisfies the following hypotheses:
    \begin{enumerate}
        \item \(f\) is continuous on the closed interval \([a,b]\).
        \item \(f\) is differentiable on the open interval \((a,b)\).
    \end{enumerate}
    Then there is a number \(c\) in \((a,b)\) such that
    \[f'(c)=\frac{f(b)-f(a)}{b-a}\]
    or, equivalently,
    \[f(b)-f(a)=f'(c)(b-a)\]
\end{theorem}
\begin{proof}
    We apply Rolle's Theorem to a new function \(h\) defined as a difference
    between \(f\) and the function whose graph is the secant line \(AB\) where
    \(A\) is the point \((a,f(a))\) and \(B\) is the point \((b,f(b))\).
    The slope of the secant line \(AB\) is
    \[m=\frac{f(b)-f(a)}{b-a}\]
    Then the equation of the secant line \(AB\) is
    \[y-f(a)=\frac{f(b)-f(a)}{b-a}(x-a)\]
    or
    \[y=f(a)+\frac{f(b)-f(a)}{b-a}(x-a)\]
    So,
    \[h(x)=f(x)-f(a)-\frac{f(b)-f(a)}{b-a}(x-a)\]
    First we must verify that \(h\) satisfies the three hypotheses of Rolle's
    Theorem.
    \begin{enumerate}
        \item The function \(h\) is continuous on \([a,b]\) because it is the
        sum of \(f\) and a first degree polynomial, both of which is
        continuous.
        \item The function \(h\) is differentiable on \((a,b)\) because both
        \(f\) and a first degree polynomial are differentiable.
        In fact, we can compute \(h'\) directly:
        \[h'(x)=f'(x)-\frac{f(b)-f(a)}{b-a}\]
        Note that \(f(a)\) and \(\dfrac{f(b)-f(a)}{b-a}\) are constants.
        \item
        \[h(a)=f(a)-f(a)-\frac{f(b)-f(a)}{b-a}(a-a)=0\]
        \[h(b)=f(b)-f(a)-\frac{f(b)-f(a)}{b-a}(b-a)
        =f(b)-f(a)-\big[f(b)-f(a)\big]=0\]
        Therefore \(h(a)=h(b)\).
        Since \(h\) satisfies the hypotheses of Rolle's Theorem,
        there is a number \(c\) in \((a,b)\) such that \(h'(c)=0\).
        Therefore
        \[0=h'(c)=f'(c)-\frac{f(b)-f(a)}{b-a}\]
        and so
        \[f'(c)=\frac{f(b)-f(a)}{b-a}\]
    \end{enumerate}
\end{proof}
In general, Lagrange's Mean Value Theorem can be interpreted as saying that
there is a number at which the instantaneous rate of change is equal to the
average rate of change over an interval.
The main significance of Lagrange's Mean Value Theorem is that it enables
us to obtain information about a function from information about its
derivative.
\begin{theorem}
    If \(f'(x)=0\) for all \(x\) in an interval \((a,b)\), then \(f\) is a
    constant on \((a,b)\).
\end{theorem}
\begin{proof}
    Let \(x_1\) and \(x_2\) be any two numbers in \((a,b)\) with \(x_1<x_2\).
    Since \(f\) is differentiable on \((a,b)\), it must be differentiable on
    \((x_1,x_2)\) and continuous on \([x_1,x_2]\).
    By applying Lagrange's Mean Value Theorem to \(f\) on the interval
    \([x_1,x_2]\), we get a number \(c\) such that \(x_1<c<x_2\) and
    \[f(x_2)-f(x_1)=f'(c)(x_2-x_1)\]
    Since \(f'(x)=0\) for all \(x\), we have \(f'(c)=0\), and so
    \[f(x_2)-f(x_1)=0\]
    or
    \[f(x_2)=f(x_1)\]
    Therefore \(f\) has the same value at any two numbers \(x_1\) and \(x_2\)
    in \((a,b)\).
    This means that \(f\) is a constant on \((a,b)\).
\end{proof}
\begin{corollary}
    If \(f'(x)=g'(x)\) for all \(x\) in an interval \((a,b)\), then \(f-g\) is
    a constant on \((a,b)\); that is; \(f(x)=g(x)+c\) where \(c\) is a
    constant.
\end{corollary}
\begin{proof}
    Since \(f'(x)=g'(x)\), we have \(f'(x)-g'(x)=0\).
    Let \(F(x)=f(x)-g(x)\).
    Then
    \[F'(x)=f'(x)-g'(x)=0\]
    for all \(x\) in \((a,b)\).
    Thus \(F\) is a constant; that is, \(f-g\) is a constant.
\end{proof}
\begin{theorem}[Cauchy's Mean Value Theorem]
    Suppose that the fucntions \(f\) and \(g\) are continuous on \([a,b]\) and
    differentiable on \((a,b)\), and \(g(x)\neq 0\) for all \(x\) in
    \((a,b)\).
    Then there is a number \(c\) in \((a,b)\) such that
    \[\frac{f'(c)}{g'(c)}=\frac{f(b)-f(a)}{g(b)-g(a)}\]
\end{theorem}
\subsection{Derivatives and the Shapes of Graphs}

\subsubsection*{First Derivatives}
Increasing/Decreasing Test:
\begin{enumerate}
    \item If \(f'(x)>0\) on an interval, then \(f\) is increasing on that
    interval.
    \item If \(f'(x)<0\) on an interval, then \(f\) is decreasing on that
    interval.
\end{enumerate}
\begin{proof}
    Let \(x_1\) and \(x_2\) be any two numbers in the interval with
    \(x_1<x_2\).
    Given \(f'(x)>0\), we know that \(f\) is differentiable on \([x_1,x_2]\).
    By Lagrange's Mean Value Theorem, there is a number \(c\) between \(x_1\)
    and \(x_2\) such that
    \[f(x_2)-f(x_1)=f'(c)(x_2-x_1)\]
    Since \(f'(c)>0\) and \(x_2-x_1>0\), thus
    \[f(x_2)-f(x_1)>0\]
    or
    \[f(x_1)<f(x_2)\]
    This shows that \(f\) is increasing.
    Part 2 is proved similarly.
\end{proof}
The First Derivative Test: Suppose that \(c\) is a critical number of a
continuous function \(f\).
\begin{enumerate}
    \item If \(f'\) changes from positive to negative at \(c\), then \(f\) has
    a local maximum at \(c\).
    \item If \(f'\) changes from negative to positive at \(c\), then \(f\) has
    a local minimum at \(c\).
    \item If \(f'\) does not change sign at \(c\), then \(f\) has no local
    maximum or minimum at \(c\).
\end{enumerate}

\subsubsection*{Second Derivatives}
\begin{definition}
    If the graph of \(f\) lies above all of its tangents on an interval \(I\),
    then it is called \textbf{concave upward} on \(I\).
    If the graph of \(f\) lies below all of its tangents on \(I\), it is
    called \textbf{concave downward} on \(I\).
\end{definition}
\begin{definition}
    A point \(P\) on a curve \(y=f(x)\) is called an \textbf{inflection point}
    if \(f\) is continuous there and the curve changes from concave upward to
    downward or from concave downward to concave upward at \(P\).
\end{definition}
Concavity Test:
\begin{enumerate}
    \item If \(f''(x)>0\) for all \(x\) in \(I\), then the graph of \(f\) is
    concave upward on \(I\).
    \item If \(f''(x)<0\) for all \(x\) in \(I\), then the graph of \(f\) is
    concave downward on \(I\).
\end{enumerate}
The Second Derivative Test: Suppose \(f''\) is continuous near \(c\).
\begin{enumerate}
    \item If \(f'(c)=0\) and \(f''(c)>0\), then \(f\) has a local minimum at
    \(c\).
    \item If \(f'(c)=0\) and \(f''(c)<0\), then \(f\) has a local maximum at
    \(c\).
\end{enumerate}
\subsection{Optimization Problems}

\begin{problem}
    A farmer has 2400 m of fencing and wants to fence off a rectangular field
    that borders a straight river.
    He needs no fence along the river.
    What are the dimensions of the field that has the largest area?
\end{problem}
\begin{solution}
    We want to maximize the area \(A\) of the rectangle.
    Let \(x\) and \(y\) be the length and width of the rectangle.
    Then the area is
    \[A=xy\]
    Given that the total length of fencing is 2400 m so
    \[x+2y=2400\]
    Thus
    \[y=1200-\frac{1}{2}x\]
    and so
    \[A=x\left(1200-\frac{1}{2}x\right)=1200x-\frac{1}{2}x^2\]
    Note that \(x\geq 0\) and \(x\leq 2400\).
    So the function we want to maximize is
    \[A(x)=1200x-\frac{1}{2}x^2,\quad 0\leq x\leq 2400\]
    The derivative is \(A'(x)=1200-x\), so to find the critical numbers we solve
    the equation
    \[1200-x=0\]
    which gives \(x=1200\).
    Since \(A(0)=0,A(1200)=720000,A(2400)=0\), the Closed Interval Method
    gives the maximum value as \(A(1200)=720000\).
    Thus the rectangular field should have a length of 1,200 m and a width of
    600 m.
\end{solution}
\subsection{Newton's Method}

We have \textbf{Newton's Method} to find an approximate numerical value of the
root of an equation.
Let \(r\) be the root of the equation.
We start with a first approximation \(x_1\).
Consider the tangent line \(L\) to the curve \(y=f(x)\) at the point
\((x_1,f(x_1))\) and we label the \(x\)-intercept of \(L\) as \(x_2\).
The slope of \(L\) is \(f(x_1)\), so its equation is
\[y-f(x_1)=f'(x_1)(x-x_1)\]
Since the \(x\)-intercept of \(L\) is \(x_2\), we set \(y=0\) and obtain
\[0-f(x_1)=f'(x_1)(x_2-x_1)\]
If \(f'(x_1)\neq 0\), we can solve this equation for \(x_2\):
\[x_2=x_1-\frac{f(x_1)}{f'(x_1)}\]
We use \(x_2\) as a second approximation to \(r\).
Next we repeat this procedure with \(x_1\) replaced by \(x_2\), using the
tangent line at \((x_2,f(x_2))\).
This gives a third approximation:
\[x_3=x_2-\frac{f(x_2)}{f'(x_2)}\]
If we keep repeating this process, we obtain a sequence of approximations
\(x_1,x_2,x_3,x_4,\dots\).
In general, if the \(n\)th approximation is \(x_n\) and \(f'(x_n)\neq 0\),
then the next approximation is given by
\[x_{n+1}=x_n-\frac{f(x_n)}{f'(x_n)}\]
If the numbers \(x_n\) become closer and closer to \(r\) as \(n\) becomes
large, then we say that the sequence converges to \(r\) and we write
\[\lim_{n\to\infty}x_n=r\]
\subsection{Antiderivatives}

\begin{definition}
    A function \(F\) is called an \textbf{antiderivative} of \(f\) on an
    interval \(I\) if \(F'(x)=f(x)\) for all \(x\) in \(I\).
\end{definition}
\begin{theorem}
    If \(F\) is an antiderivative of \(f\) on interval \(I\), then the most
    general antiderivative of \(f\) on \(I\) is
    \[F(x)+C\]
    where \(C\) is an arbitrary constant.
\end{theorem}
\begin{problem}
    Find the most general antiderivative of \(f(x)=\dfrac{1}{x}\).
\end{problem}
\begin{solution}
    We know that
    \[\frac{d}{dx}(\ln |x|)=\frac{1}{x}\]
    for all \(x\neq 0\).
    Then the antiderivative of \(f\) is \(F(x)=\ln x+C_1\) if \(x>0\)
    or \(F(x)=\ln(-x)+C_2\) if \(x<0\).
    The general antiderivative of \(f\) is \(F(x)=\ln |x|+C\) on each of the
    intervals \((-\infty,0)\) and \((0,\infty)\).
\end{solution}
\begin{problem}
    Find the most general antiderivative of \(f(x)=x^n,\quad n\neq -1\).
\end{problem}
\begin{solution}
    If \(n\neq -1\), then by the Power Rule,
    \[\frac{d}{dx}\left(\frac{x^{n+1}}{n+1}\right)=\frac{(n+1)x^n}{n+1}=x^n\]
    Thus the general antiderivative of \(f(x)=x^n\) is
    \[F(x)=\frac{x^{n+1}}{n+1}+C\]
\end{solution}

\section{Integrals}

\subsection{The Definite Integral}

\subsubsection*{The Area Problem}
\begin{definition}
    The area \(A\) of the region \(S\) that lies under the graph of the
    continuous function \(f\) is the limit of the sum of the areas of the
    approximating rectangles:
    \[A=\lim_{n\to\infty}R_n
    =\lim_{n\to\infty}\big[f(x_1)\Delta x+f(x_2)\Delta x+\cdots
    +f(x_n)\Delta x\big]\]
\end{definition}

\subsubsection*{The Distance Problem}
In general, suppose an object moves with velocity \(v=f(t)\), where
\(a\leq t\leq b\), and \(f(t)>0\).
The exact total distance \(d\) traveled is
\[d=\lim_{n\to\infty}\sum_{i=1}^n f(t_i)\Delta t\]

\subsubsection*{The Definite Integral}
In general, we start with any function \(f\) defined on \([a,b]\) and we
divide \([a,b]\) into \(n\) smaller subintervals by choosing partition points
\(x_0,x_1,x_2,\dots,x_n\) so that
\[a=x_0<x_1<x_2<\dots<x_{n-1}<x_n=b\]
The resulting collection of subintervals
\[[x_0,x_1],[x_1,x_2],[x_2,x_3],\dots,[x_{n-1},x_n]\]
is called a \textbf{partition} \(P\) of the \([a,b]\).
We use the notation \(\Delta x_i\) for the length of the \(i\)th subinterval
\([x_{i-1},x_i]\).
Thus
\[\Delta x_i=x_i-x_{i-1}\]
Then we choose \textbf{sample points} \(x_1^*,x_2^*,\dots,x_n^*\) in the
subintervals with \(x_i^*\) in the \(i\)th subinterval \([x_{i-1},x_i]\).
A \textbf{Riemann sum} associated with a partition \(P\) and a function \(f\)
is
\[\sum_{i=1}^n f(x_i^*)\Delta x_i=f(x_1^*)\Delta x_1+f(x_2^*)\Delta x_2
+\dots+f(x_n^*)\Delta x_n\]
\begin{definition}
    If \(f\) is a function defined on \([a,b]\),
    the \textbf{definite integral of} \(f\) \textbf{from} \(a\) \textbf{to}
    \(b\) is the number
    \[\int_a^b f(x)\,dx=\lim_{\max\Delta x_i\to 0}
    \sum_{i=1}^n f(x_i^*)\Delta x_i\]
    provided that this limit exists.
    If it does exist, we say that \(f\) is \textbf{integrable} on \([a,b]\).
\end{definition}
\begin{definition}
    \[\int_a^b f(x)\,dx=I\]
    means that for every \(\epsilon>0\) there is a corresponding number
    \(\delta>0\) such that
    \[\left\lvert I-\sum_{i=1}^n f(x_i^*)\Delta x_i\right\rvert<\epsilon\]
    for all partitions \(P\) of \([a,b]\) with \(\max\Delta x_i<\delta\) and
    for all possible choices of \(x_i^*\) in \([x_{i-1},x_i]\).
\end{definition}
The symbol \(\displaystyle{\int}\) was introduced by Leibniz and is called an
\textbf{integral sign}.
In the notation \(\displaystyle{\int_a^b f(x)\,dx}\), \(f(x)\) is called the
\textbf{integrand} and \(a\) and \(b\) are called the
\textbf{limits of integration}; \(a\) is the \textbf{lower limit} and \(b\) is the
\textbf{upper limit}.
The symbol \(dx\) indicates that the independent variable is \(x\).
The procedure of calculating an integral is called \textbf{integration}.
The definite integral is a number so it does not depend on \(x\).
In fact, we can use any letter in place without changing the value of the
integral:
\[\int_a^b f(x)\,dx=\int_a^b f(y)\,dy=\int_a^b f(t)\,dt\]
\begin{theorem}
    If \(f\) is continuous on \([a,b]\), or if \(f\) has only a finite number
    of jump discontinuities, then \(f\) is integrable on \([a,b]\); that is,
    the definite integral \(\displaystyle{\int_a^b f(x)\,dx}\) eixsts.
\end{theorem}
\begin{theorem}
    If \(f\) is integrable on \([a,b]\), then
    \[\int_a^b f(x)\,dx=\lim_{n\to\infty}\sum_{i=1}^nf(x_i)\Delta x\]
    where \(\Delta x=\dfrac{b-a}{n}\) and \(x_i=a+i\Delta x\).
\end{theorem}
A definite integral can be interpreted as a \textbf{net area}, that is, a
difference of areas:
\[\int_a^b f(x)\,dx=A_1-A_2\]

\subsubsection*{The Midpoint Rule}
The Midpoint Rule:
\[\int_a^b f(x)\,dx\approx \sum_{i=1}^n f(\overline{x}_i)\Delta x
=\Delta x\big[f(\overline{x}_1)+\dots+f(\overline{x}_n)\big]\]
where \(\Delta x=\dfrac{b-a}{n}\) and
\(\overline{x}_i=\dfrac{1}{2}(x_{i-1}+x_i)\).

\subsubsection*{Properties of The Definite Integrals}
If \(a>b\), then
\[\int_b^a f(x)\,dx=-\int_a^b f(x)\,dx\]
If \(a=b\), then
\[\int_a^a f(x)\,dx=0\]
Properties of the Integral: Suppose all of the following integrals exist.
\begin{enumerate}
    \item \(\displaystyle{\int_a^b c\,dx=c(b-a)}\), where \(c\) is any
    constant.
    \item \(\displaystyle{\int_a^b\big[f(x)+g(x)\big]\,dx
    =\int_a^b f(x)\,dx+\int_a^b g(x)\,dx}\)
    \item \(\displaystyle{\int_a^b cf(x)\,dx=c\int_a^b f(x)\,dx}\), where
    \(c\) is any constant.
    \item \(\displaystyle{\int_a^b\big[f(x)-g(x)\big]\,dx
    =\int_a^b f(x)\,dx-\int_a^b g(x)\,dx}\)
\end{enumerate}
\[\int_a^c f(x)\,dx+\int_c^b f(x)\,dx=\int_a^b f(x)\,dx\]
Comparison Properties of the Integral:
\begin{enumerate}
    \item If \(f(x)\geq 0\) for \(a\leq x\leq b\),
    then \(\displaystyle{\int_a^b f(x)\,dx\geq 0}\).
    \item If \(f(x)\geq g(x)\) for \(a\leq x\leq b\),
    then \(\displaystyle{\int_a^b f(x)\,dx\geq \int_a^b g(x)\,dx}\).
    \item If \(m\leq f(x)\leq M\) for \(a\leq x\leq b\),
    then
    \[m(b-a)\leq\int_a^b f(x)\,dx\leq M(b-a)\]
\end{enumerate}
\subsection{Evaluating Definite Integrals}

\begin{theorem}[Evaluation Theorem]
    If \(f\) is continuous on the interval \([a,b]\), then
    \[\int_a^b f(x)\,dx=F(b)-F(a)\]
    where \(F\) is any antiderivative of \(f\), that is, \(F'=f\).
\end{theorem}
\begin{proof}
    We divide the interval \([a,b]\) into \(n\) subintervals with endpoints \\
    \(x_0=a,x_1,x_2,\cdots,x_n=b\) and with length
    \(\Delta x=\dfrac{b-a}{n}\).
    Let \(F\) be any antiderivative of \(f\).
    Then
    \begin{align*}
        & F(b)-F(a) \\
        &= F(x_n)-F(x_0) \\
        &= F(x_n)-F(x_{n-1})+F(x_{n-1})-F(x_{n-2})+\cdots+F(x_2)-F(x_1)+F(x_1)
        -F(x_0) \\
        &= \sum_{i=1}^n\big[F(x_i)-F(x_{i-1})\big]
    \end{align*}
    Now \(F\) is continuous because it's differentiable and so we can apply
    Lagrange's Mean Value Theorem to \(F\) on each subinterval
    \([x_{i-1},x_i]\).
    Thus there exists a number \(x_i^*\) between \(x_{i-1}\) and \(x_i\) such
    that
    \[F(x_i)-F(x_{i-1})=F'(x_i^*)(x_i-x_{i-1})=f(x_i^*)\Delta x\]
    Therefore
    \[F(b)-F(a)=\sum_{i=1}^n f(x_i^*)\Delta x\]
    Now we take the limit of each side of this equation as \(n\to\infty\).
    Then
    \[F(b)-F(a)=\lim_{n\to\infty}\sum_{i=1}^n f(x_i^*)\Delta x
    =\int_a^b f(x)\,dx\]
\end{proof}
When applying the Evaluation Theorem we use the notation
\[\int_a^b f(x)\,dx=F(b)-F(a)=\big[F(x)\big]_a^b=F(x)\Big|_a^b\]

\subsubsection*{Indefinite Integrals}
The notation
\[\int f(x)\,dx\]
is used for an antiderivative of \(f\) and is called an
\textbf{indefinite intergral}.
Thus
\[\int f(x)\,dx=F(x)\]
means that
\[F'(x)=f(x)\]

\subsubsection*{Applications}
\begin{theorem}[Net Change Theorem]
    The integral of a rate of change is the net change:
    \[\int_a^b F'(x)\,dx=F(b)-F(a)\]
\end{theorem}
\subsection{The Fundamental Theorem of Calculus}

\begin{theorem}[The Fundamental Theorem of Calculus]
    Suppose \(f\) is continuous on \([a,b]\).
    If the function \(F\) is defined by
    \[F(x)=\int_a^xf(t)\,dt,\quad a\leq x\leq b\]
    then \(F\) is an antiderivative of \(f\) and \[F'(x)=f(x),\quad a<x<b\]
    and therefore\[\frac{d}{dx}\int_a^x f(t)\,dt=f(x)\]
    If \(F\) is an antiderivative of \(f\) such that \(F'=f\), then
    \[\int_a^b f(x)\,dx=F(b)-F(a)\]
\end{theorem}
\subsection{The Substitution Rule}

\subsubsection*{The Substitution Rule}
\begin{theorem}[The Substitution Rule]
    If \(u=g(x)\) is a differentiable function whose range is an interval
    \(I\) and \(f\) is continuous on \(I\), then
    \[\int f(g(x))g'(x)\,dx=\int f(u)\,du\]
    If \(g'\) is continuous on \([a,b]\) and \(f\) is continuous on \(I\),
    then
    \[\int_a^b f(g(x))g'(x)\,dx=\int_{g(a)}^{g(b)} f(u)\,du\]
\end{theorem}
\begin{proof}
    Let \(F'=f\), then by the chain rule we have
    \[\int f(g(x))g'(x)\,dx=F(g(x))+C=F(u)+C=\int f(u)\,du\]
\end{proof}
\begin{problem}
    Evaluate \(\displaystyle{\int\frac{x}{\sqrt{1-4x^2}}\,dx}\).
\end{problem}
\begin{solution}
    Let \(u=1-4x^2\iff du=-8x\,dx\) so \(xdx=-\dfrac{1}{8}\,du\).
    Therefore,
    \[\int\frac{x}{\sqrt{1-4x^2}}\,dx=-\int\frac{1}{8}u^{-(1/2)}\,du
    =-\frac{1}{8}(2\sqrt{u})+C=-\frac{1}{4}\sqrt{1-4x^2}+C\]
\end{solution}
\begin{problem}
    Evaluate \(\displaystyle{\int\tan x\,dx}\).
\end{problem}
\begin{solution}
    We have \(\displaystyle{\int\tan x\,dx=\int\frac{\sin x}{\cos x}}\,dx\),
    let \(u=\cos x \iff du=-\sin\,dx\) then
    \[\int\tan x\,dx=-\int\frac{du}{u}=-\ln|u|+C=-\ln|\cos x|+C
    =\ln\frac{1}{|\sec x|}+C=\ln|\sec x|+C\]
\end{solution}
\begin{problem}
    Calculate \(\displaystyle{\int_1^e \frac{\ln x}{x}\,dx}\).
\end{problem}
\begin{solution}
    Let \(u=\ln x\iff du=\dfrac{1}{x}\,dx\), then
    \[\int_1^e \frac{\ln x}{x}\,dx=\int_0^1 u\,du
    =\left[\frac{u^2}{2}\right]_0^1=\frac{1}{2}\]
\end{solution}

\subsubsection*{Symmetry}
\begin{theorem}
    Suppose \(f\) is continuous on \([-a,a]\), then
    \begin{itemize}
        \item If \(f(-x)=f(x)\) so \(f\) is even, then
        \(\displaystyle{\int_{-a}^a f(x)\,dx=2\int_a^a f(x)\,dx}\).
        \item If \(f(-x)=-f(x)\) so \(f\) is odd, then
        \(\displaystyle{\int_{-a}^a f(x)\,dx=0}\).
    \end{itemize}
\end{theorem}

\section{Techniques of Integration}

\subsection{Integration by Parts}

Every differentiation rule has a corresponding integration rule.
For instance, the Substitution Rule for integration corresponds to the Chain
Rule for differentiation.
The rule that corresponds to the Product Rule for differentiation is called
the rule for integration by parts.
The Product Rule states that if \(f\) and \(g\) are differentiable functions,
then
\[\frac{d}{dx}\big[f(x)g(x)\big]=f'(x)g(x)+f(x)g'(x)\]
In the notation fo indefinite integrals this equation becomes
\[f(x)g(x)=\int\big[f'(x)g(x)+f(x)g'(x)\big]\,dx\]
or
\[\int f'(x)g(x)\,dx+\int f(x)g'(x)\,dx=f(x)g(x)\]
We can rearrange this equation as
\[\int f(x)g'(x)\,dx=f(x)g(x)-\int g(x)f'(x)\,dx\]
and this equation is called the \textbf{formula for integration by parts}.
Let \(u=f(x)\) and \(v=f(x)\).
Then the differentials are \(du=f'(x)\,dx\) and \(dv=g'(x)\,dx\), so, by the
Substitution Rule, the formula for integration by parts becomes
\[\int u\,dv=uv-\int v\,du\]

\begin{problem}
    Evaluate \(\displaystyle{\int\ln x\,dx}\).
\end{problem}
\begin{solution}
    Let \(u=\ln x\iff du=\dfrac{1}{x}\,dx\) and \(dv=dx\iff v=x\).
    Then
    \[\int\ln x\,dx=x\ln x-\int dx=x\ln x-x+C\]
\end{solution}
\begin{problem}
    Evaluate \(\displaystyle{\int e^x\sin x\,dx}\).
\end{problem}
\begin{solution}
    Let \(u=\sin x\iff du=\cos x\,dx\) and \(dv=e^x\,dx\iff v=e^x\).
    then
    \[\int e^x\sin x\,dx=e^x\sin x-\int e^x\cos x\,dx\]
    Let \(u=\cos x\iff du=-\sin x\,dx\) and \(dv=e^x\,dx\iff v=e^x\).
    Then
    \[\int e^x\cos x\,dx=e^x\cos x+\int e^x\sin x\,dx\]
    Therefore
    \begin{align*}
        \int e^x\sin x\,dx &= e^x\sin x-e^x\cos x-\int e^x\sin x\,dx \\
        \int e^x\sin x\,dx &= \frac{1}{2}e^x(\sin x-\cos x) 
    \end{align*}
\end{solution}
The formula of integration by parts for definite integrals is
\[\int_a^b f(x)g'(x)\,dx=\big[f(x)g(x)\big]_a^b-\int_a^b g(x)f'(x)\,dx\]
or
\[\int_a^b v\,du=\big[uv\big]_a^b-\int_a^b v\,du\]

\begin{problem}
    Evaluate \(\displaystyle{\int_0^1\arctan x\,dx}\).
\end{problem}
\begin{solution}
    Let \(u=\arctan x\iff du=\dfrac{dx}{1+x^2}\) and \(dv=dx\iff v=x\), then
    \[\int_0^1\arctan x\,dx
    =\big[x\arctan x\big]_0^1-\int_0^1\frac{x}{1+x^2}\,dx\]
    We have \(\big[x\arctan x\big]_0^1=\arctan(1)=\dfrac{\pi}{4}\).
    Let \(u=1+x^2\iff du=2x\,dx\) so \(x\,dx=\dfrac{1}{2}\,du\), then
    \[\int_0^1\frac{x}{1+x^2}\,dx=\frac{1}{2}\int_1^2\frac{du}{u}
    =\left[\frac{1}{2}\ln|u|\right]_1^2=\frac{1}{2}\ln(2)\]
    Therefore
    \[\int_0^1\arctan x\,dx=\frac{\pi}{4}-\frac{1}{2}\ln(2)\]
\end{solution}
\begin{problem}
    Prove the reduction formula
    \[\int\sin^n x\,dx
    =-\frac{1}{n}\sin^{n-1}x\cos x+\frac{n-1}{n}\int\sin^{n-2}x\,dx\]
    where \(n\geq 2\) is an integer.
\end{problem}
\begin{solution}
    Let \(u=\sin^{n-1}x\iff du=(n-1)\sin^{n-2}x\cos x\,dx\) and \\
    \(dv=\sin x\,dx\iff v=-\cos x\).
    Then
    \begin{align*}
        \int\sin^n x\,dx
        &= -\sin^{n-1}x\cos x+(n-1)\int \sin^{n-2}x\cos^2 x\,dx \\
        &= -\sin^{n-1}x\cos x+(n-1)\int \sin^{n-2}x(1-\sin^2 x)\,dx \\
        &= -\sin^{n-1}x\cos x+(n-1)\int \sin^{n-2}x\,dx-(n-1)\int\sin^n x\,dx
    \end{align*}
    Therefore
    \begin{align*}
        (n-1+1)\int\sin^n x\,dx=n\int\sin^n x\,dx
        &= -\sin^{n-1}x\cos x+(n-1)\int \sin^{n-2}x\,dx \\
        \int\sin^n x\,dx
        &= -\frac{1}{n}\sin^{n-1}x\cos x
        +\frac{n-1}{n}\int\sin^{n-2}x\,dx 
    \end{align*}
\end{solution}
\begin{problem}
    Prove the reduction formula
    \[\int\cos^n x\,dx
    =\frac{1}{n}\cos^{n-1}x\sin x+\frac{n-1}{n}\int\cos^{n-2}x\,dx\]
    where \(n\geq 2\) is an integer.
\end{problem}
\begin{solution}
    Let \(u=\cos^{n-1}x\iff du=-(n-1)\cos^{n-2}x\sin x\,dx\) and
    \(dv=\cos x\,dx\iff v=\sin x\).
    Then
    \begin{align*}
        \int\cos^n x\,dx
        &= \cos^{n-1} x\sin x+(n-1)\int\cos^{n-2}x\sin^2 x\,dx \\
        &= \cos^{n-1} x\sin x+(n-1)\int \cos^{n-2}x(1-\cos^2 x)\,dx \\
        &= \cos^{n-1} x\sin x+(n-1)\int \cos^{n-2}x\,dx-(n-1)\int\cos^n x\,dx
    \end{align*}
    Therefore
    \begin{align*}
        (n-1+1)\int\cos^n x\,dx=n\int\cos^n x\,dx
        &= \cos^{n-1}x\sin x+(n-1)\int \cos^{n-2}x\,dx \\
        \int\cos^n x\,dx
        &= \frac{1}{n}\cos^{n-1}x\sin x
        +\frac{n-1}{n}\int\cos^{n-2}x\,dx 
    \end{align*}
\end{solution}
\begin{problem}
    Show that
    \[\int_0^{\pi/2}\sin^n x\,dx=\frac{n-1}{n}\int_0^{\pi/2}\sin^{n-2}x\,dx\]
    where \(n\geq 2\) is an integer.
\end{problem}
\begin{solution}
    We use the reduction formula then
    \begin{align*}
        \int_0^{\pi/2}\sin^n x\,dx
        &= \left[-\frac{1}{n}\sin^{n-1} x\cos x\right]_0^{\pi/2}
        +\frac{n-1}{n}\int_0^{\pi/2}\sin^{n-2} x\,dx \\
        &= -\frac{1}{n}\left(\sin^{n-1}\left(\frac{\pi}{2}\right)
        \cos\left(\frac{\pi}{2}\right)-\sin^{n-1}(0)\cos(0)\right)
        +\frac{n-1}{n}\int_0^{\pi/2}\sin^{n-2} x\,dx
    \end{align*}
    and since \(\sin(0)=0\) and \(\cos(\pi/2)=0\) so
    \[\int_0^{\pi/2}\sin^n x\,dx=-\frac{1}{n}(0)
    +\frac{n-1}{n}\int_0^{\pi/2}\sin^{n-2} x\,dx
    =\frac{n-1}{n}\int_0^{\pi/2}\sin^{n-2} x\,dx\]
\end{solution}
\begin{problem}
    Show that
    \[\int_0^{\pi/2}\sin^{2n+1} x\,dx
    =\frac{2\cdot4\cdot6\cdot\cdots\cdot2n}
    {3\cdot5\cdot7\cdot\cdots\cdot(2n+1)}\]
    where \(n\geq 2\) is an integer.
\end{problem}
\begin{solution}
    By the reduction formula
    \begin{align*}
        \int_0^{\pi/2}\sin^{2n+1} x\,dx
        &= \frac{2n}{2n+1}\int_{0}^{\pi/2}\sin^{2n-1} x\,dx
        =\frac{(2n)(2n-2)}{(2n+1)(2n-1)}\int_{0}^{\pi/2}\sin^{2n-3} x\,dx \\
        &= \frac{(2n)(2n-2)\cdots(6)(4)(2)}{(2n+1)(2n-1)\cdots(7)(5)(3)}
        \int_{0}^{\pi/2}\sin x\,dx
    \end{align*}
    and since
    \[\int_{0}^{\pi/2}\sin x\,dx=\Big[-\cos x\Big]_0^{\pi/2}
    =-\cos\left(\frac{\pi}{2}\right)+\cos(0)=0+1=1\]
    hence
    \[\int_0^{\pi/2}\sin^{2n+1} x\,dx
    =\frac{2\cdot4\cdot6\cdot\cdots\cdot2n}
    {3\cdot5\cdot7\cdot\cdots\cdot(2n+1)}\]
\end{solution}
\begin{problem}
    Show that
    \[\int_0^{\pi/2}\sin^{2n} x\,dx
    =\frac{1\cdot3\cdot5\cdot\cdots\cdot(2n-1)}
    {2\cdot4\cdot6\cdot\cdots\cdot2n}\frac{\pi}{2}\]
    where \(n\geq 2\) is an integer.
\end{problem}
\begin{solution}
    \begin{align*}
        \int_0^{\pi/2}\sin^{2n} x\,dx
        &= \frac{2n-1}{2n}\int_0^{\pi/2}\sin^{2n-2} x\,dx
        =\frac{(2n-1)(2n-3)}{(2n)(2n-2)}\int_0^{\pi/2}\sin^{2n-4} x\,dx \\
        &= \frac{(2n-1)(2n-3)\cdots(5)(3)(1)}{(2n)(2n-2)\cdots(6)(4)(2)}
        \int_0^{\pi/2}\,dx=\frac{1\cdot3\cdot5\cdot\cdots\cdot(2n-1)}
        {2\cdot4\cdot6\cdot\cdots\cdot2n}\frac{\pi}{2} 
    \end{align*}
\end{solution}
\begin{problem}
    Prove the reduction formula
    \[\int(\ln x)^n\,dx=x(\ln x)^n-n\int(\ln x)^{n-1}\,dx\]
\end{problem}
\begin{solution}
    Let \(u=(\ln x)^n\iff du=\dfrac{n}{x}(\ln x)^{n-1}\) and \(dv=dx\iff v=x\)
    then
    \[\int(\ln x)^n\,dx=x(\ln x)^n-\int x\cdot\frac{n}{x}(\ln x)^{n-1}\,dx
    =x(\ln x)^n-n\int(\ln x)^{n-1}\,dx\]
\end{solution}
\begin{problem}
    Prove the reduction formula
    \[\int x^n e^x\,dx=x^n e^x-n\int x^{n-1}e^x\,dx\]
\end{problem}
\begin{solution}
    Let \(u=x^n\iff du=nx^{n-1}\,dx\) and \(dv=e^x\,dx\iff v=e^x\) then
    \[\int x^n e^x\,dx=x^n e^x-\int nx^{n-1}e^x\,dx
    =x^n e^x-n\int x^{n-1}e^x\,dx\]
\end{solution}
\subsection{Trigonometric Integrals and Substitutions}

\subsubsection*{Trigonometric Integrals}
We can use trigonometric identities to integrate trigonometric integrals.
\begin{problem}
    Evaluate \(\displaystyle{\int\sin 5x\sin 2x\,dx}\).
\end{problem}
\begin{solution}
    \begin{align*}
        \int\sin 5x\sin 2x\,dx &= \int\frac{1}{2}(\cos(5x-2x)-\cos(5x+2x))\,dx
        =\frac{1}{2}\int(\cos 3x-\cos 7x)\,dx \\
        &= \frac{1}{2}\left(\frac{1}{3}\sin 3x-\frac{1}{7}\sin 7x\right)+C
        =\frac{1}{6}\sin 3x-\frac{1}{14}\sin 7x+C
    \end{align*}
\end{solution}
\begin{problem}
    Evaluate \(\displaystyle{\int\sin 3x\cos x\,dx}\).
\end{problem}
\begin{solution}
    \begin{align*}
        \int\sin 3x\cos x\,dx &= \int\frac{1}{2}(\sin(3x+x)+\sin(3x-x))\,dx
        =\frac{1}{2}\int(\sin 4x+\sin 2x)\,dx \\
        &= \frac{1}{2}\left(-\frac{1}{4}\cos 4x-\frac{1}{2}\cos 2x\right)+C
        =-\frac{1}{8}\cos 4x-\frac{1}{4}\cos 2x+C
    \end{align*}
\end{solution}
\begin{problem}
    Evaluate \(\displaystyle{\int\cos^3 x\,dx}\).
\end{problem}
\begin{solution}
    \[\int\cos^3 x\,dx=\int(1-\sin^2 x)\cos x\,dx=\int(1-u^2)\,du
    =u-\frac{u^3}{3}+C=\sin x-\frac{1}{3}\sin^3 x+C\]
\end{solution}
\begin{problem}
    Evaluate \(\displaystyle{\int\sin^5 x\cos^2 x\,dx}\).
\end{problem}
\begin{solution}
    \begin{align*}
        \int\sin^5 x\cos^2 x\,dx &= \int(1-\cos^2 x)^2\cos^2 x\sin x\,dx
        =-\int(1-u^2)^2u^2\,du \\
        &= -\int(u^2-2u^4+u^6)\,du
        =-\frac{1}{3}u^3+\frac{2}{5}u^5-\frac{1}{7}u^7+C \\
        &= -\frac{1}{3}\cos^3 x+\frac{2}{5}\cos^5 x-\frac{1}{7}\cos^7 x
        +C
    \end{align*}
\end{solution}
\begin{problem}
    Evaluate \(\displaystyle{\int\cos^2 x\,dx}\).
\end{problem}
\begin{solution}
    \[\int\cos^2 x\,dx=\int\frac{1}{2}(1+\cos 2x)\,dx
    =\frac{1}{2}\int(1+\cos 2x)\,dx
    =\frac{1}{2}\left(x+\frac{1}{2}\sin 2x\right)+C\]
\end{solution}
\begin{problem}
    Evaluate \(\displaystyle{\int_0^\pi\sin^2 x\,dx}\).
\end{problem}
\begin{solution}
    \begin{align*}
    \int_0^\pi\sin^2 x\,dx &= \int_0^\pi\frac{1}{2}(1-\cos 2x)\,dx
    =\frac{1}{2}\int_0^{\pi}(1+\cos 2x)\,dx \\
    &=\frac{1}{2}
    \left(\bigl[x\bigr]_0^\pi-\frac{1}{2}\bigl[\sin 2x\bigr]_0^\pi\right)
    =\frac{\pi}{2}
    \end{align*}
\end{solution}
\begin{problem}
    Evaluate \(\displaystyle{\int\sin^4 x\,dx}\).
\end{problem}
\begin{solution}
    \begin{align*}
        \int\sin^4 x\,dx &= \int\left(\frac{1}{2}(1-\cos 2x)\right)^2\,dx
        =\frac{1}{4}\int(1-2\cos 2x+\cos^2 2x)\,dx \\
        &= \frac{1}{4}
        \left(x-\sin 2x+\frac{1}{2}
        \left(x+\frac{1}{4}\sin 4x\right)\right)+C \\
        &= \frac{1}{4}
        \left(\frac{3}{2}x-\sin 2x+\frac{1}{8}\sin 4x\right)+C
    \end{align*}
\end{solution}
In general, an integral of powers of \(\sin x\) and \(\cos x\) is in the
form
\[\int\sin^m x\cos^n x\,dx\]
where \(m,n\in\Z\) and \(m,n\geq 0\).
If \(m\) is odd, then we save a factor of \(\sin x\) and express the rest in
terms of \(\cos x\) for substitution.
If \(n\) is odd, then we save a factor of \(\cos x\) and express the rest in
terms of \(\sin x\) for substitution.
If \(m\) and \(n\) are even, then we use the power reduction formulas.
\begin{problem}
    Evaluate \(\displaystyle{\int\tan^6 x\sec^4 x\,dx}\).
\end{problem}
\begin{solution}
    \begin{align*}
        \int\tan^6 x\sec^4 x\,dx &= \int\tan^6 x(\tan^2 x+1)\sec^2 x\,dx
        =\int u^6(u^2+1)\,du=\int u^8+u^6 \\
        & =\frac{u^9}{9}+\frac{u^7}{7}
        =\frac{1}{9}\tan^9 x+\frac{1}{7}\tan^7 x+C
    \end{align*}
\end{solution}
\begin{problem}
    Evaluate \(\displaystyle{\int\tan^5 x\sec^7 x\,dx}\).
\end{problem}
\begin{solution}
    \begin{align*}
        \int\tan^5 x\sec^7 x\,dx
        &= \int(\sec^2 x-1)^2\sec^6 x\sec x\tan x\,dx
        =\int(u^2-1)^2u^6\,du \\
        &= \int(u^{10}-2u^8+u^6)\,dx
        =\frac{1}{11}u^{11}-\frac{2}{9}u^9+\frac{1}{7}u^7+C \\
        &= \frac{1}{11}\sec^{11}x-\frac{2}{9}\sec^9 x+\frac{1}{7}\sec^7 x
        +C
    \end{align*}
\end{solution}
In general, an integral of powers of \(\tan x\) and \(\sec x\) is in the
form
\[\int\tan^m x\sec^n x\,dx\]
where \(m,n\in\Z\) and \(m,n\geq 0\).
If \(m\) is odd, then we save a factor of \(\sec x\tan x\) and express the
rest in terms of \(\sec x\) for substitution.
If \(n\) is even, then we save a factor of \(\sec^2 x\) and express the
rest in terms of \(\tan x\) for substitution.
\begin{problem}
    Evaluate \(\displaystyle{\int\sec x\,dx}\).
\end{problem}
\begin{solution}
    We have
    \[\int\sec x\,dx=\int\sec x\frac{\sec x+\tan x}{\sec x+\tan x}\,dx
    =\int\frac{\sec^2 x+\sec x\tan x}{\sec x+\tan x}\,dx\]
    Let \(u=\sec x+\tan x\iff du=(\sec x\tan x+\sec^2 x)\,dx\), then
    \[\int\sec x\,dx=\int\frac{\sec x\tan x+\sec^2 x}{\sec x+\tan x}\,dx
    =\int\frac{du}{u}=\ln|u|+C=\ln|\sec x+\tan x|+C\]
\end{solution}
\begin{problem}
    Evaluate \(\displaystyle{\int\tan^3 x\,dx}\).
\end{problem}
\begin{solution}
    \begin{align*}
    \int\tan^3 x\,dx &= \int\tan x(\sec^2x-1)\,dx
    =\int\tan x\sec^2\,dx-\int\tan x\,dx \\
    &= \frac{1}{2}\tan^2 x-\ln|\sec x|+C
    \end{align*}
\end{solution}
\begin{problem}
    Evaluate \(\displaystyle{\int\sec^3 x\,dx}\).
\end{problem}
\begin{solution}
    Let \(u=\sec x\iff du=\sec x\tan x\) and \(dv=\sec^2 x\,dx\iff v=\tan x\),
    then
    \[\int\sec^3 x\,dx=\sec x\tan x-\int\tan^2 x\sec x\,dx\]
    We have
    \[\int\tan^2 x\sec x\,dx=\int(\sec^2 x-1)\sec x\,dx
    =\int\sec^3 x\,dx-\int\sec x\,dx\]
    Therefore,
    \begin{align*}
        \int\sec^3 x\,dx &= \sec x\tan x-\int\sec^3 x\,dx+\int\sec x\,dx \\
        \int\sec^3 x\,dx
        &= \frac{1}{2}(\sec x\tan x+\ln|\sec x+\tan x|)+C
    \end{align*}
\end{solution}

\subsubsection*{Trigonometric Substitutions}
If an integral has the form \(\displaystyle{\int\sqrt{a^2-x^2}\,dx}\),
then we can use the substitution \(x=a\sin\theta\) where
\(-\pi/2\leq\theta\leq\pi/2\) to get
\[\sqrt{a^2-x^2}=\sqrt{a^2-a^2\sin^2\theta}=\sqrt{a^2(1-\sin^2\theta)}
=\sqrt{a^2\cos^2\theta}=|a|\cos\theta\]
If an integral has the form \(\displaystyle{\int\sqrt{a^2+x^2}\,dx}\),
then we can use the substitution \(x=a\tan\theta\) where
\(-\pi/2<\theta<\pi/2\) to get
\[\sqrt{a^2+x^2}=\sqrt{a^2+a^2\tan^2\theta}=\sqrt{a^2(\tan^2\theta+1)}
=\sqrt{a^2\sec^2\theta}=|a|\sec\theta\]
If an integral has the form \(\displaystyle{\int\sqrt{x^2-a^2}\,dx}\),
then we can use the substitution \(x=a\sec\theta\) where
\(0\leq\theta<\pi/2\) or \(\pi\leq\theta<3\pi/2\) to get
\[\sqrt{x^2-a^2}=\sqrt{a^2\sec^2\theta-a^2}=\sqrt{a^2(\sec^2\theta-1)}
=\sqrt{a^2\tan^2\theta}=|a|\tan\theta\]
\begin{problem}
    Evaluate \(\displaystyle{\int\frac{\sqrt{9-x^2}}{x^2}\,dx}\)
\end{problem}
\begin{solution}
    Let \(x=3\sin\theta\iff dx=3\cos\theta\,d\theta\), then
    \begin{align*}
        \int\frac{\sqrt{9-x^2}}{x^2}\,dx
        &= \int\frac{3\cos\theta}{9\sin^2\theta}3\cos\theta\,d\theta
        =\int\frac{\cos^2\theta}{\sin^2\theta}=\int\cot^2\theta\,d\theta
        =\int(\csc^2\theta-1)\,d\theta \\
        &=-\cot\theta-\theta+C
        =-\frac{\sqrt{9-x^2}}{x}-\arcsin\left(\frac{x}{3}\right)+C
    \end{align*}
\end{solution}
\begin{problem}
    Evaluate \(\displaystyle{\int\frac{dx}{x^2\sqrt{x^2+4}}}\).
\end{problem}
\begin{solution}
    Let \(x=2\tan\theta\iff dx=2\sec^2\theta\,d\theta\) and
    \(-\pi/2<\theta<\pi/2\), then
    \begin{align*}
        \int\frac{dx}{x^2\sqrt{x^2+4}}
        &= \int\frac{2\sec^2\theta}{4\tan^2\theta(2\sec\theta)}\,d\theta
        =\frac{1}{4}\int\frac{\sec\theta}{\tan^2\theta}\,d\theta
        =\frac{1}{4}\int\frac{\cos\theta}{\sin^2\theta}\,d\theta
        =\frac{1}{4}\int\frac{du}{u^2} \\
        &= -\frac{1}{4u}+C=-\frac{1}{4\sin\theta}+C
        =-\frac{\sqrt{x^2+4}}{4x}+C
    \end{align*}
\end{solution}
\begin{problem}
    Find the area enclosed by a circle with radius \(r\).
\end{problem}
\begin{solution}
    The area is \(\displaystyle{A=4\int_0^r\sqrt{r^2-x^2}\,dx}\).
    Let \(x=r\sin\theta\iff dx=r\cos\theta\,d\theta\) and
    \(0\leq\theta\leq\pi/2\), then
    \begin{align*}
        A &= 4\int_0^{\pi/2}r\cos\theta(r\cos\theta)\,d\theta
        =4r^2\int_0^{\pi/2}\cos^2\theta\,d\theta
        =4r^2\left(\frac{1}{2}\left(\bigl[\theta\bigr]_0^{\pi/2}
        +\frac{1}{2}\bigl[\sin2\theta\bigr]_0^{\pi/2}\right)\right) \\
        &= 4r^2\left(\frac{\pi}{4}\right)=\pi r^2
    \end{align*}
\end{solution}
\begin{problem}
    Find the area enclosed by the ellipse
    \(\displaystyle{\frac{x^2}{a^2}+\frac{y^2}{b^2}}=1\).
\end{problem}
\begin{solution}
    The area is \(\displaystyle{A=4\int_0^a\frac{b}{a}\sqrt{a^2-x^2}\,dx}\).
    Let \(x=a\sin\theta\iff dx=a\cos\theta\,d\theta\) and
    \(0\leq\theta\leq\pi/2\), then
    \[A=4\int_0^{\pi/2}\frac{b}{a}(a\cos\theta)(a\cos\theta)\,d\theta
    =4ab\int_0^{\pi/2}\cos^2\theta\,d\theta
    =4ab\left(\frac{\pi}{4}\right)=\pi ab\]
\end{solution}
\subsection{Partial Fractions}

Consider a rational function \(\displaystyle{f(x)=\frac{P(x)}{Q(x)}}\) where
\(P\) and \(Q\) are polynomials.
If \(f\) is improper, that is, \(\deg(P)\geq\deg(Q)\), then we simplify to get
\(\displaystyle{f(x)=S(x)+\frac{R(x)}{Q(x)}}\) where \(S\) and \(R\) are
polynomials.
Then we can factor \(Q\) to be irreducible and express \(\dfrac{R(x)}{Q(x)}\)
as the a sum of \textbf{partial fractions} of the form
\[\frac{A}{(ax+b)^i}\]
or
\[\frac{Ax+B}{(ax^2+bx+c)^j}\]
There are four possible cases:
\begin{enumerate}
    \item \(Q\) is a product of distinct linear factors.
    Then
    \[Q(x)=(a_1 x+b_1)(a_2 x+b_2)\cdots(a_k x+b_k)\]
    and therefore
    \[\frac{R(x)}{Q(x)}=\frac{A_1}{a_1 x+b_1}+\frac{A_2}{a_2 x+b_2}+\cdots+
    \frac{A_k}{a_k x+b_k}\]
    \begin{problem}
        Evaluate \(\displaystyle{\int\frac{x^2+2x-1}{2x^3+3x^2-2x}\,dx}\).
    \end{problem}
    \begin{solution}
        We simplify to get \(2x^3+3x^2-2x=x(2x-1)(x+2)\) then
        \begin{align*}
            \frac{x^2+2x-1}{x(2x-1)(x+2)} &=
            \frac{A}{x}+\frac{B}{2x-1}+\frac{C}{x+2} \\
            x^2+2x-1 &= A(2x-1)(x+2)+Bx(x+2)+Cx(2x-1) \\
            x=0\iff -1=-2A &\iff A=\frac{1}{2} \\
            x=\frac{1}{2}\iff \frac{1}{4}=\frac{5}{4}B &\iff B=\frac{1}{5} \\
            x=-2\iff -1=10C &\iff C=-\frac{1}{10}
        \end{align*}
        and therefore
        \begin{align*}
            \int\frac{x^2+2x-1}{2x^3+3x^2-2x}
            &= \frac{1}{2}\int\frac{dx}{x}+\frac{1}{5}\int\frac{dx}{2x-1}
            -\frac{1}{10}\int\frac{dx}{x+2} \\
            &= \frac{1}{2}\ln|x|+\frac{1}{10}\ln|2x-1|
            -\frac{1}{10}\ln|x+2|+K
        \end{align*}
    \end{solution}    
    \item \(Q\) is a product of linear factors where some are repeated.
    Suppose that the first linear factor is repeated \(r\) times, then we have
    \[\frac{A_1}{a_1 x+b_1}+\frac{A_2}{(a_1 x+b_1)^2}+\cdots+
    \frac{A_r}{(a_1 x+b_1)^r}\]
    for the first repeated linear factor and similarly for other repeated
    linear factors.
    \begin{problem}
        Evaluate \(\displaystyle{\int\frac{x^4-2x^2+4x+1}{x^3-x^2-x+1}\,dx}\).
    \end{problem}
    \begin{solution}
        We simplify to get
        \begin{align*}
            x^4-2x^2+4x+1 &=(x^2-1)^2+4x=(x+1)^2(x-1)^2+4x \\
            x^3-x^2-x+1 &= x^2(x-1)-(x-1)=(x^2-1)(x-1)=(x+1)(x-1)^2
        \end{align*}
        and so
        \begin{align*}
            \int\frac{x^4-2x^2+4x+1}{x^3-x^2-x+1}\,dx
            &= \int(x+1)\,dx+\int\frac{4x}{(x+1)(x-1)^2}\,dx
        \end{align*}
        Then
        \begin{align*}
            \frac{4x}{(x+1)(x-1)^2} &=
            \frac{A}{x+1}+\frac{B}{x-1}+\frac{C}{(x-1)^2} \\
            4x &= A(x-1)^2+B(x+1)(x-1)+C(x+1) \\
            x=-1\iff -4=4A &\iff A=-1 \\ x=1\iff 4=2C &\iff C=2 \\
            x=0\iff 0=-1-B+2 &\iff B=1
        \end{align*}
        and therefore
        \begin{align*}
            \int\frac{x^4-2x^2+4x+1}{x^3-x^2-x+1}\,dx
            &= \int(x+1)\,dx-\int\frac{dx}{x+1}+\int\frac{dx}{x-1}
            +2\int\frac{dx}{(x-1)^2} \\
            &= \frac{x^2}{2}+x-\ln|x+1|+\ln|x-1|-\frac{2}{x-1}+K
        \end{align*}
    \end{solution}    
    \item \(Q\) has irreducible quadratic factors without repeated
    factors.
    Then for every quadratic factor we have
    \[\frac{Ax+B}{ax^2+bx+c}\]
    \begin{problem}
        Evaluate \(\displaystyle{\int\frac{2x^2-x+4}{x^3+4x}\,dx}\).
    \end{problem}
    \begin{solution}
        We do partial fraction decomposition then
        \begin{align*}
            \frac{2x^2-x+4}{x^3+4x} &= \frac{2x^2-x+4}{x(x^2+4)}
            =\frac{A}{x}+\frac{Bx+C}{x^2+4} \\
            2x^2-x+4 &= A(x^2+4)+(Bx+C)x=(A+B)x^2+Cx+4A
        \end{align*}
        \[A=1, \quad B=1,\quad C=-1\]
        and so
        \begin{align*}
            \int\frac{2x^2-x+4}{x^3+4x}\,dx
            &= \int\frac{dx}{x}+\int\frac{x-1}{x^2+4}\,dx
            =\ln|x|+\frac{1}{2}\ln|x^2+4|
            +\frac{1}{2}\arctan\left(\frac{x}{2}\right)+K
        \end{align*}
    \end{solution}
    Note that
    \[\int \frac{dx}{x^2+a^2}=\frac{1}{a}\arctan\left(\frac{x}{a}\right)+C\]
    \begin{problem}
        Evaluate \(\displaystyle{\int\frac{4x^2-3x+2}{4x^2-4x+3}\,dx}\).
    \end{problem}
    \begin{solution}
        We simplify to get
        \[\int\frac{4x^2-3x+2}{4x^2-4x+3}\,dx
        =\int\frac{4x^2-4x+3+x-1}{4x^2-4x+3}\,dx
        =\int\left(1+\frac{x-1}{(2x-1)^2+2}\right)\,dx\]
        then
        \begin{align*}
            \int\frac{x-1}{(2x-1)^2+2}\,dx
            &= \frac{1}{2}\int\frac{\frac{1}{2}(u+1)-1}{u^2+2}\,dx
            =\frac{1}{4}\int\frac{u-1}{u^2+2}\,dx \\
            &= \frac{1}{4}\left(\frac{1}{2}\ln(u^2+2)
            -\frac{\sqrt{2}}{2}\arctan\left(\frac{\sqrt{2}}{2}u\right)\right) \\
            &= \frac{1}{8}\ln((2x-1)^2+2)
            -\frac{\sqrt{2}}{8}\arctan\left(\frac{\sqrt{2}}{2}(2x-1)\right)
        \end{align*}
        and so
        \[\int\frac{4x^2-3x+2}{4x^2-4x+3}\,dx
        =x+\frac{1}{8}\ln(4x^2-4x+3)
        -\frac{\sqrt{2}}{8}\arctan\left(\frac{\sqrt{2}}{2}(2x-1)\right)
        +C\]
    \end{solution}    
    \item \(Q\) has a repeated irreducible quadratic factor.
    Suppose that the first quadratic factor is repeated \(r\) times,
    then we have
    \[\frac{A_1 x+B_1}{ax^2+bx+c}+\frac{A_2 x+B_2}{(ax^2+bx+c)^2}+\cdots
    +\frac{A_r x+B_r}{(ax^2+bx+c)^r}\]
    for the first repeated quadratic factor and similarly for the others.
    \begin{problem}
        Evaluate \(\displaystyle{\int\frac{1-x+2x^2-x^3}{x(x^2+1)^2}\,dx}\).
    \end{problem}
    \begin{solution}
        We do partial fraction decomposition then
        \begin{align*}
            \frac{1-x+2x^2-x^3}{x(x^2+1)^2}
            &= \frac{A}{x}+\frac{Bx+C}{x^2+1}+\frac{Dx+E}{(x^2+1)^2} \\
            1-x+2x^2-x^3 &= A(x^2+1)^2+(Bx+C)x(x^2+1)+(Dx+E)x
        \end{align*}
        \[x=0\iff A=1\]
        and so
        \begin{align*}
            1-x+2x^2-x^3 &= x^4+2x^2+1+Bx^4+Bx^2+Cx^3+Cx+Dx^2+Ex \\
            -x-x^3 &= (B+1)x^4+Cx^3+(B+D)x^2+(C+E)x
        \end{align*}
        \[\quad B=-1\quad C=-1\quad D=1\quad E=0\]
        Therefore
        \begin{align*}
            \int\frac{1-x+2x^2-x^3}{x(x^2+1)^2}\,dx
            &= \int\frac{dx}{x}-\int\frac{x+1}{x^2+1}\,dx
            +\int\frac{x}{(x^2+1)^2}\,dx \\
            &= \ln|x|-\frac{1}{2}\ln(x^2+1)-\arctan x
            -\frac{1}{2(x^2+1)}+K
        \end{align*}
    \end{solution}
\end{enumerate}
\subsection{Approximate Integration}

Most of the functions we study here in calculus are
\textbf{elementary functions}.
These are the polynomials, rational functions, power functions, exponential
functions, logarithmic functions, trigonometric and inverse trigonometric
functions, and all functions that can be obtained from these by the five
operations of addition, subtraction, multiplication, division, and
composition.
If \(f\) is an elementary function, then \(f'\) is an elementary function but
\(\displaystyle{\int f(x)\,dx}\) need not be an elementary function.
Consider \(f(x)=e^{x^2}\).
Since \(f\) is continuous, its integral exists, and if we define the function
\(F\) by
\[F(x)=\int_0^x e^{t^2}\,dt\]
that we know that by the Fundamental Theorem of Calculus that
\[F'(x)=e^{x^2}\]
Thus \(f(x)=e^{x^2}\) has an antiderivative \(F\), but it can be proved that
\(F\) is not an elementary function.
This means that we cannot evaluate \(\displaystyle{\int e^{x^2}\,dx}\) in
terms of elementary functions.
The same can be said of the following integrals:
\begin{align*}
    & \int\frac{e^x}{x}\,dx && \int\sin(x^2)\,dx & \int\cos(e^x)\,dx \\
    & \int\sqrt{x^3+1}\,dx && \int\frac{1}{\ln x}\,dx & \int\frac{\sin x}{x}\,dx
\end{align*}
In fact, the majority of elementary functions do not have elementary
antiderivatives. \\
Trapezoidal Rule:
\[\int_a^b f(x)\,dx\approx T_n
=\frac{\Delta x}{2}\big[f(x_0)+2f(x_1)+f(x_2)+\cdots+2f(x_{n-1})+f(x_n)\big]\]
where \(\Delta x=\dfrac{b-a}{n}\) and \(x_i=a+i\Delta x\). \\
Error Bounds: Suppose \(|f''(x)|\leq K\) for \(a\leq x\leq b\).
If \(E_T\) and \(E_M\) are the errors in the Trapezoidal and Midpoint Rules,
then
\[|E_T|\leq\frac{K(b-a)^3}{12n^2}\]
and
\[|E_M|\leq\frac{K(b-a)^3}{24n^2}\]

\subsubsection*{Simpson's Rule}
Simpson's Rule:
\[\int_a^b f(x)\,dx\approx S_n
=\frac{\Delta x}{3}\big[f(x_0)+4f(x_1)+2f(x)2+4f(x_3)+\cdots+2f(x_{n-2})
+4f(x_{n-1})+f(x_n)\big]\]
where \(n\) is even and \(\Delta x=\dfrac{b-a}{n}\). \\
Error Bounds for Simpson's Rule: Suppose \(|f^{(4)}(x)|\leq K\) for
\(a\leq x\leq b\).
If \(E_S\) is the error involved in using Simpson's Rule, then
\[|E_S|\leq\frac{K(b-a)^5}{180n^4}\]
\subsection{Improper Integrals}
We extend the concept of a definite integral
\(\displaystyle{\int_a^b f(x)\,dx}\) to the case where the interval is
infinite and also to the case where \(f\) has an infinite discontinuity in
\([a,b]\).
In either case the integral is called an \textbf{improper integral}.

\subsubsection*{Infinite Intervals}
\begin{definition}
    \begin{enumerate}
        \item If \(\displaystyle{\int_a^t f(x)\,dx}\) exists for every number
        \(t\geq a\), then
        \[\int_a^\infty f(x)\,dx=\lim_{t\to\infty}\int_a^t f(x)\,dx\]
        provided this limit exists (as a finite number).
        \item If \(\displaystyle{\int_t^b f(x)\,dx}\) exists for every number
        \(t\leq b\), then
        \[\int_{-\infty}^b f(x)\,dx=\lim_{t\to -\infty}\int_t^b f(x)\,dx\]
        provided this limit exists (as a finite number).
    \end{enumerate}
    The improper integrals \(\displaystyle{\int_a^\infty f(x)\,dx}\) and
    \(\displaystyle{\int_{-\infty}^b f(x)\,dx}\) are called
    \textbf{convergent} if the corresponding limit exists and
    \textbf{divergent} if the limit does not exist. \\
    If both \(\displaystyle{\int_a^\infty f(x)\,dx}\) and
    \(\displaystyle{\int_{-\infty}^a f(x)\,dx}\) are convergent, then we define
    \[\int_{-\infty}^{\infty}f(x)\,dx
    =\int_{-\infty}^a f(x)\,dx+\int_a^\infty f(x)\,dx\]
    for any real number \(a\).
\end{definition}
\(\displaystyle{\int_1^\infty \frac{1}{x^p}\,dx}\) is convergent if \(p>1\)
and divergent if \(p\leq 1\).

\subsubsection*{Discontinuous Integrands}
\begin{definition}
    \begin{enumerate}
        \item If \(f\) is continuous on \([a,b)\) and is discontinuous at
        \(b\), then
        \[\int_a^b f(x)\,dx=\lim_{t\to b^-}\int_a^t f(x)\,dx\]
        if the limit exists.
        \item If \(f\) is continuous on \((a,b]\) and is discontinuous at
        \(a\), then
        \[\int_a^b f(x)\,dx=\lim_{t\to a^+}\int_t^b f(x)\,dx\]
        if the limit exists.
    \end{enumerate}
    The improper integral \(\displaystyle{\int_a^b f(x)\,dx}\) is
    \textbf{convergent} if the corresponding limit exists and
    \textbf{divergent} if the limit does not exist. \\
    If \(f\) has a discontinuity at \(c\), where \(a<c<b\), and both
    \(\displaystyle{\int_a^c f(x)\,dx}\) and
    \(\displaystyle{\int_c^b f(x)\,dx}\) are convergent,
    then
    \[\int_a^b f(x)\,dx=\int_a^c f(x)\,dx+\int_c^b f(x)\,dx\]
\end{definition}

\subsubsection*{Comparison Test for Improper Integrals}
\begin{theorem}[Comparison Theorem]
    Suppose that \(f\) and \(g\) are continuous functions with
    \(f(x)\geq g(x)\geq 0\) for \(x\geq a\).
    \begin{enumerate}
        \item If \(\displaystyle{\int_a^\infty f(x)\,dx}\) is convergent, then
        \(\displaystyle{\int_a^\infty g(x)\,dx}\) is convergent.
        \item If \(\displaystyle{\int_a^\infty g(x)\,dx}\) is divergent, then
        \(\displaystyle{\int_a^\infty f(x)\,dx}\) is divergent.
    \end{enumerate}    
\end{theorem}

\section{Applications of Integration}

\subsection{Areas between Curves}

Consider the region \(S\) that lies between two curves \(y=f(x)\) and
\(y=g(x)\) and between the vertical lines \(x=a\) and \(x=b\), where \(f\)
and \(g\) are continuous functions and \(f(x)\geq g(x)\) for all \(x\) in
\([a,b]\).
We define the \textbf{area} \(A\) of \(S\) as as the limiting value of the sum
of the areas of the approximating rectangles.
\[A=\lim_{n\to\infty}\sum_{i=1}^n\big[f(x_i^*)-g(x_i^*)\big]\Delta x\]
The area \(A\) of the region bounded by the curves \(y=f(x),y=g(x)\) and the
lines \(x=a,x=b\), where \(f\) and \(g\) are continuous and \(f(x)\geq g(x)\)
for all \(x\) in \([a,b]\), is
\[A=\int_a^b \big[f(x)-g(x)\big]\,dx\]
If a region is bounded by curves with equations \(x=f(y),x=g(y),y=c,y=d\),
where \(f\) and \(g\) are continuous and \(f(y)\geq g(y)\) for
\(c\leq y\leq d\), then its area is
\[A=\int_c^d \big[f(y)-g(y)\big]\,dy\]
\subsection{Volumes}

\begin{definition}
    Let \(S\) be a solid that lies between \(x=a\) and \(x=b\).
    If the cross-sectional area of \(S\) in the plane \(P_x\), through \(x\)
    and perpendicular to the \(x\)-axis, is \(A(x)\), where \(A\) is an
    integrable function, then the \textbf{volume} of \(S\) is
    \[V=\lim_{\max\Delta x_i\to 0}\sum_{i=1}^n A(x_i^*)\Delta x_i
    =\int_a^b A(x)\,dx\]
\end{definition}
In general, we calculate the volume of a \textbf{solid of revolution} obtained
by revolving a region about a line by using the basic defining formula
\[V=\int_a^b A(x)\,dx\]
or
\[V=\int_c^d A(y)\,dy\]

\subsubsection*{Volumes by Cylindrical Shells}

The volume of the solid \(S\) obtained by rotating about the \(y\)-axis the
region under the curve \(y=f(x)\) from \(a\) to \(b\), is
\[V=\int_a^b 2\pi xf(x)\,dx\]
where \(0\leq a<b\).
\subsection{Arc Length}

Suppose that a curve \(C\) is defined by the equation \(y=f(x)\), where \(f\)
is continuous and \(a\leq x\leq b\).
We define the \textbf{length} \(L\) of the curve \(C\) with equation
\(y=f(x),a\leq x\leq b\), as the limit of the lengths of the inscribed
polygons (if the limit exists):
\[L=\lim_{n\to\infty}\sum_{i=1}^n |P_{i-1}P_i|\]
We can derive an integral formula for \(L\) in the case where \(f\) has a
continuous derivative.
Such a function \(f\) is called \textbf{smooth} because a small change in
\(x\) produces a small change in \(f'(x)\).
\[L=\lim_{n\to\infty}\sum_{i=1}^n\sqrt{1+\big[f'(x_i^*)\big]^2}\Delta x\]
The \textbf{Arc Length Formula}: If \(f'\) is continuous on \([a,b]\), then
the length of the curve \(y=f(x),a\leq x\leq b\), is
\[L=\int_a^b\sqrt{1+\big[f'(x)\big]^2}\,dx\]
In Leibniz's notation:
\[L=\int_a^b\sqrt{1+\left(\frac{dy}{dx}\right)^2}\,dx\]
If a curve has the equation \(x=g(y),c\leq y\leq d\), and \(g'(y)\) is
continuous, then the Arc Length Formula is:
\[L=\int_c^d\sqrt{1+\big[g'(y)\big]^2}\,dy
=\int_c^d\sqrt{1+\left(\frac{dx}{dy}\right)^2}\,dy\]

\subsubsection*{The Arc Length Function}
If a smooth curve \(C\) has the equation \(y=f(x),a\leq x\leq b\), let
\(s(x)\) be the distance along \(C\) from the initial starting point
\(P_0(a,f(a))\) to the point \(Q(x,f(x))\).
Then \(s\) is a function, called the \textbf{arc length function}, and
\[s(x)=\int_a^x\sqrt{1+\big[f'(t)\big]^2}\,dt\]
By the Fundamental Theorem of Calculus,
\[\frac{ds}{dx}=\sqrt{1+\big[f'(x)\big]^2}
=\sqrt{1+\left(\frac{dy}{dx}\right)^2}\]
The differential of arc length is
\[ds=\sqrt{1+\left(\frac{dy}{dx}\right)^2}\,dx\]
and this equation is sometimes written in the symmetric form
\[(ds)^2=(dx)^2+(dy)^2\]
If we write \(\displaystyle{L=\int ds}\), then we can solve to get
\[ds=\sqrt{1+\left(\frac{dx}{dy}\right)^2}\,dy\]
\subsection{Area of a Surface of Revolution}

A surface of revolution is formed when a curve is rotated about a line. \\
If \(f\) is positive and has a continuous derivative, we define the
\textbf{surface area} of the surface obtained by rotating the curve
\(y=f(x),a\leq x\leq b,\) about the \(x\)-axis as
\[S=\int_a^b 2\pi f(x)\sqrt{1+\big[f'(x)\big]^2}\,dx\]
In Leibniz's notation:
\[S=\int_a^b 2\pi y\sqrt{1+\left(\frac{dy}{dx}\right)^2}\,dx\]
If the curve is described as \(x=g(y),c\leq y\leq d\), then the formula for
surface area is
\[S=\int_c^d 2\pi y\sqrt{1+\left(\frac{dx}{dy}\right)^2}\,dy\]
These formulas can be summarized symbolically as
\[S=\int 2\pi y\,ds\]
For rotation about the \(y\)-axis, the surface area formula is
\[S=\int 2\pi x\,ds\]

\section{Series}

\subsection{Sequences}

An infinite \textbf{sequence} \(\{a_n\}\) is a list of numbers written in a
definite order:
\[a_1,a_2,a_3\dots,a_n,a_{n+1},\dots\]
where \(a_n\) is the \(n\)th term of the sequence.
Consider the sequence
\[a_n=\left\{\frac{1}{2},\frac{1}{4},\frac{1}{8},\frac{1}{16},\frac{1}{32},
\cdots\right\}\]
We can rewrite it as
\[a_n=\left\{\frac{1}{2},\frac{1}{2^2},\frac{1}{2^3},\frac{1}{2^4},
\frac{1}{2^5},\cdots\right\}\]
so the formula for the \(n\)th term is
\[a_n=\frac{1}{2^n}\]
for \(n=1,2,3,\cdots\) and
\[\{a_n\}=\left\{\frac{1}{2^n}\right\}_{n=1}^\infty\]
\begin{definition}
    A sequence \(\{a_n\}\) has the \textbf{limit} \(L\) and we write
    \[\lim_{n\to\infty}a_n=L\]
    or \(a_n\to L\) as \(n\to\infty\) if we can make the terms \(a_n\) as
    close to \(L\) as we like by taking \(n\) sufficiently large.
    If \(\displaystyle{\lim_{n\to\infty}a_n}\) exists, then the sequence is
    \textbf{convergent}, otherwise it is \textbf{divergent}.
\end{definition}
The precise definition of a limit of a sequence is:
\begin{definition}
    A sequence \(\{a_n\}\) has the \textbf{limit} \(L\) and we write
    \[\lim_{n\to\infty}a_n=L\]
    if for every \(\epsilon>0\) there is an corresponding integer \(N\) such
    that
    \[n>N\implies |a_n-L|<\epsilon\]
\end{definition}
\begin{theorem}
    If \(\displaystyle{\lim_{x\to\infty}f(x)=L}\) and \(f(n)=a_n\) when \(n\) is an integer,
    then \(\displaystyle{\lim_{n\to\infty}a_n=L}\). 
\end{theorem}
In particular, we have
\[\lim_{n\to\infty}\frac{1}{n^r}=0\]
when \(r>0\).
If \(a_n\) becomes large as \(n\) becomes large, we use the notation
\[\lim_{n\to\infty}a_n=\infty\]
\begin{definition}
    \[\lim_{n\to\infty}a_n=\infty\]
    means that for every positive number \(M\) there is a positive integer
    \(N\) such that
    \[n>N\implies a_n>M\]
\end{definition}
If \(\displaystyle{\lim_{n\to\infty}a_n=\infty}\) then the sequence
\(\{a_n\}\) is divergent and we say that \(\{a_n\}\) diverges to \(\infty\).
If \(a_n\) and \(b_n\) are convergent sequences and \(c\) is a constant, then
\begin{enumerate}
    \item
    \(\displaystyle{\lim_{n\to\infty}(a_n+b_n)
    =\lim_{n\to\infty}a_n+\lim_{n\to\infty}b_n}\)
    \item
    \(\displaystyle{\lim_{n\to\infty}(a_n-b_n)
    =\lim_{n\to\infty}a_n-\lim_{n\to\infty}b_n}\)
    \item \(\displaystyle{\lim_{n\to\infty}c\cdot a_n=c\lim_{n\to\infty}a_n}\)
    \item \(\displaystyle{\lim_{n\to\infty}c=c}\)
    \item
    \(\displaystyle{\lim_{n\to\infty}(a_n\cdot b_n)
    =\lim_{n\to\infty}a_n\cdot\lim_{n\to\infty}b_n}\)
    \item
    \(\displaystyle{\lim_{n\to\infty}\frac{a_n}{b_n}
    =\frac{\displaystyle{\lim_{n\to\infty}a_n}}
    {\displaystyle{\lim_{n\to\infty}b_n}}}\)
    if \(\displaystyle{\lim_{n\to\infty}b_n\neq 0}\).
    \item
    \(\displaystyle{\lim_{n\to\infty}a_n^p=\big[\lim_{n\to\infty}a_n\big]^p}\)
    if \(p>0\) and \(a_n>0\).
\end{enumerate}
\begin{theorem}
    If \(a_n\leq b_n\leq c_n\) for \(n\geq n_0\) and
    \(\displaystyle{\lim_{n\to\infty}a_n=\lim_{n\to\infty}c_n}=L\), then
    \(\displaystyle{\lim_{n\to\infty}b_n=L}\).
\end{theorem}
\begin{theorem}
    If \(\displaystyle{\lim_{n\to\infty}|a_n|=0}\), then
    \(\displaystyle{\lim_{n\to\infty}a_n=0}\).
\end{theorem}
\begin{theorem}[Continuity and Convergence Theorem]
    If \(\displaystyle{\lim_{n\to\infty}a_n=L}\) and the function \(f\) is
    continuous at \(L\), then
    \[\lim_{n\to\infty}f(a_n)=f(L)\]
\end{theorem}
The sequence \(\{r^n\}\) is convergent if \(-1<r\leq 1\) and divergent for all
other values of \(r\).
\[\lim_{n\to\infty}r^n=0\]
if \(-1<r<1\) and
\[\lim_{n\to\infty}r^n=1\]
if \(r=1\).
\begin{definition}
    A sequence \(\{a_n\}\) is \textbf{increasing} if \(a_n<a_{n+1}\) for all
    \(n\geq 1\), that is, \(a_1<a_2<a_3<\cdots\).
    it is \textbf{decreasing} if \(a_n>a_{n+1}\) for all \(n\geq 1\).
    A sequence is \textbf{monotonic} if it is either increasing or decreasing.
\end{definition}
\begin{definition}
    A sequence \(\{a_n\}\) is \textbf{bounded above} if there is a number
    \(M\) such that
    \[a_n\leq M\]
    for all \(n\geq 1\).
    It is \textbf{bounded below} if there is a number
    \(m\) such that
    \[m\leq a_n\]
    for all \(n\geq 1\).
    If it is bounded above and below, then it is a \textbf{bounded sequence}.
\end{definition}
\begin{theorem}[Monotonic Sequence Theorem]
    Every bounded, monotonic sequence is convergent.
\end{theorem}
The proof of the Monotonic Sequence Theorem is based on the
\textbf{Completeness Axiom} of the set of real numbers \(\R\).
The Completeness Axiom states if \(S\) is a nonempty set of real numbers that
has an upper bound \(M\) (\(x\leq M\) for all \(x\in S\)), then \(S\) has a
\textbf{least upper bound} \(b\).
(This means that \(b\) is an upper bound of \(S\),
but if \(M\) is any other upper bound, then \(b\leq M\).)
The Completeness Axiom is an expression of the fact that there is no gap or
hole in the real number line.
\begin{proof}
    Suppose \(\{a_n\}\) is an incerasing sequence.
    Since \(\{a_n\}\) is bounded, the set \\ \(S=\{a_n\mid n\geq 1\}\) has an
    upper bound.
    By the Completeness Axiom it has a least upper bound \(L\).
    Given \(\epsilon>0\), \(L-\epsilon\) is not an upper bound for \(S\)
    (since \(L\) is the least upper bound).
    Therefore
    \[a_N>L-\epsilon\]
    for some integer \(N\).
    But the sequence is incerasing so \(a_n\geq a_N\) for every \(n>N\).
    Thus if \(n>N\) we have
    \begin{align*}
        a_n &> L-\epsilon \\
        a_n+\epsilon &> L \\
        L &< a_n +\epsilon
    \end{align*}
    so
    \[0\leq L-a_n<\epsilon\]
    since \(a_n\leq L\).
    Thus
    \[|L-a_n|<\epsilon\]
    which implies that
    \[|a_n-L|<\epsilon\]
    whenever \(n>N\) so \(\displaystyle{\lim_{n\to\infty}a_n=L}\).
    A similar proof (using the greatest lower bound) works if \(\{a_n\}\) is
    decreasing.
\end{proof}
The proof of the Monotonic Sequence Theorem shows that an increasing sequence
that is bounded above is convergent and a decreasing sequence that is bounded
below is convergent.
\subsection{Series}

In general, if we try to add the terms of an infinite sequence \(\{a_n\}\) we
get an expression of the form
\[a_1+a_2+a_3+\cdots+a_n+\cdots\]
which is called an \textbf{infinite series} (or just a \textbf{series}) and is
denoted by
\[\sum_{n=1}^\infty a_n\]
\begin{definition}
    Given a series
    \(\displaystyle{\sum_{n=1}^{\infty}a_n=a_1+a_2+a_3\cdots}\), let \(s_n\)
    be its \(n\)th partial sum:
    \[s_n=\sum_{k=1}^n a_i=a_1+a_2+\cdots+a_n\]
    If the sequence \(\{s_n\}\) is \textbf{convergent} and we write
    \(\displaystyle{\lim_{n\to\infty}s_n=s}\) exists as a real number, then the
    series \(\displaystyle{\sum_{n=1}^\infty a_n}\) is convergent and
    \[\sum_{n=1}^{\infty}a_n=a_1+a_2+\dots+a_n+\cdots=s\]
    The number \(s\) is the \textbf{sum} of the series.
    If the sequence \(\{s_n\}\) is divergent, then the series is
    \textbf{divergent}.
\end{definition}
Thus the sum of a series is the limit of the sequence of partial sums.
Notice that
\[\sum_{n=1}^\infty a_n=\lim_{n\to\infty}\sum_{k=1}^n a_i\]
Consider the \textbf{geometric series}
\[\sum_{n=1}^\infty ar^{n-1}=a+ar+ar^2+ar^3+\cdots+ar^{n-1}+\cdots,
\quad a\neq 0\]
with common ratio \(r\).

If \(r=1\), then \(s_n=a+a+\cdots+a=na\).
Thus \(\displaystyle{\lim_{n\to\infty}s_n=\pm\infty}\) so the limit does not
exist and the geometric series diverges when \(r=1\).

If \(r\neq 1\), then we have
\begin{align*}
    s_n &= a+ar+ar^2+\cdots+ar^{n-1} \\
    rs_n &= ar+ar^2+ar^3\cdots+ar^{n-1}+ar^n \\
    s_n-rs_n &= a-ar^n \\
    (1-r)s_n &= a(1-r^n) \\
    s_n &= \frac{a(1-r^n)}{1-r}
\end{align*}

If \(-1<r<1\), since \(\displaystyle{\lim_{n\to\infty}r^n}=0\), so
\[\lim_{n\to\infty}s_n=\lim_{n\to\infty}\frac{a(1-r^n)}{1-r}
=\frac{a}{1-r}-\frac{a}{1-r}\cdot\lim_{n\to\infty}r^n=\frac{a}{1-r}\]
Thus when \(|r|<1\) the geometric series is convergent and its sum is
\(\dfrac{a}{1-r}\).

If \(r\leq -1\) or \(r>1\), then the sequence \(\{r^n\}\) is divergent so
\(\displaystyle{\lim_{n\to\infty}s_n}\) does not exist thus the geometric
series diverges.

The geometric series
\[\sum_{n=1}^\infty ar^{n-1}=a+ar+ar^2+\cdots\]
is convergent if \(|r|<1\) and
its sum is
\[\sum_{n=1}^\infty ar^{n-1}=\frac{a}{1-r},\quad|r|<1\]
If \(|r|\geq 1\), then the geometric series is divergent.
\begin{problem}
    Find the sum of the series \(\displaystyle{\sum_{n=0}^\infty x^n}\), where
    \(|x|<1\).
\end{problem}
\begin{solution}
    Notice that
    \[\sum_{n=0}^\infty x^n=\sum_{n=1}^{\infty}x^{n-1}=1+x+x^2+x^3+\cdots\]
    This is the geometric series with \(a=1\) and \(r=x\).
    Since \(|r|=|x|<1\), it converges and gives
    \[\sum_{n=0}^\infty x^n=\frac{1}{1-x}\]
\end{solution}
\begin{problem}
    Show that the series \(\displaystyle{\sum_{n=1}^\infty\frac{1}{n(n+1)}}\)
    is convergent, and find its sum.
\end{problem}
\begin{solution}
    From the definition of convergent series we compute the partial sums.
    \[s_n=\sum_{k=1}^n\frac{1}{k(k+1)}
    =\frac{1}{1\cdot 2}+\frac{1}{2\cdot 3}+\frac{1}{3\cdot 4}+\cdots
    +\frac{1}{n(n+1)}\]
    We can simplify this expression if we use the partial fraction
    decomposition
    \[\frac{1}{k(k+1)}=\frac{1}{k}-\frac{1}{k+1}\]
    Thus we have
    \begin{align*}
        s_n &= \sum_{k=1}^n\frac{1}{k(k+1)}
        =\sum_{k=1}^n\left(\frac{1}{k}-\frac{1}{k+1}\right) \\
        &= \left(1-\frac{1}{2}\right)+\left(\frac{1}{2}-\frac{1}{3}\right)
        +\left(\frac{1}{3}-\frac{1}{4}\right)+\cdots
        +\left(\frac{1}{n}-\frac{1}{n+1}\right) \\
        &= 1-\frac{1}{n+1}
    \end{align*}
    Notice that the terms cancel in pairs.
    This series is an example of a \textbf{telescoping series}.
    Because of all the cancellations, the sum collapses into just two terms.
    Then
    \[\lim_{n\to\infty}s_n=\lim_{n\to\infty}\left(1-\frac{1}{n+1}\right)=1-0
    =1\]
    Therefore the given series is convergent and
    \[\sum_{n=1}^\infty\frac{1}{n(n+1)}=1\]
\end{solution}
\begin{problem}
    Show that the \textbf{harmonic series}
    \[\sum_{n=1}^\infty \frac{1}{n}=1+\frac{1}{2}+\frac{1}{3}+\frac{1}{4}
    +\cdots\]
    is divergent.
\end{problem}
\begin{solution}
    For this particular series it is convenient to consider the partial sums
    \[s_2,s_4,s_8,s_{16},s_{32},\dots\]
    and show that they become large.
    \begin{align*}
        s_2 &= 1+\frac{1}{2} \\
        s_4 &= 1+\frac{1}{2}+\left(\frac{1}{3}+\frac{1}{4}\right)
        >1+\frac{1}{2}+\left(\frac{1}{4}+\frac{1}{4}\right)=1+\frac{2}{2} \\
        s_8 &= 1+\frac{1}{2}+\left(\frac{1}{3}+\frac{1}{4}\right)
        +\left(\frac{1}{5}+\frac{1}{6}+\frac{1}{7}+\frac{1}{8}\right) \\
        &> 1+\frac{1}{2}+\left(\frac{1}{4}+\frac{1}{4}\right)
        +\left(\frac{1}{8}+\frac{1}{8}+\frac{1}{8}+\frac{1}{8}\right)
        =1+\frac{1}{2}+\frac{1}{2}+\frac{1}{2}=1+\frac{3}{2} \\
        s_{16} &= 1+\frac{1}{2}+\left(\frac{1}{3}+\frac{1}{4}\right)
        +\left(\frac{1}{5}+\cdots+\frac{1}{8}\right)
        +\left(\frac{1}{9}+\cdots+\frac{1}{16}\right) \\
        &> 1+\frac{1}{2}+\left(\frac{1}{4}+\frac{1}{4}\right)
        +\left(\frac{1}{8}+\cdots+\frac{1}{8}\right)
        +\left(\frac{1}{16}+\cdots+\frac{1}{16}\right) \\
        &= 1+\frac{1}{2}+\frac{1}{2}+\frac{1}{2}+\frac{1}{2}=1+\frac{4}{2}
    \end{align*}
    Similarly, \(s_{32}>1+\frac{5}{2},s_{64}>1+\frac{6}{2}\), and in general
    \[s_{2^n}>1+\frac{n}{2}\]
    This shows that \(s_{2^n}\to\infty\) as \(n\to\infty\) and so \(\{s_n\}\)
    is divergent.
    Therefore the harmonic series diverges.
\end{solution}

\begin{theorem}
    If the series \(\displaystyle{\sum_{n=1}^\infty a_n}\) is convergent, then
    \(\displaystyle{\lim_{n\to\infty}a_n=0}\).
\end{theorem}
\begin{proof}
    Let \(s_n=a_1+a_2+\cdots+a_n\).
    Then \(a_n=s_n-s_{n-1}\).
    Since \(\displaystyle{\sum_{n=1}^\infty a_n}\) is convergent, the sequence
    \(\{s_n\}\) is convergent.
    Let \(\displaystyle{\lim_{n\to\infty}s_n=s}\).
    Since \(n-1\to\infty\) as \(n\to\infty\), we also have
    \(\displaystyle{\lim_{n\to\infty}s_{n-1}=s}\).
    Therefore
    \[\lim_{n\to\infty}a_n=\lim_{n\to\infty}(s_n-s_{n-1})
    =\lim_{n\to\infty}s_n-\lim_{n\to\infty}s_{n-1}=s-s=0\]
\end{proof}
With any series \(\displaystyle{\sum_{n=1}^\infty a_n}\) we associate two
sequences: the sequence \(\{s_n\}\) of its partial sums and the sequence
\({a_n}\) of its terms.
If \(\displaystyle{\sum_{n=1}^\infty a_n}\) is convergent, then the limit of
the sequence \(\{s_n\}\) is \(s\) (the sum of the series) and the limit of the
sequence \(\{a_n\}\) is 0.
\begin{theorem}[Test for Divergence]
    If \(\displaystyle{\lim_{n\to\infty}a_n}\) does not exist or if
    \(\displaystyle{\lim_{n\to\infty}a_n\neq 0}\), then the series
    \(\displaystyle{\sum_{n=1}^\infty a_n}\) is divergent.
\end{theorem}
The Test of Divergence works because if the series is not divergent, then it
is convergent, and so \(\displaystyle{\lim_{n\to\infty}a_n=0}\).
\begin{theorem}
    If \(\displaystyle{\sum_{n=1}^\infty a_n}\) and
    \(\displaystyle{\sum_{n=1}^\infty b_n}\) are convergent series, then so
    are the following series:
    \begin{enumerate}
        \item
        \(\displaystyle{\sum_{n=1}^\infty c\cdot a_n
        =c\sum_{n=1}^\infty a_n}\) where \(c\) is a constant.
        \item
        \(\displaystyle{\sum_{n=1}^\infty(a_n+b_n)
        =\sum_{n=1}^\infty a_n+\sum_{n=1}^\infty b_n}\)
        \item
        \(\displaystyle{\sum_{n=1}^\infty(a_n-b_n)
        =\sum_{n=1}^\infty a_n-\sum_{n=1}^\infty b_n}\)
    \end{enumerate}
\end{theorem}
A finite number of terms does not affect the convergence or divergence of a
series.
If it is known that the series \(\displaystyle{\sum_{n=N+1}^\infty}a_n\)
converges, then the full series
\[\sum_{n=1}^\infty a_n=\sum_{n=1}^N a_n+\sum_{n=N+1}^\infty a_n\]
is also convergent.
\subsection{The Integral and Comparison Tests}

\subsubsection*{Testing with an Integral}

\begin{theorem}[Integral Test]
    Suppose \(f\) is continuous, positive, decreasing function on
    \([1,\infty)\) and let \(a_n=f(n)\).
    Then the series \(\displaystyle{\sum_{n=1}^\infty a_n}\) is convergent if
    and only if the improper integral
    \(\displaystyle{\int_1^\infty f(x)\,dx}\) is convergent.
    \begin{enumerate}
        \item If \(\displaystyle{\int_1^\infty f(x)\,dx}\) is convergent, then
        \(\displaystyle{\sum_{n=1}^\infty a_n}\) is convergent.
        \item If \(\displaystyle{\int_1^\infty f(x)\,dx}\) is divergent, then
        \(\displaystyle{\sum_{n=1}^\infty a_n}\) is divergent.
    \end{enumerate}
\end{theorem}
The \(p\)-series \(\displaystyle{\sum_{n=1}^\infty \frac{1}{n^p}}\) is
convergent if \(p>1\) and divergent if \(p\leq 1\).

\subsubsection*{Testing by Comparing}

\begin{theorem}[Comparison Test]
    Suppose that \(\displaystyle{\sum_{n=1}^\infty a_n}\) and
    \(\displaystyle{\sum_{n=1}^\infty b_n}\) are series with positive terms.
    \begin{enumerate}
        \item If \(\displaystyle{\sum_{n=1}^\infty b_n}\) is convergent and
        \(a_n\leq b_n\) for all \(n\), then
        \(\displaystyle{\sum_{n=1}^\infty a_n}\) is also convergent.
        \item If \(\displaystyle{\sum_{n=1}^\infty b_n}\) is divergent and
        \(a_n\geq b_n\) for all \(n\), then
        \(\displaystyle{\sum_{n=1}^\infty a_n}\) is also divergent.
    \end{enumerate}
\end{theorem}

\subsubsection*{Proof of the Integral Test}
\subsection{Other Convergence Tests}

\subsubsection*{Alternating Series}

An \textbf{alternating series} is a series whose terms are alternately
positive and negative.

If the terms of an alternating series decrease to 0 in absolute value, then
the series converges.
\begin{theorem}[Alternating Series Test]
    If the alternating series
    \[\sum_{n=1}^\infty (-1)^{n-1}b_n=b_1-b_2+b_3-b_4+b_5-b_6+\cdots,
    \quad b_n>0\]
    satisfies
    \[b_{n+1}\leq b_n\]
    for all \(n\) and
    \[\lim_{n\to\infty}b_n=0\]
    then the series is convergent.
\end{theorem}
\begin{theorem}[Alternating Series Estimation Theorem]
    If \(\displaystyle{s=\sum_{n=1}^\infty}(-1)^{n-1}b_n\) is the sum of an
    alternating series that satisfies
    \[0\leq b_{n+1}\leq b_n\]
    and
    \[\lim_{n\to\infty}b_n=0\]
    then
    \[|R_n|=|s-s_n|\leq b_{n+1}\]
\end{theorem}

\subsubsection*{Absolute Convergence}

\begin{definition}
    A series \(\displaystyle{\sum_{n=1}^\infty}a_n\) is called
    \textbf{absolutely convergent} if the series of absolute values
    \(\displaystyle{\sum_{n=1}^\infty}|a_n|\) is convergent.
\end{definition}
\begin{definition}
    A series \(\displaystyle{\sum_{n=1}^\infty}a_n\) is called
    \textbf{conditionally convergent} if it is convergent but not absolutely
    convergent.
\end{definition}
\begin{theorem}
    If a series \(\displaystyle{\sum_{n=1}^\infty}a_n\) is absolutely
    convergent, then it is convergent.
\end{theorem}

\subsubsection*{The Ratio Test}

\begin{theorem}[Ratio Test]
    \begin{enumerate}
        \item If
        \(\displaystyle{\lim_{n\to\infty}\left|\frac{a_{n+1}}{a_n}\right|}
        =L<1\),
        then the series \(\displaystyle{\sum_{n=1}^\infty}a_n\) is absolutely
        convergent (and therefore convergent).
        \item If
        \(\displaystyle{\lim_{n\to\infty}\left|\frac{a_{n+1}}{a_n}\right|}
        =L>1\)
        or
        \(\displaystyle{\lim_{n\to\infty}\left|\frac{a_{n+1}}{a_n}\right|}
        =\infty\),
        then the series \(\displaystyle{\sum_{n=1}^\infty}a_n\) is divergent.
        \item If
        \(\displaystyle{\lim_{n\to\infty}\left|\frac{a_{n+1}}{a_n}\right|}
        =1\),
        the Ratio Test is inconclusive; that is, no conclusion can be drawn
        about the convergence or divergence of
        \(\displaystyle{\sum_{n=1}^\infty}a_n\).
    \end{enumerate}
\end{theorem}

\begin{theorem}[Root Test]
    \begin{enumerate}
        \item If
        \(\displaystyle{\lim_{n\to\infty}\sqrt[n]{|a_n|}}=L<1\),
        then the series \(\displaystyle{\sum_{n=1}^\infty}a_n\) is absolutely
        convergent (and therefore convergent).
        \item If
        \(\displaystyle{\lim_{n\to\infty}\sqrt[n]{|a_n|}}=L>1\)
        or
        \(\displaystyle{\lim_{n\to\infty}\sqrt[n]{|a_n|}}
        =\infty\),
        then the series \(\displaystyle{\sum_{n=1}^\infty}a_n\) is divergent.
        \item If
        \(\displaystyle{\lim_{n\to\infty}\sqrt[n]{|a_n|}}=1\),
        the Root Test is inconclusive.
    \end{enumerate}
\end{theorem}
\subsection{Power Series}

A \textbf{power series} is a series of the form
\[\sum_{n=0}^\infty c_n x^n=c_0+c_1 x +c_2 x^2+c_3 x^3+\cdots\]

A power series is a series in which each term is a power function.
A \textbf{trigonometric series}
\[\sum_{n=0}^\infty(a_n\cos nx+b_n\sin nx)\]
is a series whose terms are trigonometric functions.

In general, a power series centered at \(a\) is a series of the form
\[\sum_{n=0}^{\infty}c_n(x-a)^n=a_0+c_1(x-a)+c_2(x-a)^2+\cdots\]
where \(x\) is a variable and the coefficient \(c_n\) is a constant.
The sum of the series is a function
\[f(x)=c_0+c_1x+c_2x^2+\cdots+c_nx^n+\cdots\]
whose domain is the set of all \(x\) for which the series converges.
\begin{theorem}
    For a given power series
    \(\displaystyle{\sum_{n=0}^\infty c_n(x-a)^n}\) there are only three
    possibilities:
    \begin{enumerate}
        \item The series converges only when \(x=a\).
        \item The series converges for all \(x\in\R\).
        \item There is a positive number \(R\) such that the series converges
        if \(|x-a|<R\) and diverges if \(|x-a|>R\).
    \end{enumerate}
\end{theorem}
The number \(R\) in cases 3 is the \textbf{radius of convergence} of the power
series.
The radius of convergence is \(R=0\) in case 1 and \(R=\infty\) in case 2.
The \textbf{interval of convergence} of a power series is the interval that
consists of all values of \(x\) for which the series converges.
\subsection{Representing Functions as Power Series}

We start with the equation
\[\frac{1}{1-x}=1+x+x^2+x^3+\cdots=\sum_{n=0}^\infty x^n,\quad|x|<1\]
and find power series representations of similar functions.

\subsubsection*{Differentiation and Integration of Power Series}

We can differentiate and integrate a function represented by the sum of a
power series term by term.
This is called \textbf{term-by-term differentiation and integration}.
\begin{theorem}
    If the power series \(\displaystyle{\sum_{n=0}^\infty c_n(x-a)^n}\) has radius of convergence \(R>0\),
    then the function \(f\) defined by
    \[f(x)=c_0+c_1(x-a)+c_2(x-a)^2+\cdots=\sum_{n=0}^\infty c_n(x-a)^n\]
    is differentiable (and therefore continuous) on the interval
    \((a-R,a+R)\) and
    \[f'(x)=c_1+2c_2(x-a)+3c_3(x-a)^2+\cdots
    =\sum_{n=1}^\infty nc_n(x-a)^{n-1}\]
    and
    \[\int f(x)\,dx
    =C+c_0(x-a)+c_1\frac{(x-a)^2}{2}+c_2\frac{(x-a)^3}{3}+\cdots
    =C+\sum_{n=0}^\infty c_n\frac{(x-a)^{n+1}}{n+1}\]
    The radii of convergence of these power series are both \(R\).
\end{theorem}
In Leibniz's notation:
\[\frac{d}{dx}\left[\sum_{n=0}^\infty c_n(x-a)^n\right]
=\sum_{n=0}^\infty\frac{d}{dx}\big[c_n(x-a)^n\big]\]
\[\int\left[\sum_{n=0}^\infty c_n(x-a)^n\right]\,dx
=\sum_{n=0}^\infty\int c_n(x-a)^n\,dx\]
\[\ln(1+x)=x-\frac{x^2}{2}+\frac{x^3}{3}-\frac{x^4}{4}+\cdots
=\sum_{n=1}^\infty(-1)^{n-1}\frac{x^n}{n},\quad|x|<1,\quad R=1\]
\[\arctan x=x-\frac{x^3}{3}+\frac{x^5}{5}-\frac{x^7}{7}+\cdots
=\sum_{n=0}^\infty(-1)^n\frac{x^{2n+1}}{2n+1},\quad R=1\]
\subsection{Taylor and Maclaurin Series}

\begin{theorem}
    If \(f\) has a power series expansion at \(a\), that is, if
    \[f(x)=\sum_{n=0}^\infty c_n(x-a)^n,\quad|x-a|<R\]
    then its coefficients are given by the formula
    \[c_n=\frac{f^{(n)}(a)}{n!}\]
\end{theorem}
The \textbf{Taylor series} of a function \(f\) centered at \(a\) is
\[f(x)=\sum_{n=0}^{\infty}\frac{f^{(n)}(a)}{n!}(x-a)^n
=f(a)+\frac{f'(a)}{1!}(x-a)+\frac{f''(a)}{2!}(x-a)^2+\frac{f'''(a)}{3!}(x-a)^2
+\cdots\]
For the special case \(a=0\), the Taylor series becomes the
\textbf{Maclaurin series}
\[f(x)=\sum_{n=0}^{\infty}\frac{f^{(n)}(0)}{n!}x^n
=f(0)+\frac{f'(0)}{1!}x+\frac{f''(0)}{2!}x^2+\cdots\]
\begin{problem}
    Find the Maclaurin series of the function \(f(x)=e^x\) and its radius of
    convergence.
\end{problem}
\begin{solution}
    Since \(f^{(n)}(x)=e^x\) and \(f^{(n)}(0)=e^0=1\) for all \(n\)
    therefore
    \[e^x=\sum_{n=0}^\infty\frac{x^n}{n!}
    =1+x+\frac{x^2}{2!}+\frac{x^3}{3!}+\cdots\]
    Let \(a_n=\dfrac{x^n}{n!}\), then
    \[\left|\frac{a_{n+1}}{a_n}\right|
    =\left|\frac{x^{n+1}}{(n+1)!}\frac{n!}{x^n}\right|=\frac{|x|}{n+1}\]
    and
    \[\lim_{n\to\infty}\frac{|x|}{n+1}=0<1\]
    so by the Ratio Test the series converges for all \(x\in\R\) and the
    radius of convergence is \(R=\infty\).
\end{solution}
The \(n\)th degree \textbf{Taylor polynomial} \(T_n\) of \(f\) at \(a\) is
\begin{align*}
    T_n(x) &= \sum_{k=0}^{n}\frac{f^{(k)}(a)}{k!}(x-a)^k \\
    &= f(a)+\frac{f'(a)}{1!}(x-a)+\frac{f''(a)}{2!}(x-a)^2+\cdots
    +\frac{f^{(n)}(a)}{n!}(x-a)^n
\end{align*}
In general, \(f(x)\) is the sum of its Taylor series if
\[f(x)=\lim_{n\to\infty}T_n(x)\]
Let
\[R_n(x)=f(x)-T_n(x)\]
so that
\[f(x)=T_n(x)+R_n(x)\]
then \(R_n(x)\) is the \textbf{remainder} of the Taylor series.
If \(\displaystyle{\lim_{n\to\infty}R_n(x)=0}\), then
\[\lim_{n\to\infty}T_n(x)=\lim_{n\to\infty}\Big[f(x)-R_n(x)\Big]=f(x)\]
\begin{theorem}
    If \(f(x)=T_n(x)+R_n(x)\) where \(T_n\) is the \(n\)th degree Taylor
    polynomial of \(f\) at \(a\) and
    \[\lim_{n\to\infty}R_n(x)=0\]
    for \(|x-a|<R\), then \(f\) is equal to the sum of its Taylor series on
    the interval \(|x-a|<R\).
\end{theorem}
\begin{theorem}[Taylor's Formula]
    If \(f\) has \(n+1\) derivatives in an interval \(I\) that contains the
    number \(a\), then for \(x\) in \(I\) there is a number \(z\) strictly
    between \(x\) and \(a\) such that the remainder term in the Taylor series
    is
    \[R_n(x)=\frac{f^{(n+1)}(z)}{(n+1)!}(x-a)^{n+1}\]
\end{theorem}
This expression for \(R_n(x)\) is
\textbf{Lagrange's form of the remainder term}.
\[\lim_{n\to\infty}\frac{x^n}{n!}=0\] for every real number \(x\in\R\).
\begin{problem}
    Prove that \(e^x\) is equal to the sum of its Taylor series.
\end{problem}
\begin{solution}
    If \(f(x)=e^x\), then \(f^{n+1}(x)=e^x\), so the remainder term in
    Taylor's Formula is
    \[R_n(x)=\frac{e^z}{(n+1)!}x^{n+1}\]
    where \(z\) is a number strictly between \(0\) and \(x\).
    Note that \(z\) depends on \(n\).
    If \(x>0\), then \(0<z<x\), so \(e^z<e^x\).
    Therefore
    \[0<R_n(x)=\frac{e^z}{(n+1)!}x^{n+1}<e^x\frac{x^{n+1}}{(n+1)!}\to 0\]
    so \(R_n(x)\to 0\) as \(n\to\infty\) by the Squeeze Theorem.
    If \(x<0\), then \(x<z<0\), so \(e^z<e^0=1\) and
    \[|R_n(x)|<\frac{|x|^{n+1}}{(n+1)!}\to 0\]
    Again \(R_n(x)\to 0\).
    Thus \(e^x\) is equal to the sum of its Taylor series.
\end{solution}
\[e=\sum_{n=0}^\infty\frac{1}{n!}=1+\frac{1}{1!}+\frac{1}{2!}+\frac{1}{3!}+\cdots\]

\[\sin x=x-\frac{x^3}{3!}+\frac{x^5}{5!}-\frac{x^7}{7!}+\cdots
=\sum_{n=0}^\infty(-1)^n\frac{x^{2n+1}}{(2n+1)!},\quad R=\infty\]

\[\cos x=1-\frac{x^2}{2!}+\frac{x^4}{4!}-\frac{x^6}{6!}+\cdots
=\sum_{n=0}^\infty(-1)^n\frac{x^{2n}}{(2n)!},\quad R=\infty\]

If \(k\) is any real number and \(|x|<1\), then the \textbf{binomial series}
is
\[(1+x)^k=\sum_{n=0}^\infty\binom{k}{n}x^n=1+kx+\frac{k(k-1)}{2!}x^2
+\frac{k(k-1)(k-2)}{3!}x^3+\cdots\]
where the \textbf{binomial coefficient} is
\[\binom{k}{n}=\frac{k(k-1)(k-2)\cdots(k-n+1)}{n!}\]
\begin{problem}
    Evaluate \(\displaystyle{\int e^{-x^2}}\,dx\) as an infinite series.
\end{problem}
\begin{solution}
    For all \(x\in\R\) we have
    \[e^{-x^2}=\sum_{n=0}^\infty\frac{(-x^2)^n}{n!}
    =\sum_{n=0}^\infty(-1)^n\frac{x^{2n}}{n!}
    =1-\frac{x^2}{1!}+\frac{x^4}{2!}-\frac{x^6}{3!}+\cdots\]
    Then we integrate term by term:
    \begin{align*}
        \int e^{-x^2}\,dx &= \int \left(1-\frac{x^2}{1!}+\frac{x^4}{2!}
        -\frac{x^6}{3!}+\cdots+(-1)^n\frac{x^{2n}}{n!}+\cdots\right)\,dx \\
        &= C+x-\frac{x^3}{3\cdot 1!}+\frac{x^5}{5\cdot 2!}
        +\frac{x^7}{7\cdot 3!}+\cdots+(-1)^n\frac{x^{2n+1}}{(2n+1)n!}+\cdots
    \end{align*}
    This series converges for all \(x\in\R\).
\end{solution}
Power series can be added, subtracted, multiplied, and divided like
polynomials.
\subsection{Applications of Taylor Polynomial}

\subsubsection*{Approximation Functions by Polynomials}

The Taylor polynomial of \(f\) at \(a\) can be used as an approximation to
\(f\):
\[f(x)\approx T_n(x)\]
The size of the error of the approximation is:
\[|R_n(x)|=|f(x)-T_n(x)|\]

\subsubsection*{Applications to Physics}

Physicist uses a Taylor polynomial as an approximation to the function.
\begin{problem}
    In Einstein's theory of special relativity the mass of an object moving
    with velocity \(v\) is
    \[m=\frac{m_0}{\sqrt{1-\dfrac{v^2}{c^2}}}\]
    where \(m_0\) is the mass of the object when at rest and \(c\) is the
    speed of light.
    The kinetic energy of the object is the difference between its total
    energy and its energy at rest:
    \[K=mc^2-m_0c^2\]
    Show that when \(v\) is very small compared with \(c\), this
    expression for \(K\) agrees with classical Newtonian physics:
    \[K=\frac{1}{2}m_0v^2\]
\end{problem}
\begin{solution}
    Using the expressions given for \(K\) and \(m\), we get
    \begin{align*}
        K &= mc^2-m_0c^2=\frac{m_0c^2}{\sqrt{1-\dfrac{v^2}{c^2}}}-m_0c^2 \\
        &= m_0c^2\left[\left(1-\frac{v^2}{c^2}\right)^{-1/2}-1\right]
    \end{align*}
    With \(x=-\dfrac{v^2}{c^2}\), then we have the binomial series with
    \(k=-\dfrac{1}{2}\).
    Notice that \(|x|<1\) because \(v<c\).
    Therefore we have
    \[(1+x)^{-1/2}=1-\frac{1}{2}x+\frac{3}{8}x^2-\frac{5}{16}x^3+\cdots\]
    and
    \[K=m_0c^2\left(\frac{1}{2}\cdot\frac{v^2}{c^2}
    +\frac{3}{8}\cdot\frac{v^2}{c^2}+\frac{5}{16}\cdot\frac{v^2}{c^2}\right)\]
    If \(v\) is much smaller than \(c\), then all terms after the first are
    very small when compared with the first term.
    If we omit them, we get
    \[K\approx m_0c^2\left(\frac{1}{2}\cdot\frac{v^2}{c^2}\right)
    =\frac{1}{2}m_0v^2\]
\end{solution}

% \section{Parametric Equations and Polar Coordinates}

% \subsection{Curves Defined by Parametric Equations}

\subsubsection*{Parametric Equations}
Suppose that \(x\) and \(y\) are both given as functions of the variable
\(t\), the \textbf{parameter}, by the equations
\begin{align*}
    x &= f(t) & y &= g(t)
\end{align*}
which are the \textbf{parametric equations}.
As \(t\) varies, the point \((x,y)=(f(t),g(t))\) varies and traces out a curve
which is the parametric curve.
In general, the curve with parametric equations
\begin{align*}
    x &= f(t) & y &= g(t) & a\leq t\leq b
\end{align*}
has \textbf{initial point} \((f(a),g(a))\) and \textbf{terminal point}
\((f(b),g(b))\).

\subsubsection*{The Cycloid}
% \subsection{Calculus with Parametric Curves}

\subsubsection*{Tangents}
Suppose \(f\) and \(g\) are differentiable functions and we want to find the
tangent line at a point on the parametric curve \(x=f(t),y=g(t)\), where \(y\)
is a differentiable function of \(x\).
Then the chain rule gives
\[\frac{dy}{dt}=\frac{dy}{dx}\cdot\frac{dx}{dt}\]
If \(dx/dt\neq0\), then
\[\frac{dy}{dx}=\frac{dy/dt}{dx/dt}\]
It follows that
\[\frac{d^2y}{dx^2}=\frac{d}{dx}\left(\frac{dy}{dx}\right)
=\frac{\dfrac{d}{dt}\left(\dfrac{dy}{dx}\right)}{\dfrac{dx}{dt}}\]

\subsubsection*{Areas}
If the curve \(y=F(x)\) from \(a\) to \(b\) is traced out once by the
parametric equations \(x=f(t),y=g(t),\alpha\leq t\leq\beta\), then the area
under the curve is
\[A=\int_{a}^{b}y\,dx=\int_{\alpha}^{\beta}g(t)f'(t)\,dt\]

\subsubsection*{Arc Length}
\begin{theorem}
    If a curve \(C\) is described by the parametric equations
    \(x=f(t),y=g(t),\alpha\leq t\leq\beta\), where \(f'\) and \(g'\) are
    continuous on \([\alpha,\beta]\) and \(C\) is traversed exactly once at
    \(t\) increases from \(\alpha\) to \(\beta\), then the length of \(C\) is
    \[L=\int_{\alpha}^{\beta}
    \sqrt{\left(\frac{dx}{dt}\right)^2+\left(\frac{dy}{dt}\right)^2}\,dt\]
\end{theorem}

\subsubsection*{Surface Area}
Suppose that a curve \(C\) is given by the parametric equations
\(x=f(t),y=g(t),\alpha\leq t\leq\beta\) where \(f',g'\) are continuous,
\(g(t)\geq0\), and \(C\) is traversed exactly once as \(t\) increases from
\(\alpha\) to \(\beta\).
If \(C\) is rotated about the \(x\)-axis, then the area of the resulting
surface is given by
\[S=\int_{\alpha}^{\beta}2\pi y
\sqrt{\left(\frac{dx}{dt}\right)^2+\left(\frac{dy}{dt}\right)^2}\,dt\]
% \subsection{Polar Coordinates}

We choose a point \(O\) in the plane called the \textbf{pole} (or origin).
Then we draw a ray (half-line) starting at \(O\) called the
\textbf{polar axis}.
This axis is usually drawn horizontally to the right and
corresponds to the positive \(x\)-axis in Cartesian coordinates.
If \(P\) is any other point in the plane, let \(r\) be the distance from \(O\)
to \(P\) and let \(\theta\) be the angle between the polar axis and line \(OP\).
Then the point \(P\) is represented by the ordered pair \((r,\theta)\) called
the polar coordinates of \(P\).
We use the convention that an angle is positive if measured in the
counterclockwise direction from the polar axis and negative in the clockwise
direction.
If \(P=O\), then \(r=0\) and we agree that \((0,\theta)\) represents the pole
for any value of \(\theta\).
We extend the meaning of polar coordinates \((r,\theta)\) to the case \(r\) is
negative by agreeing that, the points \((-r,\theta)\) and \((r,\theta)\) lie
on the same line through \(O\) and at the same distance \(|r|\) from \(O\),
but on opposite sides of \(O\).
If \(r>0\), the point \((r,\theta)\) lies in the same quadrant as \(\theta\);
if \(r<0\), it lies in the quadrant on the opposite side of the pole.
Note that \((-r,\theta)\) represents the same point as \((r,\theta+\pi)\).
Suppose that point \(P\) has Cartesian coordinates \((x,y)\) and polar
coordinates \((r,\theta)\).
To find Cartesian coordinates \((x,y)\) when polar coordinates \((r,\theta)\)
are known, we use the equations
\begin{align*}
    x &= r\cos\theta & y &= r\sin\theta
\end{align*}
To find polar coordinates \((x,y)\) when Cartesian coordinates \((r,\theta)\)
are known, we use the equations
\begin{align*}
    r^2 &= x^2+y^2 & \tan\theta &= \frac{y}{x}
\end{align*}

\subsubsection*{Polar Curves}
The graph of a polar equation \(r=f(\theta)\), or more generally
\(F(r,\theta)=0\), consists of all points \(P\) that have at least one polar
representation \((r,\theta)\) whose coordinates satisfy the equation.

\subsubsection*{Symmetry}
% \subsection{Calculus in Polar Coordinates}

\subsubsection*{Tangents}
To find a tangent line for polar curve \(r=f(\theta)\),
we find the slope of the parametric curve.
We have
\[\frac{dy}{dx}=\frac{dy/d\theta}{dx/d\theta}
=\frac{\frac{dr}{d\theta}\cos\theta-r\sin\theta}
{\frac{dr}{d\theta}\sin\theta+r\cos\theta}\]

\subsubsection*{Areas}

\subsubsection*{Arc Lengths}
Let \(\theta\) be a parameter, then the parametric equations for a polar curve
\(r=f(\theta)\) are
\begin{align*}
    x &= r\cos\theta=f(\theta)\cos\theta & y &= r\sin\theta=f(\theta)\sin\theta
\end{align*}
To find the length of the polar curve \(r=f(\theta),a\leq\theta\leq b\),
we differentiate with respect to \(\theta\) (using the product rule):
\begin{align*}
    \frac{dx}{d\theta} &= \frac{dr}{d\theta}\cos\theta-r\sin\theta
    & \frac{dy}{d\theta} &= \frac{dr}{d\theta}\sin\theta+r\cos\theta
\end{align*}
Then we have
\begin{align*}
    \left(\frac{dx}{d\theta}\right)^2+\left(\frac{dy}{d\theta}\right)^2
    &= \left(\frac{dr}{d\theta}\right)^2\cos^2\theta
    -2r\frac{dr}{d\theta}\cos\theta\sin\theta+r^2\sin^2\theta \\
    &+ \left(\frac{dr}{d\theta}\right)^2\sin^2\theta
    +2r\frac{dr}{d\theta}\sin\theta\cos\theta+r^2\cos^2\theta \\
    &= \left(\frac{dr}{d\theta}\right)^2+r^2
\end{align*}
Assuming that \(f'\) is continuous, the length of a curve with polar equation
\(r=f(\theta),\alpha\leq\theta\leq\beta\), is
\[L
=\int_{\alpha}^{\beta}\sqrt{r^2+\left(\frac{dr}{d\theta}\right)^2}\,d\theta\]

% \subsection{Conic Sections in Polar Coordinates}

\subsubsection*{Conics in Cartesian Coordinates}

\subsubsection*{Conics in Polar Coordinates}

% \section{Differential Equations}

% \subsection{Ordinary Differential Equations}

A \textbf{differential equation }is an equation that contains an unknown
function and one or more of its derivatives.
An \textbf{ordinary differential equation} (ODE) is a differential equation
that relates one or more functions of a single variable and their ordinary
derivatives.The importance of differential equations lies in the fact that
when a scientist or engineer formulates a physical law in mathematical terms,
it frequently turns out to be a differential equation.

The \textbf{order} of a differential equation is the order of the highest
derivative that occurs in the equation.

A function \(f\) is called a \textbf{solution} of a differential equation if
the equation is satisfied when \(y=f(x)\) and its derivatives are substituted
into the equation.

When we are asked to solve a differential equation we are expected to find all
possible solutions of the equation.

In many physical problems we need to find the particular solution that
satisfies a condition of the form \(y(x_0)=y_0\).
This is called an \textbf{initial condition}, and the problem of finding a
solution of the differential equation that satisfies the initial condition is
called an \textbf{initial value problem} (IVP).

% Newton's second law of motion, the force \(F\) acting on an object with mass
% \(m\) and acceleration \(a\) is \(F=ma\) can be written as an ordinary
% differential equation
% \[F=m\frac{dv}{dt}\]
% which is a first order differential equation, or
% \[F=m\frac{d^2s}{dt^2}\]
%  which is a second order differential equation.
% By Newton's second law of motion, the net force \(F\) acting on a falling
% object with mass \(m\), velocity \(v\), and air resistance force \(\gamma v\)
% can be modeled by the differential equation
% \[F=m\frac{dv}{dt}=mg-\gamma v\]
% where \(g\) is the acceleration due to gravity and \(\gamma\) is the drag
% coefficient.
% A function \(f\) is a \textbf{solution} of a differential equation if the
% function and its derivatives satisfy the equation for all values of \(x\) in
% some open interval \(a<x<b\).
% It is possible that there are many solutions of a differential equation.
% An initial condition is a condition \(y(x_0)=y_0\) or
% \(y^{(n)}(x_0)=y_n\) on the solution.
% An \textbf{initial value problem} is solving a differential equation with initial conditions.
% The interval of validity is the largest possible interval on which
% the solution is valid and contains \(x_0\) in the initial conditions.
% The general solution of a differential equation is the set of all solutions
% and the particular solution is the solution that satisfies the initial
% conditions.
% An explicit solution is any solution in the form \(y=y(x)\), otherwise it is
% an implicit solution.
% The particular solution of the differential equation
% \[\frac{dy}{dx}=y\]
% with initial
% condition \(y(0)=1\) is \(y=e^x\) since
% \[\frac{dy}{dx}=\frac{d}{dx}e^x=e^x=y\]
% and \(y(0)=e^0=1\).
% The \textbf{existence} and \textbf{uniqueness} problem asks that given a
% differential equation, does there exist a solution and if any is there only
% one solution.
% \subsection{Direction Fields and Euler's Method}

\subsubsection*{Direction Fields}
Suppose we are given a first order differential equation of the form \(y'=F(x,y)\).
If a solution curve, or an integral curve, passes through the point
\((x_0,y_0)\), then its slope at the point is \(y'(x_0)=F(x_0,y_0)\).
If we draw short line segments with slopes \(F(x,y)\) at serval points
\((x,y)\), then the result is a \textbf{direction field}, or a
\textbf{slope field}.

\subsubsection*{Euler's Method}
% \subsection{Separable Equations}

\subsubsection*{Separable Equations}
A \textbf{separable equation} is a first order differential equation that can
be written in the differential form
\[M(x)\,dx+N(y)\,dy=0\]
or can be written in the form
\[\frac{dy}{dx}=f(x)g(y)\]
We can separate the variables if \(g(y)\neq0\), let \(h(y)=1/g(y)\) then
\[\frac{dy}{dx}=\frac{f(x)}{h(y)}\]
We write the equation in the differential form
\[h(y)\,dy=f(x)\,dx\]
Then we integrate both sides of the equation:
\[\int h(y)\,dy=\int f(x)\,dx\]
Now we have an implicit solution of the differential equation and sometimes we
can solve for an explicit solution.
We can justify the method of separation of variables by using the chain rule
to show that
\begin{align*}
    \frac{d}{dx}\int h(y)\,dy &= \frac{d}{dx}\int f(x)\,dx \\
    \frac{d}{dy}\int h(y)\,dy\frac{dy}{dx} &= f(x) \\
    h(y)\frac{dy}{dx} &= f(x) \\
    \frac{dy}{dx} &= \frac{f(x)}{h(y)}=f(x)g(y)
\end{align*}
\begin{problem}
    Solve the differential equation \(\dfrac{dy}{dx}=-xy\).
\end{problem}
\begin{solution}
    Notice that \(y=0\) is a trivial solution then we solve the differential
    equation for non-trivial solutions \(y\neq 0\).
    We use separation of variables then
    \begin{align*}
        \dfrac{dy}{dx} &= -xy \\
        \int\frac{dy}{y} &= -\int x\,dx \\
        \ln |y|+C_1 &= -\frac{x^2}{2}+C_2
    \end{align*}
    Let \(C=C_2-C_1\), then
    \begin{align*}
        \ln |y| &= -\frac{x^2}{2}+C_2-C_1=-\frac{x^2}{2}+C \\
        |y| &= e^{(-x^2/2)+C}=e^Ce^{-x^2/2} \\
        y &= \pm e^Ce^{-x^2/2}=Ae^{-x^2/2}
    \end{align*}
    where \(A\in\R\) is an arbitrary constant.
\end{solution}
\begin{problem}
    Solve the differential equation \(\dfrac{dy}{dx}=\dfrac{x^2}{y^2}\) with
    the initial condition \(y(0)=2\).
\end{problem}
\begin{solution}
    We use separation of variables to find the general solution then
    \begin{align*}
        \dfrac{dy}{dx} &= \dfrac{x^2}{y^2} \\
        \int y^2\,dy &= \int x^2\,dx \\
        \frac{y^3}{3}+C_1 &= \frac{x^3}{3}+C_2 \\
        \frac{y^3}{3} &=\frac{x^3}{3}+C,\quad C=C_2-C_1 \\
        y^3 &= x^3+3C \\
        y &= \sqrt[3]{x^3+K},\quad K=3C
    \end{align*}
    We consider the initial condition \(y(0)=2\) to find the particular
    solution then
    \[y(0)=\sqrt[3]{0+K}\iff 2=\sqrt[3]{K}\iff K=8\]
    The solution of the initial value problem is \(y=\sqrt[3]{x^3+8}\).
\end{solution}
\begin{problem}
    Solve the differential equation \(\dfrac{dy}{dx}=\dfrac{6x^2}{2y+\cos y}\).
\end{problem}
\begin{solution}
    We have
    \begin{align*}
        \dfrac{dy}{dx} &= \dfrac{6x^2}{2y+\cos y} \\
        \int(2y+\cos y)\,dy &= \int 6x^2\,dx \\
        y^2+\sin y &= 2x^3+C
    \end{align*}
    where \(C\in\R\) is an arbitrary constant.
\end{solution}
\begin{problem}
    Solve the differential equation \(y'=x^2y\).
\end{problem}
\begin{solution}
    We have
    \begin{align*}
        \frac{dy}{dx} &= x^2y \\ \int \frac{dy}{y} &= \int x^2\,dx \\
        \ln |y| &= \frac{x^3}{3}+C \\ |y| &= e^{(x^3/3)+C}\\ y &= Ae^{x^3/3}
    \end{align*}
    where \(A\in\R\) is an arbitrary constant.
\end{solution}

\subsubsection*{Homogeneous Equations}
A \textbf{homogeneous} equation is in the form
\[\frac{dy}{dx}=f\left(\frac{y}{x}\right)\]
We can transform a homogeneous equation into a separable equation by a change
of variable.
Consider the equation
\[\frac{dy}{dx}=\frac{y-4x}{x-y}\]
and we can show that
\[\frac{dy}{dx}=\frac{(y/x)-4}{1-(y/x)}\]
thus the equation is homogeneous.
Let \(v=y/x\iff y=vx\) so
\begin{align*}
    \frac{dy}{dx}=\frac{dv}{dx}x+v &= \frac{v-4}{1-v} \\
    \frac{dv}{dx}x
    &= \frac{v-4}{1-v}-v=\frac{v-4-v(1-v)}{1-v}=\frac{v^2-4}{1-v}
\end{align*}
and the equation is separable then
\begin{align*}
    \int\frac{1-v}{v^2-4}\,dv &= \int\frac{dx}{x}
\end{align*}
Since
\[\int\frac{1-v}{v^2-4}\,dv=\int\frac{dv}{v^2-4}-\int\frac{v}{v^2-4}\,dv\]
then
\begin{align*}
    \frac{1}{v^2-4}=\frac{1}{(v+2)(v-2)} &= \frac{A}{v+2}+\frac{B}{v-2} \\
    1 &= A(v-2)+B(v+2) \\
    v=-2\iff -4A=1\iff A &= -\frac{1}{4} \\
    v=2\iff 4B=1\iff B &= \frac{1}{4}
\end{align*}
so
\[\int\frac{dv}{v^2-4}
=-\frac{1}{4}\int\frac{dv}{v+2}+\frac{1}{4}\int\frac{dv}{v-2}
=-\frac{1}{4}\ln|v+2|+\frac{1}{4}\ln|v-2|\]
and
\[\int\frac{v}{v^2-4}\,dv=\frac{1}{2}\ln|v^2-4|\]
therefore
\[\int\frac{1-v}{v^2-4}\,dv
=-\frac{1}{4}\ln|v+2|+\frac{1}{4}\ln|v-2|-\frac{1}{2}\ln|v^2-4|\]
Then
\begin{align*}
    -\frac{1}{4}\ln|v+2|+\frac{1}{4}\ln|v-2|-\frac{1}{2}\ln|v^2-4|
    &= \ln|x|+C_1 \\
    \ln|v-2|-\ln|v+2|-2\ln|v^2-4| &= 4\ln|x|+C_2,\quad C_2=4C_1 \\
    \ln\left|\frac{v-2}{v+2}\right|-\ln((v^2-4)^2)
    =\ln\left|\frac{v-2}{(v+2)(v+2)^2(v-2)^2}\right| &= \ln(x^4)+C_2 \\
    \ln\left|\frac{1}{(v+2)^3(v-2)}\right| &= \ln(x^4)+C_2 \\
    \frac{1}{(v+2)^3(v-2)} &= C_3x^4,\quad C_3=\pm e^{C_2} \\
    (v+2)^3(v-2)x^4 &= C,\quad C=1/C_3 \\
    (vx+2x)^3(vx-2x) &= C
\end{align*}
Therefore the solution is
\[(y+2x)^3(y-2x)=C\]
\begin{problem}
    Solve the differential equation
    \[\frac{dy}{dx}=\frac{x^2+xy+y^2}{x^2}\]
\end{problem}
\begin{solution}
    Since
    \[\frac{dy}{dx}=\frac{x^2+xy+y^2}{x^2}
    =1+\frac{y}{x}+\left(\frac{y}{x}\right)^2\]
    hence the equation is homogeneous.
    Let \(y=vx\), then
    \begin{align*}
        \frac{dy}{dx}=\frac{dv}{dx}x+v &= 1+v+v^2 \\
        \frac{dv}{dx}x &= 1+v^2
    \end{align*}
    and by separation of variables
    \begin{align*}
        \int\frac{dv}{1+v^2} &= \int\frac{dx}{x} \\
        \arctan v &= \ln x+C \\
        \arctan\left(\frac{y}{x}\right)-\ln x &= C \qedhere
    \end{align*}
\end{solution}

\subsubsection*{Orthogonal Trajectories}
An \textbf{orthogonal trajectory} of a family of curves is a curve that
intersects each curve of the family orthogonally, that is, at right angles.
\begin{problem}
    Find the orthogonal trajectories of the family of curves \(x=ky^2\), where
    \(k\) is an arbitrary constant.
\end{problem}
\begin{solution}
    The curves \(x=ky^2\) form a family of parabolas whose axis of symmetry is
    the \(x\)-axis.
    We differentiate \(x=ky^2\) and have
    \begin{align*}
        1 &= 2ky\frac{dy}{dx} \\
        \frac{dy}{dx} &= \frac{1}{2ky}
    \end{align*}
    and since \(k=x/y^2\) so we get
    \[\frac{dy}{dx}=\frac{1}{2(x/y^2)y}=\frac{y}{2x}\]
    which is the slope of the tangent line at any point \((x,y)\) on one of
    the parabolas.
    On an orthogonal trajectory the slope of the tangent line must be the
    negative reciprocal of this slope.
    Therefore the orthogonal trajectories must satisfy the differential
    equation
    \[\frac{dy}{dx}=-\frac{2x}{y}\]
    This differential equation is separable, and we solve it as follows:
    \begin{align*}
        \int y\,dy &= -\int 2x\,dx \\
        \frac{y^2}{2} &= -x^2+C \\
        x^2+\frac{y^2}{2} &= C
    \end{align*}
    where \(C\) is an arbitrary positive constant.
    Thus the orthogonal trajectories are the family of ellipses given by the
    equation \(x^2+(y^2/2)=C\).
\end{solution}

\subsubsection*{Mixing Problems}
%\subsection{Population Growth}
Let \(y=y(t)\) be a function representing the value of a quantity \(y\) at
time \(t\) such that
\[\frac{dy}{dt}=ky\]
where \(k\) is a constant, then the differential equation is the law of
natural growth if \(k>0\) or the law of natural decay if \(k<0\).
Since the differential equation is separable, then
\begin{align*}
    \frac{dy}{dt} &= ky \\
    \int\frac{dy}{y} &= \int k\,dt \\
    \ln|y| &= kt+C \\
    |y| &= e^{kt+C}=e^Ce^{kt} \\
    y &= \pm e^Ce^{kt}=Ae^{kt}
\end{align*}
so \(y=Ae^{kt}\) is the general solution of the differential equation.
It follows that
\[\frac{dy}{dt}=kAe^{kt}=ky\]
and \(y(0)=Ae^0=A\) is the initial value of the function \(y(t)\).

\subsubsection*{Logistic Growth}
If \(M\) is the carrying capacity and \(0<y<M\), then the logistic
differential equation is
\[\frac{dy}{dt}=ky(M-y)\]
We use separation of variables then
\begin{align*}
    \frac{dy}{dt} &= ky(M-y) \\
    \int\frac{dy}{y(M-y)} &= \int k\,dt
\end{align*}
and using partial fractions
\begin{align*}
    \frac{1}{y(M-y)} &= \frac{A}{y}+\frac{B}{M-y} \\
    1 &= A(M-y)+By \\
    y=0 &\iff A=\frac{1}{M} \\
    y=M &\iff B=\frac{1}{M}
\end{align*}
therefore
\begin{align*}
    \frac{1}{M}\int\left(\frac{1}{y}+\frac{1}{M-y}\right)\,dy &= \int k\,dt \\
    \frac{1}{M}(\ln|y|-\ln|M-y|) &= kt+C_1 \\
    \ln\frac{y}{M-y} &= kMt+C_2 \\
    \frac{y}{M-y} &= e^{kMt+C_2}=e^{C_2}e^{kMt}=Ae^{kMt}
\end{align*}
If the population at time \(t=0\) is \(y(0)=y_0\),
then \(A=y_0/(M-y_0)\) and so
\begin{align*}
    \frac{y}{M-y} &= \frac{y_0}{M-y_0}e^{kMt} \\
    (M-y_0)y &= y_0e^{kMt}(M-y)=y_0Me^{kMt}-y_0e^{kMt}y \\
    (M-y_0)y+y_0e^{kMt}y=(M-y_0+y_0e^{kMt})y &= y_0Me^{kMt} \\
    y &= \frac{y_0Me^{kMt}}{M-y_0+y_0e^{kMt}}
\end{align*}
then
\[y=\frac{y_0M}{(M-y_0+y_0e^{kMt})e^{-kMt}}=\frac{y_0M}{y_0+(M-y_0)e^{-kMt}}\]
is the solution of the differential equation and
\[\lim_{t\to\infty}y(t)=\frac{y_0M}{y_0+0}=M\]
We can show that
\begin{align*}
    \frac{d^2y}{dt^2} &= \frac{d}{dt}(kMy-ky^2)=kM\frac{dy}{dt}-2ky\frac{dy}{dt}
    =k(M-2y)\frac{dy}{dt}=k(M-2y)ky(M-y) \\
    &= k^2y(M-y)(M-2y)
\end{align*}
then
\[k^2y(M-y)(M-2y)=0\]
and
\begin{align*}
    k^2y &= 0 & M-y &= 0 & M-2y &= 0 \\
    y &= 0 & y &= M & y &= \frac{M}{2}
\end{align*}
so
\[\left.\frac{dy}{dt}\right|_{y=0}=k(0)(M-0)=0\]
\[\left.\frac{dy}{dt}\right|_{y=M}=kM(M-M)=0\]
\[\left.\frac{dy}{dt}\right|_{y=M/2}=k\frac{M}{2}\left(M-\frac{M}{2}\right)
=\frac{kM^2}{4}\]
We deduce that a population grows fastest when it reaches half its carrying
capacity.

\begin{problem}
    An object of mass \(m\) is moving horizontally through a medium which resists the
    motion with a force that is a function of the velocity
    \[m\frac{d^2s}{dt^2}=m\frac{dv}{dt}=f(v)\]
    where \(v=v(t)\) and \(s=s(t)\) represent the velocity and position of the
    object at time \(t\), respectively.
    Let \(v(0)=v_0\) and \(s(0)=s_0\) be the initial values of \(v\) and \(s\).
    Suppose that the resisting force is proportional to the velocity for small
    values of \(v\) so
    \[f(v)=-kv\]
    where \(k\) is a positive constant.
    For large values of \(v\) a better model is
    \[f(v)=-kv^2\]
    where \(k\) is a positive constant.
    Determine \(v\) and \(s\) at any time \(t\) for each of the two models.
    What is the total distance that the object travels in each case?
\end{problem}
\begin{solution}
    For small values of \(v\) we have
    \begin{align*}
        m\frac{dv}{dt} &= -kv \\
        \int\frac{dv}{v} &= -\int\frac{k}{m}\,dt \\
        \ln(|v|) &= -\frac{k}{m}t+C \\
        |v| &= e^{-(k/m)t+C}=e^Ce^{-(k/m)t} \\
        v &= \pm e^Ce^{-(k/m)t}=Ae^{-(k/m)t}
    \end{align*}
    and since \(v(0)=A=v_0\) hence \(v=v_0e^{-(k/m)t}\).
    Then
    \[s=\int v_0e^{-(k/m)t}\,dt=-\frac{m}{k}v_0e^{-(k/m)t}+C\]
    and since \(s(0)=-(m/k)v_0+C=s_0\iff C=s_0+(m/k)v_0\) hence
    \[s=-\frac{m}{k}v_0e^{-(k/m)t}+s_0+\frac{m}{k}v_0
    =s_0+\frac{m}{k}v_0(1-e^{-(k/m)t})\]
    The total distance traveled by the object is
    \begin{align*}
        \lim_{t\to\infty}(s(t)-s(0))
        &= \lim_{t\to\infty}\left(s_0+\frac{m}{k}v_0(1-e^{-(k/m)t})-s_0\right)
        =\lim_{t\to\infty}\frac{m}{k}v_0(1-e^{-(k/m)t}) \\
        &= \frac{m}{k}v_0(1-0)=\frac{m}{k}v_0
    \end{align*}
    For large values of \(v\) we have
    \begin{align*}
        m\frac{dv}{dt} &= -kv^2 \\
        \int\frac{dv}{v^2} &= -\int\frac{k}{m}\,dt \\
        -\frac{1}{v} &= -\frac{k}{m}t+C_1 \\
        \frac{1}{v} &= \frac{k}{m}t+C=\frac{kt+Cm}{m} \\
        v &= \frac{m}{kt+Cm}
    \end{align*}
    and since \(v(0)=m/(Cm)=v_0\iff C=1/v_0\) hence
    \[v=\frac{m}{kt+(m/v_0)}\]
    Then
    \begin{align*}
        s &= \int\frac{m}{kt+(m/v_0)}\,dt=\frac{m}{k}\int\frac{du}{u}
        =\frac{m}{k}\ln(|u|)+C
        =\frac{m}{k}\ln\left(\left|kt+\frac{m}{v_0}\right|\right)+C
    \end{align*}
    and since \(s(0)=(m/k)\ln(|m/v_0|)+C=s_0\iff C=s_0-(m/k)\ln(|m/v_0|)\)
    hence
    \begin{align*}
        s &= \frac{m}{k}\ln\left(\left|kt+\frac{m}{v_0}\right|\right)+s_0
        -\frac{m}{k}\ln\left(\left|\frac{m}{v_0}\right|\right)
        =s_0+\frac{m}{k}\left(\ln\left(\left|kt+\frac{m}{v_0}\right|\right)
        -\ln\left(\left|\frac{m}{v_0}\right|\right)\right) \\
        &= s_0+\frac{m}{k}
        \ln\left(\left|\frac{kt+(m/v_0)}{m/v_0}\right|\right)
        =s_0+\frac{m}{k}\ln\left(\left|\frac{k}{m}v_0t+1\right|\right)
    \end{align*}
    Since \(m\) is the mass so \(m\) is positive and given \(k\) is positive
    therefore
    \[s=s_0+\frac{m}{k}\ln\left(\frac{k}{m}v_0t+1\right)\]
    and the distance traveled by the object is
    \[\lim_{t\to\infty}(s(t)-s(0))
    =\lim_{t\to\infty}\left(s_0
    +\frac{m}{k}\ln\left(\frac{k}{m}v_0t+1\right)-s_0\right)=
    \lim_{t\to\infty}\frac{m}{k}\ln\left(\frac{k}{m}v_0t+1\right)
    =\infty\qedhere\]
\end{solution}
\begin{problem}
    According to Newton's law of universal gravitation,
    the gravitational force on an object of mass \(m\) that has been
    projected vertically upward from the earth's surface is
    \[F=\frac{mgR^2}{(x+R)^2}\]
    where \(x=x(t)\) is the object's distance
    above the surface at time \(t\), \(R\) is the
    earth's radius, and \(g\) is the acceleration due to gravity.
    Also, by Newton's second law,
    \[F=ma=m\frac{dv}{dt}=-\frac{mgR^2}{(x+R)^2}\]
    Suppose a rocket is fired vertically upward with an initial velocity \(v_0\).
    Let \(h\) be the maximum height above the surface reached by the object.
    Show that \(v_0=\sqrt{\dfrac{2gRh}{R+h}}\) and compute
    \(v_e=\lim_{h\to\infty}v_0\), the escape velocity of the earth, using
    \(R=6,378\) km and \(g=9.8\ \text{m/s}^2\).
\end{problem}
\begin{solution}
    By the chain rule
    \[\frac{dv}{dt}=\frac{dx}{dt}\frac{dv}{dx}=v\frac{dv}{dx}\]
    then
    \begin{align*}
        m\frac{dv}{dt}=mv\frac{dv}{dx} &= -\frac{mgR^2}{(x+R)^2} \\
        v\,dv &= -\frac{gR^2}{(x+R)^2}\,dx
    \end{align*}
    Since the height is zero when \(v=v_0\) and the height is maximum when
    \(v=0\) then
    \begin{align*}
        \int_{v_0}^0 v\,dv &= -\int_0^h \frac{gR^2}{(x+R)^2}\,dx \\
        \left[\frac{v^2}{2}\right]_{v_0}^0
        &= \left[\frac{gR^2}{x+R}\right]_0^h \\
        -\frac{(v_0)^2}{2} &= gR^2\left(\frac{1}{R+h}-\frac{1}{R}\right)
        =gR^2\left(\frac{R-(R+h)}{R(R+h)}\right)=-\frac{gRh}{R+h} \\
        v_0= &= \sqrt{\dfrac{2gRh}{R+h}}
    \end{align*}
    The escape velocity of the earth is
    \begin{align*}
        v_e &= \lim_{h\to\infty}\sqrt{\dfrac{2gRh}{R+h}}
        =\sqrt{\lim_{h\to\infty}\frac{2gR}{(R/h)+1}}=\sqrt{\frac{2gR}{0+1}}
        =\sqrt{2gR} \\
        &=\sqrt{2(9.8)(6.378\times10^6)}\ \text{m/s}
        \approx 11.1807\ \text{km/s}
    \end{align*}
\end{solution}

\section*{The End}
\end{document}