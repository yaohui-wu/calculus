\subsection{Lagrange's Mean Value Theorem}

\begin{theorem}[Rolle's Theorem]
    Let \(f\) be a function that satisfies the following three hypotheses:
    \begin{enumerate}
        \item \(f\) is continuous on the closed interval \([a,b]\).
        \item \(f\) is differentiable on the open interval \((a,b)\).
        \item \(f(a)=f(b)\)
    \end{enumerate}
    Then there is a number \(c\) in \((a,b)\) such that \(f'(c)=0\).
\end{theorem}
\begin{proof}
    There are three cases: \\
    Case 1: \(f(x)=k\), a constant \\
    Then \(f'(x)=0\) so the number \(c\) can be any number in \((a,b)\). \\
    Case 2: \(f(x)>f(a)\) for some \(x\) in \((a,b)\) \\
    By the Extreme Value Theorem, \(f\) has a maximum value somewhere in
    \([a,b]\).
    Since \(f(a)=f(b)\), it must attain this maximum value at a number \(c\)
    in the open interval \((a,b)\).
    Then \(f\) has a local maximum at \(c\).
    Therefore \(f'(c)=0\) by Fermat's Theorem. \\
    Case 3: \(f(x)<f(a)\) for some \(x\) in \((a,b)\) \\
    By the Extreme Value Theorem, \(f\) has a minimum value somewhere in
    \([a,b]\), and since \(f(a)=f(b)\), it attains this minimum value at a
    number \(c\) in \((a,b)\).
    Then \(f'(c)=0\) by Fermat's Theorem.
\end{proof}
\begin{theorem}[Lagrange's Mean Value Theorem]
    Let \(f\) be a function that satisfies the following hypotheses:
    \begin{enumerate}
        \item \(f\) is continuous on the closed interval \([a,b]\).
        \item \(f\) is differentiable on the open interval \((a,b)\).
    \end{enumerate}
    Then there is a number \(c\) in \((a,b)\) such that
    \[f'(c)=\frac{f(b)-f(a)}{b-a}\]
    or, equivalently,
    \[f(b)-f(a)=f'(c)(b-a)\]
\end{theorem}
\begin{proof}
    We apply Rolle's Theorem to a new function \(h\) defined as a difference
    between \(f\) and the function whose graph is the secant line \(AB\) where
    \(A\) is the point \((a,f(a))\) and \(B\) is the point \((b,f(b))\).
    The slope of the secant line \(AB\) is
    \[m=\frac{f(b)-f(a)}{b-a}\]
    Then the equation of the secant line \(AB\) is
    \[y-f(a)=\frac{f(b)-f(a)}{b-a}(x-a)\]
    or
    \[y=f(a)+\frac{f(b)-f(a)}{b-a}(x-a)\]
    So,
    \[h(x)=f(x)-f(a)-\frac{f(b)-f(a)}{b-a}(x-a)\]
    First we must verify that \(h\) satisfies the three hypotheses of Rolle's
    Theorem.
    \begin{enumerate}
        \item The function \(h\) is continuous on \([a,b]\) because it is the
        sum of \(f\) and a first degree polynomial, both of which is
        continuous.
        \item The function \(h\) is differentiable on \((a,b)\) because both
        \(f\) and a first degree polynomial are differentiable.
        In fact, we can compute \(h'\) directly:
        \[h'(x)=f'(x)-\frac{f(b)-f(a)}{b-a}\]
        Note that \(f(a)\) and \(\dfrac{f(b)-f(a)}{b-a}\) are constants.
        \item
        \[h(a)=f(a)-f(a)-\frac{f(b)-f(a)}{b-a}(a-a)=0\]
        \[h(b)=f(b)-f(a)-\frac{f(b)-f(a)}{b-a}(b-a)
        =f(b)-f(a)-\big[f(b)-f(a)\big]=0\]
        Therefore \(h(a)=h(b)\).
        Since \(h\) satisfies the hypotheses of Rolle's Theorem,
        there is a number \(c\) in \((a,b)\) such that \(h'(c)=0\).
        Therefore
        \[0=h'(c)=f'(c)-\frac{f(b)-f(a)}{b-a}\]
        and so
        \[f'(c)=\frac{f(b)-f(a)}{b-a}\]
    \end{enumerate}
\end{proof}
In general, Lagrange's Mean Value Theorem can be interpreted as saying that
there is a number at which the instantaneous rate of change is equal to the
average rate of change over an interval.
The main significance of Lagrange's Mean Value Theorem is that it enables
us to obtain information about a function from information about its
derivative.
\begin{theorem}
    If \(f'(x)=0\) for all \(x\) in an interval \((a,b)\), then \(f\) is a
    constant on \((a,b)\).
\end{theorem}
\begin{proof}
    Let \(x_1\) and \(x_2\) be any two numbers in \((a,b)\) with \(x_1<x_2\).
    Since \(f\) is differentiable on \((a,b)\), it must be differentiable on
    \((x_1,x_2)\) and continuous on \([x_1,x_2]\).
    By applying Lagrange's Mean Value Theorem to \(f\) on the interval
    \([x_1,x_2]\), we get a number \(c\) such that \(x_1<c<x_2\) and
    \[f(x_2)-f(x_1)=f'(c)(x_2-x_1)\]
    Since \(f'(x)=0\) for all \(x\), we have \(f'(c)=0\), and so
    \[f(x_2)-f(x_1)=0\]
    or
    \[f(x_2)=f(x_1)\]
    Therefore \(f\) has the same value at any two numbers \(x_1\) and \(x_2\)
    in \((a,b)\).
    This means that \(f\) is a constant on \((a,b)\).
\end{proof}
\begin{corollary}
    If \(f'(x)=g'(x)\) for all \(x\) in an interval \((a,b)\), then \(f-g\) is
    a constant on \((a,b)\); that is; \(f(x)=g(x)+c\) where \(c\) is a
    constant.
\end{corollary}
\begin{proof}
    Since \(f'(x)=g'(x)\), we have \(f'(x)-g'(x)=0\).
    Let \(F(x)=f(x)-g(x)\).
    Then
    \[F'(x)=f'(x)-g'(x)=0\]
    for all \(x\) in \((a,b)\).
    Thus \(F\) is a constant; that is, \(f-g\) is a constant.
\end{proof}
\begin{theorem}[Cauchy's Mean Value Theorem]
    Suppose that the fucntions \(f\) and \(g\) are continuous on \([a,b]\) and
    differentiable on \((a,b)\), and \(g(x)\neq 0\) for all \(x\) in
    \((a,b)\).
    Then there is a number \(c\) in \((a,b)\) such that
    \[\frac{f'(c)}{g'(c)}=\frac{f(b)-f(a)}{g(b)-g(a)}\]
\end{theorem}