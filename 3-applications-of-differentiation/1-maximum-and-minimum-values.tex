\subsection{Maximum and Minimum Values}

\begin{definition}
    Let \(c\) be a number in the domain \(D\) of a function \(f\).
    Then \(f(c)\) is the
    \begin{itemize}
        \item \textbf{absolute maximum} value of \(f\) on \(D\) if
        \(f(c)\geq f(x)\) for all \(x\) in \(D\).
        \item \textbf{absolute minimum} value of \(f\) on \(D\) if
        \(f(c)\leq f(x)\) for all \(x\) in \(D\).
    \end{itemize}
\end{definition}
An absolute maximum or minimum is sometimes called a \textbf{global} maximum
or minimum.
The maximum and minimum values of \(f\) are called \textbf{extreme values} of
\(f\).
\begin{definition}
    The number \(f(c)\) is a
    \begin{itemize}
        \item \textbf{local maximum} value of \(f\) if \(f(c)\geq f(x)\) when
        \(x\) is near \(c\).
        \item \textbf{local minimum} value of \(f\) if \(f(c)\leq f(x)\) when
        \(x\) is near \(c\).
    \end{itemize}
\end{definition}
\begin{theorem}[Extreme Value Theorem]
    If \(f\) is continuous on a closed interval \([a,b]\), then \(f\) attains
    an absolute maximum value \(f(c)\) and absolute minimum value \(f(d)\) at
    some numbers \(c\) and \(d\) in \([a,b]\).
\end{theorem}
\begin{theorem}[Fermat's Theorem]
    If \(f\) has a local maximum or minimum at \(c\), and if \(f'(c)\) exists,
    then \(f'(c)=0\).
\end{theorem}
\begin{proof}
    Suppose that \(f\) has a local maximum at \(c\).
    Then \(f(c)\geq f(x)\) if \(x\) is sufficiently close to \(c\).
    This implies that if \(h\) is sufficiently close to 0, with \(h\) being
    positive or negative, then
    \[f(c)\geq f(c+h)\]
    and theorefore
    \[f(c+h)-f(c)\leq 0\]
    If \(h>0\) and \(h\) is sufficiently small, we have
    \[\frac{f(c+h)-f(c)}{h}\leq 0\]
    Taking the right-hand limit, we get
    \[\lim_{h\to 0^+}\frac{f(c+h)-f(c)}{h}\leq\lim_{h\to 0^+}0=0\]
    But since \(f'(c)\) exists, we have
    \[f'(c)=\lim_{h\to 0}\frac{f(c+h)-f(c)}{h}
    =\lim_{h\to 0^+}\frac{f(c+h)-f(c)}{h}\]
    and so we have shown that \(f'(c)\leq 0\).
    If \(h<0\), then
    \[\frac{f(c+h)-f(c)}{h}\geq 0\]
    Taking the left-hand limit, we have
    \[f'(c)=\lim_{h\to 0}\frac{f(c+h)-f(c)}{h}
    =\lim_{h\to 0^-}\frac{f(c+h)-f(c)}{h}\geq 0\]
    We have shown that \(f'(c)\leq 0\) and \(f'(c)\geq 0\).
    Therefore \(f'(c)=0\).
    The case of a local minimum can be proved in a similar manner.
\end{proof}
\begin{definition}
    A \textbf{critical number} of a function \(f\) is a number \(c\) in the
    domain of \(f\) such that either \(f'(c)=0\) or \(f'(c)\) does not exist.
\end{definition}
If \(f\) has a local maximum or minimum at \(c\), then \(c\) is a critical
number of \(f\).
The Closed Interval Method: To find the absolute maximum and minimum values of
a continuous function \(f\) on a closed interval \([a,b]\):
\begin{enumerate}
    \item Find the values of \(f\) at the critical numbers of \(f\) in
    \((a,b)\).
    \item Find the values of \(f\) at the endpoints of the interval.
    \item The largest of these values is the absolute maximum value; the
    smallest of these values is the absolute minimum value.
\end{enumerate}