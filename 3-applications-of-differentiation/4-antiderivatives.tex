\subsection{Antiderivatives}

\subsubsection*{The Antiderivative of a Function}
\begin{definition}
    A function \(F\) is an \textbf{antiderivative} of \(f\) on an interval
    \(I\) if \(F'(x)=f(x)\) for all \(x\) in \(I\).
\end{definition}
\begin{theorem}
    If \(F\) is an antiderivative of \(f\) on interval \(I\), then the general
    antiderivative of \(f\) on \(I\) is \(F(x)+C\) where \(C\) is an arbitrary
    constant.
\end{theorem}
\begin{problem}
    Find the general antiderivative of \(f(x)=1/x\).
\end{problem}
\begin{solution}
    We know that \[\frac{d}{dx}\ln |x|=\frac{1}{x}\] for all \(x\neq 0\).
    Then the antiderivative of \(f\) is \(F(x)=\ln x+C_1\) if \(x>0\)
    or \(F(x)=\ln(-x)+C_2\) if \(x<0\).
    Therefore, the general antiderivative of \(f\) is \(F(x)=\ln |x|+C\).
\end{solution}