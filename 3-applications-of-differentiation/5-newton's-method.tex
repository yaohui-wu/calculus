\subsection{Newton's Method}

We have \textbf{Newton's Method} to find an approximate numerical value of the
root of an equation.
Let \(r\) be the root of the equation.
We start with a first approximation \(x_1\).
Consider the tangent line \(L\) to the curve \(y=f(x)\) at the point
\((x_1,f(x_1))\) and we label the \(x\)-intercept of \(L\) as \(x_2\).
The slope of \(L\) is \(f(x_1)\), so its equation is
\[y-f(x_1)=f'(x_1)(x-x_1)\]
Since the \(x\)-intercept of \(L\) is \(x_2\), we set \(y=0\) and obtain
\[0-f(x_1)=f'(x_1)(x_2-x_1)\]
If \(f'(x_1)\neq 0\), we can solve this equation for \(x_2\):
\[x_2=x_1-\frac{f(x_1)}{f'(x_1)}\]
We use \(x_2\) as a second approximation to \(r\).
Next we repeat this procedure with \(x_1\) replaced by \(x_2\), using the
tangent line at \((x_2,f(x_2))\).
This gives a third approximation:
\[x_3=x_2-\frac{f(x_2)}{f'(x_2)}\]
If we keep repeating this process, we obtain a sequence of approximations
\(x_1,x_2,x_3,x_4,\dots\).
In general, if the \(n\)th approximation is \(x_n\) and \(f'(x_n)\neq 0\),
then the next approximation is given by
\[x_{n+1}=x_n-\frac{f(x_n)}{f'(x_n)}\]
If the numbers \(x_n\) become closer and closer to \(r\) as \(n\) becomes
large, then we say that the sequence converges to \(r\) and we write
\[\lim_{n\to\infty}x_n=r\]