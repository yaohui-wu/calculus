\subsection{Trigonometric Integrals and Substitutions}

\subsubsection*{Trigonometric Integrals}
We can use trigonometric identities to integrate trigonometric integrals.
\begin{problem}
    Evaluate \(\displaystyle{\int\sin 5x\sin 2x\,dx}\).
\end{problem}
\begin{solution}
    \begin{align*}
        \int\sin 5x\sin 2x\,dx &= \int\frac{1}{2}(\cos(5x-2x)-\cos(5x+2x))\,dx
        =\frac{1}{2}\int(\cos 3x-\cos 7x)\,dx \\
        &= \frac{1}{2}\left(\frac{1}{3}\sin 3x-\frac{1}{7}\sin 7x\right)+C
        =\frac{1}{6}\sin 3x-\frac{1}{14}\sin 7x+C
    \end{align*}
\end{solution}
\begin{problem}
    Evaluate \(\displaystyle{\int\sin 3x\cos x\,dx}\).
\end{problem}
\begin{solution}
    \begin{align*}
        \int\sin 3x\cos x\,dx &= \int\frac{1}{2}(\sin(3x+x)+\sin(3x-x))\,dx
        =\frac{1}{2}\int(\sin 4x+\sin 2x)\,dx \\
        &= \frac{1}{2}\left(-\frac{1}{4}\cos 4x-\frac{1}{2}\cos 2x\right)+C
        =-\frac{1}{8}\cos 4x-\frac{1}{4}\cos 2x+C
    \end{align*}
\end{solution}
\begin{problem}
    Evaluate \(\displaystyle{\int\cos^3 x\,dx}\).
\end{problem}
\begin{solution}
    \[\int\cos^3 x\,dx=\int(1-\sin^2 x)\cos x\,dx=\int(1-u^2)\,du
    =u-\frac{u^3}{3}+C=\sin x-\frac{1}{3}\sin^3 x+C\]
\end{solution}
\begin{problem}
    Evaluate \(\displaystyle{\int\sin^5 x\cos^2 x\,dx}\).
\end{problem}
\begin{solution}
    \begin{align*}
        \int\sin^5 x\cos^2 x\,dx &= \int(1-\cos^2 x)^2\cos^2 x\sin x\,dx
        =-\int(1-u^2)^2u^2\,du \\
        &= -\int(u^2-2u^4+u^6)\,du
        =-\frac{1}{3}u^3+\frac{2}{5}u^5-\frac{1}{7}u^7+C \\
        &= -\frac{1}{3}\cos^3 x+\frac{2}{5}\cos^5 x-\frac{1}{7}\cos^7 x
        +C
    \end{align*}
\end{solution}
\begin{problem}
    Evaluate \(\displaystyle{\int\cos^2 x\,dx}\).
\end{problem}
\begin{solution}
    \[\int\cos^2 x\,dx=\int\frac{1}{2}(1+\cos 2x)\,dx
    =\frac{1}{2}\int(1+\cos 2x)\,dx
    =\frac{1}{2}\left(x+\frac{1}{2}\sin 2x\right)+C\]
\end{solution}
\begin{problem}
    Evaluate \(\displaystyle{\int_0^\pi\sin^2 x\,dx}\).
\end{problem}
\begin{solution}
    \begin{align*}
    \int_0^\pi\sin^2 x\,dx &= \int_0^\pi\frac{1}{2}(1-\cos 2x)\,dx
    =\frac{1}{2}\int_0^{\pi}(1+\cos 2x)\,dx \\
    &=\frac{1}{2}
    \left(\bigl[x\bigr]_0^\pi-\frac{1}{2}\bigl[\sin 2x\bigr]_0^\pi\right)
    =\frac{\pi}{2}
    \end{align*}
\end{solution}
\begin{problem}
    Evaluate \(\displaystyle{\int\sin^4 x\,dx}\).
\end{problem}
\begin{solution}
    \begin{align*}
        \int\sin^4 x\,dx &= \int\left(\frac{1}{2}(1-\cos 2x)\right)^2\,dx
        =\frac{1}{4}\int(1-2\cos 2x+\cos^2 2x)\,dx \\
        &= \frac{1}{4}
        \left(x-\sin 2x+\frac{1}{2}
        \left(x+\frac{1}{4}\sin 4x\right)\right)+C \\
        &= \frac{1}{4}
        \left(\frac{3}{2}x-\sin 2x+\frac{1}{8}\sin 4x\right)+C
    \end{align*}
\end{solution}
In general, an integral of powers of \(\sin x\) and \(\cos x\) is in the
form
\[\int\sin^m x\cos^n x\,dx\]
where \(m,n\in\Z\) and \(m,n\geq 0\).
If \(m\) is odd, then we save a factor of \(\sin x\) and express the rest in
terms of \(\cos x\) for substitution.
If \(n\) is odd, then we save a factor of \(\cos x\) and express the rest in
terms of \(\sin x\) for substitution.
If \(m\) and \(n\) are even, then we use the power reduction formulas.
\begin{problem}
    Evaluate \(\displaystyle{\int\tan^6 x\sec^4 x\,dx}\).
\end{problem}
\begin{solution}
    \begin{align*}
        \int\tan^6 x\sec^4 x\,dx &= \int\tan^6 x(\tan^2 x+1)\sec^2 x\,dx
        =\int u^6(u^2+1)\,du=\int u^8+u^6 \\
        & =\frac{u^9}{9}+\frac{u^7}{7}
        =\frac{1}{9}\tan^9 x+\frac{1}{7}\tan^7 x+C
    \end{align*}
\end{solution}
\begin{problem}
    Evaluate \(\displaystyle{\int\tan^5 x\sec^7 x\,dx}\).
\end{problem}
\begin{solution}
    \begin{align*}
        \int\tan^5 x\sec^7 x\,dx
        &= \int(\sec^2 x-1)^2\sec^6 x\sec x\tan x\,dx
        =\int(u^2-1)^2u^6\,du \\
        &= \int(u^{10}-2u^8+u^6)\,dx
        =\frac{1}{11}u^{11}-\frac{2}{9}u^9+\frac{1}{7}u^7+C \\
        &= \frac{1}{11}\sec^{11}x-\frac{2}{9}\sec^9 x+\frac{1}{7}\sec^7 x
        +C
    \end{align*}
\end{solution}
In general, an integral of powers of \(\tan x\) and \(\sec x\) is in the
form
\[\int\tan^m x\sec^n x\,dx\]
where \(m,n\in\Z\) and \(m,n\geq 0\).
If \(m\) is odd, then we save a factor of \(\sec x\tan x\) and express the
rest in terms of \(\sec x\) for substitution.
If \(n\) is even, then we save a factor of \(\sec^2 x\) and express the
rest in terms of \(\tan x\) for substitution.
\begin{problem}
    Evaluate \(\displaystyle{\int\sec x\,dx}\).
\end{problem}
\begin{solution}
    We have
    \[\int\sec x\,dx=\int\sec x\frac{\sec x+\tan x}{\sec x+\tan x}\,dx
    =\int\frac{\sec^2 x+\sec x\tan x}{\sec x+\tan x}\,dx\]
    Let \(u=\sec x+\tan x\iff du=(\sec x\tan x+\sec^2 x)\,dx\), then
    \[\int\sec x\,dx=\int\frac{\sec x\tan x+\sec^2 x}{\sec x+\tan x}\,dx
    =\int\frac{du}{u}=\ln|u|+C=\ln|\sec x+\tan x|+C\]
\end{solution}
\begin{problem}
    Evaluate \(\displaystyle{\int\tan^3 x\,dx}\).
\end{problem}
\begin{solution}
    \begin{align*}
    \int\tan^3 x\,dx &= \int\tan x(\sec^2x-1)\,dx
    =\int\tan x\sec^2\,dx-\int\tan x\,dx \\
    &= \frac{1}{2}\tan^2 x-\ln|\sec x|+C
    \end{align*}
\end{solution}
\begin{problem}
    Evaluate \(\displaystyle{\int\sec^3 x\,dx}\).
\end{problem}
\begin{solution}
    Let \(u=\sec x\iff du=\sec x\tan x\) and \(dv=\sec^2 x\,dx\iff v=\tan x\),
    then
    \[\int\sec^3 x\,dx=\sec x\tan x-\int\tan^2 x\sec x\,dx\]
    We have
    \[\int\tan^2 x\sec x\,dx=\int(\sec^2 x-1)\sec x\,dx
    =\int\sec^3 x\,dx-\int\sec x\,dx\]
    Therefore,
    \begin{align*}
        \int\sec^3 x\,dx &= \sec x\tan x-\int\sec^3 x\,dx+\int\sec x\,dx \\
        \int\sec^3 x\,dx
        &= \frac{1}{2}(\sec x\tan x+\ln|\sec x+\tan x|)+C
    \end{align*}
\end{solution}

\subsubsection*{Trigonometric Substitutions}
If an integral has the form \(\displaystyle{\int\sqrt{a^2-x^2}\,dx}\),
then we can use the substitution \(x=a\sin\theta\) where
\(-\pi/2\leq\theta\leq\pi/2\) to get
\[\sqrt{a^2-x^2}=\sqrt{a^2-a^2\sin^2\theta}=\sqrt{a^2(1-\sin^2\theta)}
=\sqrt{a^2\cos^2\theta}=|a|\cos\theta\]
If an integral has the form \(\displaystyle{\int\sqrt{a^2+x^2}\,dx}\),
then we can use the substitution \(x=a\tan\theta\) where
\(-\pi/2<\theta<\pi/2\) to get
\[\sqrt{a^2+x^2}=\sqrt{a^2+a^2\tan^2\theta}=\sqrt{a^2(\tan^2\theta+1)}
=\sqrt{a^2\sec^2\theta}=|a|\sec\theta\]
If an integral has the form \(\displaystyle{\int\sqrt{x^2-a^2}\,dx}\),
then we can use the substitution \(x=a\sec\theta\) where
\(0\leq\theta<\pi/2\) or \(\pi\leq\theta<3\pi/2\) to get
\[\sqrt{x^2-a^2}=\sqrt{a^2\sec^2\theta-a^2}=\sqrt{a^2(\sec^2\theta-1)}
=\sqrt{a^2\tan^2\theta}=|a|\tan\theta\]
\begin{problem}
    Evaluate \(\displaystyle{\int\frac{\sqrt{9-x^2}}{x^2}\,dx}\)
\end{problem}
\begin{solution}
    Let \(x=3\sin\theta\iff dx=3\cos\theta\,d\theta\), then
    \begin{align*}
        \int\frac{\sqrt{9-x^2}}{x^2}\,dx
        &= \int\frac{3\cos\theta}{9\sin^2\theta}3\cos\theta\,d\theta
        =\int\frac{\cos^2\theta}{\sin^2\theta}=\int\cot^2\theta\,d\theta
        =\int(\csc^2\theta-1)\,d\theta \\
        &=-\cot\theta-\theta+C
        =-\frac{\sqrt{9-x^2}}{x}-\arcsin\left(\frac{x}{3}\right)+C
    \end{align*}
\end{solution}
\begin{problem}
    Evaluate \(\displaystyle{\int\frac{dx}{x^2\sqrt{x^2+4}}}\).
\end{problem}
\begin{solution}
    Let \(x=2\tan\theta\iff dx=2\sec^2\theta\,d\theta\) and
    \(-\pi/2<\theta<\pi/2\), then
    \begin{align*}
        \int\frac{dx}{x^2\sqrt{x^2+4}}
        &= \int\frac{2\sec^2\theta}{4\tan^2\theta(2\sec\theta)}\,d\theta
        =\frac{1}{4}\int\frac{\sec\theta}{\tan^2\theta}\,d\theta
        =\frac{1}{4}\int\frac{\cos\theta}{\sin^2\theta}\,d\theta
        =\frac{1}{4}\int\frac{du}{u^2} \\
        &= -\frac{1}{4u}+C=-\frac{1}{4\sin\theta}+C
        =-\frac{\sqrt{x^2+4}}{4x}+C
    \end{align*}
\end{solution}
\begin{problem}
    Find the area enclosed by a circle with radius \(r\).
\end{problem}
\begin{solution}
    The area is \(\displaystyle{A=4\int_0^r\sqrt{r^2-x^2}\,dx}\).
    Let \(x=r\sin\theta\iff dx=r\cos\theta\,d\theta\) and
    \(0\leq\theta\leq\pi/2\), then
    \begin{align*}
        A &= 4\int_0^{\pi/2}r\cos\theta(r\cos\theta)\,d\theta
        =4r^2\int_0^{\pi/2}\cos^2\theta\,d\theta
        =4r^2\left(\frac{1}{2}\left(\bigl[\theta\bigr]_0^{\pi/2}
        +\frac{1}{2}\bigl[\sin2\theta\bigr]_0^{\pi/2}\right)\right) \\
        &= 4r^2\left(\frac{\pi}{4}\right)=\pi r^2
    \end{align*}
\end{solution}
\begin{problem}
    Find the area enclosed by the ellipse
    \(\displaystyle{\frac{x^2}{a^2}+\frac{y^2}{b^2}}=1\).
\end{problem}
\begin{solution}
    The area is \(\displaystyle{A=4\int_0^a\frac{b}{a}\sqrt{a^2-x^2}\,dx}\).
    Let \(x=a\sin\theta\iff dx=a\cos\theta\,d\theta\) and
    \(0\leq\theta\leq\pi/2\), then
    \[A=4\int_0^{\pi/2}\frac{b}{a}(a\cos\theta)(a\cos\theta)\,d\theta
    =4ab\int_0^{\pi/2}\cos^2\theta\,d\theta
    =4ab\left(\frac{\pi}{4}\right)=\pi ab\]
\end{solution}