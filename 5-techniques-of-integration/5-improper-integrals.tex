\subsection{Improper Integrals}
We extend the concept of a definite integral
\(\displaystyle{\int_a^b f(x)\,dx}\) to the case where the interval is
infinite and also to the case where \(f\) has an infinite discontinuity in
\([a,b]\).
In either case the integral is called an \textbf{improper integral}.

\subsubsection*{Infinite Intervals}
\begin{definition}
    \begin{enumerate}
        \item If \(\displaystyle{\int_a^t f(x)\,dx}\) exists for every number
        \(t\geq a\), then
        \[\int_a^\infty f(x)\,dx=\lim_{t\to\infty}\int_a^t f(x)\,dx\]
        provided this limit exists (as a finite number).
        \item If \(\displaystyle{\int_t^b f(x)\,dx}\) exists for every number
        \(t\leq b\), then
        \[\int_{-\infty}^b f(x)\,dx=\lim_{t\to -\infty}\int_t^b f(x)\,dx\]
        provided this limit exists (as a finite number).
    \end{enumerate}
    The improper integrals \(\displaystyle{\int_a^\infty f(x)\,dx}\) and
    \(\displaystyle{\int_{-\infty}^b f(x)\,dx}\) are called
    \textbf{convergent} if the corresponding limit exists and
    \textbf{divergent} if the limit does not exist. \\
    If both \(\displaystyle{\int_a^\infty f(x)\,dx}\) and
    \(\displaystyle{\int_{-\infty}^a f(x)\,dx}\) are convergent, then we define
    \[\int_{-\infty}^{\infty}f(x)\,dx
    =\int_{-\infty}^a f(x)\,dx+\int_a^\infty f(x)\,dx\]
    for any real number \(a\).
\end{definition}
\(\displaystyle{\int_1^\infty \frac{1}{x^p}\,dx}\) is convergent if \(p>1\)
and divergent if \(p\leq 1\).

\subsubsection*{Discontinuous Integrands}
\begin{definition}
    \begin{enumerate}
        \item If \(f\) is continuous on \([a,b)\) and is discontinuous at
        \(b\), then
        \[\int_a^b f(x)\,dx=\lim_{t\to b^-}\int_a^t f(x)\,dx\]
        if the limit exists.
        \item If \(f\) is continuous on \((a,b]\) and is discontinuous at
        \(a\), then
        \[\int_a^b f(x)\,dx=\lim_{t\to a^+}\int_t^b f(x)\,dx\]
        if the limit exists.
    \end{enumerate}
    The improper integral \(\displaystyle{\int_a^b f(x)\,dx}\) is
    \textbf{convergent} if the corresponding limit exists and
    \textbf{divergent} if the limit does not exist. \\
    If \(f\) has a discontinuity at \(c\), where \(a<c<b\), and both
    \(\displaystyle{\int_a^c f(x)\,dx}\) and
    \(\displaystyle{\int_c^b f(x)\,dx}\) are convergent,
    then
    \[\int_a^b f(x)\,dx=\int_a^c f(x)\,dx+\int_c^b f(x)\,dx\]
\end{definition}

\subsubsection*{Comparison Test for Improper Integrals}
\begin{theorem}[Comparison Theorem]
    Suppose that \(f\) and \(g\) are continuous functions with
    \(f(x)\geq g(x)\geq 0\) for \(x\geq a\).
    \begin{enumerate}
        \item If \(\displaystyle{\int_a^\infty f(x)\,dx}\) is convergent, then
        \(\displaystyle{\int_a^\infty g(x)\,dx}\) is convergent.
        \item If \(\displaystyle{\int_a^\infty g(x)\,dx}\) is divergent, then
        \(\displaystyle{\int_a^\infty f(x)\,dx}\) is divergent.
    \end{enumerate}    
\end{theorem}