\subsection{Integration by Parts}

Every differentiation rule has a corresponding integration rule.
For instance, the Substitution Rule for integration corresponds to the Chain
Rule for differentiation.
The rule that corresponds to the Product Rule for differentiation is called
the rule for integration by parts.
The Product Rule states that if \(f\) and \(g\) are differentiable functions,
then
\[\frac{d}{dx}\big[f(x)g(x)\big]=f'(x)g(x)+f(x)g'(x)\]
In the notation fo indefinite integrals this equation becomes
\[f(x)g(x)=\int\big[f'(x)g(x)+f(x)g'(x)\big]\,dx\]
or
\[\int f'(x)g(x)\,dx+\int f(x)g'(x)\,dx=f(x)g(x)\]
We can rearrange this equation as
\[\int f(x)g'(x)\,dx=f(x)g(x)-\int g(x)f'(x)\,dx\]
and this equation is called the \textbf{formula for integration by parts}.
Let \(u=f(x)\) and \(v=f(x)\).
Then the differentials are \(du=f'(x)\,dx\) and \(dv=g'(x)\,dx\), so, by the
Substitution Rule, the formula for integration by parts becomes
\[\int u\,dv=uv-\int v\,du\]

\begin{problem}
    Evaluate \(\displaystyle{\int\ln x\,dx}\).
\end{problem}
\begin{solution}
    Let \(u=\ln x\iff du=\dfrac{1}{x}\,dx\) and \(dv=dx\iff v=x\).
    Then
    \[\int\ln x\,dx=x\ln x-\int dx=x\ln x-x+C\]
\end{solution}
\begin{problem}
    Evaluate \(\displaystyle{\int e^x\sin x\,dx}\).
\end{problem}
\begin{solution}
    Let \(u=\sin x\iff du=\cos x\,dx\) and \(dv=e^x\,dx\iff v=e^x\).
    then
    \[\int e^x\sin x\,dx=e^x\sin x-\int e^x\cos x\,dx\]
    Let \(u=\cos x\iff du=-\sin x\,dx\) and \(dv=e^x\,dx\iff v=e^x\).
    Then
    \[\int e^x\cos x\,dx=e^x\cos x+\int e^x\sin x\,dx\]
    Therefore
    \begin{align*}
        \int e^x\sin x\,dx &= e^x\sin x-e^x\cos x-\int e^x\sin x\,dx \\
        \int e^x\sin x\,dx &= \frac{1}{2}e^x(\sin x-\cos x) 
    \end{align*}
\end{solution}
The formula of integration by parts for definite integrals is
\[\int_a^b f(x)g'(x)\,dx=\big[f(x)g(x)\big]_a^b-\int_a^b g(x)f'(x)\,dx\]
or
\[\int_a^b v\,du=\big[uv\big]_a^b-\int_a^b v\,du\]

\begin{problem}
    Evaluate \(\displaystyle{\int_0^1\arctan x\,dx}\).
\end{problem}
\begin{solution}
    Let \(u=\arctan x\iff du=\dfrac{dx}{1+x^2}\) and \(dv=dx\iff v=x\), then
    \[\int_0^1\arctan x\,dx
    =\big[x\arctan x\big]_0^1-\int_0^1\frac{x}{1+x^2}\,dx\]
    We have \(\big[x\arctan x\big]_0^1=\arctan(1)=\dfrac{\pi}{4}\).
    Let \(u=1+x^2\iff du=2x\,dx\) so \(x\,dx=\dfrac{1}{2}\,du\), then
    \[\int_0^1\frac{x}{1+x^2}\,dx=\frac{1}{2}\int_1^2\frac{du}{u}
    =\left[\frac{1}{2}\ln|u|\right]_1^2=\frac{1}{2}\ln(2)\]
    Therefore
    \[\int_0^1\arctan x\,dx=\frac{\pi}{4}-\frac{1}{2}\ln(2)\]
\end{solution}
\begin{problem}
    Prove the reduction formula
    \[\int\sin^n x\,dx
    =-\frac{1}{n}\sin^{n-1}x\cos x+\frac{n-1}{n}\int\sin^{n-2}x\,dx\]
    where \(n\geq 2\) is an integer.
\end{problem}
\begin{solution}
    Let \(u=\sin^{n-1}x\iff du=(n-1)\sin^{n-2}x\cos x\,dx\) and \\
    \(dv=\sin x\,dx\iff v=-\cos x\).
    Then
    \begin{align*}
        \int\sin^n x\,dx
        &= -\sin^{n-1}x\cos x+(n-1)\int \sin^{n-2}x\cos^2 x\,dx \\
        &= -\sin^{n-1}x\cos x+(n-1)\int \sin^{n-2}x(1-\sin^2 x)\,dx \\
        &= -\sin^{n-1}x\cos x+(n-1)\int \sin^{n-2}x\,dx-(n-1)\int\sin^n x\,dx
    \end{align*}
    Therefore
    \begin{align*}
        (n-1+1)\int\sin^n x\,dx=n\int\sin^n x\,dx
        &= -\sin^{n-1}x\cos x+(n-1)\int \sin^{n-2}x\,dx \\
        \int\sin^n x\,dx
        &= -\frac{1}{n}\sin^{n-1}x\cos x
        +\frac{n-1}{n}\int\sin^{n-2}x\,dx 
    \end{align*}
\end{solution}
\begin{problem}
    Prove the reduction formula
    \[\int\cos^n x\,dx
    =\frac{1}{n}\cos^{n-1}x\sin x+\frac{n-1}{n}\int\cos^{n-2}x\,dx\]
    where \(n\geq 2\) is an integer.
\end{problem}
\begin{solution}
    Let \(u=\cos^{n-1}x\iff du=-(n-1)\cos^{n-2}x\sin x\,dx\) and
    \(dv=\cos x\,dx\iff v=\sin x\).
    Then
    \begin{align*}
        \int\cos^n x\,dx
        &= \cos^{n-1} x\sin x+(n-1)\int\cos^{n-2}x\sin^2 x\,dx \\
        &= \cos^{n-1} x\sin x+(n-1)\int \cos^{n-2}x(1-\cos^2 x)\,dx \\
        &= \cos^{n-1} x\sin x+(n-1)\int \cos^{n-2}x\,dx-(n-1)\int\cos^n x\,dx
    \end{align*}
    Therefore
    \begin{align*}
        (n-1+1)\int\cos^n x\,dx=n\int\cos^n x\,dx
        &= \cos^{n-1}x\sin x+(n-1)\int \cos^{n-2}x\,dx \\
        \int\cos^n x\,dx
        &= \frac{1}{n}\cos^{n-1}x\sin x
        +\frac{n-1}{n}\int\cos^{n-2}x\,dx 
    \end{align*}
\end{solution}
\begin{problem}
    Show that
    \[\int_0^{\pi/2}\sin^n x\,dx=\frac{n-1}{n}\int_0^{\pi/2}\sin^{n-2}x\,dx\]
    where \(n\geq 2\) is an integer.
\end{problem}
\begin{solution}
    We use the reduction formula then
    \begin{align*}
        \int_0^{\pi/2}\sin^n x\,dx
        &= \left[-\frac{1}{n}\sin^{n-1} x\cos x\right]_0^{\pi/2}
        +\frac{n-1}{n}\int_0^{\pi/2}\sin^{n-2} x\,dx \\
        &= -\frac{1}{n}\left(\sin^{n-1}\left(\frac{\pi}{2}\right)
        \cos\left(\frac{\pi}{2}\right)-\sin^{n-1}(0)\cos(0)\right)
        +\frac{n-1}{n}\int_0^{\pi/2}\sin^{n-2} x\,dx
    \end{align*}
    and since \(\sin(0)=0\) and \(\cos(\pi/2)=0\) so
    \[\int_0^{\pi/2}\sin^n x\,dx=-\frac{1}{n}(0)
    +\frac{n-1}{n}\int_0^{\pi/2}\sin^{n-2} x\,dx
    =\frac{n-1}{n}\int_0^{\pi/2}\sin^{n-2} x\,dx\]
\end{solution}
\begin{problem}
    Show that
    \[\int_0^{\pi/2}\sin^{2n+1} x\,dx
    =\frac{2\cdot4\cdot6\cdot\cdots\cdot2n}
    {3\cdot5\cdot7\cdot\cdots\cdot(2n+1)}\]
    where \(n\geq 2\) is an integer.
\end{problem}
\begin{solution}
    By the reduction formula
    \begin{align*}
        \int_0^{\pi/2}\sin^{2n+1} x\,dx
        &= \frac{2n}{2n+1}\int_{0}^{\pi/2}\sin^{2n-1} x\,dx
        =\frac{(2n)(2n-2)}{(2n+1)(2n-1)}\int_{0}^{\pi/2}\sin^{2n-3} x\,dx \\
        &= \frac{(2n)(2n-2)\cdots(6)(4)(2)}{(2n+1)(2n-1)\cdots(7)(5)(3)}
        \int_{0}^{\pi/2}\sin x\,dx
    \end{align*}
    and since
    \[\int_{0}^{\pi/2}\sin x\,dx=\Big[-\cos x\Big]_0^{\pi/2}
    =-\cos\left(\frac{\pi}{2}\right)+\cos(0)=0+1=1\]
    hence
    \[\int_0^{\pi/2}\sin^{2n+1} x\,dx
    =\frac{2\cdot4\cdot6\cdot\cdots\cdot2n}
    {3\cdot5\cdot7\cdot\cdots\cdot(2n+1)}\]
\end{solution}
\begin{problem}
    Show that
    \[\int_0^{\pi/2}\sin^{2n} x\,dx
    =\frac{1\cdot3\cdot5\cdot\cdots\cdot(2n-1)}
    {2\cdot4\cdot6\cdot\cdots\cdot2n}\frac{\pi}{2}\]
    where \(n\geq 2\) is an integer.
\end{problem}
\begin{solution}
    \begin{align*}
        \int_0^{\pi/2}\sin^{2n} x\,dx
        &= \frac{2n-1}{2n}\int_0^{\pi/2}\sin^{2n-2} x\,dx
        =\frac{(2n-1)(2n-3)}{(2n)(2n-2)}\int_0^{\pi/2}\sin^{2n-4} x\,dx \\
        &= \frac{(2n-1)(2n-3)\cdots(5)(3)(1)}{(2n)(2n-2)\cdots(6)(4)(2)}
        \int_0^{\pi/2}\,dx=\frac{1\cdot3\cdot5\cdot\cdots\cdot(2n-1)}
        {2\cdot4\cdot6\cdot\cdots\cdot2n}\frac{\pi}{2} 
    \end{align*}
\end{solution}
\begin{problem}
    Prove the reduction formula
    \[\int(\ln x)^n\,dx=x(\ln x)^n-n\int(\ln x)^{n-1}\,dx\]
\end{problem}
\begin{solution}
    Let \(u=(\ln x)^n\iff du=\dfrac{n}{x}(\ln x)^{n-1}\) and \(dv=dx\iff v=x\)
    then
    \[\int(\ln x)^n\,dx=x(\ln x)^n-\int x\cdot\frac{n}{x}(\ln x)^{n-1}\,dx
    =x(\ln x)^n-n\int(\ln x)^{n-1}\,dx\]
\end{solution}
\begin{problem}
    Prove the reduction formula
    \[\int x^n e^x\,dx=x^n e^x-n\int x^{n-1}e^x\,dx\]
\end{problem}
\begin{solution}
    Let \(u=x^n\iff du=nx^{n-1}\,dx\) and \(dv=e^x\,dx\iff v=e^x\) then
    \[\int x^n e^x\,dx=x^n e^x-\int nx^{n-1}e^x\,dx
    =x^n e^x-n\int x^{n-1}e^x\,dx\]
\end{solution}