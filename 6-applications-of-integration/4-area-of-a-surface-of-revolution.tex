\subsection{Area of a Surface of Revolution}

A surface of revolution is formed when a curve is rotated about a line. \\
If \(f\) is positive and has a continuous derivative, we define the
\textbf{surface area} of the surface obtained by rotating the curve
\(y=f(x),a\leq x\leq b,\) about the \(x\)-axis as
\[S=\int_a^b 2\pi f(x)\sqrt{1+\big[f'(x)\big]^2}\,dx\]
In Leibniz's notation:
\[S=\int_a^b 2\pi y\sqrt{1+\left(\frac{dy}{dx}\right)^2}\,dx\]
If the curve is described as \(x=g(y),c\leq y\leq d\), then the formula for
surface area is
\[S=\int_c^d 2\pi y\sqrt{1+\left(\frac{dx}{dy}\right)^2}\,dy\]
These formulas can be summarized symbolically as
\[S=\int 2\pi y\,ds\]
For rotation about the \(y\)-axis, the surface area formula is
\[S=\int 2\pi x\,ds\]