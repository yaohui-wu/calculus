\subsection{Volumes}

\begin{definition}
    Let \(S\) be a solid that lies between \(x=a\) and \(x=b\).
    If the cross-sectional area of \(S\) in the plane \(P_x\), through \(x\)
    and perpendicular to the \(x\)-axis, is \(A(x)\), where \(A\) is an
    integrable function, then the \textbf{volume} of \(S\) is
    \[V=\lim_{\max\Delta x_i\to 0}\sum_{i=1}^n A(x_i^*)\Delta x_i
    =\int_a^b A(x)\,dx\]
\end{definition}
In general, we calculate the volume of a \textbf{solid of revolution} obtained
by revolving a region about a line by using the basic defining formula
\[V=\int_a^b A(x)\,dx\]
or
\[V=\int_c^d A(y)\,dy\]

\subsubsection*{Volumes by Cylindrical Shells}

The volume of the solid \(S\) obtained by rotating about the \(y\)-axis the
region under the curve \(y=f(x)\) from \(a\) to \(b\), is
\[V=\int_a^b 2\pi xf(x)\,dx\]
where \(0\leq a<b\).